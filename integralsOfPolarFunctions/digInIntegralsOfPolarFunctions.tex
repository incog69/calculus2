\documentclass{ximera}

%\usepackage{todonotes}

\newcommand{\todo}{}

\usepackage{tkz-euclide}
\tikzset{>=stealth} %% cool arrow head
\tikzset{shorten <>/.style={ shorten >=#1, shorten <=#1 } } %% allows shorter vectors

\usetikzlibrary{backgrounds} %% for boxes around graphs
\usetikzlibrary{shapes,positioning}  %% Clouds and stars
\usetikzlibrary{matrix} %% for matrix
\usepgfplotslibrary{polar} %% for polar plots
\usetkzobj{all}
\usepackage[makeroom]{cancel} %% for strike outs
%\usepackage{mathtools} %% for pretty underbrace % Breaks Ximera
\usepackage{multicol}





\usepackage{array}
\setlength{\extrarowheight}{+.1cm}   
\newdimen\digitwidth
\settowidth\digitwidth{9}
\def\divrule#1#2{
\noalign{\moveright#1\digitwidth
\vbox{\hrule width#2\digitwidth}}}





\newcommand{\RR}{\mathbb R}
\newcommand{\R}{\mathbb R}
\newcommand{\N}{\mathbb N}
\newcommand{\Z}{\mathbb Z}

%\renewcommand{\d}{\,d\!}
\renewcommand{\d}{\mathop{}\!d}
\newcommand{\dd}[2][]{\frac{\d #1}{\d #2}}
\renewcommand{\l}{\ell}
\newcommand{\ddx}{\frac{d}{\d x}}

\newcommand{\zeroOverZero}{\ensuremath{\boldsymbol{\tfrac{0}{0}}}}
\newcommand{\inftyOverInfty}{\ensuremath{\boldsymbol{\tfrac{\infty}{\infty}}}}
\newcommand{\zeroOverInfty}{\ensuremath{\boldsymbol{\tfrac{0}{\infty}}}}
\newcommand{\zeroTimesInfty}{\ensuremath{\small\boldsymbol{0\cdot \infty}}}
\newcommand{\inftyMinusInfty}{\ensuremath{\small\boldsymbol{\infty - \infty}}}
\newcommand{\oneToInfty}{\ensuremath{\boldsymbol{1^\infty}}}
\newcommand{\zeroToZero}{\ensuremath{\boldsymbol{0^0}}}
\newcommand{\inftyToZero}{\ensuremath{\boldsymbol{\infty^0}}}


\newcommand{\numOverZero}{\ensuremath{\boldsymbol{\tfrac{\#}{0}}}}
\newcommand{\dfn}{\textbf}
%\newcommand{\unit}{\,\mathrm}
\newcommand{\unit}{\mathop{}\!\mathrm}
\newcommand{\eval}[1]{\bigg[ #1 \bigg]}
\newcommand{\seq}[1]{\left( #1 \right)}
\renewcommand{\epsilon}{\varepsilon}
\renewcommand{\iff}{\Leftrightarrow}

\DeclareMathOperator{\arccot}{arccot}
\DeclareMathOperator{\arcsec}{arcsec}
\DeclareMathOperator{\arccsc}{arccsc}
\DeclareMathOperator{\si}{Si}

\newcommand{\tightoverset}[2]{%
  \mathop{#2}\limits^{\vbox to -.5ex{\kern-0.75ex\hbox{$#1$}\vss}}}
\newcommand{\arrowvec}[1]{\tightoverset{\scriptstyle\rightharpoonup}{#1}}
\renewcommand{\vec}{\mathbf}


\colorlet{textColor}{black} 
\colorlet{background}{white}
\colorlet{penColor}{blue!50!black} % Color of a curve in a plot
\colorlet{penColor2}{red!50!black}% Color of a curve in a plot
\colorlet{penColor3}{red!50!blue} % Color of a curve in a plot
\colorlet{penColor4}{green!50!black} % Color of a curve in a plot
\colorlet{penColor5}{orange!80!black} % Color of a curve in a plot
\colorlet{fill1}{penColor!20} % Color of fill in a plot
\colorlet{fill2}{penColor2!20} % Color of fill in a plot
\colorlet{fillp}{fill1} % Color of positive area
\colorlet{filln}{penColor2!20} % Color of negative area
\colorlet{fill3}{penColor3!20} % Fill
\colorlet{fill4}{penColor4!20} % Fill
\colorlet{fill5}{penColor5!20} % Fill
\colorlet{gridColor}{gray!50} % Color of grid in a plot

\newcommand{\surfaceColor}{violet}
\newcommand{\surfaceColorTwo}{redyellow}
\newcommand{\sliceColor}{greenyellow}




\pgfmathdeclarefunction{gauss}{2}{% gives gaussian
  \pgfmathparse{1/(#2*sqrt(2*pi))*exp(-((x-#1)^2)/(2*#2^2))}%
}


%%%%%%%%%%%%%
%% Vectors
%%%%%%%%%%%%%

%% Simple horiz vectors
\renewcommand{\vector}[1]{\left\langle #1\right\rangle}


%% %% Complex Horiz Vectors with angle brackets
%% \makeatletter
%% \renewcommand{\vector}[2][ , ]{\left\langle%
%%   \def\nextitem{\def\nextitem{#1}}%
%%   \@for \el:=#2\do{\nextitem\el}\right\rangle%
%% }
%% \makeatother

%% %% Vertical Vectors
%% \def\vector#1{\begin{bmatrix}\vecListA#1,,\end{bmatrix}}
%% \def\vecListA#1,{\if,#1,\else #1\cr \expandafter \vecListA \fi}

%%%%%%%%%%%%%
%% End of vectors
%%%%%%%%%%%%%

%\newcommand{\fullwidth}{}
%\newcommand{\normalwidth}{}



%% makes a snazzy t-chart for evaluating functions
%\newenvironment{tchart}{\rowcolors{2}{}{background!90!textColor}\array}{\endarray}

%%This is to help with formatting on future title pages.
\newenvironment{sectionOutcomes}{}{} 



%% Flowchart stuff
%\tikzstyle{startstop} = [rectangle, rounded corners, minimum width=3cm, minimum height=1cm,text centered, draw=black]
%\tikzstyle{question} = [rectangle, minimum width=3cm, minimum height=1cm, text centered, draw=black]
%\tikzstyle{decision} = [trapezium, trapezium left angle=70, trapezium right angle=110, minimum width=3cm, minimum height=1cm, text centered, draw=black]
%\tikzstyle{question} = [rectangle, rounded corners, minimum width=3cm, minimum height=1cm,text centered, draw=black]
%\tikzstyle{process} = [rectangle, minimum width=3cm, minimum height=1cm, text centered, draw=black]
%\tikzstyle{decision} = [trapezium, trapezium left angle=70, trapezium right angle=110, minimum width=3cm, minimum height=1cm, text centered, draw=black]


\outcome{Find the area of a region bound by a polar curve.}
\outcome{Find the intersection points of two polar curves.}
\outcome{Find the area of a region bound by two polar curves.} 

\title[Dig-In:]{Integrals of polar functions}

\begin{document}
\begin{abstract}
  We integrate polar functions.
\end{abstract}
\maketitle


When using rectangular coordinates, the equations $x=h$ and $y=k$ defined vertical and horzontal lines, respectively, and combinations of these lines create rectangles (hence the name ``rectangular coordinates''). It is then somewhat natural to use rectangles to approximate area as we did when learning about the definite integral.\index{polar!functions!area}

When using polar coordinates, the equations $\theta=\alpha$ and $r=c$ form lines through the origin and circles centered at the origin, respectively, and combinations of these curves form sectors of circles. It is then somewhat natural to calculate the area of regions defined by polar functions by first approximating with sectors of circles. 

Consider Figure \ref{fig:polararea} (a) where a region defined by $r=f(\theta)$ on $[\alpha,\beta]$ is given. (Note how the ``sides'' of the region are the lines $\theta=\alpha$ and $\theta=\beta$, whereas in rectangular coordinates the ``sides'' of regions were often the vertical lines $x=a$ and $x=b$.)
\mnote{.8}{\textbf{Note:} Recall that the area of a sector of a circle with radius $r$ subtended by an angle $\theta$ is $A = \frac12\theta r^2$.

\hfill
\begin{tikzpicture}[x=30pt,y=30pt,thick]
			\draw (2,0) arc (0:50:2) -- (0,0);
			\draw [] (0,0) -- (2,0) node [pos=.5,below] {$r$};
			\draw [fill=black] (0,0) circle (1pt);
			%\draw (1.95,1.0) node {$s$};
			\draw (0,0) node [shift={(15pt,8pt)}] {$\theta$};
			\end{tikzpicture}
\hfill\null
}

Partition the interval $[\alpha,\beta]$ into $n$ equally spaced subintervals as $\alpha = \theta_1 < \theta_2 <\cdots <\theta_{n+1}=\beta$. The length of each subinterval is $\Delta\theta = (\beta-\alpha)/n$, representing a small change in angle. The area of the region defined by the $i\,^\text{th}$ subinterval $[\theta_i,\theta_{i+1}]$ can be approximated with a sector of a circle with radius $f(c_i)$, for some $c_i$ in $[\theta_i,\theta_{i+1}]$. The area of this sector is $\frac12f(c_i)^2\Delta\theta$. This is shown in part (b) of the figure, where $[\alpha,\beta]$ has been divided into 4 subintervals. We approximate the area of the whole region by summing the areas of all sectors:
$$\text{Area} \approx \sum_{i=1}^n \frac12f(c_i)^2\Delta\theta.$$
This is a Riemann sum. By taking the limit of the sum as $n\to\infty$, we find the exact area of the region in the form of a definite integral.
\mtable{.5}{Computing the area of a polar region.}{fig:polararea}{%
\begin{tabular}{c}
\myincludegraphics{figures/figpolarea1}\\
(a)\\[10pt]
\myincludegraphics{figures/figpolarea2}\\
(b)
\end{tabular}
}
\enlargethispage{2\baselineskip}
%\clearpage

\theorem{thm:polar_area}{Area of a Polar Region}
{Let $f$ be continuous and non-negative on $[\alpha,\beta]$, where $0\leq \beta-\alpha\leq 2\pi$. The area  $A$ of the region bounded by the curve $r=f(\theta)$ and the lines $\theta=\alpha$ and $\theta=\beta$ is 
$$
A \ =\  \frac12\int_\alpha^\beta f(\theta)^2 \ d\theta\  =\  \frac12\int_\alpha^\beta r^{\,2} \ d\theta$$
}

The theorem states that $0\leq \beta-\alpha\leq 2\pi$. This ensures that region does not overlap itself, giving a result that does not correspond directly to the area.\\

\example{ex_polcalc3}{Area of a polar region}{
Find the area of the circle defined by $r=\cos \theta$.}
{This is a direct application of Theorem \ref{thm:polar_area}. The circle is traced out on $[0,\pi]$, leading to the integral
\begin{align*}
\text{Area} &= \frac12\int_0^\pi \cos^2\theta\ d  \theta \\
						&= \frac12\int_0^\pi \frac{1+\cos(2\theta)}{2}\ d\theta\\
						&= \frac14\big(\theta +\frac12\sin(2\theta)\big)\Bigg|_0^\pi\\
						&= \frac14\pi.
\end{align*}
Of course, we already knew the area of a circle with radius $1/2$. We did this example to demonstrate that the area formula is correct.
}\\

\mnote{.7}{\textbf{Note:} Example \ref{ex_polcalc3} requires the use of the integral $\ds\int \cos^2\theta\ d\theta$. This is handled well by using the power reducing formula as found in the back of this text. Due to the nature of the area formula, integrating $\cos^2\theta$ and $\sin^2\theta$ is required often. We offer here these indefinite integrals as a time--saving measure.\\
$$\int\cos^2\theta\ d\theta = \frac12\theta+\frac14\sin(2\theta)+C$$
$$\int\sin^2\theta\ d\theta = \frac12\theta-\frac14\sin(2\theta)+C$$
}

\example{ex_polcalc4}{Area of a polar region}{
Find the area of the cardiod $r=1+\cos\theta$ bound between $\theta=\pi/6$ and $\theta=\pi/3$, as shown in Figure \ref{fig:polcalc4}.
\mfigure{.3}{Finding the area of the shaded region of a cardiod in Example \ref{ex_polcalc4}.}{fig:polcalc4}{figures/figpolcalc4}
}
{This is again a direct appliation of Theorem \ref{thm:polar_area}. 
\begin{align*}
\text{Area} &= \frac12\int_{\pi/6}^{\pi/3} (1+\cos\theta)^2\ d\theta\\
				&= \frac12\int_{\pi/6}^{\pi/3} (1+2\cos\theta+\cos^2\theta)\ d\theta\\
				&= \frac12\left(\theta+2\sin\theta+\frac12\theta+\frac14\sin(2\theta)\right)\Bigg|_{\pi/6}^{\pi/3} \\
				&= \frac18\big(\pi+4\sqrt{3}-4\big) \approx 0.7587.
				\end{align*}
\vskip-\baselineskip
}\\

\noindent\textbf{Area Between Curves}\\

Our study of area in the context of rectangular functions led naturally to finding area bounded between curves. We consider the same in the context of polar functions. \index{polar!functions!area between curves}

Consider the shaded region shown in Figure \ref{fig:polarea3}. We can find the area of this region by computing the area bounded by $r_2=f_2(\theta)$ and subtracting the area bounded by $r_1=f_1(\theta)$ on $[\alpha,\beta]$. Thus
$$\text{Area}\ = \ \frac12\int_\alpha^\beta r_2^{\,2}\ d\theta - \frac12\int_\alpha^\beta r_1^{\,2}\ d\theta = \frac12\int_\alpha^\beta \big(r_2^{\,2}-r_1^{\,2}\big)\ d\theta.$$
\mfigure{.6}{Illustrating area bound between two polar curves.}{fig:polarea3}{figures/figpolarea3}

\keyidea{idea:area_between_polar}{Area Between Polar Curves}
{The area $A$ of the region bounded by $r_1=f_1(\theta)$ and $r_2=f_2(\theta)$, $\theta=\alpha$ and $\theta=\beta$, where $f_1(\theta)\leq f_2(\theta)$ on $[\alpha,\beta]$, is
$$A = \frac12\int_\alpha^\beta \big(r_2^{\,2}-r_1^{\,2}\big)\ d\theta.$$
}

\enlargethispage{2\baselineskip}
\example{ex_polcalc5}{Area between polar curves}{
Find the area bounded between the curves $r=1+\cos \theta$ and $r=3\cos\theta$, as shown in Figure \ref{fig:polcalc5}.
\mfigure{.3}{Finding the area between polar curves in Example \ref{ex_polcalc5}.}{fig:polcalc5}{figures/figpolcalc5}
}
{We need to find the points of intersection between these two functions. Setting them equal to each other, we find:
\begin{align*}
1+\cos\theta &= 3\cos \theta \\
 \cos\theta &=1/2\\
\theta &= \pm \pi/3
\end{align*}
Thus we integrate $\frac12\big((3\cos\theta)^2-(1+\cos\theta)^2\big)$ on $[-\pi/3,\pi/3]$.
\begin{align*}
\text{Area} &= \frac12\int_{-\pi/3}^{\pi/3} \big((3\cos\theta)^2-(1+\cos\theta)^2\big)\ d\theta\\
		&= \frac12\int_{-\pi/3}^{\pi/3} \big( 8\cos^2\theta-2\cos\theta-1\big)\ d\theta \\
		&= \big(2\sin(2\theta) - 2\sin\theta+3\theta\big)\Bigg|_{-\pi/3}^{\pi/3}\\
		&= 2\pi.
\end{align*}
Amazingly enough, the area between these curves has a ``nice'' value.
}\\

\example{ex_polcalc6}{Area defined by polar curves}{
Find the area bounded between the polar curves $r=1$ and $r=2\cos(2\theta)$, as shown in Figure \ref{fig:polcalc6} (a).}
{We need to find the point of intersection between the two curves. Setting the two functions equal to each other, we have
$$2\cos(2\theta) = 1 \quad \Rightarrow \quad \cos(2\theta) = \frac12 \quad \Rightarrow \quad 2\theta = \pi/3\quad \Rightarrow \quad \theta=\pi/6.$$
\mtable{.4}{Graphing the region bounded by the functions in Example \ref{ex_polcalc6}.}{fig:polcalc6}{%
\begin{tabular}{c}
\myincludegraphics{figures/figpolcalc6}\\
(a)\\[10pt]
\myincludegraphics{figures/figpolcalc6a}\\
(b)\\
\end{tabular}
}
In part (b) of the figure, we zoom in on the region and note that it is not really bounded \textit{between} two polar curves, but rather \textit{by} two polar curves, along with $\theta=0$. The dashed line breaks the region into its component parts. Below the dashed line, the region is defined by $r=1$, $\theta=0$ and $\theta = \pi/6$. (Note: the dashed line lies on the line $\theta=\pi/6$.) Above the dashed line the region is bounded by $r=2\cos(2\theta)$ and $\theta =\pi/6$. Since we have two separate regions, we find the area using two separate integrals.

Call the area below the dashed line $A_1$ and the area above the dashed line $A_2$. They are determined by the following integrals:
$$A_1 = \frac12\int_0^{\pi/6} (1)^2\ d\theta\qquad  A_2 = \frac12\int_{\pi/6}^{\pi/4} \big(2\cos(2\theta)\big)^2\ d\theta.$$
(The upper bound of the integral computing $A_2$ is $\pi/4$ as $r=2\cos(2\theta)$ is at the pole when $\theta=\pi/4$.)

We omit the integration details and let the reader verify that $A_1 = \pi/12$ and $A_2 = \pi/12-\sqrt{3}/8$; the total area is $A = \pi/6-\sqrt{3}/8$.
}\\


\end{document}
















