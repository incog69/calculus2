\documentclass{ximera}

%\usepackage{todonotes}

\newcommand{\todo}{}

\usepackage{tkz-euclide}
\tikzset{>=stealth} %% cool arrow head
\tikzset{shorten <>/.style={ shorten >=#1, shorten <=#1 } } %% allows shorter vectors

\usetikzlibrary{backgrounds} %% for boxes around graphs
\usetikzlibrary{shapes,positioning}  %% Clouds and stars
\usetikzlibrary{matrix} %% for matrix
\usepgfplotslibrary{polar} %% for polar plots
\usetkzobj{all}
\usepackage[makeroom]{cancel} %% for strike outs
%\usepackage{mathtools} %% for pretty underbrace % Breaks Ximera
\usepackage{multicol}





\usepackage{array}
\setlength{\extrarowheight}{+.1cm}   
\newdimen\digitwidth
\settowidth\digitwidth{9}
\def\divrule#1#2{
\noalign{\moveright#1\digitwidth
\vbox{\hrule width#2\digitwidth}}}





\newcommand{\RR}{\mathbb R}
\newcommand{\R}{\mathbb R}
\newcommand{\N}{\mathbb N}
\newcommand{\Z}{\mathbb Z}

%\renewcommand{\d}{\,d\!}
\renewcommand{\d}{\mathop{}\!d}
\newcommand{\dd}[2][]{\frac{\d #1}{\d #2}}
\renewcommand{\l}{\ell}
\newcommand{\ddx}{\frac{d}{\d x}}

\newcommand{\zeroOverZero}{\ensuremath{\boldsymbol{\tfrac{0}{0}}}}
\newcommand{\inftyOverInfty}{\ensuremath{\boldsymbol{\tfrac{\infty}{\infty}}}}
\newcommand{\zeroOverInfty}{\ensuremath{\boldsymbol{\tfrac{0}{\infty}}}}
\newcommand{\zeroTimesInfty}{\ensuremath{\small\boldsymbol{0\cdot \infty}}}
\newcommand{\inftyMinusInfty}{\ensuremath{\small\boldsymbol{\infty - \infty}}}
\newcommand{\oneToInfty}{\ensuremath{\boldsymbol{1^\infty}}}
\newcommand{\zeroToZero}{\ensuremath{\boldsymbol{0^0}}}
\newcommand{\inftyToZero}{\ensuremath{\boldsymbol{\infty^0}}}


\newcommand{\numOverZero}{\ensuremath{\boldsymbol{\tfrac{\#}{0}}}}
\newcommand{\dfn}{\textbf}
%\newcommand{\unit}{\,\mathrm}
\newcommand{\unit}{\mathop{}\!\mathrm}
\newcommand{\eval}[1]{\bigg[ #1 \bigg]}
\newcommand{\seq}[1]{\left( #1 \right)}
\renewcommand{\epsilon}{\varepsilon}
\renewcommand{\iff}{\Leftrightarrow}

\DeclareMathOperator{\arccot}{arccot}
\DeclareMathOperator{\arcsec}{arcsec}
\DeclareMathOperator{\arccsc}{arccsc}
\DeclareMathOperator{\si}{Si}

\newcommand{\tightoverset}[2]{%
  \mathop{#2}\limits^{\vbox to -.5ex{\kern-0.75ex\hbox{$#1$}\vss}}}
\newcommand{\arrowvec}[1]{\tightoverset{\scriptstyle\rightharpoonup}{#1}}
\renewcommand{\vec}{\mathbf}


\colorlet{textColor}{black} 
\colorlet{background}{white}
\colorlet{penColor}{blue!50!black} % Color of a curve in a plot
\colorlet{penColor2}{red!50!black}% Color of a curve in a plot
\colorlet{penColor3}{red!50!blue} % Color of a curve in a plot
\colorlet{penColor4}{green!50!black} % Color of a curve in a plot
\colorlet{penColor5}{orange!80!black} % Color of a curve in a plot
\colorlet{fill1}{penColor!20} % Color of fill in a plot
\colorlet{fill2}{penColor2!20} % Color of fill in a plot
\colorlet{fillp}{fill1} % Color of positive area
\colorlet{filln}{penColor2!20} % Color of negative area
\colorlet{fill3}{penColor3!20} % Fill
\colorlet{fill4}{penColor4!20} % Fill
\colorlet{fill5}{penColor5!20} % Fill
\colorlet{gridColor}{gray!50} % Color of grid in a plot

\newcommand{\surfaceColor}{violet}
\newcommand{\surfaceColorTwo}{redyellow}
\newcommand{\sliceColor}{greenyellow}




\pgfmathdeclarefunction{gauss}{2}{% gives gaussian
  \pgfmathparse{1/(#2*sqrt(2*pi))*exp(-((x-#1)^2)/(2*#2^2))}%
}


%%%%%%%%%%%%%
%% Vectors
%%%%%%%%%%%%%

%% Simple horiz vectors
\renewcommand{\vector}[1]{\left\langle #1\right\rangle}


%% %% Complex Horiz Vectors with angle brackets
%% \makeatletter
%% \renewcommand{\vector}[2][ , ]{\left\langle%
%%   \def\nextitem{\def\nextitem{#1}}%
%%   \@for \el:=#2\do{\nextitem\el}\right\rangle%
%% }
%% \makeatother

%% %% Vertical Vectors
%% \def\vector#1{\begin{bmatrix}\vecListA#1,,\end{bmatrix}}
%% \def\vecListA#1,{\if,#1,\else #1\cr \expandafter \vecListA \fi}

%%%%%%%%%%%%%
%% End of vectors
%%%%%%%%%%%%%

%\newcommand{\fullwidth}{}
%\newcommand{\normalwidth}{}



%% makes a snazzy t-chart for evaluating functions
%\newenvironment{tchart}{\rowcolors{2}{}{background!90!textColor}\array}{\endarray}

%%This is to help with formatting on future title pages.
\newenvironment{sectionOutcomes}{}{} 



%% Flowchart stuff
%\tikzstyle{startstop} = [rectangle, rounded corners, minimum width=3cm, minimum height=1cm,text centered, draw=black]
%\tikzstyle{question} = [rectangle, minimum width=3cm, minimum height=1cm, text centered, draw=black]
%\tikzstyle{decision} = [trapezium, trapezium left angle=70, trapezium right angle=110, minimum width=3cm, minimum height=1cm, text centered, draw=black]
%\tikzstyle{question} = [rectangle, rounded corners, minimum width=3cm, minimum height=1cm,text centered, draw=black]
%\tikzstyle{process} = [rectangle, minimum width=3cm, minimum height=1cm, text centered, draw=black]
%\tikzstyle{decision} = [trapezium, trapezium left angle=70, trapezium right angle=110, minimum width=3cm, minimum height=1cm, text centered, draw=black]


\title[Dig-In:]{Trigonometric substitution}

\begin{document}
\begin{abstract}
  Quadratic functions can integrated by substitution with the
  appropriate trigonometric function.
\end{abstract}
\maketitle



One useful model of the definite integral is that of a ``machine''
that computes ``signed area under the curve.'' In particular, this
metaphore allows us to compute
\[
\int_{-1}^1 \sqrt{1-x^2} \d x
\]
via geometry. Check out a graph of $y= \sqrt{1-x^2}$:
\begin{image}
  \begin{tikzpicture}
    \begin{axis}[
        width=6in,
        %height=3in,
        unit vector ratio*=1 1 1,            
        xmin=-1.1, xmax=1.1,ymin=-.1,ymax=1.1,
        axis lines =center, xlabel=$x$, ylabel=$y$,
        every axis y label/.style={at=(current axis.above origin),anchor=south},
        every axis x label/.style={at=(current axis.right of origin),anchor=west},
        axis on top,
      ] 
      \addplot [draw=none, fill=fillp,samples=200,domain=-1:1] {sqrt(1-x^2)} \closedcycle;
      
      \addplot [penColor,very thick,samples=200,domain=-1:1] {sqrt(1-x^2)};
    \end{axis}
  \end{tikzpicture}
\end{image}

\begin{question}
  Using the graph above (and basic facts from geometry)
  \[
  \int_{-1}^1 \sqrt{1-x^2} \d x = \answer{\pi/2}
  \]
\end{question}

We have since learned a number of integration techniques, including
substitution and integration by parts, yet we are still unable to
evaluate the above integral without resorting to a geometric
interpretation.

This section introduces \dfn{trigonometric substitution}, a method of
integration that fills this gap in our integration skill.

It will apply more generally to most integrals which involve a quadratic function of $x$.

This technique works on the same principle as substitution. However,
whereas with substitution we were typically transforming the integral
left to right in the diagram below

\begin{image}
  \begin{tikzpicture}[scale=1,every node/.style={transform shape}]
    \draw [->, line width=10, penColor!10!background] (1,0)--(-0.5,0);
    \node at (0,0) {
      $\int_a^b f'(g(x)) g'(x) \d x =\int_{g(a)}^{g(b)} f'(g) \d g$
    };
  \end{tikzpicture}
\end{image} 

\begin{image}
  \begin{tikzpicture}[scale=1,every node/.style={transform shape}]
    \draw [->, line width=10, penColor!10!background] (-.5,0)--(1,0);
    \node at (0,0) {
      $\int_a^b f'(g(x)) g'(x) \d x =\int_{g(a)}^{g(b)} f'(g) \d g$
    };
  \end{tikzpicture}
\end{image}

We start by demonstrating this method in evaluating the integral
above by computing the antiderivative of $\sqrt{1-x^2}$:

\begin{example}
  Compute:
  \[
  \int_{-1}^1 \sqrt{1-x^2} \d x
  \]
  \begin{explanation}
  
  Evaluate $\int_{-1}^1\sqrt{1-x^2}\d x$. We begin by noting that
    $\sin^2\theta + \cos^2\theta = 1$, and hence $\cos^2\theta =
    1-\sin^2\theta$. If we let $x=\sin\theta$, then $1-x^2 =
    1-\sin^2\theta = \cos^2\theta$.

Setting $x=\sin \theta$ gives $\d x = 3\cos\theta\ \d \theta$. We are
almost ready to substitute. We also wish to change our bounds of
integration. The bound $x=-1$ corresponds to $\theta = -\pi/2$ (for
when $\theta = -\pi/2$, $x=\sin \theta = -1$). Likewise, the bound of
$x=1$ is replaced by the bound $\theta = \pi/2$. Thus

\begin{align*}
  \int_{-1}^1\sqrt{1-x^2} \d x &= \int_{-\pi/2}^{\pi/2} \sqrt{1-\sin^2\theta} (\cos\theta)\ \d \theta \\
  &= \int_{-\pi/2}^{\pi/2} 3\sqrt{\cos^2\theta} \cos\theta\ \d \theta \\
  &=\int_{-\pi/2}^{\pi/2} |\cos \theta| \cos\theta\ d\theta.
\end{align*}

On $[-\pi/2,\pi/2]$, $\cos \theta$ is always positive, so we can drop the absolute value bars, then employ a half angle identity

    \begin{align*}
      &= \int_{-\pi/2}^{\pi/2} \cos^2 \theta \d \theta \\
      &= \int_{-\pi/2}^{\pi/2} \frac{1}{2}\big(1+\cos(2\theta)\big) \d \theta\\
      &= \frac{1}{2} \eval{\theta +\frac{1}{2}\sin(2\theta)}_{-\pi/2}^{\pi/2}\\
      &=\frac{\pi}{2}
    \end{align*}
    
    This matches our answer from before.
  \end{explanation}
\end{example}


We now describe in detail Trigonometric Substitution. This method excels when dealing with integrands that contain a quadratic function.   For instance, each of the following integrals is a prime candidate for Trig Sub:

\begin{itemize}
\item $\int \frac{1}{(2+4x+3x^2)^5} \d x$
\item $\int \frac{x^2}{\sqrt{9-x^2}} \d x$
\item $\int \sqrt{1+x^2} \d x$
\end{itemize}

The key idea is always to complete the square, and then choose a trig function which allows the use of a pythagorean identity to simplify the expression. 

\begin{example}

\[
\int_{-\frac{1}{2}}^1 \frac{1}{\sqrt{4x^2+4x+10}} \d x
\]

First we complete the square on the quadratic:

\begin{align*}
	4x^2+4x+10 &= 4(x^2+x)+10\\
		&=4(x^2+x+\frac{1}{4} - \frac{1}{4})+10\\
		&=4(\answer{x+\frac{1}{2}})^2+9\\
		&=(\answer{2x+1})^2+9
\end{align*}

Now we would like to choose to let $2x+1$ equal some trig function so that $(2x+1)^2+9$ will simplify after a Pythagorean identity.  We know that $1+\tan^2(\theta) = \sec^2(\theta)$, so a wise choice would be $2x+1 = 3\tan(\theta)$.  In this case we have 

\[
\d x = \answer{\frac{3}{2} \sec^2(\theta)} \d \theta
\]

The lower and upper limits of integration after this substitution will be 

\[
\theta = \answer{0} \textrm{ to } \theta = \answer{\frac{\pi}{2}}
\]

So making this substitution we have

\begin{align*}
\int_{-\frac{1}{2}}^1 \frac{1}{\sqrt{4x^2+4x+10}} \d x &= \int_{-\frac{1}{2}}^1 \frac{1}{\sqrt{(2x+1)^2+9}} \d x\\
&= \int_0^\frac{\pi}{2} \frac{1}{\sqrt{(3\tan(\theta))^2+9}} \sec^2(\theta) \d \theta\\
&= \int_0^\frac{\pi}{2} \frac{1}{3\sqrt{\tan^2(\theta)+1}} \sec^2(\theta) \d \theta\\
&= \int_0^\frac{\pi}{2} \frac{1}{3\sqrt{\tan^2(\theta)+1}} \sec^2(\theta) \d \theta\\
&= \int_0^\frac{\pi}{2} \frac{1}{3\sqrt{\sec^2(\theta)}} \sec^2(\theta) \d \theta\\
&= \int_0^\frac{\pi}{2} \frac{1}{3 \left| \sec(\theta) \right|} \sec^2(\theta) \d \theta\\
&= \int_0^\frac{\pi}{2} \frac{1}{3 \sec(\theta)} \sec^2(\theta) \d \theta \textrm{ b/c $\sec{\theta} > 0$ on $[0,\frac{\pi}{2}]$}\\
&= \int_0^\frac{\pi}{2} \frac{1}{3} \sec(\theta) \d \theta\\
&= \frac{1}{3} \eval{\ln(\left| \answer{\sec(\theta)+\tan(\theta)} \right| }_0^\frac{\pi}{2}\\
&=\frac{1}{3} ( \ln(\sqrt{2}+1) - \ln(1))\\
&=\frac{\ln(\sqrt{2}+1)}{3}
\end{align*}
 
\end{example}

In the last example, we choose to use a substitution involving $\tan$ because our quadratic looked like $a^2+u^2$ after completing the square, so $ u = a\tan(\theta)$ gave us a nice Pythagorean identity.

In general:

\begin{itemize}
\item If the quadratic has the form $a^2 + u^2$ after completing the square, make the substitution $u = a\tan(\theta)$
\item If the quadratic has the form $a^2 - u^2$ after completing the square, make the substitution $u = a\sin(\theta)$ 
\item If the quadratic has the form $u^2 - a^2$ after completing the square, make the substitution $u = a\sec(\theta)$
\end{itemize}

(The choices $a\cot(\theta)$,  $a\cos(\theta)$, and $a \csc(\theta)$ could be used with equal efficiency, but there is no reason to ever use them.)

Let's try another practice problem, this time an indefinite integral:


\begin{example}
	\[
	\int \frac{1}{x(4x^2 - 25)}
	\]
	
	Here, the quadratic term is already in the appropriate form $(2x)^2-5^2$.  According to the bullet points above, we should make the substitution
	
	\[
	2x = \answer{5\sec(\theta)}
	\]
	
	\begin{hint}
		$2x = 5\sec(\theta)$
	\end{hint}
	
	When we make this substitution, we will have
	
	\[
	\d x = \answer{\frac{5}{2} \sec(\theta) \tan(\theta)} \d \theta
	\]
	
	So we have
	
	\begin{align*}
		\int \frac{1}{x(4x^2 - 25)} &= \int \frac{1}{\frac{5}{2} \sec(\theta) (25\sec^2(\theta) - 25)}\frac{5}{2} \sec(\theta) \tan(\theta) \d \theta \\
				&=\frac{1}{25} \int \frac{\tan(\theta)}{\answer{\tan^2(\theta)}} \d \theta\\
				&=\frac{1}{25} \int \cot(\theta) \d \theta\\
				&=\frac{1}{25} \ln(\left| \sin(\theta) \right| )+C
	\end{align*}
	
	Now all the integration is done, but we have to express out answer in terms of $x$, not $\theta$.  
	
	Remember that $\sec(\theta)  =\frac{2}{5} x$. So let's draw a right triangle to figure out $\sin(\theta)$ in terms of $x$.
	
	\begin{image}
      \begin{tikzpicture}

        
        \draw[penColor, very thick] (0,0) -- (3,0);
        \draw[penColor, very thick] (3,0) -- (3,4);
        \draw[penColor, very thick] (0,0) -- (3,4);
        \node [textColor] at (1.4,-0.3) {$1$};
        \node [textColor] at (1.4,2.5) {$\frac{2x}{5}$};
        \node [textColor] at (0.5,0.3) {$\theta$};
        \node [textColor] at (3.8,2) {Opposite};
      \end{tikzpicture}
    \end{image}

	We want to find $\sin(\theta)$ in terms of $x$.  So the first thing is to find the length of the opposite to $\theta$.  By the pythagorean theorem, we have
	
	\[
	\textrm{Opposite} =  \answer{\sqrt{\frac{4x^2}{25}-1}}
	\]
	
	So after a bit of algebra
	
	\[
	\sin(\theta) = \frac{1}{x} \sqrt{x^2 - \frac{25}{4}}
	\]
	
	Thus the integral is (after yet more algebra)
	
	\[
	\int \frac{1}{x(4x^2 - 25)} \d x = \frac{1}{50} \ln(\left| \frac{4x^2-25}{x^2} \right|)+C
	\]
	
	\begin{warning}
		Note that the absolute value signs are crucially important here.  Without them, the antiderivative would not even be defined at at $x=1$, for example.
	\end{warning}
\end{example}

\end{document}}




\end{document}
