\documentclass{ximera}

%\usepackage{todonotes}

\newcommand{\todo}{}

\usepackage{tkz-euclide}
\tikzset{>=stealth} %% cool arrow head
\tikzset{shorten <>/.style={ shorten >=#1, shorten <=#1 } } %% allows shorter vectors

\usetikzlibrary{backgrounds} %% for boxes around graphs
\usetikzlibrary{shapes,positioning}  %% Clouds and stars
\usetikzlibrary{matrix} %% for matrix
\usepgfplotslibrary{polar} %% for polar plots
\usetkzobj{all}
\usepackage[makeroom]{cancel} %% for strike outs
%\usepackage{mathtools} %% for pretty underbrace % Breaks Ximera
\usepackage{multicol}





\usepackage{array}
\setlength{\extrarowheight}{+.1cm}   
\newdimen\digitwidth
\settowidth\digitwidth{9}
\def\divrule#1#2{
\noalign{\moveright#1\digitwidth
\vbox{\hrule width#2\digitwidth}}}





\newcommand{\RR}{\mathbb R}
\newcommand{\R}{\mathbb R}
\newcommand{\N}{\mathbb N}
\newcommand{\Z}{\mathbb Z}

%\renewcommand{\d}{\,d\!}
\renewcommand{\d}{\mathop{}\!d}
\newcommand{\dd}[2][]{\frac{\d #1}{\d #2}}
\renewcommand{\l}{\ell}
\newcommand{\ddx}{\frac{d}{\d x}}

\newcommand{\zeroOverZero}{\ensuremath{\boldsymbol{\tfrac{0}{0}}}}
\newcommand{\inftyOverInfty}{\ensuremath{\boldsymbol{\tfrac{\infty}{\infty}}}}
\newcommand{\zeroOverInfty}{\ensuremath{\boldsymbol{\tfrac{0}{\infty}}}}
\newcommand{\zeroTimesInfty}{\ensuremath{\small\boldsymbol{0\cdot \infty}}}
\newcommand{\inftyMinusInfty}{\ensuremath{\small\boldsymbol{\infty - \infty}}}
\newcommand{\oneToInfty}{\ensuremath{\boldsymbol{1^\infty}}}
\newcommand{\zeroToZero}{\ensuremath{\boldsymbol{0^0}}}
\newcommand{\inftyToZero}{\ensuremath{\boldsymbol{\infty^0}}}


\newcommand{\numOverZero}{\ensuremath{\boldsymbol{\tfrac{\#}{0}}}}
\newcommand{\dfn}{\textbf}
%\newcommand{\unit}{\,\mathrm}
\newcommand{\unit}{\mathop{}\!\mathrm}
\newcommand{\eval}[1]{\bigg[ #1 \bigg]}
\newcommand{\seq}[1]{\left( #1 \right)}
\renewcommand{\epsilon}{\varepsilon}
\renewcommand{\iff}{\Leftrightarrow}

\DeclareMathOperator{\arccot}{arccot}
\DeclareMathOperator{\arcsec}{arcsec}
\DeclareMathOperator{\arccsc}{arccsc}
\DeclareMathOperator{\si}{Si}

\newcommand{\tightoverset}[2]{%
  \mathop{#2}\limits^{\vbox to -.5ex{\kern-0.75ex\hbox{$#1$}\vss}}}
\newcommand{\arrowvec}[1]{\tightoverset{\scriptstyle\rightharpoonup}{#1}}
\renewcommand{\vec}{\mathbf}


\colorlet{textColor}{black} 
\colorlet{background}{white}
\colorlet{penColor}{blue!50!black} % Color of a curve in a plot
\colorlet{penColor2}{red!50!black}% Color of a curve in a plot
\colorlet{penColor3}{red!50!blue} % Color of a curve in a plot
\colorlet{penColor4}{green!50!black} % Color of a curve in a plot
\colorlet{penColor5}{orange!80!black} % Color of a curve in a plot
\colorlet{fill1}{penColor!20} % Color of fill in a plot
\colorlet{fill2}{penColor2!20} % Color of fill in a plot
\colorlet{fillp}{fill1} % Color of positive area
\colorlet{filln}{penColor2!20} % Color of negative area
\colorlet{fill3}{penColor3!20} % Fill
\colorlet{fill4}{penColor4!20} % Fill
\colorlet{fill5}{penColor5!20} % Fill
\colorlet{gridColor}{gray!50} % Color of grid in a plot

\newcommand{\surfaceColor}{violet}
\newcommand{\surfaceColorTwo}{redyellow}
\newcommand{\sliceColor}{greenyellow}




\pgfmathdeclarefunction{gauss}{2}{% gives gaussian
  \pgfmathparse{1/(#2*sqrt(2*pi))*exp(-((x-#1)^2)/(2*#2^2))}%
}


%%%%%%%%%%%%%
%% Vectors
%%%%%%%%%%%%%

%% Simple horiz vectors
\renewcommand{\vector}[1]{\left\langle #1\right\rangle}


%% %% Complex Horiz Vectors with angle brackets
%% \makeatletter
%% \renewcommand{\vector}[2][ , ]{\left\langle%
%%   \def\nextitem{\def\nextitem{#1}}%
%%   \@for \el:=#2\do{\nextitem\el}\right\rangle%
%% }
%% \makeatother

%% %% Vertical Vectors
%% \def\vector#1{\begin{bmatrix}\vecListA#1,,\end{bmatrix}}
%% \def\vecListA#1,{\if,#1,\else #1\cr \expandafter \vecListA \fi}

%%%%%%%%%%%%%
%% End of vectors
%%%%%%%%%%%%%

%\newcommand{\fullwidth}{}
%\newcommand{\normalwidth}{}



%% makes a snazzy t-chart for evaluating functions
%\newenvironment{tchart}{\rowcolors{2}{}{background!90!textColor}\array}{\endarray}

%%This is to help with formatting on future title pages.
\newenvironment{sectionOutcomes}{}{} 



%% Flowchart stuff
%\tikzstyle{startstop} = [rectangle, rounded corners, minimum width=3cm, minimum height=1cm,text centered, draw=black]
%\tikzstyle{question} = [rectangle, minimum width=3cm, minimum height=1cm, text centered, draw=black]
%\tikzstyle{decision} = [trapezium, trapezium left angle=70, trapezium right angle=110, minimum width=3cm, minimum height=1cm, text centered, draw=black]
%\tikzstyle{question} = [rectangle, rounded corners, minimum width=3cm, minimum height=1cm,text centered, draw=black]
%\tikzstyle{process} = [rectangle, minimum width=3cm, minimum height=1cm, text centered, draw=black]
%\tikzstyle{decision} = [trapezium, trapezium left angle=70, trapezium right angle=110, minimum width=3cm, minimum height=1cm, text centered, draw=black]


\outcome{Give the definition of a power series.}
\outcome{Find the interval and radius of convergence of a power series.}
\outcome{Express functions as power series.}
\outcome{Express power series as closed-form functions.}
\outcome{Differentiate and integrate power series.}

\title[Dig-In:]{Power series}

\begin{document}
\begin{abstract}
  Infinite series can represent functions.
\end{abstract}
\maketitle

If we refuse to truncate a Taylor polynomial, and instead allow it to
be a series (an infinite sum) we call it a \textit{power series}. 

\begin{definition}
  A \dfn{power series} is a series of the form
  \[
  \sum_{k=0}^\infty a_k(x-c)^k
  \]
  where the $a_k$'s are the \dfn{coefficients} and $c$ is the
  \dfn{center}.
\end{definition}

Here are four basic power series (centered at zero) that every
mathematician knows:

\begin{align*}
           e^x &= 1 + x + \frac{x^2}{2!} + \frac{x^3}{3!} + \cdots\\
       \sin(x) &= x - \frac{x^3}{3!} + \frac{x^5}{5!} -\frac{x^7}{7!} + \cdots\\
       \cos(x) &= 1-\frac{x^2}{2!} + \frac{x^4}{4!} -\frac{x^6}{6!} + \cdots\\
 \frac{1}{1-x} &= 1+ x+ x^2 + x^3 + \cdots \qquad |x|< 1
\end{align*}

Using power series we can ``read-off'' properties of functions. For instance:
\begin{itemize}
\item We can easily see that $e^0 =1$, $\sin(0)=0$, and $\cos(0) =1$.
\item Since every power of $x$ in the power series for sine is odd, we
  can see that sine is an odd function. Likewise, since every power of
  $x$ in the power series for cosine is even, we can see cosine is an
  even function.
\item Limits like
  \[
  \lim_{x\to 0}\frac{\sin(x)}{x} = 1\qquad\text{and}\qquad \lim_{x\to 0} \frac{\cos(x)-1}{x} = 0
  \]
  are ``easy'' to compute, since they can be rewritten as:
  \begin{align*}
    \lim_{x\to 0}\frac{\sin(x)}{x} &=\lim_{x\to 0} \frac{x - \frac{x^3}{3!} + \frac{x^5}{5!} -\frac{x^7}{7!} + \cdots}{x}\\
    &=\lim_{x\to 0} \left(1 - \frac{x^2}{3!} + \frac{x^4}{5!} -\frac{x^6}{7!} + \cdots\right)\\
    &=1,
  \end{align*}
  and
  \begin{align*}
    \lim_{x\to 0} \frac{\cos(x)-1}{x}&=\lim_{x\to 0} \frac{1-\frac{x^2}{2!} + \frac{x^4}{4!} -\frac{x^6}{6!} + \cdots-1}{x}\\
    &=\lim_{x\to 0} \left(-\frac{x}{2!} + \frac{x^3}{4!} -\frac{x^5}{6!} + \cdots\right)\\
    &=0.
  \end{align*}
\item Power series give us methods to \textit{actually compute} values
  for these functions.
\end{itemize}

\section{Convergence of power series}

You may have noticed a small caveat above:
\[
\frac{1}{1-x} = 1+ x+ x^2 + x^3 + \cdots \qquad |x|< 1
\]
I'm talking about the ``$|x|<1$.'' This restriction is required because if
our formula is true, then for any number $r$,
\[
\frac{1}{1-r} = 1+ r+ r^2 + r^3 + \cdots \qquad |r|< 1
\]
Ah! The expression on the right-hand side of the equation above is a
\index{geometric series}geometric series! As we've learned, geometric
series only converge when the common ratio (in this case $r$) is
between $-1$ and $1$ noninclusive.

\begin{question}
  True or False:
  \[
  \frac{4}{3} = \frac{1}{1-(1/4)} = 1 + \left(\frac{1}{4}\right)+ \left(\frac{1}{4}\right)^2+ \left(\frac{1}{4}\right)^3 + \cdots
  \]
  \begin{prompt}
  \begin{multipleChoice}
    \choice[correct]{true}
    \choice{false}
  \end{multipleChoice}
  \end{prompt}
  \begin{question}
    True or False:
    \[
    \frac{1}{-3} = \frac{1}{1-4} = 1 + 4+ 4^2+ 4^3 + \cdots
    \]
    \begin{prompt}
      \begin{multipleChoice}
        \choice{true}
        \choice[correct]{false}
      \end{multipleChoice}
    \end{prompt}
\end{question}
\end{question}

Our next theorem tells us what possible scenarios we could encounter
when investigating convergence of power series:

\begin{theorem}[Convergence of Power Series]\index{convergence of power series}\index{power series!convergence}
  Consider the power series
  \[
  \sum_{n=0}^\infty a_n(x-c)^n.
  \]
  Exactly one of the following is true:
\begin{enumerate}
\item The series converges only at $x=c$.
\item There is an $R>0$ such that the series converges for all $x$ in	
  $(c-R,c+R)$ and diverges for all $x<c-R$ and $x>c+R$.
\item The series converges for all $x$.
\end{enumerate}
\end{theorem}

\begin{question}
  True or False: A power series
  \[
  \sum_{k=0}^\infty a_k(x-c)^k
  \]
  \textbf{always} converges when $x=c$.
  \begin{prompt}
    \begin{multipleChoice}
      \choice[correct]{true}
      \choice{false}
    \end{multipleChoice}
  \end{prompt}
  \begin{feedback}
    If $x=c$, then
    \begin{align*}
      \sum_{k=0}^\infty a_k(x-c)^k &= \sum_{k=0}^\infty a_k(c-c)^k \\
      &= \sum_{k=0}^\infty a_k(0)^k.
    \end{align*}
  \end{feedback}
  \begin{question}
    True or False: If 
    \[
    f(x) = \sum_{k=0}^\infty a_k(x-c)^k,
    \]
    then $f(c) = 0$. 
  \begin{prompt}
    \begin{multipleChoice}
      \choice{true}
      \choice[correct]{false}
    \end{multipleChoice}
  \end{prompt}
  \begin{feedback}
    \begin{align*}
      f(c) &= \sum_{k=0}^\infty a_k(c-c)^k \\
      &= \sum_{k=0}^\infty a_k(0)^k\\
      &= a_0
    \end{align*}
    since $0^k = 0$ when $k\ne 0$ and $0^0 = 1$.
  \end{feedback} 
\end{question}
\end{question}

Because power series can define functions, we no longer exclusively
talk about convergence at a point, instead we talk about the
\textit{radius} and \textit{interval} of convergence.

\begin{definition}
  \hfil
  \begin{itemize}
    \item If a power series converges absolutely for all $x$, then its
      \dfn{radius of convergence} is said to be $\infty$ and the
      \dfn{interval of convergence} is $(-\infty,\infty)$.
    \item If a power series converges absolutely for all $x$ in
      $(c-R,c+R)$ and diverges for all $x<c-R$ and $x>c+R$, then its
      \dfn{radius of convergence} is said to be $R$ and the
      \dfn{interval of convergence} is one of the following:
      \[
      (c-R,c+R),\quad [c-R,c+R),\quad (c-R,c+R],\quad [c-R,c+R].
      \]
  \end{itemize}
\end{definition}

How do we check for radius of convergence? Two old friends can come to
the rescue: The ratio and the root tests.

\begin{example}
  Consider the power series:
  \[
  \sum_{n=0}^\infty \frac{x^n}{n!} = 1 + x + \frac{x^2}{2!} + \frac{x^3}{3!} + \cdots\\
  \]
  Determine the radius and interval of convergence.
  \begin{explanation}
    For this power series we will use the ratio test. To apply the
    ratio test, we must look at the absolute value of the terms in the
    series:
    \begin{align*}
      \lim_{n\to\infty} \frac{\frac{|x|^{n+1}}{(n+1)!}}{\frac{|x|^n}{n!}}
      &= \lim_{n\to\infty} \frac{|x|^{n+1}n!}{(n+1)!|x|^n}\\
      &= \lim_{n\to\infty} \frac{|x|}{n}.
    \end{align*}
    Now, for any \textbf{fixed} value of $x$, we have that
    \[
    \lim_{n\to\infty} \frac{|x|}{n} = 0,
    \]
    hence the radius of convergence for $\sum_{n=0}^\infty
    \frac{x^n}{n!}$ is $R=\infty$, and the interval of covergence is $(-\infty, \infty)$.
  \end{explanation}
\end{example}

While the ratio and root test are good for determining the radius of
convergence of a power series, they are useless for determining
convergence at the end-points of the interval. Let's see an example:

\begin{example}
  Consider the power series:
  \[
  \sum_{n=1}^\infty \frac{(x-1)^n}{n \cdot 9^n} = \frac{(x-1)}{9} + \frac{(x-1)^2}{2\cdot 9^2} + \frac{(x-1)^3}{3\cdot 9^3} + \cdots\\
  \]
  Determine the radius and interval of convergence.
  \begin{explanation}
    Here let's start with the root test. Again, we must look at the absolute value of the terms in the series:
    \begin{align*}
      \lim_{n\to\infty}\sqrt[n]{\frac{|x-1|^n}{n \cdot 9^n}} &= \lim_{n\to\infty}\sqrt[n]{\frac{|x-1|^n}{n \cdot 9^n}}\\
      &= \lim_{n\to\infty} \frac{|x-1|}{\sqrt[n]{n} \cdot 9}\\
      &= \frac{|x-1|}{\sqrt[n]{n} \cdot 9} \lim_{n\to\infty} n^{-1/n}.
    \end{align*}
    Using logarithms and L'H\^opital's rule, we can show that
    \[
    \lim_{n\to\infty} n^{-1/n} = 1.
    \]
    Hence
    \[
    \lim_{n\to\infty}\sqrt[n]{\frac{|x-1|^n}{n \cdot 9^n}} = \frac{|x-1|}{9}.
    \]
    This limit is between $-1$ and $1$ when
    \begin{align*}
    \frac{|x-1|}{9} &<1\\
    |x-1| &< 9.
    \end{align*}
    However,
    \[
    |x-1| < 9 \qquad\text{means that}\qquad -9 < x-1 < 9
    \]
    and so adding $1$ to all sides of the inequality, we need $x$ such that
    \[
    -8 < x < 10.
    \]
    Since our power series is centered at $x=1$, the radius of
    convergence is $R=9$. However, the root test (and ratio test) tell
    us nothing about the end points. For this we need to investigate
    the following \textit{two} series:
    \[
    \sum_{n=1}^\infty \frac{(-8-1)^n}{n \cdot 9^n}\qquad\text{and}\qquad \sum_{n=1}^\infty \frac{(10-1)^n}{n \cdot 9^n}
    \]
    For the first, where $x=-8$, note
    \begin{align*}
      \sum_{n=1}^\infty \frac{(-8-1)^n}{n \cdot 9^n} &= \sum_{n=1}^\infty \frac{(-9)^n}{n \cdot 9^n}\\
      &= \sum_{n=1}^\infty \frac{(-1)^n}{n}.
    \end{align*}
    Ah, this is the alternating harmonic series, which we know converges. So our power series converges at $x= -8$.
    For the second, where $x=10$, note
    \begin{align*}
      \sum_{n=1}^\infty \frac{(10-1)^n}{n \cdot 9^n} &= \sum_{n=1}^\infty \frac{(9)^n}{n \cdot 9^n}\\
      &= \sum_{n=1}^\infty \frac{1}{n}.
    \end{align*}
    This is the harmonic series, which we know diverges. So our
    power series diverges at $x= 10$. Hence the interval of
    convergence for $\sum_{n=1}^\infty \frac{(x-1)^n}{n \cdot 9^n}$ is
    $[-8,10)$.
  \end{explanation}
\end{example}

Let's do an example of a power series that only converges at a single
point.

\begin{example}
  Consider the power series:
  \[
  \sum_{n=0}^\infty n!(x+7)^n = 1 + (x+7) + 2(x+7)^2 + 6(x+7)^3 + \cdots
  \]
  Determine the radius and interval of convergence.
  \begin{explanation}
    Here we'll use the ratio test, looking at the absolute value of
    the terms in the series:
    \[
    \lim_{n\to\infty} \frac{(n+1)!|x+7|^{n+1}}{n!|x+7|^n}= \lim_{n\to\infty} (n+1)|x+7|
    \]
    This limit diverges unless $x=-7$, the center of the power
    series. The the radius of convergence is $R=0$, and there is no
    interval of convergence, since the series only converges at a
    single point.
  \end{explanation}
\end{example}



\section{New power series from old}

With the basic power series above, we can produce new power series via
algebraic manipulation.

\begin{theorem}[Algebra of Power Series]\index{power series!algebra of}
  Let
  \begin{align*}
    f(x) &= \sum_{n=0}^\infty a_nx^n\\
    g(x) &= \sum_{n=0}^\infty b_nx^n
  \end{align*}
  converge absolutely for $|x|<R$, and let $h(x)$ be continuous.
  \begin{itemize}
	\item $f(x)\pm g(x) = \sum_{n=0}^\infty (a_n\pm b_n)x^n$ \quad for $|x|<R$.
	%\item $f(x)g(x) = \left(\sum_{n=0}^\infty a_nx^n\right)\left(\sum_{n=0}^\infty b_nx^n\right) = \sum_{n=0}^\infty\big(a_0b_n+a_1b_{n-1}+\ldots a_nb_0\big)x^n$ for $|x|<R$.
	\item $\begin{aligned}[t]
	f(x)g(x) &= \left(\sum_{n=0}^\infty a_nx^n\right)\left(\sum_{n=0}^\infty b_nx^n\right)\\
	      &= \sum_{n=0}^\infty\big(a_0b_n+a_1b_{n-1}+\dots + a_nb_0\big)x^n
		\end{aligned}$ for $|x|<R$.%\hfill
	
	\item $f\big(h(x)\big) = \sum_{n=0}^\infty a_n\big(h(x)\big)^n$ \quad for $|h(x)|<R$.

  \end{itemize}
\end{theorem}



\begin{theorem}[Derivatives and Indefinite Integrals of Power Series Functions]\index{integration!of power series}\index{derivative!power series}\index{power series!derivatives and integrals}
  Let
  \[
  f(x) = \sum_{n=0}^\infty a_n(x-c)^n
  \]
  be a function defined by a power series, with radius of convergence $R$.
  \begin{itemize}
  \item $f(x)$ is continuous and differentiable on $(c-R,c+R)$.
  \item	$f'(x) = \sum_{n=1}^\infty a_n\cdot n\cdot (x-c)^{n-1}$, with radius of convergence $R$.
  \item	$\int f(x) \d x = C+\sum_{n=0}^\infty a_n\frac{(x-c)^{n+1}}{n+1}$, with radius of convergence $R$.
  \end{itemize}
\end{theorem}

A few notes about the theorem above:
\begin{itemize}
\item The theorem states that differentiation and integration do not
  change the radius of convergence. It does not state anything about
  the \textit{interval} of convergence. They are not always the same.
\item Notice how the summation for $f'(x)$ starts with $n=1$. This is
  because the constant term $a_0$ of $f(x)$ goes to $0$.
\item Differentiation and integration are simply calculated
  term-by-term using the power rule.
\end{itemize}





\end{document}
