\documentclass{ximera}

\title{Limits}

\newcommand{\defnword}[1]{\textbf{#1}}
\newcommand{\Z}{\mathbb{Z}}
\newcommand{\N}{\mathbb{N}}
\newcommand{\nth}{\mbox{th}}
\renewcommand{\index}[1]{}

\begin{document}

\begin{abstract}
  Limits address the question ``What happens after a while?''
\end{abstract}

\maketitle

We've seen a lot of sequences, and already there are a few things we
might notice.  For instance, the arithmetic progression
$$
1,\quad 8,\quad 15,\quad 22,\quad 29,\quad 36,\quad 43,\quad 50,\quad 57,\quad 64,\quad 71,\quad 78,\quad 85,\quad 92,\quad \ldots
$$
just keeps getting bigger and bigger.  No matter how large a number
you think of, if I add enough $7$'s to $1$, eventually I will surpass
the giant number you thought of.  On the other hand, the terms in a geometric progression where each term is half the previous term, namely
$$
\frac{1}{2},\quad \frac{1}{4},\quad \frac{1}{8},\quad \frac{1}{16},\quad \frac{1}{32},\quad \frac{1}{64},\quad \frac{1}{128},\quad \frac{1}{256},\quad \frac{1}{512},\quad \frac{1}{1024},\quad \ldots ,
$$
are getting closer and closer to zero.  No matter how close you stand
near but not at zero, eventually this geometric sequence gets even closer than you
are to zero.

\youtube{https://www.youtube.com/watch?v=PRTjvMA2nCY}

These two sequences have very different stories.  One shoots off to
infinity; the other zooms in towards zero.  Mathematics is not just
about numbers; mathematics provides tools for talking about the
qualitative features of the numbers we deal with.  What about the two
sequences we just considered?  They are qualitatively very different.  The first ``goes to''
infinity; the second ``goes to'' zero.

In short, given a sequence, it is helpful to be able to say something
qualitative about it; we may want to address the question such as
``what happens after a while?'' Formally, when faced with a sequence,
we are interested in the limit
$$\lim_{i\to \infty} f(i) = \lim_{i\to\infty} a_i.$$
In Calculus One, we studied a similar question about
$$\lim_{x\to\infty} f(x)$$
when $x$ is a variable taking on real values; now, in Calculus Two, we
simply want to restrict the ``input'' values to be integers. No
significant difference is required in the definition of limit, except
that we specify, perhaps implicitly, that the variable is an integer.

\begin{definition} \relax\index{limit of a sequence}
\label{definition:limit-of-a-sequence}
Suppose that $\left(a_n\right)$ is a sequence.
To say that $\displaystyle\lim_{n\to \infty}a_n=L$ is to say that \\
\null\quad for every $\epsilon>0$, \\
\null\quad\quad there is an $N > 0$, \\
\null\quad so that whenever $n>N$, \\
\null\quad\quad we have $|a_n-L|<\epsilon$. \\
If $\displaystyle\lim_{n\to\infty}a_n=L$ we say that the sequence
\defnword{converges}\index{convergent
  sequence}\index{sequence!convergent}.  If there is no finite value $L$ so
that $\displaystyle\lim_{n\to\infty}a_n = L$, then we say that the limit
\defnword{does not exist}, or equivalently that the sequence
\defnword{diverges}\index{divergent sequence}\index{sequence!divergent}.
\end{definition} 

\begin{question}
  To say that the sequence \(a_n\) converges to \(L\) means what?  In other words, what is the definition of the statement \(\displaystyle\lim_{n \to \infty} a_n = L\)?

    \begin{hint}
      We are trying to make precise the idea that, eventually, all the terms of the sequence \(a_n\) are as close as we want to \(L\).
    \end{hint}
    \begin{hint}
      To measure closeness to \(L\), we will use a positive real number \(\epsilon\).
    \end{hint}
    \begin{hint}
      We must achieve any desired degree of closeness, so we will make a statement which is true for any positive real number \(\epsilon\).
    \end{hint}
    \begin{hint}
      In other words, the definition will begin ``For every positive real number \(\epsilon > 0\)\ldots''
    \end{hint}
    \begin{hint}
      We now must make precise the idea of ``eventually'' close.
    \end{hint}
    \begin{hint}
      We use a whole number \(N\) to capture the idea of ``sufficiently large'' values of \(n\).
    \end{hint}
    \begin{hint}
      Specifically, the definition will begin ``For every positive real number \(\epsilon > 0\), there exists an \(N \in \mathbb{N}\)\ldots''
    \end{hint}
    \begin{hint}
      The ``sufficiently large'' value of \(n\) is any value which is at least as large as \(N\).
    \end{hint}
    \begin{hint}
      So we will only consider those \(n\) for which \(n \geq N\).
    \end{hint}
    \begin{hint}
      Thus the definition goes ``For every positive real number \(\epsilon > 0\), there exists an \(N \in \mathbb{N}\) so that whenever \(n \geq N\)\ldots''
    \end{hint}
    \begin{hint}
      What happens ``eventually'' is that terms of the sequence are close to \(L\).  How close?  Within \(\epsilon\).
    \end{hint}
    \begin{hint}
      The quantity \(|a_n - L|\) is the distance between \(a_n\) and \(L\).
    \end{hint}
    \begin{hint}
      To say that \(a_n\) is within \(\epsilon\) of \(L\) is to say that \(|a_n - L| < \epsilon\).
    \end{hint}
    \begin{hint}
      Therefore the definition is ``For every positive real number \(\epsilon > 0\) there exists an \(N \in \mathbb{N}\) so that whenever \(n \geq N\), we have \( |a_n - L| < \epsilon \).''
    \end{hint}

    \begin{multipleChoice}
      \choice[correct]{For every positive real number \(\epsilon > 0\) there exists an \(N \in \mathbb{N}\) so that whenever \(n \geq N\), we have \( |a_n - L| < \epsilon \).}
      \choice{For every real number \(\epsilon > 0\) there exists an \(N \in \mathbb{N}\) so that \( |a_N - L| < \epsilon \).}
      \choice{For every real number \(\epsilon \in \mathbb{R}\) there exists an \(N \in \mathbb{N}\) so that whenever \(n \geq N\), we have \( |a_n - L| < \epsilon \).}
      \choice{For every whole number \(N > 0\) there exists a positive real number \(\epsilon > 0\) so that whenever \(n \geq N\), we have \( |a_n - L| < \epsilon \).}
      \choice{For every whole number \(N > 0\) there exists a real number \(\epsilon \in \mathbb{R}\) so that whenever \(n \geq N\), we have \( |a_n - L| < \epsilon \).}
    \end{multipleChoice}
              

  The definition of limit can be written as if it were poetry with
  line breaks and all.  Like the best of poems, it deserves to be
  memorized, performed, internalized.  Humanity struggled for millenia
  to find the wisdom contained in this definition.
\end{question}

\begin{warning}
  In the case that $\lim_{n \to \infty} a_n = \infty$, we say that
  $(a_n)$ diverges, or perhaps more precisely, we say $(a_n)$ diverges to
  infinity.  The only time we say that a sequence converges is when
  the limit exists and is equal to a \textit{finite} value.
\end{warning}

\youtube{https://www.youtube.com/watch?v=0UCRZAsIkXM}

One way to compute the limit of a sequence is to compute the limit of
a function.
\begin{theorem}
  \label{theorem:compute-limit-of-sequence-via-function}
  Let $f(x)$ be a real-valued function.  If $a_n = f(n)$ defines a
  sequence $(a_n)$ and if $\displaystyle\lim_{x\to\infty}f(x)=L$ in the sense of Calculus
  One, then $\displaystyle\lim_{n\to\infty} a_n=L$ as well.
\end{theorem}

\begin{example}
\label{example:find-n-for-epsilon}
Since $\displaystyle\lim_{x\to\infty}(1/x)=0$, it is
clear that also $\displaystyle\lim_{n\to\infty}(1/n)=0$; in other words, the sequence of numbers
$${1\over1},\quad {1\over2},\quad {1\over3},\quad {1\over4},\quad {1\over5},\quad {1\over6},\quad \ldots$$
get closer and closer to 0, or more precisely, as close as you want to get to zero, after a while, all the terms in the sequence are that close.

More precisely, no matter what $\epsilon > 0$ we pick, we can find an
$N$ big enough so that, whenever $n > N$, we have that $1/n$ is within
$\epsilon$ of the claimed limit, zero.  This can be made concrete:
let's suppose we set $\epsilon = 0.17$.  What is a suitable choice for
$N$ in response?  If we choose $N = 5$, then whenever $n > 5$ we have
$0 < 1/n < 0.17$.
\end{example}

\youtube{https://www.youtube.com/watch?v=lCW8BBBQRyc}

\begin{question}
  Consider the sequence given by the rule \[b_{n} = \displaystyle\displaystyle\frac{ 3 \, n + 9 }{ 4 \, n + 20 }.\]  For which value of \(N\) is it the case that whenever \(n \geq N\) we have that \(b_{n}\) is within \(1/50\) of \(3/4\)?

    \begin{hint}
      There can only be one right answer.
    \end{hint}
    \begin{hint}
      So the answer is either \(N = 69\) or ``none of these.''
    \end{hint}
    \begin{hint}
      Note that \(b_{69} = \displaystyle\frac{27}{37}\).
    \end{hint}
    \begin{hint}
      Consequently, \(\left| b_{69} - \displaystyle \displaystyle\frac{3}{4} \right| = \displaystyle\frac{3}{148} \geq \displaystyle\frac{1}{50}\).
    \end{hint}
    \begin{hint}
      So the answer must be ``none of these.''
    \end{hint}

    \begin{multipleChoice}
      \choice[correct]{None of these choices for \(N\) is large enough.}
      \choice{\(N = 69 \)}
      \choice{\(N = 67 \)}
      \choice{\(N = 65 \)}
      \choice{\(N = 63 \)}
    \end{multipleChoice}

\end{question}

But it is important to note that the converse of this theorem is not
true; \label{sidenote:raining-converse}the \defnword{converse} of a
statement is what you get when you swap the assumption and the
conclusion; the converse of ``if it is raining, then it is cloudy'' is
the statement ``if it is cloudy, then it is raining.''  Which of those
statements is true?.

To show the converse is not true, it is enough to provide a single
example where it fails.  Here's the counterexample.  Recall an
instance of (a potential) general rule being broken is called a
\defnword{counterexample}.  This is a popular term among
mathematicians and philosophers.

\begin{example}
  Consider the sequence $(a_n)$ given by the rule $a_n = f(n)=\sin(n\pi)$.  This is the sequence
$$
  \sin(0\pi),\quad \sin(1\pi),\quad\sin(2\pi),\quad\sin(3\pi),\quad\ldots,
$$
which is just the sequence $0, 0, 0, 0, \ldots$ since $\sin(n\pi)=0$
whenever $n$ is an integer.  Since the sequence is just the constant sequence, we have
$$
\lim_{n\to\infty} f(n)= \lim_{n\to\infty} 0 = 0. 
$$But $\displaystyle\lim_{x\to\infty}f(x)$, when $x$ is real, does not exist: as $x$ gets
bigger and bigger, the values $\sin(x\pi)$ do not get closer and
closer to a single value, but instead oscillate between $-1$ and $1$.
\end{example} 

Here's some general advice. If you want to know $\displaystyle\lim_{n\to\infty}
a_n$, you might first think of a function $f(x)$ where $a_n = f(n)$,
and then attempt to compute $\displaystyle\lim_{x\to\infty}f(x)$.  If the limit
of the function exists, then it is equal to the limit of the sequence.
But, if for some reason $\displaystyle\lim_{x\to\infty}f(x)$ does not exist, it
may nevertheless still be the case that $\displaystyle\lim_{n\to\infty}f(n)$
exists---you'll just have to figure out another way to compute it.


\section{New from old}

Given a sequence, one way to build a new sequence is to start with the
old sequence, but then throw away a whole bunch of terms.  For
instance, if we started with the sequence of perfect squares
$$
1,\quad 4,\quad 9,\quad 16,\quad 25,\quad 36,\quad 49,\quad 64,\quad 81,\quad\ldots
$$
we could throw away all the odd-indexed terms, and be left with
$$
4,\quad 16,\quad 36,\quad 64,\quad 100,\quad 144,\quad 196,\quad 256,\quad 324,\quad 400,\quad 484,\quad\ldots
$$
We say that this latter sequence is a
\defnword{subsequence}\index{sequence!subsequence}\index{subsequence}
of the original sequence.  Here is a precise definition.

\begin{definition}
  Suppose $(a_n)$ is a sequence with initial index $N$, and suppose we have a sequence of integers $(n_i)$ so that
  $$
  N \leq n_1 < n_2 < n_3 < n_4 < n_5 < \cdots 
  $$
  Then the sequence $(b_i)$ given by $b_i = a_{n_i}$ is said to be a \defnword{subsequence}\index{sequence!subsequence}\index{subsequence}
  of the sequence $a_n$.
\end{definition}

\youtube{http://www.youtube.com/watch?v=OYRRFRzf12o}

Limits are telling the story of ``what happens'' to a sequence.  If
the terms of a sequence can be made as close as desired to a limiting
value $L$, then the subsequence must share that same fate.

\begin{theorem}
  \label{theorem:subsequence-same-limit}
  If $(b_i)$ is a subsequence of the convergent sequence $(a_n)$, then
  $\ds\lim_{i \to \infty} b_i = \ds\lim_{n \to \infty} a_n$.
\end{theorem}

Of course, just because a subsequence converges does not mean that the
larger sequence converges, too.  We'll see this again in more detail,
but we'll discuss it briefly now.

\begin{example}
Find a convergent subsequence of the sequence $(a_n)$ given by the rule $a_n = (-1)^n$.
\end{example}

\begin{solution}
Note that the sequence $(a_n)$ does not converge.  But by considering the sequence of indexes $n_i = 2 \cdot i$, we can build a subsequence
$$
b_i = a_{n_i} = a_{2i} = (-1)^{2i} = 1,
$$
which is a constant sequence, so it converges to 1.
\end{solution}

There are other subsequences of $a_n = (-1)^n$ which converge but do
\textit{not} converge to one.  For instance, the subsequence of odd
indexed terms is the constant sequence $c_n = -1$, which converges to
$-1$.  For that matter, the fact that there are convergent
subsequences with distinct limits perhaps explains why the original
sequence $(a_n)$ does not converge.  Let's formalize this.

\begin{corollary}
  \label{corollary:different-subsequences-then-diverge}

  Suppose $(b_i)$ and $(c_i)$ are convergent subsequences of the sequence $(a_n)$, but
  $$
  \ds\lim_{i \to \infty} b_i \neq \ds\lim_{i \to \infty} c_i.
  $$
  Then the sequence $(a_n)$ does not converge.
\end{corollary}

\begin{proof}
  Suppose, on the contrary, the sequence $(a_n)$ did converge.  Then
  by Theorem~\xrefn{theorem:subsequence-same-limit}, the subsequence
  $(b_i)$ would converge, too, and
  $$
  \ds\lim_{i \to \infty} b_i = \ds\lim_{n \to \infty} a_n.
  $$
  Again by Theorem~\xrefn{theorem:subsequence-same-limit}, the subsequence
  $(c_i)$ would converge, too, and
  $$
  \ds\lim_{i \to \infty} c_i = \ds\lim_{n \to \infty} a_n.
  $$
  But then $\ds\lim_{i \to \infty} b_i = \ds\lim_{i \to \infty} c_i$,
  which is exactly what we are supposing doesn't happen!  To avoid
  this contradiction, it must be that our original assumption that
  $(a_n)$ converged was incorrect; in short, the sequence $(a_n)$ does
  not converge.
\end{proof}


\section{Features of sequences}

\subsection{Monotonicity}

And sometimes we don't even care about limits, but we'd simply like
some terminology with which to describe features we might notice about
sequences.  Here is some of that terminology.

(For instance, how much money I have on day $n$ is a sequence; I
probably hope that sequence is an increasing sequence!)

\begin{definition}
  A sequence is called
  \defnword{increasing}\index{sequence!increasing} (or sometimes
  \defnword{strictly increasing}) if $\ds a_n<a_{n+1}$ for all $n$.
  It is called {\dfont
    non-decreasing\index{sequence!non-decreasing}\/} if $\ds a_n\le
  a_{n+1}$ for all $n$.

  Similarly a sequence is {\dfont
    decreasing\index{sequence!decreasing}\/} (or, by some people,
  \defnword{strictly decreasing}) if $\ds a_n>a_{n+1}$ for all $n$ and
  {\dfont non-increasing\index{sequence!non-increasing}\/} if $\ds
  a_n\ge a_{n+1}$ for all $n$.
\end{definition}
To make matters worse, the people who insist on saying ``strictly
increasing'' may---much to everybody's confusion---insist on calling a
non-decreasing sequence ``increasing.'' I'm not going to play their
game; I'll be careful to say ``non-decreasing'' when I mean a sequence
which is getting larger or staying the same.

To make matters better, lots of facts are true for sequences which are
either increasing or decreasing; to talk about this situation without
constantly saying ``either increasing or decreasing,'' we can make up
a single word to cover both cases.
\begin{definition}
  If a sequence is increasing, non-decreasing, decreasing, or
  non-increasing, it is said to be {\dfont
    monotonic\index{sequence!monotonic}\/}.
\end{definition}

\youtube{https://www.youtube.com/watch?v=FMaKP0hmytU}

Let's see some examples of sequences which are monotonic.
\begin{example}
The sequence $\ds a_n = {2^n-1\over2^n}$ which starts
$$
  {1\over2},\quad {3\over4},\quad {7\over8},\quad {15\over16},\quad \ldots,
$$
is increasing.  On the other hand, the sequence $\ds b_n = {n+1\over n}$, which starts
$$ 
  {2\over1},\quad{3\over2},\quad{4\over3},\quad{5\over4},\quad\ldots,
$$
is decreasing.
\end{example}

\begin{question}
  Which of the following could be the initial terms of a monotonic sequence?

    \begin{hint}
      A monotonic sequence is a sequence which is either increasing, decreasing, non-increasing, or non-decreasing.
    \end{hint}
    \begin{hint}
      Looking at available choices, mostly these sequences are increasing.
    \end{hint}
    \begin{hint}
      For example, \(12 < 13\).
    \end{hint}
    \begin{hint}
      Can \(5,  11,  16,  12,  18,  22 \) be the beginning of a monotonic sequence?  No, because mostly these terms are increasing, but \(16\) and \(12\) break the pattern.
    \end{hint}
    \begin{hint}
      Can \(5,  11,  16,  22,  16,  22 \) be the beginning of a monotonic sequence?  No, because mostly these terms are increasing, but \(22\) and \(16\) break the pattern.
    \end{hint}
    \begin{hint}
      Can \(5,  6,  10,  12,  15,  10 \) be the beginning of a monotonic sequence?  No, because mostly these terms are increasing, but \(15\) and \(10\) break the pattern.
    \end{hint}
    \begin{hint}
      Can \(5,  11,  12,  13,  15,  16 \) possibly be the beginning of a monotonic sequence?  Yes, because this portion of the sequence is increasing, so it is possible that the rest of sequence continues that pattern.
    \end{hint}


    \begin{multipleChoice}
      \choice[correct]{\(5,\quad 11,\quad 12,\quad 13,\quad 15,\quad 16,\quad\ldots \)}
      \choice{\(5,\quad 3,\quad 6,\quad 8,\quad 12,\quad 15,\quad\ldots \)}
      \choice{\(5,\quad 10,\quad 4,\quad 10,\quad 15,\quad 19,\quad\ldots \)}
      \choice{\(5,\quad 11,\quad 16,\quad 12,\quad 18,\quad 22,\quad\ldots \)}
      \choice{\(5,\quad 11,\quad 16,\quad 22,\quad 16,\quad 22,\quad\ldots \)}
      
    \end{multipleChoice}
    
\end{question}

\subsection{Boundedness}

Sometimes we can't say exactly which number a sequence approaches, but
we can at least say that the sequence doesn't get too big or too
small.

\begin{definition}
  \label{definition:sequence-bounded}
  A sequence $(a_n)$ is \defnword{bounded
    above}\index{sequence!bounded above} if there is some number
$M$ so that for all $n$, we have $\ds a_n\le M$.  Likewise, a sequence
$(a_n)$ is {\dfont bounded below\index{sequence!bounded below}\/} if
there is some number $M$ so that for every $n$, we have $\ds a_n\ge M$.

If a sequence is both bounded above and bounded below, the sequence is said
to be {\dfont bounded\index{sequence!bounded}\/}.
\end{definition}

\youtube{https://www.youtube.com/watch?v=FC4TCSk-O24}

\begin{question}
  To say that the sequence \(a_n\) is ``bounded below'' is to say what?

    \begin{hint}
      The definition begins by asserting the existence of some bound \(M\).
    \end{hint}
    \begin{hint}
      The bound is a real number, meaning \(M \in \mathbb{R}\).
    \end{hint}
    \begin{hint}
      So the definition begins ``there exists an \(M \in \mathbb{R}\)\ldots''
    \end{hint}
    \begin{hint}
      The bound must hold for all terms in the sequence.
    \end{hint}
    \begin{hint}
      So we will assert something for all indexes \(n \in \mathbb{N}\).
    \end{hint}
    \begin{hint}
      The definition begins ``there exists an \(M \in \mathbb{R}\), so that for all \(n \in \mathbb{N}\), \ldots''
    \end{hint}
    \begin{hint}
      For a particular term \(a_n\) to be bounded below by \(M\) just means that \(a_n \geq M\).
    \end{hint}
    \begin{hint}
      Altogether then, being bounded below means ``there exists an \(M \in \mathbb{R}\), so that for all \(n \in \mathbb{N}\), we have \(a_n \geq M\).''  Sometimes you might see this ending with \( a_n > M \), but that is a difference which does not affect which sequences are bounded below.
      
    \end{hint}
    
    \begin{multipleChoice}
      \choice[correct]{There exists an \(M \in \mathbb{R}\), so that for all \(n \in \mathbb{N}\), we have \(a_n \geq M\).}
      \choice{There exists an \(n \in \mathbb{N}\), so that for all \(M \in \mathbb{R}\), we have \(a_n \geq M\).}
      \choice{For all \(M \in \mathbb{R}\), there exists an \(n \in \mathbb{N}\), so that \(a_n \geq M\).}
      \choice{For all \(n \in \mathbb{N}\), there exists an \(M \in \mathbb{R}\), so that \(a_n \geq M\).}
      \choice{There exists an \(M \in \mathbb{R}\), so that for all \(n \in \mathbb{N}\), we have \(a_n \leq M\).}
    \end{multipleChoice}

\end{question}
            
If a sequence $\ds
\{a_n\}_{n=0}^\infty$ is increasing or non-decreasing it is bounded
below (by $\ds a_0$), and if it is decreasing or non-increasing it is
bounded above (by $\ds a_0$).

\begin{question}
  Consider the sequence \(b_{n} = -n^{2} + 5 \, n - 5\).  Is the sequence bounded above?  Bounded below?

    \begin{hint}
      Consider the coefficient on \(n^{2}\) in \(b_{n} = -n^{2} + 5 \, n - 5\), which is \(-1\).
    \end{hint}
    \begin{hint}
      Since the leading term's coefficient is negative, when \(n\) is large, \(b_{n}\) is very negative.
    \end{hint}
    \begin{hint}
      Consequently, the sequence is not bounded below.
    \end{hint}
    \begin{hint}
      At least for \(n\) large, the sequence is decreasing.
    \end{hint}
    \begin{hint}
      Consequently, the sequence is bounded above.
    \end{hint}
    \begin{hint}
      Altogether then, the sequence is bounded above, but not below.
    \end{hint}

    \begin{multipleChoice}
      \choice[correct]{Bounded above, but not bounded below.}
      \choice{Bounded below, but not bounded above.}
      \choice{Bounded above and bounded below.}
      \choice{Bounded neither above nor below.}
    \end{multipleChoice}

\end{question}


\subsection{Monotone convergence}

inally, with all this new terminology we can state an important
theorem.

\begin{theorem} If the sequence $a_n$ is bounded and monotonic, then
  $\lim_{n \to \infty} a_n$ exists.  \label{thm:bounded-monotonic}
\end{theorem}
In short, bounded monotonic sequences converge---though we can't
necessarily describe the number to which they converge.

\youtube{https://www.youtube.com/watch?v=BmHt0-r_8IQ}

We will not prove this theorem in the textbook.\sidenote{Proving this
  theorem is, honestly, the purview of a course in \textit{analysis},
  the theoretical underpinnings of calculus.  That's not to say it
  couldn't be done in this course, but I intend this to be a ``first
  glance'' at sequences---so much will be left unsaid.} Nevertheless,
it is not hard to believe: suppose that a sequence is increasing and
bounded, so each term is larger than the one before, yet never larger
than some fixed value $M$. The terms must then get closer and closer
to some value between $\ds a_0$ and $M$. It certainly need not be $M$,
since $M$ may be a ``too generous'' upper bound; the limit will be the
smallest number that is above\sidenote{This concept of the ``smallest
  number above all the terms'' is an incredibly important one; it is
  the idea of a
  \href{http://en.wikipedia.org/wiki/Least-upper-bound_property}{least
    upper bound} that underlies the real numbers.} all of the terms
$\ds a_n$.  Let's try an example!

\begin{example}
  \label{example:sequence-bounded}
  All of the terms $\ds (2^i-1)/2^i$ are less than 2, and the sequence
  is increasing.  As we have seen, the limit of the sequence is 1---1
  is the smallest number that is bigger than all the terms in the
  sequence.  Similarly, all of the terms $(n+1)/n$ are bigger than
  $1/2$, and the limit is 1---1 is the largest number that is smaller
  than the terms of the sequence.
\end{example}

\youtube{https://www.youtube.com/watch?v=SySsJ9S6g6c}

We don't actually need to know that a sequence is monotonic to apply
this theorem---it is enough to know that the sequence is
``eventually'' monotonic,\sidenote{After all, the limit only depends on
  what is happening after some large index, so throwing away the
  beginning of a sequence won't affect its convergence or its limit.}
that is, that at some point it becomes increasing or decreasing.  For
example, the sequence $10$, $9$, $8$, $15$, $3$, $21$, $4$, $3/4$,
$7/8$, $15/16$, $31/32,\ldots$ is not increasing, because among the
first few terms it is not. But starting with the term $3/4$ it is
increasing, so if the pattern continues and the sequence is bounded,
the theorem tells us that the ``tail'' $3/4, 7/8, 15/16, 31/32,\ldots$
converges.  Since convergence depends only on what happens as $n$ gets
large, adding a few terms at the beginning can't turn a convergent
sequence into a divergent one.

\begin{example}
\label{example:nth-root-of-n}
Show that the sequence $(a_n)$ given by $a_n = n^{1/n}$ converges.
\end{example}

\begin{warning}
You may be worried about my saying that $\log 3 > 1$.  If
  $\log$ were the common (base~10) logarithm, this would be wrong, but
  as far as I'm concerned, there is only one log, the natural log.
  Since $3 > e$, we may conclude that $\log 3 > 1$.
\end{warning}

\begin{worked-solution}
  We might first show that this sequence is decreasing, that is, we show
  that for all $n$,
  $$
  n^{1/n} > (n+1)^{1/(n+1)}.
  $$
  But this isn't true!  Take a look
  \begin{align*}
    a_1 &= 1, \\
    a_2 &= \sqrt{2} \approx 1.4142, \\
    a_3 &= \sqrt[3]{3} \approx 1.4422, \\
    a_4 &= \sqrt[4]{4} \approx 1.4142, \\
    a_5 &= \sqrt[5]{5} \approx 1.3797, \\
    a_6 &= \sqrt[6]{6} \approx 1.3480, \\
    a_7 &= \sqrt[7]{7} \approx 1.3205, \\
    a_8 &= \sqrt[8]{8} \approx 1.2968, \mbox{ and}\\
    a_9 &= \sqrt[9]{9} \approx 1.2765. \\
  \end{align*}
  But it does seem that this sequence perhaps is decreasing after the
  first few terms.  Can we justify this?

  Yes!  Consider the real function $\ds f(x)=x^{1/x}$ when $x\ge1$.
  We compute the derivative---perhaps via ``logarithmic differentiation''---to find
  $$
  f'(x)=\frac{x^{1/x} \, (1-\log x)}{x^2}.
  $$
  Note that when $x\ge 3$, the derivative $f'(x)$ is negative.  Since the function $f$ is decreasing, we can conclude that the sequence is decreasing---well, at least for $n \geq 3$.

  Since all terms of the sequence are positive, the sequence is
  decreasing and bounded when $n \ge 3$, and so the sequence converges.
\end{worked-solution}

\marginnote{As it happens, you could compute the limit in
  Example~\xrefn{example:nth-root-of-n}, but our given solution shows that
  it converges even without knowing the limit!}

\begin{example}
Show that the sequence $a_n = \ds\frac{n!}{n^n}$ converges.
\end{example}

\begin{solution}
  Let's get an idea of what is going on by computing the first few terms.
% print(join(['a_' + str(k) + '=' + latex(f(x=k)) + ' ' + '\\approx ' + str(n(f(x=k),digits=5)) for k in range(1,9)],',\quad '))
\begin{align*}
a_1&= 1,\quad a_2= \frac{1}{2},\quad a_3= \frac{2}{9} \approx 0.22222,\quad a_4= \frac{3}{32} \approx 0.093750, \\
a_5&= \frac{24}{625} \approx 0.038400, \quad a_6= \frac{5}{324} \approx 0.015432, \\
a_7&= \frac{720}{117649} \approx 0.0061199,\quad a_8= \frac{315}{131072} \approx 0.0024033.
\end{align*}
  The sequence appears to be decreasing.  To formally show this, we would need to show $\ds a_{n+1}< a_n$, but we will instead show that
$$
\frac{a_{n+1}}{a_n} < 1,
$$
which amounts to the same thing.  It is helpful trick here to think of
the ratio between subsequent terms, since the factorials end up
canceling nicely.  In particular,
\begin{align*}
  {a_{n+1}\over a_n} &= {(n+1)!\over (n+1)^{n+1}}{n^n\over n!} \\
  &= {(n+1)!\over n!}{n^n\over (n+1)^{n+1}} \\
  &= {n+1\over n+1}\left({n\over n+1}\right)^n
  &= \left({n\over n+1}\right)^n < 1.
\end{align*}
  Note that the sequence is bounded below, since every term is positive.

  Because the sequence is decreasing and bounded below, it converges.
  Indeed, Exercise~\xrefn{exercise:factorial-limit} asks you to
  compute the limit.
\end{solution}

These sorts of arguments involving the ratio of subsequent terms will
come up again in a big way when we consider the ratio test.  Stay
tuned!

\subsection{Sample problem}

\begin{question}
  Consider the sequence \(a_{n}\).  Suppose you know that for all \(n > 1\), \[ -6 \leq a_{n} \leq 0 \] and \(a_{1} = 2\), and \(a_{2} = -1\), and that the sequence is nonincreasing.  Does the sequence converge?

    \begin{hint}
      Since the sequence is nonincreasing, the sequence is monotone.
    \end{hint}
    \begin{hint}
      Since for all \(n \geq 1\), we have \(a_{n} \geq -6\), the sequence is bounded below.
    \end{hint}
    \begin{hint}
      So by the Monotone Convergence Theorem, the sequence converges to some value; let us call it \(L\).
    \end{hint}
    \begin{hint}
      Now consider the direction in which the sequence is heading.
    \end{hint}
    \begin{hint}
      Since the sequence is nonincreasing, for all \(n \geq 2\), we have  \(-6 \leq a_{n} \leq -1\).
    \end{hint}
    \begin{hint}
      The limit \(L\) must be in that interval as well.
    \end{hint}
    \begin{hint}
      Therefore the sequence converges to a value \(L\) so that \(-6 \leq L \leq -1\).
    \end{hint}

    \begin{multipleChoice}
      \choice[correct]{Yes, with limit between \(-6\) and \(-1\).}
      \choice{No, the sequence does not converge.}
      \choice{Yes, with limit between \(-1\) and \(0\).}
    \end{multipleChoice}
    
\end{question}

\end{document}
