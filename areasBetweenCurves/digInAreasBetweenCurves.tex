\documentclass{ximera}
%\usepackage{todonotes}

\newcommand{\todo}{}

\usepackage{tkz-euclide}
\tikzset{>=stealth} %% cool arrow head
\tikzset{shorten <>/.style={ shorten >=#1, shorten <=#1 } } %% allows shorter vectors

\usetikzlibrary{backgrounds} %% for boxes around graphs
\usetikzlibrary{shapes,positioning}  %% Clouds and stars
\usetikzlibrary{matrix} %% for matrix
\usepgfplotslibrary{polar} %% for polar plots
\usetkzobj{all}
\usepackage[makeroom]{cancel} %% for strike outs
%\usepackage{mathtools} %% for pretty underbrace % Breaks Ximera
\usepackage{multicol}





\usepackage{array}
\setlength{\extrarowheight}{+.1cm}   
\newdimen\digitwidth
\settowidth\digitwidth{9}
\def\divrule#1#2{
\noalign{\moveright#1\digitwidth
\vbox{\hrule width#2\digitwidth}}}





\newcommand{\RR}{\mathbb R}
\newcommand{\R}{\mathbb R}
\newcommand{\N}{\mathbb N}
\newcommand{\Z}{\mathbb Z}

%\renewcommand{\d}{\,d\!}
\renewcommand{\d}{\mathop{}\!d}
\newcommand{\dd}[2][]{\frac{\d #1}{\d #2}}
\renewcommand{\l}{\ell}
\newcommand{\ddx}{\frac{d}{\d x}}

\newcommand{\zeroOverZero}{\ensuremath{\boldsymbol{\tfrac{0}{0}}}}
\newcommand{\inftyOverInfty}{\ensuremath{\boldsymbol{\tfrac{\infty}{\infty}}}}
\newcommand{\zeroOverInfty}{\ensuremath{\boldsymbol{\tfrac{0}{\infty}}}}
\newcommand{\zeroTimesInfty}{\ensuremath{\small\boldsymbol{0\cdot \infty}}}
\newcommand{\inftyMinusInfty}{\ensuremath{\small\boldsymbol{\infty - \infty}}}
\newcommand{\oneToInfty}{\ensuremath{\boldsymbol{1^\infty}}}
\newcommand{\zeroToZero}{\ensuremath{\boldsymbol{0^0}}}
\newcommand{\inftyToZero}{\ensuremath{\boldsymbol{\infty^0}}}


\newcommand{\numOverZero}{\ensuremath{\boldsymbol{\tfrac{\#}{0}}}}
\newcommand{\dfn}{\textbf}
%\newcommand{\unit}{\,\mathrm}
\newcommand{\unit}{\mathop{}\!\mathrm}
\newcommand{\eval}[1]{\bigg[ #1 \bigg]}
\newcommand{\seq}[1]{\left( #1 \right)}
\renewcommand{\epsilon}{\varepsilon}
\renewcommand{\iff}{\Leftrightarrow}

\DeclareMathOperator{\arccot}{arccot}
\DeclareMathOperator{\arcsec}{arcsec}
\DeclareMathOperator{\arccsc}{arccsc}
\DeclareMathOperator{\si}{Si}

\newcommand{\tightoverset}[2]{%
  \mathop{#2}\limits^{\vbox to -.5ex{\kern-0.75ex\hbox{$#1$}\vss}}}
\newcommand{\arrowvec}[1]{\tightoverset{\scriptstyle\rightharpoonup}{#1}}
\renewcommand{\vec}{\mathbf}


\colorlet{textColor}{black} 
\colorlet{background}{white}
\colorlet{penColor}{blue!50!black} % Color of a curve in a plot
\colorlet{penColor2}{red!50!black}% Color of a curve in a plot
\colorlet{penColor3}{red!50!blue} % Color of a curve in a plot
\colorlet{penColor4}{green!50!black} % Color of a curve in a plot
\colorlet{penColor5}{orange!80!black} % Color of a curve in a plot
\colorlet{fill1}{penColor!20} % Color of fill in a plot
\colorlet{fill2}{penColor2!20} % Color of fill in a plot
\colorlet{fillp}{fill1} % Color of positive area
\colorlet{filln}{penColor2!20} % Color of negative area
\colorlet{fill3}{penColor3!20} % Fill
\colorlet{fill4}{penColor4!20} % Fill
\colorlet{fill5}{penColor5!20} % Fill
\colorlet{gridColor}{gray!50} % Color of grid in a plot

\newcommand{\surfaceColor}{violet}
\newcommand{\surfaceColorTwo}{redyellow}
\newcommand{\sliceColor}{greenyellow}




\pgfmathdeclarefunction{gauss}{2}{% gives gaussian
  \pgfmathparse{1/(#2*sqrt(2*pi))*exp(-((x-#1)^2)/(2*#2^2))}%
}


%%%%%%%%%%%%%
%% Vectors
%%%%%%%%%%%%%

%% Simple horiz vectors
\renewcommand{\vector}[1]{\left\langle #1\right\rangle}


%% %% Complex Horiz Vectors with angle brackets
%% \makeatletter
%% \renewcommand{\vector}[2][ , ]{\left\langle%
%%   \def\nextitem{\def\nextitem{#1}}%
%%   \@for \el:=#2\do{\nextitem\el}\right\rangle%
%% }
%% \makeatother

%% %% Vertical Vectors
%% \def\vector#1{\begin{bmatrix}\vecListA#1,,\end{bmatrix}}
%% \def\vecListA#1,{\if,#1,\else #1\cr \expandafter \vecListA \fi}

%%%%%%%%%%%%%
%% End of vectors
%%%%%%%%%%%%%

%\newcommand{\fullwidth}{}
%\newcommand{\normalwidth}{}



%% makes a snazzy t-chart for evaluating functions
%\newenvironment{tchart}{\rowcolors{2}{}{background!90!textColor}\array}{\endarray}

%%This is to help with formatting on future title pages.
\newenvironment{sectionOutcomes}{}{} 



%% Flowchart stuff
%\tikzstyle{startstop} = [rectangle, rounded corners, minimum width=3cm, minimum height=1cm,text centered, draw=black]
%\tikzstyle{question} = [rectangle, minimum width=3cm, minimum height=1cm, text centered, draw=black]
%\tikzstyle{decision} = [trapezium, trapezium left angle=70, trapezium right angle=110, minimum width=3cm, minimum height=1cm, text centered, draw=black]
%\tikzstyle{question} = [rectangle, rounded corners, minimum width=3cm, minimum height=1cm,text centered, draw=black]
%\tikzstyle{process} = [rectangle, minimum width=3cm, minimum height=1cm, text centered, draw=black]
%\tikzstyle{decision} = [trapezium, trapezium left angle=70, trapezium right angle=110, minimum width=3cm, minimum height=1cm, text centered, draw=black]


\title[Dig-In:]{Area between curves}

\begin{document}
\begin{abstract}
  We find out how to compute the area of a region between two curves using the definite integral.
\end{abstract}
\maketitle


We have seen how integration can be used to find signed area between a
curve and the $x$-axis. With very little change we can find some areas
between curves. Let's see an example:

\begin{example} Find the area below $f(x)= -x^2+4x+3$ and above
$g(x)=-x^3+7x^2-10x+5$ over the interval $1 \le x \le 2$. 


\begin{image}
\begin{tikzpicture}
	\begin{axis}[
            domain=0:3, ymax=14,xmax=3,ymin=0, xmin=0,
            axis lines =left, xlabel=$x$, ylabel=$y$,
            every axis y label/.style={at=(current axis.above origin),anchor=south},
            every axis x label/.style={at=(current axis.right of origin),anchor=west},
            axis on top,
          ]
          \addplot [draw=none,fill=fillp,domain=1:2] {-x^2+4*x+3} \closedcycle;
          \addplot [draw=none,fill=background,domain=1:2] {-x^3 + 7*x^2-10*x+5} \closedcycle;
          \addplot [draw=penColor,very thick] {-x^2+4*x+3};
          \addplot [draw=penColor2,very thick] {-x^3 + 7*x^2-10*x+5};
          \node at (axis cs:1,6.7) [penColor] {$f$};
          \node at (axis cs:2,4) [penColor2] {$g$};
        \end{axis}
\end{tikzpicture}
\end{image}

\begin{explanation}
Above, we show the two curves together, with the desired area shaded.

It is clear from the figure that the area we want is the area under
$f(x)$ minus the area under $g(x)$, which is to say

\[
\int_1^2 f(x)\d x-\int_1^2 g(x)\d x = \int_1^2 \left(f(x)-g(x)\right)\d x.
\]

\[ 
\int_1^2 f(x)-g(x)\d x = \answer[given]{\frac{49}{12}} 
\]
\begin{hint}

\[
\int_1^2 -x^2+4x+3-(-x^3+7x^2-10x+5) \d x
\]

\begin{align*}
  &=\int_1^2 x^3-8x^2+14x-2 \d x \\
  &=\eval{\frac{x^4}{4}-\frac{8x^3}{3}+7x^2-2x}_1^2 \\
  &= \frac{16}{4} - \frac{64}{3}+28-4-(\frac{1}{4}- \frac{8}{3}+7-2) \\
  &=23-\frac{56}{3}-\frac{1}{4}\\
  &=\frac{49}{12}
\end{align*}
\end{hint}

\end{explanation}

\end{example}




In our first example, one curve was higher than the other over the
entire interval. This does not always happen.

\begin{example} Find the area between $f(x)= -x^2+4x$ and
$g(x)=x^2-6x+5$ over the interval $0 \le x \le 1$.


\begin{image}
\begin{tikzpicture}
	\begin{axis}[
            domain=-1:2, ymax=6,xmax=1.5,ymin=-0.3, xmin=-.5,
            axis lines =center, xlabel=$x$, ylabel=$y$,
	   xtick={0.5635,1},
            xticklabels={$a$,$1$}, 
            every axis y label/.style={at=(current axis.above origin),anchor=south},
            every axis x label/.style={at=(current axis.right of origin),anchor=west},
            axis on top,
          ]
          \addplot [draw=none,fill=fillp,domain=0:0.56] {x^2-6*x+5} \closedcycle;
          \addplot [draw=none,fill=background,domain=0:0.56] {-x^2+4*x} \closedcycle;
          \addplot [draw=none,fill=fillp,domain=0.56:1] {-x^2+4*x} \closedcycle;
          \addplot [draw=none,fill=background,domain=0.56:1] {x^2-6*x+5} \closedcycle;
          \addplot [draw=penColor,very thick] {-x^2+4*x};
          \addplot [draw=penColor2,very thick] {x^2 - 6*x+5};
          \node at (axis cs:1.25,3.1) [penColor] {$f$};
          \node at (axis cs:.25,4.3) [penColor2] {$g$};
          \addplot [textColor,dashed] plot coordinates {(0.5635,0) (0.5635,1.9364)};

        \end{axis}
\end{tikzpicture}
\end{image}


\begin{explanation}
 Generally we should interpret ``area'' in the usual sense, as a necessarily
positive quantity. Since the two curves cross, we need to compute two
areas and add them. First we find the intersection point of the
curves:

Letting $a$ be the x-coordinate of the point of intersection, we have $a = \answer{\frac{5-\sqrt{15}}{2}}$

\begin{hint}
\begin{align*}
  -a^2+4a &= a^2-6a+5 \\
  0 &= 2a^2-10a+5 \\
  a &= \frac{10 \pm \sqrt{100-40}}{4}\\
 a &= \frac{5 \pm \sqrt{15}}{2}.
\end{align*}

Of the two solutions, only $\frac{5-\sqrt{15}}{2}$ is within the region of intererst.
\end{hint}

Then the total area is 

\[
\textrm{Area} = \int_0^a g(x) - f(x) \d x + \int_a^1 f(x) - g(x) \d x
\]

since $g$ is above $f$ on the interval [0,a], and $f$ is above $g$ on the interval $[a,1]$.

Computing this area we have,

\[
	\textrm{Area} = \answer{5\sqrt{15} - \frac{52}{3}}
\]

\begin{hint}

\begin{align*}
\int_0^a &g(x) - f(x) \d x + \int_a^1 f(x) - g(x) \d x \\ 
 &= \int_0^a x^2-6x+5-(-x^2+4x)\d x +\int_a^1 -x^2+4x-(x^2-6x+5)\d x
\end{align*}
\begin{align*}
  &=\int_0^a 2x^2-10x+5\d x+\int_a^1 -2x^2+10x-5\d x \\
  &= \eval{\frac{2x^3}{3}-5x^2+5x}_0^a + \eval{ \frac{-2x^3}{3}+5x^2-5x}_a^1 \\
  &=5\sqrt{15}-\frac{52}{3}
\end{align*}
	
\end{hint}


\end{explanation}

\end{example}

In both of our examples above, we gave you the limits of integration
by bounding the $x$-values between $0$ and $1$. However, in some problems
you will have to do more work to determine these bounds.

\begin{example}
Find the area between $f(x)= -x^2+4x$ and
$g(x)=x^2-6x+5$.


\begin{image}
\begin{tikzpicture}
	\begin{axis}[
            domain=0:5, ymax=5,xmax=5,ymin=-5, xmin=0,
            axis lines =center, xlabel=$x$, ylabel=$y$,
 	   xtick={0.5635},
            xticklabels={$a$}, 
            every axis y label/.style={at=(current axis.above origin),anchor=south},
            every axis x label/.style={at=(current axis.right of origin),anchor=west},
            axis on top,
          ]
          \addplot [draw=none,fill=fillp,domain=.56:4] {-x^2+4*x} \closedcycle;
          \addplot [draw=none,fill=fillp,domain=.56:4.44] {x^2-6*x+5} \closedcycle;
          \addplot [draw=none,fill=background,domain=4:5] {-x^2+4*x} \closedcycle;
          \addplot [draw=none,fill=background,domain=0:1] {x^2-6*x+5} \closedcycle;
          %\addplot [draw=none,fill=fillp,domain=.56:4] {-x^2+4*x} \closedcycle;       
          \addplot [draw=penColor,very thick,smooth] {-x^2+4*x};
          \addplot [draw=penColor2,very thick,smooth] {x^2-6*x+5};
          
          \node at (axis cs:2,4.4) [penColor] {$f$};
          \node at (axis cs:1,-1) [penColor2] {$g$};
	 \node at (axis cs:4.43649,0.3) [textColor] {b};
	 \addplot [textColor,dashed] plot coordinates {(0.5635,0) (0.5635,1.9364)};
          \addplot [textColor,dashed] plot coordinates {(4.43649,0) (4.43649,-1.9364)};
        \end{axis}
\end{tikzpicture}
\end{image}

\begin{explanation}
Here we are not given a specific interval, so it must be the case that
there is a ``natural'' region involved. Since the curves are both
parabolas, the only reasonable interpretation is the region between
the two intersection points, which we found in the previous example to be 
$a=(5-\sqrt{15})/2$ and $b=(5+\sqrt{15})/2$.


Since $f \geq g$ on $[a,b]$, the total area is 

\[
\int_a^b f(x) - g(x) \d x = \answer{4\sqrt{15}}
\]


\begin{hint}
\begin{align*}
  \int_a^b -x^2+4x-(x^2-6x+5)\d x
  &=\int_a^b -2x^2+10x-5\d x \\
  &=\eval{-\frac{2x^3}{3}+5x^2-5x}_a^b \\
  &=5\sqrt{15},
\end{align*}
\end{hint}

%BADBAD should we have an example where it is better to integrate with respect to y?  How about one with a piecewise defined function, or bounded by three curves?


\end{explanation}
\end{example}

\end{document}
