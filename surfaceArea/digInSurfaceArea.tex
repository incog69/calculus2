\documentclass{ximera}

%\usepackage{todonotes}

\newcommand{\todo}{}

\usepackage{tkz-euclide}
\tikzset{>=stealth} %% cool arrow head
\tikzset{shorten <>/.style={ shorten >=#1, shorten <=#1 } } %% allows shorter vectors

\usetikzlibrary{backgrounds} %% for boxes around graphs
\usetikzlibrary{shapes,positioning}  %% Clouds and stars
\usetikzlibrary{matrix} %% for matrix
\usepgfplotslibrary{polar} %% for polar plots
\usetkzobj{all}
\usepackage[makeroom]{cancel} %% for strike outs
%\usepackage{mathtools} %% for pretty underbrace % Breaks Ximera
\usepackage{multicol}





\usepackage{array}
\setlength{\extrarowheight}{+.1cm}   
\newdimen\digitwidth
\settowidth\digitwidth{9}
\def\divrule#1#2{
\noalign{\moveright#1\digitwidth
\vbox{\hrule width#2\digitwidth}}}





\newcommand{\RR}{\mathbb R}
\newcommand{\R}{\mathbb R}
\newcommand{\N}{\mathbb N}
\newcommand{\Z}{\mathbb Z}

%\renewcommand{\d}{\,d\!}
\renewcommand{\d}{\mathop{}\!d}
\newcommand{\dd}[2][]{\frac{\d #1}{\d #2}}
\renewcommand{\l}{\ell}
\newcommand{\ddx}{\frac{d}{\d x}}

\newcommand{\zeroOverZero}{\ensuremath{\boldsymbol{\tfrac{0}{0}}}}
\newcommand{\inftyOverInfty}{\ensuremath{\boldsymbol{\tfrac{\infty}{\infty}}}}
\newcommand{\zeroOverInfty}{\ensuremath{\boldsymbol{\tfrac{0}{\infty}}}}
\newcommand{\zeroTimesInfty}{\ensuremath{\small\boldsymbol{0\cdot \infty}}}
\newcommand{\inftyMinusInfty}{\ensuremath{\small\boldsymbol{\infty - \infty}}}
\newcommand{\oneToInfty}{\ensuremath{\boldsymbol{1^\infty}}}
\newcommand{\zeroToZero}{\ensuremath{\boldsymbol{0^0}}}
\newcommand{\inftyToZero}{\ensuremath{\boldsymbol{\infty^0}}}


\newcommand{\numOverZero}{\ensuremath{\boldsymbol{\tfrac{\#}{0}}}}
\newcommand{\dfn}{\textbf}
%\newcommand{\unit}{\,\mathrm}
\newcommand{\unit}{\mathop{}\!\mathrm}
\newcommand{\eval}[1]{\bigg[ #1 \bigg]}
\newcommand{\seq}[1]{\left( #1 \right)}
\renewcommand{\epsilon}{\varepsilon}
\renewcommand{\iff}{\Leftrightarrow}

\DeclareMathOperator{\arccot}{arccot}
\DeclareMathOperator{\arcsec}{arcsec}
\DeclareMathOperator{\arccsc}{arccsc}
\DeclareMathOperator{\si}{Si}

\newcommand{\tightoverset}[2]{%
  \mathop{#2}\limits^{\vbox to -.5ex{\kern-0.75ex\hbox{$#1$}\vss}}}
\newcommand{\arrowvec}[1]{\tightoverset{\scriptstyle\rightharpoonup}{#1}}
\renewcommand{\vec}{\mathbf}


\colorlet{textColor}{black} 
\colorlet{background}{white}
\colorlet{penColor}{blue!50!black} % Color of a curve in a plot
\colorlet{penColor2}{red!50!black}% Color of a curve in a plot
\colorlet{penColor3}{red!50!blue} % Color of a curve in a plot
\colorlet{penColor4}{green!50!black} % Color of a curve in a plot
\colorlet{penColor5}{orange!80!black} % Color of a curve in a plot
\colorlet{fill1}{penColor!20} % Color of fill in a plot
\colorlet{fill2}{penColor2!20} % Color of fill in a plot
\colorlet{fillp}{fill1} % Color of positive area
\colorlet{filln}{penColor2!20} % Color of negative area
\colorlet{fill3}{penColor3!20} % Fill
\colorlet{fill4}{penColor4!20} % Fill
\colorlet{fill5}{penColor5!20} % Fill
\colorlet{gridColor}{gray!50} % Color of grid in a plot

\newcommand{\surfaceColor}{violet}
\newcommand{\surfaceColorTwo}{redyellow}
\newcommand{\sliceColor}{greenyellow}




\pgfmathdeclarefunction{gauss}{2}{% gives gaussian
  \pgfmathparse{1/(#2*sqrt(2*pi))*exp(-((x-#1)^2)/(2*#2^2))}%
}


%%%%%%%%%%%%%
%% Vectors
%%%%%%%%%%%%%

%% Simple horiz vectors
\renewcommand{\vector}[1]{\left\langle #1\right\rangle}


%% %% Complex Horiz Vectors with angle brackets
%% \makeatletter
%% \renewcommand{\vector}[2][ , ]{\left\langle%
%%   \def\nextitem{\def\nextitem{#1}}%
%%   \@for \el:=#2\do{\nextitem\el}\right\rangle%
%% }
%% \makeatother

%% %% Vertical Vectors
%% \def\vector#1{\begin{bmatrix}\vecListA#1,,\end{bmatrix}}
%% \def\vecListA#1,{\if,#1,\else #1\cr \expandafter \vecListA \fi}

%%%%%%%%%%%%%
%% End of vectors
%%%%%%%%%%%%%

%\newcommand{\fullwidth}{}
%\newcommand{\normalwidth}{}



%% makes a snazzy t-chart for evaluating functions
%\newenvironment{tchart}{\rowcolors{2}{}{background!90!textColor}\array}{\endarray}

%%This is to help with formatting on future title pages.
\newenvironment{sectionOutcomes}{}{} 



%% Flowchart stuff
%\tikzstyle{startstop} = [rectangle, rounded corners, minimum width=3cm, minimum height=1cm,text centered, draw=black]
%\tikzstyle{question} = [rectangle, minimum width=3cm, minimum height=1cm, text centered, draw=black]
%\tikzstyle{decision} = [trapezium, trapezium left angle=70, trapezium right angle=110, minimum width=3cm, minimum height=1cm, text centered, draw=black]
%\tikzstyle{question} = [rectangle, rounded corners, minimum width=3cm, minimum height=1cm,text centered, draw=black]
%\tikzstyle{process} = [rectangle, minimum width=3cm, minimum height=1cm, text centered, draw=black]
%\tikzstyle{decision} = [trapezium, trapezium left angle=70, trapezium right angle=110, minimum width=3cm, minimum height=1cm, text centered, draw=black]


\title[Dig-In:]{Surface area}

\outcome{Know the definition of a frustum.}
\outcome{Know how to compute the surface area of a frustum.}
\outcome{Set-up an integral to compute the area of a surface of revolution in terms of $x$.}
\outcome{Set-up an integral to compute the area of a surface of revolution in terms of $y$.}
\outcome{Compute the area of a surface of revolution.}


\begin{document}
\begin{abstract}
We compute surface area.
\end{abstract}
\maketitle

%% Adoped from APEX

We have already seen how a curve $y=f(x)$ on $[a,b]$ can be revolved
around an axis to form a solid. Instead of computing the volume of
this solid, we'll compute the area of its surface.

\section{The area of a frustum}

%% Herb Clemens! IT IS BEAUTIFUL
To compute the area of a \dfn{surface of revolution}, we will show
that this area is equal to the sum of areas of things that we can
lay out flat. The argument for this goes way back to the great
physicist and mathematician, \index{Archimedes of Alexandria}\textit{Archimedes of Alexandria}. To follow his
argument, we have to begin by computing the area of a `lamp shade' or
\textit{frustum}.

\begin{definition}
  A \dfn{frustum} of a cone is a section of a cone bounded by two
  planes, where both planes are perpendicular to the height of the
  cone.
  \begin{image}[1in]
    \begin{tikzpicture}
      \draw[penColor,very thick,] (-1,0) arc (180:360:1cm and 0.5cm);
      \draw[penColor,very thick,] (-1,0) arc (180:0:1cm and 0.5cm);
      \draw[penColor,very thick,] (-2,-3) arc (180:370:2cm and 1cm);
      \draw[penColor,very thick,dashed] (-2,-3) arc (180:10:2cm and 1cm);
      \draw[penColor,very thick,](-2,-2.9)  -- (-1,0);
      \draw[penColor,very thick,](2,-2.9)   -- (1,0);
      %\node [left] at (2,-1) {$s$};
      \shade[left color=blue!5!white,right color=blue!60!white,opacity=0.3] (-1,0) arc (180:360:1cm and 0.5cm) -- (2,-3) arc (360:180:2cm and 1cm) -- cycle;
      \shade[left color=blue!60!black,right color=blue!5!black,opacity=0.5] (0,0) circle (1cm and 0.5cm);
    \end{tikzpicture}
  \end{image}
\end{definition}

And of course, this writer can think of nothing more interesting than
the area of a frustum:

\begin{theorem}
  The surface area of the frustum; meaning the lateral sides, not the top nor the bottom;
  \begin{image}[1in]
    \begin{tikzpicture}
      \draw[penColor,very thick,] (-1,0) arc (180:360:1cm and 0.5cm);
      \draw[penColor,very thick,] (-1,0) arc (180:0:1cm and 0.5cm);
      \draw[penColor,very thick,] (-2,-3) arc (180:370:2cm and 1cm);
      \draw[penColor,very thick,dashed] (-2,-3) arc (180:10:2cm and 1cm);
      \draw[penColor,very thick,] (-2,-2.9)  -- (-1,0);
      \draw[penColor,very thick,] (2,-2.9)   -- (1,0);
      
      \draw[penColor,very thick,dashed] (0,0)   -- (1,0);
      \draw[penColor,very thick,dashed] (0,-3)   -- (2,-3);

      %\draw[decoration={brace,raise=.1cm},decorate,thin] (0,0)   -- (1,0);
      %\draw[decoration={brace,raise=.1cm},decorate,thin] (0,-3)   -- (2,-3);
      \draw[decoration={brace,mirror,raise=.1cm},decorate,thin] (2,-2.9)--(1,0);
      
      \node [right] at (1.7,-1.4) {$s$};
      \node [above] at (.5,0) {$r$};
      \node [above] at (1,-3) {$R$};               
    \end{tikzpicture}
  \end{image}
  is given by
  \[
  \text{area of frustum} = 2\pi\cdot \frac{r+R}{2}\cdot s.
  \]
  \begin{explanation}
    We can think of a frustum as approximated by an arrangement of $n$
    congruent trapezoids:
    \begin{image}[1in]
      \begin{tikzpicture}
        
      \draw [penColor,very thick,domain=0:360, samples=11] 
      plot ({cos(\x)}, {.5*sin(\x)} );

      \draw [penColor,very thick,domain=0:360, samples=11] 
        plot ({2*cos(\x)}, {sin(\x)-3} );

      \foreach \x in {0,36,72,...,360}
               {
                 \draw[penColor,very thick,] ({cos(\x)},{.5*sin(\x)}) -- ({2*cos(\x)},{sin(\x)-3}) ;
               }
      \end{tikzpicture}
    \end{image}
    Let
    \begin{itemize}
    \item $n$ denote the number of trapezoids,
    \item $t_{n}$ denote the length of the top of each trapezoid,
    \item $h_{n}$ denote the height of each trapezoid,
    \item $b_{n}$ denote the length of the bottom of each trapezoid,
    \end{itemize}
    then using basic geometry, we can show that each of the trapezoids
    \begin{image}[1in]
      \begin{tikzpicture}
        
        \draw [gray,domain=0:360, samples=11] 
        plot ({cos(\x)}, {.5*sin(\x)} );
        
        \draw [gray,domain=0:360, samples=11] 
        plot ({2*cos(\x)}, {sin(\x)-3} );
        
        \foreach \x in {0,36,72,...,360}
                 {
                   \draw[gray] ({cos(\x)},{.5*sin(\x)}) -- ({2*cos(\x)},{sin(\x)-3}) ;
                 }
                 \draw[penColor,very thick]
                 ({cos(252)}, {.5*sin(252)}) --  ({cos(288)}, {.5*sin(288)}) --
                 ({2*cos(288)}, {sin(288)-3}) -- ({2*cos(252)}, {sin(252)-3}) -- ({cos(252)}, {.5*sin(252)});
                 
                 \draw[penColor,very thick,dashed]
                 ({cos(288)}, {.5*sin(288)}) --({cos(288)}, {sin(288)-3});

                 \node [below] at ({(cos(288)+cos(252))},-4) {$b_n$};
                 \node [above] at ({(.25*cos(288)+.25*cos(252))},-.5) {$t_n$};
                 \node [left] at ({cos(288)},-2.25) {$h_n$};
      \end{tikzpicture}
    \end{image}
    have area
    \[
    \left(  \frac{t_{n}+b_n}{2}\right)\cdot h_n.
    \]
    The area of the frustum approximated by the sum of these $n$
    trapezoids, so the area of the frustum is approximated by
    \[
    \answer[given]{n}\cdot\left(\frac{t_{n}+b_n}{2}\right)\cdot h_n
    =\left(\frac{n\cdot t_{n}+n\cdot b_n}{2}\right)\cdot h_n.
    \]
    As $n$ goes to infinity, where $n$ is the number of congruent
    trapezoids,
    \begin{image}
      \begin{tikzpicture}
        
      \draw [penColor,very thick,domain=0:360, samples=11] 
      plot ({cos(\x)}, {.5*sin(\x)} );

      \draw [penColor,very thick,domain=0:360, samples=11] 
        plot ({2*cos(\x)}, {sin(\x)-3} );

      \foreach \x in {0,36,72,...,360}
               {
                 \draw[penColor,very thick,] ({cos(\x)},{.5*sin(\x)}) -- ({2*cos(\x)},{sin(\x)-3}) ;
               }
               

      \draw[penColor,very thick,] (4,0) arc (180:360:1cm and 0.5cm);
      \draw[penColor,very thick,] (4,0) arc (180:0:1cm and 0.5cm);
      \draw[penColor,very thick,] (3,-3) arc (180:370:2cm and 1cm);
      \draw[penColor,very thick,dashed] (3,-3) arc (180:10:2cm and 1cm);
      \draw[penColor,very thick,] (3,-2.9)  -- (4,0);
      \draw[penColor,very thick,] (7,-2.9)   -- (6,0);

      \draw[decoration={brace,mirror,raise=.1cm},decorate,thin] (7,-2.9)--(6,0);
      \node [right] at (6.7,-1.4) {$s$};

      \node[scale=2] at (-3,-2) {$\lim_{n\to \infty}$}; 
      \node[scale=2] at (2.5,-2) {$=$};
      \end{tikzpicture}
    \end{image}
    Hence,
    \[
    \lim_{n\to \infty} \left(\left(\frac{n\cdot t_{n}+n\cdot b_n}{2}\right)\cdot h_n\right) = \text{area of the frustum.}
    \]
    However, if
    \begin{itemize}
    \item $c$ is the circumference of the top circle,
    \item $s$ is the slant height of the frustum as shown in the above
      figure, and 
    \item $C$ is the circumference of the bottom circle,    
    \end{itemize}
    then
    \begin{align*}
      \lim_{n\to \infty} n \cdot t_{n}  &  =c,\\
      \lim_{n\to \infty} h_{n}  &  =s,\\
      \lim_{n\to \infty} n\cdot b_{n}  &  =C,
    \end{align*}
    and by way of limit laws\index{limit laws} we find
    \[
    \lim_{n\to \infty} \left(\left(\frac{n\cdot t_{n}+n\cdot b_n}{2}\right)\cdot h_n\right) = \frac{c + C}{2}\cdot s.
    \]
    Now, letting
    \begin{itemize}
    \item $r$ be the radius of the circle defining the top of the frustum,
    \item $s$ be the slant height of the frustum, and
    \item $R$ be the radius of the circle defining the base of the frustum,
    \end{itemize}
    we see that
    \begin{align*}
      \text{area of frustum}  &= \answer[given]{\frac{c + C}{2}\cdot s} \\
      &= \frac{\answer[given]{2\pi r + 2\pi R}}{2}\cdot s\\
      &=2\pi \cdot \frac{r+R}{2}\cdot s.
    \end{align*}
  \end{explanation}
\end{theorem}
Now we are ready to compute the area of a surface of revolution.

\section{The area of a surface of revolution}
\index{area of a surface of revolution}

Let's consider a function $f$ with a continuous derivative, and the
surface of revolution formed by this curve by rotating around the
$x$-axis:
\begin{image}
\begin{tikzpicture}
  \begin{axis}[
      xmin=1.5, xmax=3.5,
      domain=1.5:3.5,
      ymin=-4.5, ymax=4.5,
      clip=false,
      axis lines =center,
      xlabel=$x$, ylabel=$y$, every axis y label/.style={at=(current axis.above origin),anchor=south},
      every axis x label/.style={at=(current axis.right of origin),anchor=west},
      axis on top,
      xtick={2,2.5},
      xticklabels={$x$,$x+\d x$}
    ]
    \addplot [penColor,very thick,smooth,domain=2:2.5,fill=fill1!50!white,draw=none] {16-19*x+8*x^2-x^3} \closedcycle;
    \addplot [penColor,very thick,smooth,domain=2:2.5,fill=fill1!50!white,draw=none] {-16+19*x-8*x^2+x^3} \closedcycle;

    \draw[penColor,very thick,fill=fill1!20!white] (axis cs:2,0) ellipse (20 and 200);
    %\draw[penColor,very thick,fill=fill1] (axis cs:2.5,0) ellipse (20 and 287.5);
    \draw[penColor,very thick,dashed] (axis cs: 2.5,-2.875) arc (270:90:20 and 287.5);
    \draw[penColor,very thick,fill=fill1!50!white] (axis cs: 2.5,-2.875) arc (270:450:20 and 287.5);

    \addplot [penColor,very thick,smooth]{16-19*x+8*x^2-x^3};
    \addplot [penColor,very thick,smooth,domain=2:2.5]{-16+19*x-8*x^2+x^3};

    \addplot[dashed,->] plot coordinates {(2,2) (2.5,2)};
    \addplot[dashed,->] plot coordinates {(2.5,2) (2.5,2.875)};
    \addplot[->,ultra thick, penColor2] plot coordinates {(2,2) (2.5,2.875)};

    %\node[anchor=north] at (axis cs:2.25,2) {$\d x$};
    %\node[anchor=west] at (axis cs:2.5,2.4375) {$\d y$};
    \node[anchor=south] at (axis cs:2.25,2.6875) {$\d s$};

    \addplot[dashed] plot coordinates {(2,0) (2,2)};
    \addplot[dashed] plot coordinates {(2.5,0) (2.5,2.875)};
    
    \node[penColor] at (axis cs:3,3.5) {$f$};
  \end{axis}
\end{tikzpicture}
\end{image}
Thus the surface area of this frustum is
\[
\d A = 2\pi\frac{f(x)+f(x+\d x)}{2} \d s
\]
and have seen that
\[
\d s =  \sqrt{1+f'(x)^2}\d x
\]
and that as $\d x \to 0$, $f(x+\d x) \to f(x)$, hence
\[
\d A = 2\pi f(x) \sqrt{1+f'(x)^2}\d x.
\]
The upshot of all this work is that if $f$ is differentiable on an
open interval containing $[a,b]$ where $f'$ is also continuous on
$[a,b]$, then the area of the surface of revolution formed by
revolving the graph of $y=f(x)$, where $f(x)\ge 0$, about the $x$-axis
is
\[
\text{Area} = \int_a^b 2\pi f(x)\sqrt{1+f'(x)^2}\d x.
\]


\begin{example}
  Consider the surface of revolution found by revolving $f(x) =
  \sqrt{x}$ around the $x$-axis. Find its area on the interval
  $[0,1]$.
  \begin{explanation}
    Let's look at a picture, with our infinitesimal frustum added:
    \begin{image}
      \begin{tikzpicture}
        \begin{axis}[
            xmin=-.5, xmax=1.5,
            domain=0:1,
            ymin=-1.5, ymax=1.5,
            clip=false,
            axis lines =center,
            xlabel=$x$, ylabel=$y$, every axis y label/.style={at=(current axis.above origin),anchor=south},
            every axis x label/.style={at=(current axis.right of origin),anchor=west},
            axis on top,
          ]
          \addplot [fill=fill1!50!white,draw=none,domain=.4:.6] {sqrt(x)} \closedcycle;
          \addplot [fill=fill1!50!white,draw=none,domain=.4:.6] {-sqrt(x)} \closedcycle;
          
          \draw[penColor,very thick,fill=fill1!20!white] (axis cs:.4,0) ellipse (7 and 63);
          \draw[penColor,very thick,dashed] (axis cs: .6,-.77) arc (270:90:7 and 77);
          \draw[penColor,very thick,fill=fill1!50!white] (axis cs: .6,-.77) arc (270:450:7 and 77);
                    
          \addplot [penColor,very thick,smooth,samples=100]{sqrt(x)};
          \addplot [penColor,very thick,smooth,domain=.4:.6]{-sqrt(x)};
          
          \addplot[dashed,->] plot coordinates {(.4,.63) (.6,.63)};
          \addplot[dashed,->] plot coordinates {(.6,.63) (.6,.77)};
          \addplot[->,ultra thick, penColor2] plot coordinates {(.4,.63) (.6,.77)};
          
          %\node[anchor=north] at (axis cs:.5,.63) {$\d x$};
          %\node[anchor=west] at (axis cs:2.5,2.4375) {$\d y$};
          \node[anchor=south] at (axis cs:.5,.7) {$\d s$};
          
          \addplot[dashed] plot coordinates {(.6,0) (.6,.77)};
          \addplot[dashed] plot coordinates {(.4,0) (.4,.63)};
          
          \node[penColor] at (axis cs:.8,1) {$f$};
        \end{axis}
      \end{tikzpicture}
    \end{image}
    Write with me:
    \[
    f'(x) = \answer[given]{\frac{x^{-1/2}}{2}}
    \]
    So
    \begin{align*}
      \text{Area} &= \int_0^1 2\pi \sqrt{x} \sqrt{1 + \left(\answer[given]{\frac{x^{-1/2}}{2}}\right)^2} \d x\\
      &= \int_0^1 2\pi \sqrt{x} \sqrt{1 + \frac{1}{4x}} \d x\\
      &= \int_0^1 2\pi \sqrt{x+\frac{1}{4}} \d x\\
      &= \eval{\answer[given]{\frac{4\pi}{3}(x+1/4)^{3/2}}}_0^1\\
      &= \frac{4\pi}{3}(1+1/4)^{3/2}  - \frac{4\pi}{3}(1/4)^{3/2}.
    \end{align*}
  \end{explanation}
\end{example}


On the other hand, if $f^{-1}$ is differentiable on
an open interval containing $[a,b]$ where $\dd[x]{y}$ is also continuous on
$[a,b]$, then the area of the surface of revolution formed by
revolving the graph of $x=f^{-1}(y)$, where $f^{-2}(y)\ge 0$, about the $y$-axis
is
\[
\text{Area} = \int_a^b 2\pi f^{-1}(y)\sqrt{1+\left(\dd[x]{y}\right)^2}\d x.
\]
Let's see an example:

\begin{example}
  Consider the surface of revolution found by revolving $f(x) = x^2$
  around the $y$-axis. Find its area on the interval when $0\le y\le
  1$.
  \begin{explanation}
    Let's look at a picture, with our infinitesimal frustum added:
    \begin{image}
      \begin{tikzpicture}
        \begin{axis}[
            xmin=-1.5, xmax=1.5,
            domain=-1:1,
            ymin=-.5, ymax=1.5,
            clip=false,
            axis lines =center,
            xlabel=$x$, ylabel=$y$, every axis y label/.style={at=(current axis.above origin),anchor=south},
            every axis x label/.style={at=(current axis.right of origin),anchor=west},
            axis on top,
          ]
          \addplot [fill=fill1!50!white,draw=none,domain=-.77:.77] {.6} \closedcycle;
          \addplot [fill=white,draw=none,domain=-.77:.77] {.4} \closedcycle;
          \addplot [fill=white,draw=none,domain=-.77:.77] {x^2} \closedcycle;
          
          \draw[penColor,very thick,fill=fill1!20!white] (axis cs: 0,.6) ellipse (77 and 7);
          \draw[penColor,very thick,fill=fill1!50!white] (axis cs:.63,.4) arc (360:180:63 and 7);
          \draw[penColor,very thick,dashed,fill=fill1!50!white] (axis cs: .63,.4) arc (0:180:63 and 7);
          \addplot [penColor,very thick,smooth]{x^2};
                    
          \addplot[dashed,->] plot coordinates {(.63,.4) (.63,.6)};
          \addplot[dashed,->] plot coordinates {(.63,.6) (.77,.6)};
          \addplot[->,ultra thick, penColor2] plot coordinates {(.63,.4) (.77,.6)};
          
          %\node[anchor=south] at (axis cs:.7 ,.63) {$\d x$};
          %\node[anchor=west] at (axis cs:2.5,2.4375) {$\d y$};
          \node[anchor=west] at (axis cs:.7,.5) {$\d s$};
          
          \addplot[dashed] plot coordinates {(0,.6) (.77,.6)};
          \addplot[dashed] plot coordinates {(0,.4) (.63,.4)};
          
          \node[penColor] at (axis cs:.8,1) {$f$};
        \end{axis}
      \end{tikzpicture}
    \end{image}
    When $f$ is restricted to nonnegative numbers, 
    \[
    f^{-1}(y) = \answer[given]{\sqrt{y}}
    \]
    and
    \[
    \dd[x]{y} = \answer[given]{\frac{y^{-1/2}}{2}}
    \]
    So
    \[
    \text{Area} = \int_0^1 2\pi \sqrt{y} \sqrt{1 + \left(\answer[given]{\frac{y^{-1/2}}{2}}\right)^2} \d y
    \]
    but this is the integral we already computed, just with all the $y$'s replaced by $x$'s!   
  \end{explanation}
\end{example}


\begin{question}
  The final answers in the previous two examples are:
  \begin{multipleChoice}
    \choice[correct]{equal}
    \choice{not equal}
  \end{multipleChoice}
  \begin{feedback}
    Since they are computing the same surface area, they are necessarily equal.
  \end{feedback}
\end{question}


As our final example, we will compute the surface area of the sphere.

% Adapted from Guichard/Mike Wills material

\begin{example}
  Compute the surface area of a sphere of radius $R$. 
\begin{explanation}
  The sphere can be obtained by rotating the graph of
  $f(x)=\sqrt{R^2 - x^2}$ about the $x$-axis.
    \begin{image}
      \begin{tikzpicture}
        \begin{axis}[
            xmin=-1.2, xmax=1.2,
            domain=-1:1,
            ymin=-1.2, ymax=1.2,
            clip=false,
            xtick = {-1,1},
            xticklabels = {$-R$,$R$},
            ytick = {-1,1},
            yticklabels = {$-R$,$R$},
            axis lines =center,
            xlabel=$x$, ylabel=$y$, every axis y label/.style={at=(current axis.above origin),anchor=south},
            every axis x label/.style={at=(current axis.right of origin),anchor=west},
            axis on top,
          ]
          \addplot [fill=fill1!50!white,draw=none,domain=.4:.6] {sqrt(1-x^2)} \closedcycle;
          \addplot [fill=fill1!50!white,draw=none,domain=.4:.6] {-sqrt(1-x^2)} \closedcycle;
          
          \draw[penColor,very thick,fill=fill1!20!white] (axis cs:.4,0) ellipse (7 and 92);
          \draw[penColor,very thick,dashed] (axis cs: .6,-.8) arc (270:90:7 and 80);
          \draw[penColor,very thick,fill=fill1!50!white] (axis cs: .6,-.8) arc (270:450:7 and 80);
                    
          %\addplot [penColor,very thick,smooth,samples=100]{sqrt(1-x^2)};
          \draw[penColor,very thick] (axis cs: 1,0) arc (0:180:100 and 100);
          \addplot [penColor,very thick,smooth,domain=.4:.6]{-sqrt(1-x^2)};
          
          \addplot[dashed,->] plot coordinates {(.4,.92) (.6,.92)};
          \addplot[dashed,->] plot coordinates {(.6,.92) (.6,.8)};
          \addplot[->,ultra thick, penColor2] plot coordinates {(.4,.92) (.6,.8)};
          
          %\node[anchor=south] at (axis cs:.5,.92) {$\d x$};
          %\node[anchor=west] at (axis cs:2.5,2.4375) {$\d y$};
          \node[anchor=north] at (axis cs:.5,.8) {$\d s$};
          
          \addplot[dashed] plot coordinates {(.6,0) (.6,.8)};
          \addplot[dashed] plot coordinates {(.4,0) (.4,.92)};
          
          \node[penColor] at (axis cs:-.8,.8) {$f$};
        \end{axis}
      \end{tikzpicture}
    \end{image}
    Write with me
    \[
    f'(x) = \answer[given]{-x/\sqrt{R^2-x^2}},
    \]
  so the area of the surface of revolution is given by:
  \begin{align*}
    A&=\int_{-R }^R 2\pi \sqrt{R^2 - x^2}\sqrt{1+\answer[given]{\frac{x^2}{R^2-x^2}}}\d x \\
    &=\int_{-R }^R 2\pi \sqrt{R^2 - x^2}\sqrt{\frac{R^2}{R^2-x^2}}\d x \\
    &=\int_{-R }^R 2\pi R \d x\\
    &=\eval{\answer[given]{2 \pi R x}}_{-R}^R\\
    &=\answer[given]{4\pi R^2}
  \end{align*}
\end{explanation}
\end{example}







\end{document}
