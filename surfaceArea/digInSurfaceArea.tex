\documentclass{ximera}

%\usepackage{todonotes}

\newcommand{\todo}{}

\usepackage{tkz-euclide}
\tikzset{>=stealth} %% cool arrow head
\tikzset{shorten <>/.style={ shorten >=#1, shorten <=#1 } } %% allows shorter vectors

\usetikzlibrary{backgrounds} %% for boxes around graphs
\usetikzlibrary{shapes,positioning}  %% Clouds and stars
\usetikzlibrary{matrix} %% for matrix
\usepgfplotslibrary{polar} %% for polar plots
\usetkzobj{all}
\usepackage[makeroom]{cancel} %% for strike outs
%\usepackage{mathtools} %% for pretty underbrace % Breaks Ximera
\usepackage{multicol}





\usepackage{array}
\setlength{\extrarowheight}{+.1cm}   
\newdimen\digitwidth
\settowidth\digitwidth{9}
\def\divrule#1#2{
\noalign{\moveright#1\digitwidth
\vbox{\hrule width#2\digitwidth}}}





\newcommand{\RR}{\mathbb R}
\newcommand{\R}{\mathbb R}
\newcommand{\N}{\mathbb N}
\newcommand{\Z}{\mathbb Z}

%\renewcommand{\d}{\,d\!}
\renewcommand{\d}{\mathop{}\!d}
\newcommand{\dd}[2][]{\frac{\d #1}{\d #2}}
\renewcommand{\l}{\ell}
\newcommand{\ddx}{\frac{d}{\d x}}

\newcommand{\zeroOverZero}{\ensuremath{\boldsymbol{\tfrac{0}{0}}}}
\newcommand{\inftyOverInfty}{\ensuremath{\boldsymbol{\tfrac{\infty}{\infty}}}}
\newcommand{\zeroOverInfty}{\ensuremath{\boldsymbol{\tfrac{0}{\infty}}}}
\newcommand{\zeroTimesInfty}{\ensuremath{\small\boldsymbol{0\cdot \infty}}}
\newcommand{\inftyMinusInfty}{\ensuremath{\small\boldsymbol{\infty - \infty}}}
\newcommand{\oneToInfty}{\ensuremath{\boldsymbol{1^\infty}}}
\newcommand{\zeroToZero}{\ensuremath{\boldsymbol{0^0}}}
\newcommand{\inftyToZero}{\ensuremath{\boldsymbol{\infty^0}}}


\newcommand{\numOverZero}{\ensuremath{\boldsymbol{\tfrac{\#}{0}}}}
\newcommand{\dfn}{\textbf}
%\newcommand{\unit}{\,\mathrm}
\newcommand{\unit}{\mathop{}\!\mathrm}
\newcommand{\eval}[1]{\bigg[ #1 \bigg]}
\newcommand{\seq}[1]{\left( #1 \right)}
\renewcommand{\epsilon}{\varepsilon}
\renewcommand{\iff}{\Leftrightarrow}

\DeclareMathOperator{\arccot}{arccot}
\DeclareMathOperator{\arcsec}{arcsec}
\DeclareMathOperator{\arccsc}{arccsc}
\DeclareMathOperator{\si}{Si}

\newcommand{\tightoverset}[2]{%
  \mathop{#2}\limits^{\vbox to -.5ex{\kern-0.75ex\hbox{$#1$}\vss}}}
\newcommand{\arrowvec}[1]{\tightoverset{\scriptstyle\rightharpoonup}{#1}}
\renewcommand{\vec}{\mathbf}


\colorlet{textColor}{black} 
\colorlet{background}{white}
\colorlet{penColor}{blue!50!black} % Color of a curve in a plot
\colorlet{penColor2}{red!50!black}% Color of a curve in a plot
\colorlet{penColor3}{red!50!blue} % Color of a curve in a plot
\colorlet{penColor4}{green!50!black} % Color of a curve in a plot
\colorlet{penColor5}{orange!80!black} % Color of a curve in a plot
\colorlet{fill1}{penColor!20} % Color of fill in a plot
\colorlet{fill2}{penColor2!20} % Color of fill in a plot
\colorlet{fillp}{fill1} % Color of positive area
\colorlet{filln}{penColor2!20} % Color of negative area
\colorlet{fill3}{penColor3!20} % Fill
\colorlet{fill4}{penColor4!20} % Fill
\colorlet{fill5}{penColor5!20} % Fill
\colorlet{gridColor}{gray!50} % Color of grid in a plot

\newcommand{\surfaceColor}{violet}
\newcommand{\surfaceColorTwo}{redyellow}
\newcommand{\sliceColor}{greenyellow}




\pgfmathdeclarefunction{gauss}{2}{% gives gaussian
  \pgfmathparse{1/(#2*sqrt(2*pi))*exp(-((x-#1)^2)/(2*#2^2))}%
}


%%%%%%%%%%%%%
%% Vectors
%%%%%%%%%%%%%

%% Simple horiz vectors
\renewcommand{\vector}[1]{\left\langle #1\right\rangle}


%% %% Complex Horiz Vectors with angle brackets
%% \makeatletter
%% \renewcommand{\vector}[2][ , ]{\left\langle%
%%   \def\nextitem{\def\nextitem{#1}}%
%%   \@for \el:=#2\do{\nextitem\el}\right\rangle%
%% }
%% \makeatother

%% %% Vertical Vectors
%% \def\vector#1{\begin{bmatrix}\vecListA#1,,\end{bmatrix}}
%% \def\vecListA#1,{\if,#1,\else #1\cr \expandafter \vecListA \fi}

%%%%%%%%%%%%%
%% End of vectors
%%%%%%%%%%%%%

%\newcommand{\fullwidth}{}
%\newcommand{\normalwidth}{}



%% makes a snazzy t-chart for evaluating functions
%\newenvironment{tchart}{\rowcolors{2}{}{background!90!textColor}\array}{\endarray}

%%This is to help with formatting on future title pages.
\newenvironment{sectionOutcomes}{}{} 



%% Flowchart stuff
%\tikzstyle{startstop} = [rectangle, rounded corners, minimum width=3cm, minimum height=1cm,text centered, draw=black]
%\tikzstyle{question} = [rectangle, minimum width=3cm, minimum height=1cm, text centered, draw=black]
%\tikzstyle{decision} = [trapezium, trapezium left angle=70, trapezium right angle=110, minimum width=3cm, minimum height=1cm, text centered, draw=black]
%\tikzstyle{question} = [rectangle, rounded corners, minimum width=3cm, minimum height=1cm,text centered, draw=black]
%\tikzstyle{process} = [rectangle, minimum width=3cm, minimum height=1cm, text centered, draw=black]
%\tikzstyle{decision} = [trapezium, trapezium left angle=70, trapezium right angle=110, minimum width=3cm, minimum height=1cm, text centered, draw=black]


\title[Dig-In:]{Surface area}

\begin{document}
\begin{abstract}
We compute surface area.
\end{abstract}
\maketitle

%% Adoped from APEX

We have already seen how a curve $y=f(x)$ on $[a,b]$ can be revolved
around an axis to form a solid. Instead of computing the volume of
this solid, we'll compute the area of its surface.

%% Herb Clemens! IT IS BEAUTIFUL
To compute the area of a \dfn{surface of revolution}, we will show
that its area is equal to the area of the sum of things that we can
lay out flat. The argument for this goes way back to the great
physicist and mathematician, \index{Archimedes of Alexandria}\textit{Archimedes of Alexandria}. To follow his
argument, we have to begin by computing the area of a `lamp shade' or
\textit{frustum}.

\begin{definition}
  A \dfn{frustum} of a cone is a section of a cone bounded by two
  planes, where both planes are perpendicular to the height of the
  cone.
  \begin{image}
    \begin{tikzpicture}
      \draw[penColor,very thick,] (-1,0) arc (180:360:1cm and 0.5cm);
      \draw[penColor,very thick,] (-1,0) arc (180:0:1cm and 0.5cm);
      \draw[penColor,very thick,] (-2,-3) arc (180:370:2cm and 1cm);
      \draw[penColor,very thick,dashed] (-2,-3) arc (180:10:2cm and 1cm);
      \draw[penColor,very thick,](-2,-2.9)  -- (-1,0);
      \draw[penColor,very thick,](2,-2.9)   -- (1,0);
      %\node [left] at (2,-1) {$s$};
      \shade[left color=blue!5!white,right color=blue!60!white,opacity=0.3] (-1,0) arc (180:360:1cm and 0.5cm) -- (2,-3) arc (360:180:2cm and 1cm) -- cycle;
      \shade[left color=blue!60!black,right color=blue!5!black,opacity=0.5] (0,0) circle (1cm and 0.5cm);
    \end{tikzpicture}
  \end{image}
\end{definition}

And of course, this writer can think of nothing more interesting than
the area of a frustum.

\begin{theorem}
  The surface area of the frustum; meaning the lateral sides, not the top nor the bottom;
  \begin{image}
    \begin{tikzpicture}
      \draw[penColor,very thick,] (-1,0) arc (180:360:1cm and 0.5cm);
      \draw[penColor,very thick,] (-1,0) arc (180:0:1cm and 0.5cm);
      \draw[penColor,very thick,] (-2,-3) arc (180:370:2cm and 1cm);
      \draw[penColor,very thick,dashed] (-2,-3) arc (180:10:2cm and 1cm);
      \draw[penColor,very thick,] (-2,-2.9)  -- (-1,0);
      \draw[penColor,very thick,] (2,-2.9)   -- (1,0);
      
      \draw[penColor,very thick,dashed] (0,0)   -- (1,0);
      \draw[penColor,very thick,dashed] (0,-3)   -- (2,-3);

      %\draw[decoration={brace,raise=.1cm},decorate,thin] (0,0)   -- (1,0);
      %\draw[decoration={brace,raise=.1cm},decorate,thin] (0,-3)   -- (2,-3);
      \draw[decoration={brace,mirror,raise=.1cm},decorate,thin] (2,-2.9)--(1,0);
      
      \node [left] at (2,-1) {$s$};
      \node [above] at (.5,0) {$r$};
      \node [above] at (1,-3) {$R$};               
    \end{tikzpicture}
  \end{image}
  is given by
  \[
  A = 2\pi\cdot s\cdot \frac{r+R}{2}.
  \]
  \begin{explanation}
    We can think of a frustum as approximated by an arrangement of $n$
    trapezoids:
    \begin{image}
      \begin{tikzpicture}
        
      \draw [penColor,very thick,domain=0:360, samples=11] 
      plot ({cos(\x)}, {.5*sin(\x)} );

      \draw [penColor,very thick,domain=0:360, samples=11] 
        plot ({2*cos(\x)}, {sin(\x)-3} );

      \foreach \x in {0,36,72,...,360}
               {
                 \draw[penColor,very thick,] ({cos(\x)},{.5*sin(\x)}) -- ({2*cos(\x)},{sin(\x)-3}) ;
               }
      \end{tikzpicture}
    \end{image}
    Let
    \begin{itemize}
    \item $n$ denote the number of trapezoids,
    \item $t_{n}$ denote the length of the top of each trapezoid,
    \item $h_{n}$ denote the height of each trapezoid,
    \item $b_{n}$ denote the length of the bottom of each trapezoid,
    \end{itemize}
    then using basic geometry we can show that each of the trapezoids
    %% \begin{image}
    %%   \begin{tikzpicture}
    %%     \draw[penColor,very thick] (0,0) -- (1,2) -- (3,2) -- (4,0) -- (0,0);
    %%     \draw[penColor,very thick,dashed] (3,0) -- (3,2);
              
    %%     \node [below] at (2,0) {$b$};
    %%     \node [above] at (2,2) {$t$};
    %%     \node [left] at (3,1) {$h$};
    %%   \end{tikzpicture}
    %% \end{image}
    %% is given by
    %% \[
    %% A = h \cdot \frac{b+t}{2}.
    %% \]
    \begin{image}
      \begin{tikzpicture}
        
        \draw [gray,domain=0:360, samples=11] 
        plot ({cos(\x)}, {.5*sin(\x)} );
        
        \draw [gray,domain=0:360, samples=11] 
        plot ({2*cos(\x)}, {sin(\x)-3} );
        
        \foreach \x in {0,36,72,...,360}
                 {
                   \draw[gray] ({cos(\x)},{.5*sin(\x)}) -- ({2*cos(\x)},{sin(\x)-3}) ;
                 }
                 \draw[penColor,very thick]
                 ({cos(252)}, {.5*sin(252)}) --  ({cos(288)}, {.5*sin(288)}) --
                 ({2*cos(288)}, {sin(288)-3}) -- ({2*cos(252)}, {sin(252)-3}) -- ({cos(252)}, {.5*sin(252)});
                 
                 \draw[penColor,very thick,dashed]
                 ({cos(288)}, {.5*sin(288)}) --({cos(288)}, {sin(288)-3});

                 \node [below] at ({(cos(288)+cos(252))},-4) {$b_n$};
                 \node [above] at ({(.25*cos(288)+.25*cos(252))},-.5) {$t_n$};
                 \node [left] at ({cos(288)},-2.25) {$h_n$};
      \end{tikzpicture}
    \end{image}
    have area
    \[
    h_{n}\cdot \left(  \frac{b_{n}+t_{n}}{2}\right).
    \]
    The area of the frustum approximated by all of these $n$
    trapezoids, so the area of the frustum is approximated by
    \[
    n\cdot h_n\cdot\left(\frac{b_{n}+t_{n}}{2}\right)
    \]
    \[
    =h_n\cdot \left(\frac{n\cdot b_{n}+n\cdot t_{n}}{2}\right).
    \]
    As $n$ goes to infinity, the area of the approximation approaches the area of
    the collar. But if 
    \begin{itemize}
    \item $c_{b}$ is the circumference of the bottom circle
    \item $c_{t}$ is the circumference of the top circle and 
    \item $s$ is the slant height of the collar as shown in the above
      figure,
    \end{itemize}
    then
    \begin{align*}
      \lim_{i\to \infty} n\cdot b_{n}  &  =c_{b}\\
      \lim_{n\to \infty} n \cdot t_{n}  &  =c_{t}\\
      \lim_{n\to \infty} h_{n}  &  =s.
    \end{align*}
    Let
    \begin{itemize}
    \item $r_b$ be the radius of the circle defining the base of the collar,
    \item $r_t$ be the radius of the circle defining the top of the collar,
    \item $r_a$ be the average of $r_b$ and $r_t$,
    \item $s$ be the slant height of the collar.
    \end{itemize}
    Explain why the area of the collar is
    \[
    \pi \cdot \left( r_{b}+r_{t}\right)\cdot s =2\pi\cdot r_{a}\cdot s
    \]
  \end{explanation}
\end{theorem}




Thus the surface area of this sample frustum of the cone is
approximately
\[
2\pi\frac{f(x_i)+f(x_{i+1})}2\sqrt{1+f'(c_i)^2}\d x_i.
\]
Since $f$ is a continuous function, the Intermediate Value Theorem\index{Intermediate Value Theorem}
states there is some $d_i$ in $[x_i,x_{i+1}]$ such that
\[
f(d_i) = \frac{f(x_i)+f(x_{i+1})}{2};
\]
we can use this to rewrite the above equation as
\[
2\pi f(d_i)\sqrt{1+f'(c_i)^2}\d x_i.
\]
Summing over all the subintervals we get the total surface area to be
approximately
\[
\text{Surface Area}\approx \sum_{i=1}^n 2\pi f(d_i)\sqrt{1+f'(c_i)^2}\d x_i,
\]
which is a Riemann Sum. Taking the limit as the subinterval lengths go
to zero gives us the exact surface area, given in the following Key
Idea.

%\keyidea{idea:surface_area}{Surface Area of a Solid of Revolution}
Let $f$ be differentiable on an open interval containing $[a,b]$ where
$f'$ is also continuous on $[a,b]$.
\index{integration!surface area}\index{surface area!solid of revolution}
\begin{enumerate}
\item The surface area of the solid formed by revolving the graph of
  $y=f(x)$, where $f(x)\geq0$, about the $x$-axis is
  \[
  \text{Surface Area} = 2\pi\int_a^b f(x)\sqrt{1+f'(x)^2}\d x.
  \]
\item The surface area of the solid formed by revolving the graph of
  $y=f(x)$ about the $y$-axis, where $a,b\ge 0$, is
  \[
  \text{Surface Area} = 2\pi\int_a^b x\sqrt{1+f'(x)^2}\d x.
  \]
\end{enumerate}


(When revolving $y=f(x)$ about the $y$-axis, the radii of the
resulting frustum are $x_i$ and $x_{i+1}$; their average value is
simply the midpoint of the interval. In the limit, this midpoint is
just $x$. This gives the second part of Key Idea
\ref{idea:surface_area}.)

\begin{example}
Find the surface area of the solid formed by revolving $y=\sin x$ on
$[0,\pi]$ around the $x$-axis, as shown in Figure \ref{fig:sa1}.
Revolving $y=\sin x$ on $[0,\pi]$ about the $x$-axis.  The setup is
relatively straightforward. Using Key Idea \ref{idea:surface_area}, we
have the surface area $SA$ is:
\begin{align*}
SA  &=	2\pi\int_0^\pi \sin x\sqrt{1+\cos^2x}\d x \\
		&=	-2\pi\frac12\left.\left(\sinh^{-1}(\cos x)+\cos x\sqrt{1+\cos^2x}\right)\right|_0^\pi \\
		&= 2\pi\left(\sqrt{2}+\sinh^{-1} 1\right) \\
		&\approx 14.42\ \text{units}^2.
\end{align*}
The integration step above is nontrivial, utilizing an integration method called Trigonometric Substitution. 

It is interesting to see that the surface area of a solid, whose shape is defined by a trigonometric function, involves both a square root and an inverse hyperbolic trigonometric function.
\end{example}

\begin{example}
Find the surface area of the solid formed by revolving the curve
$y=x^2$ on $[0,1]$ about:
		\begin{enumerate}
		\item		the $x$-axis
		\item		the $y$-axis.
		\end{enumerate}
                Like the integral in Example \ref{ex_sa1}, this requires Trigonometric Substitution.
\begin{enumerate}
	\item		The integral is straightforward to setup:
	\begin{align*}
	SA &= 2\pi\int_0^1 x^2\sqrt{1+(2x)^2}\d x\\
		&= \left.\frac{\pi}{32}\left(2(8x^3+x)\sqrt{1+4x^2}-\sinh^{-1}(2x)\right)\right|_0^1\\
		&=\frac{\pi}{32}\left(18\sqrt{5}-\sinh^{-1}2\right)\\
		&\approx 3.81\ \text{units}^2.
	\end{align*}
	The solid formed by revolving $y=x^2$ around the $x$-axis is graphed in Figure \ref{fig:sa2} (a).
	
	\item	 Since we are revolving around the $y$-axis, the ``radius'' of the solid is not $f(x)$ but rather $x$. Thus the integral to compute the surface area is: This integral can be solved using substitution. Set $u=1+4x^2$; the new bounds are $u=1$ to $u=5$. We then have 
	\begin{align*}
	SA &= 2\pi\int_0^1x\sqrt{1+(2x)^2}\d x.
		&=	\frac{\pi}4\int_1^5 \sqrt{u}\d u \\
		&= \left.\frac{\pi}{4}\frac23 u^{3/2}\right|_1^5\\
		&= \frac{\pi}6\left(5\sqrt{5}-1\right)\\
		&\approx 5.33\ \text{units}^2.
	\end{align*}
 The solid formed by revolving $y=x^2$ about the $y$-axis is graphed in Figure \ref{fig:sa2} (b).	
\end{enumerate}

\end{example}



% Adapted from Guichard/Mike Wills material

\begin{example}
  We compute the surface area of a sphere of radius $r$.  The sphere
  can be obtained by rotating the graph of $f(x)=\sqrt{r^2 - x^2}$
  about the $x$-axis.  The derivative $f'$ is $-x/\sqrt{r^2-x^2}$,
  so the surface area is given by
  \begin{align*}
    A&=2\pi \int_{-r }^r \sqrt{r^2 - x^2}\sqrt{1+\frac{x^2}{r^2-x^2}}\d x \\
    &=2\pi \int_{-r }^r \sqrt{r^2 - x^2}\sqrt{\frac{r^2}{r^2-x^2}}\d x \\
    &=2\pi \int_{-r }^r r\d x\\
    &=2\pi r\int_{-r }^r 1\d x\\
    &=4\pi r^2
  \end{align*}
\end{example}
\end{document}
