\documentclass{ximera}

%\usepackage{todonotes}

\newcommand{\todo}{}

\usepackage{tkz-euclide}
\tikzset{>=stealth} %% cool arrow head
\tikzset{shorten <>/.style={ shorten >=#1, shorten <=#1 } } %% allows shorter vectors

\usetikzlibrary{backgrounds} %% for boxes around graphs
\usetikzlibrary{shapes,positioning}  %% Clouds and stars
\usetikzlibrary{matrix} %% for matrix
\usepgfplotslibrary{polar} %% for polar plots
\usetkzobj{all}
\usepackage[makeroom]{cancel} %% for strike outs
%\usepackage{mathtools} %% for pretty underbrace % Breaks Ximera
\usepackage{multicol}





\usepackage{array}
\setlength{\extrarowheight}{+.1cm}   
\newdimen\digitwidth
\settowidth\digitwidth{9}
\def\divrule#1#2{
\noalign{\moveright#1\digitwidth
\vbox{\hrule width#2\digitwidth}}}





\newcommand{\RR}{\mathbb R}
\newcommand{\R}{\mathbb R}
\newcommand{\N}{\mathbb N}
\newcommand{\Z}{\mathbb Z}

%\renewcommand{\d}{\,d\!}
\renewcommand{\d}{\mathop{}\!d}
\newcommand{\dd}[2][]{\frac{\d #1}{\d #2}}
\renewcommand{\l}{\ell}
\newcommand{\ddx}{\frac{d}{\d x}}

\newcommand{\zeroOverZero}{\ensuremath{\boldsymbol{\tfrac{0}{0}}}}
\newcommand{\inftyOverInfty}{\ensuremath{\boldsymbol{\tfrac{\infty}{\infty}}}}
\newcommand{\zeroOverInfty}{\ensuremath{\boldsymbol{\tfrac{0}{\infty}}}}
\newcommand{\zeroTimesInfty}{\ensuremath{\small\boldsymbol{0\cdot \infty}}}
\newcommand{\inftyMinusInfty}{\ensuremath{\small\boldsymbol{\infty - \infty}}}
\newcommand{\oneToInfty}{\ensuremath{\boldsymbol{1^\infty}}}
\newcommand{\zeroToZero}{\ensuremath{\boldsymbol{0^0}}}
\newcommand{\inftyToZero}{\ensuremath{\boldsymbol{\infty^0}}}


\newcommand{\numOverZero}{\ensuremath{\boldsymbol{\tfrac{\#}{0}}}}
\newcommand{\dfn}{\textbf}
%\newcommand{\unit}{\,\mathrm}
\newcommand{\unit}{\mathop{}\!\mathrm}
\newcommand{\eval}[1]{\bigg[ #1 \bigg]}
\newcommand{\seq}[1]{\left( #1 \right)}
\renewcommand{\epsilon}{\varepsilon}
\renewcommand{\iff}{\Leftrightarrow}

\DeclareMathOperator{\arccot}{arccot}
\DeclareMathOperator{\arcsec}{arcsec}
\DeclareMathOperator{\arccsc}{arccsc}
\DeclareMathOperator{\si}{Si}

\newcommand{\tightoverset}[2]{%
  \mathop{#2}\limits^{\vbox to -.5ex{\kern-0.75ex\hbox{$#1$}\vss}}}
\newcommand{\arrowvec}[1]{\tightoverset{\scriptstyle\rightharpoonup}{#1}}
\renewcommand{\vec}{\mathbf}


\colorlet{textColor}{black} 
\colorlet{background}{white}
\colorlet{penColor}{blue!50!black} % Color of a curve in a plot
\colorlet{penColor2}{red!50!black}% Color of a curve in a plot
\colorlet{penColor3}{red!50!blue} % Color of a curve in a plot
\colorlet{penColor4}{green!50!black} % Color of a curve in a plot
\colorlet{penColor5}{orange!80!black} % Color of a curve in a plot
\colorlet{fill1}{penColor!20} % Color of fill in a plot
\colorlet{fill2}{penColor2!20} % Color of fill in a plot
\colorlet{fillp}{fill1} % Color of positive area
\colorlet{filln}{penColor2!20} % Color of negative area
\colorlet{fill3}{penColor3!20} % Fill
\colorlet{fill4}{penColor4!20} % Fill
\colorlet{fill5}{penColor5!20} % Fill
\colorlet{gridColor}{gray!50} % Color of grid in a plot

\newcommand{\surfaceColor}{violet}
\newcommand{\surfaceColorTwo}{redyellow}
\newcommand{\sliceColor}{greenyellow}




\pgfmathdeclarefunction{gauss}{2}{% gives gaussian
  \pgfmathparse{1/(#2*sqrt(2*pi))*exp(-((x-#1)^2)/(2*#2^2))}%
}


%%%%%%%%%%%%%
%% Vectors
%%%%%%%%%%%%%

%% Simple horiz vectors
\renewcommand{\vector}[1]{\left\langle #1\right\rangle}


%% %% Complex Horiz Vectors with angle brackets
%% \makeatletter
%% \renewcommand{\vector}[2][ , ]{\left\langle%
%%   \def\nextitem{\def\nextitem{#1}}%
%%   \@for \el:=#2\do{\nextitem\el}\right\rangle%
%% }
%% \makeatother

%% %% Vertical Vectors
%% \def\vector#1{\begin{bmatrix}\vecListA#1,,\end{bmatrix}}
%% \def\vecListA#1,{\if,#1,\else #1\cr \expandafter \vecListA \fi}

%%%%%%%%%%%%%
%% End of vectors
%%%%%%%%%%%%%

%\newcommand{\fullwidth}{}
%\newcommand{\normalwidth}{}



%% makes a snazzy t-chart for evaluating functions
%\newenvironment{tchart}{\rowcolors{2}{}{background!90!textColor}\array}{\endarray}

%%This is to help with formatting on future title pages.
\newenvironment{sectionOutcomes}{}{} 



%% Flowchart stuff
%\tikzstyle{startstop} = [rectangle, rounded corners, minimum width=3cm, minimum height=1cm,text centered, draw=black]
%\tikzstyle{question} = [rectangle, minimum width=3cm, minimum height=1cm, text centered, draw=black]
%\tikzstyle{decision} = [trapezium, trapezium left angle=70, trapezium right angle=110, minimum width=3cm, minimum height=1cm, text centered, draw=black]
%\tikzstyle{question} = [rectangle, rounded corners, minimum width=3cm, minimum height=1cm,text centered, draw=black]
%\tikzstyle{process} = [rectangle, minimum width=3cm, minimum height=1cm, text centered, draw=black]
%\tikzstyle{decision} = [trapezium, trapezium left angle=70, trapezium right angle=110, minimum width=3cm, minimum height=1cm, text centered, draw=black]


\outcome{Use the root test to determine if a series diverges or converges.}
\outcome{Determine when to use the ratio or root test.}

\title[Dig-In:]{The root test}

\begin{document}
\begin{abstract}
Some infinite series can be compared to geometric series.
\end{abstract}
\maketitle

We learned that the ratio test is a powerful technique based on the
concept of recognizing when a series is ``approximately'' geometric.
If
\[
\lim_{k \to \infty} \frac{a_{k+1}}{a_k} = r,
\]
then the ``tail'' of the series looks like a geometric series of ratio
$r$, and follows the same convergence and divergence behavior as a
geometric series.  The \textit{root test} uses a similar idea in a
slightly different situation.
\begin{theorem}[The Root Test]\index{root test}
  If $\sum^\infty a_k$ is an infinite series of positive terms, and $\lim_{k \to \infty} \sqrt[k]{a_k} = r$, then 
  \begin{itemize}
  \item If $0 \leq r < 1$, then the series converges.
  \item If $r>1$, then the series diverges.
  \item If $r = 1$, then we learn nothing:  the series could diverge or converge.
  \end{itemize}
\end{theorem}
Notice that the conclusion of the root test follows exactly the same
form as the ratio test.  It does so for exactly the same reason:
\begin{quote}
  If $\sqrt[k]{a_k} \approx r$ for large $k$, then $a_k \approx r^k$
  for large $k$, which exactly says that the tail of $a_k$ looks like
  a geometric series with ratio $r$.
\end{quote}
Again, we do not give a formal proof in this course (but if you are
interested you can find a proof online!)

\begin{example}
  Consider 
  \[
  \sum_{k=4}^\infty \frac{k^5}{k^k}
  \]
  Discuss the convergence of this series.
  \begin{explanation}
    We will attempt to use the root test. Setting
    $a_k=\frac{k^5}{k^k}$, write with me
    \[
    \lim_{k \to \infty} \sqrt[k]{a_k} = \answer[given]{0}	
    \]
    so the root test
	  \wordChoice{
	   \choice[correct]{says the series is convergent}
	   \choice{says the series is divergent}
	   \choice{gives no information in this case, but we know the series is convergent through some other method}
	   \choice{gives no information in this case, but we know the series is divergent through some other method}}.		
	  \begin{hint}
            \begin{align*}
	      \lim_{k \to \infty} \sqrt[k]{a_k} &= \lim_{k \to \infty} \sqrt[k]{\frac{k^5}{k^k}}\\
	      &=\lim_{k \to \infty} \frac{(k^{\frac{1}{k}})^5}{k}\\
	      &=0 \quad\text{since $\lim_{k \to \infty} k^\frac{1}{k} = 1$}
	    \end{align*}
	    So the series is convergent by the root test.
	  \end{hint}
  \end{explanation}
\end{example}

\begin{example}
  Consider 
  \[
  \sum_{k=3}^\infty \frac{2^k}{k^2}
  \]
  Discuss the convergence of this series.
  \begin{explanation}
    We will attempt to use the root test. Setting
    $a_k=\frac{2^k}{k^2}$, write with me
    \[
    \lim_{k \to \infty} \sqrt[k]{\frac{2^k}{k^2}} = \answer[given]{2}	
    \]
    so the root test
	  \wordChoice{
	   \choice{says the series is convergent}
	   \choice[correct]{says the series is divergent}
	   \choice{gives no information in this case, but we know the series is convergent through some other method}
	   \choice{gives no information in this case, but we know the series is divergent through some other method}}.		
	  \begin{hint}
            \begin{align*}
	      \lim_{k \to \infty} \sqrt[k]{a_k} &= \lim_{k \to \infty} \sqrt[k]{\frac{2^k}{k^2}}\\
	      &=\lim_{k \to \infty} \frac{2}{(k^{1/k})^2}\\
	      &=2 \quad\text{since $\lim_{k \to \infty} k^\frac{1}{k} = 1$}
	    \end{align*}
	    So the series is divergent by the root test.
	  \end{hint}
  \end{explanation}
\end{example}

\begin{example}
  Consider 
  \[
  \sum_{k=1}^\infty \left(\frac{k^2-k}{k^2+k}\right)^k
  \]
  Discuss the convergence of this series.
  \begin{explanation}
    We will attempt to use the root test. Setting
    $a_k=\left(\frac{k^2-k}{k^2+k}\right)^k$, write with me
    \[
    \lim_{k \to \infty} \sqrt[k]{\left(\frac{k^2-k}{k^2+k}\right)^k} = \answer[given]{1}	
    \]
    so the root test
	  \wordChoice{
	   \choice{says the series is convergent}
	   \choice{says the series is divergent}
	   \choice{gives no information in this case, but we know the series is convergent through some other method}
	   \choice[correct]{gives no information in this case, but we know the series is divergent through some other method}}.		
	  \begin{hint}
            \begin{align*}
	      \lim_{k \to \infty} \sqrt[k]{a_k} &= \lim_{k \to \infty} \sqrt[k]{\left(\frac{k^2-k}{k^2+k}\right)^k}\\
	      &=\lim_{k \to \infty}\frac{k^2-k}{k^2+k}\\
	      &=1
	    \end{align*}
	    So the root test gives no information.  However, we know this series diverges by the divergence test.
	  \end{hint}
  \end{explanation}
\end{example}


\begin{example}
  Consider 
  \[
  \sum_{k=1}^\infty \frac{1}{k^2}
  \]
  Discuss the convergence of this series.
  \begin{explanation}
    We will attempt to use the root test. Setting
    $a_k=\frac{1}{k^2}$, write with me
    \[
    \lim_{k \to \infty} \sqrt[k]{\frac{1}{k^2}} = \answer[given]{1}	
    \]
    so the root test
	  \wordChoice{
	   \choice{says the series is convergent}
	   \choice{says the series is divergent}
	   \choice[correct]{gives no information in this case, but we know the series is convergent through some other method}
	   \choice{gives no information in this case, but we know the series is divergent through some other method}}.		
	  \begin{hint}
            \begin{align*}
	      \lim_{k \to \infty} \sqrt[k]{a_k} &= \lim_{k \to \infty} \sqrt[k]{\frac{1}{k^2}}\\
	      &=\lim_{k \to \infty}\left(k^{1/k}\right)^{-2}\\
	      &=1 \quad\text{since $\lim_{k \to \infty} k^\frac{1}{k} = 1$}
	    \end{align*}
	    So the root test gives no information.  However, we know
            this series converges by the $p$-test.
	  \end{hint}
  \end{explanation}
\end{example}

Generally the root test is most useful when you have a lot of powers,
and no factorials.  Anytime you see a factorial is a pretty good sign
to try the ratio test.







\end{document}


