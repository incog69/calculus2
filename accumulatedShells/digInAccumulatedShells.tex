\documentclass{ximera}

%\usepackage{todonotes}

\newcommand{\todo}{}

\usepackage{tkz-euclide}
\tikzset{>=stealth} %% cool arrow head
\tikzset{shorten <>/.style={ shorten >=#1, shorten <=#1 } } %% allows shorter vectors

\usetikzlibrary{backgrounds} %% for boxes around graphs
\usetikzlibrary{shapes,positioning}  %% Clouds and stars
\usetikzlibrary{matrix} %% for matrix
\usepgfplotslibrary{polar} %% for polar plots
\usetkzobj{all}
\usepackage[makeroom]{cancel} %% for strike outs
%\usepackage{mathtools} %% for pretty underbrace % Breaks Ximera
\usepackage{multicol}





\usepackage{array}
\setlength{\extrarowheight}{+.1cm}   
\newdimen\digitwidth
\settowidth\digitwidth{9}
\def\divrule#1#2{
\noalign{\moveright#1\digitwidth
\vbox{\hrule width#2\digitwidth}}}





\newcommand{\RR}{\mathbb R}
\newcommand{\R}{\mathbb R}
\newcommand{\N}{\mathbb N}
\newcommand{\Z}{\mathbb Z}

%\renewcommand{\d}{\,d\!}
\renewcommand{\d}{\mathop{}\!d}
\newcommand{\dd}[2][]{\frac{\d #1}{\d #2}}
\renewcommand{\l}{\ell}
\newcommand{\ddx}{\frac{d}{\d x}}

\newcommand{\zeroOverZero}{\ensuremath{\boldsymbol{\tfrac{0}{0}}}}
\newcommand{\inftyOverInfty}{\ensuremath{\boldsymbol{\tfrac{\infty}{\infty}}}}
\newcommand{\zeroOverInfty}{\ensuremath{\boldsymbol{\tfrac{0}{\infty}}}}
\newcommand{\zeroTimesInfty}{\ensuremath{\small\boldsymbol{0\cdot \infty}}}
\newcommand{\inftyMinusInfty}{\ensuremath{\small\boldsymbol{\infty - \infty}}}
\newcommand{\oneToInfty}{\ensuremath{\boldsymbol{1^\infty}}}
\newcommand{\zeroToZero}{\ensuremath{\boldsymbol{0^0}}}
\newcommand{\inftyToZero}{\ensuremath{\boldsymbol{\infty^0}}}


\newcommand{\numOverZero}{\ensuremath{\boldsymbol{\tfrac{\#}{0}}}}
\newcommand{\dfn}{\textbf}
%\newcommand{\unit}{\,\mathrm}
\newcommand{\unit}{\mathop{}\!\mathrm}
\newcommand{\eval}[1]{\bigg[ #1 \bigg]}
\newcommand{\seq}[1]{\left( #1 \right)}
\renewcommand{\epsilon}{\varepsilon}
\renewcommand{\iff}{\Leftrightarrow}

\DeclareMathOperator{\arccot}{arccot}
\DeclareMathOperator{\arcsec}{arcsec}
\DeclareMathOperator{\arccsc}{arccsc}
\DeclareMathOperator{\si}{Si}

\newcommand{\tightoverset}[2]{%
  \mathop{#2}\limits^{\vbox to -.5ex{\kern-0.75ex\hbox{$#1$}\vss}}}
\newcommand{\arrowvec}[1]{\tightoverset{\scriptstyle\rightharpoonup}{#1}}
\renewcommand{\vec}{\mathbf}


\colorlet{textColor}{black} 
\colorlet{background}{white}
\colorlet{penColor}{blue!50!black} % Color of a curve in a plot
\colorlet{penColor2}{red!50!black}% Color of a curve in a plot
\colorlet{penColor3}{red!50!blue} % Color of a curve in a plot
\colorlet{penColor4}{green!50!black} % Color of a curve in a plot
\colorlet{penColor5}{orange!80!black} % Color of a curve in a plot
\colorlet{fill1}{penColor!20} % Color of fill in a plot
\colorlet{fill2}{penColor2!20} % Color of fill in a plot
\colorlet{fillp}{fill1} % Color of positive area
\colorlet{filln}{penColor2!20} % Color of negative area
\colorlet{fill3}{penColor3!20} % Fill
\colorlet{fill4}{penColor4!20} % Fill
\colorlet{fill5}{penColor5!20} % Fill
\colorlet{gridColor}{gray!50} % Color of grid in a plot

\newcommand{\surfaceColor}{violet}
\newcommand{\surfaceColorTwo}{redyellow}
\newcommand{\sliceColor}{greenyellow}




\pgfmathdeclarefunction{gauss}{2}{% gives gaussian
  \pgfmathparse{1/(#2*sqrt(2*pi))*exp(-((x-#1)^2)/(2*#2^2))}%
}


%%%%%%%%%%%%%
%% Vectors
%%%%%%%%%%%%%

%% Simple horiz vectors
\renewcommand{\vector}[1]{\left\langle #1\right\rangle}


%% %% Complex Horiz Vectors with angle brackets
%% \makeatletter
%% \renewcommand{\vector}[2][ , ]{\left\langle%
%%   \def\nextitem{\def\nextitem{#1}}%
%%   \@for \el:=#2\do{\nextitem\el}\right\rangle%
%% }
%% \makeatother

%% %% Vertical Vectors
%% \def\vector#1{\begin{bmatrix}\vecListA#1,,\end{bmatrix}}
%% \def\vecListA#1,{\if,#1,\else #1\cr \expandafter \vecListA \fi}

%%%%%%%%%%%%%
%% End of vectors
%%%%%%%%%%%%%

%\newcommand{\fullwidth}{}
%\newcommand{\normalwidth}{}



%% makes a snazzy t-chart for evaluating functions
%\newenvironment{tchart}{\rowcolors{2}{}{background!90!textColor}\array}{\endarray}

%%This is to help with formatting on future title pages.
\newenvironment{sectionOutcomes}{}{} 



%% Flowchart stuff
%\tikzstyle{startstop} = [rectangle, rounded corners, minimum width=3cm, minimum height=1cm,text centered, draw=black]
%\tikzstyle{question} = [rectangle, minimum width=3cm, minimum height=1cm, text centered, draw=black]
%\tikzstyle{decision} = [trapezium, trapezium left angle=70, trapezium right angle=110, minimum width=3cm, minimum height=1cm, text centered, draw=black]
%\tikzstyle{question} = [rectangle, rounded corners, minimum width=3cm, minimum height=1cm,text centered, draw=black]
%\tikzstyle{process} = [rectangle, minimum width=3cm, minimum height=1cm, text centered, draw=black]
%\tikzstyle{decision} = [trapezium, trapezium left angle=70, trapezium right angle=110, minimum width=3cm, minimum height=1cm, text centered, draw=black]


\title[Dig-In:]{Accumulated shells}

\begin{document}
\begin{abstract}
SOMETHING
\end{abstract}
\maketitle


\begin{image}
\begin{tikzpicture}
  \begin{axis}[
      xmin=0, xmax=3,domain=0:3,
      clip=false,
      axis lines =center, xlabel=$x$, ylabel=$y$,
      every axis y label/.style={at=(current axis.above origin),anchor=south},
      every axis x label/.style={at=(current axis.right of origin),anchor=west},
      axis on top,
    ] 
    \addplot [penColor2,very thick] {x+1};
    \addplot [penColor,very thick] {(x-1)^2};

    \node at (axis cs:.5,1.7) [penColor2] {$f(x)$};
    \node at (axis cs:2.5,1.7) [penColor] {$g(x)$};
    \addplot [very thick, penColor4] plot coordinates {(.25,.56) (1.75,.56)};
    \addplot [very thick, penColor4] plot coordinates {(1,2) (2.41,2)};
    \end{axis}
\end{tikzpicture}
%% \caption{A plot of $f(x) = x+1$ and $g(x) = (x-1)^2$ with the two
%%   types of ``washers'' indicated.}
%% \label{figure:washerHARD}
\end{image}

Suppose the region between $f(x)=x+1$ and $g(x)=(x-1)^2$ is rotated
around the $y$-axis. It is possible, but inconvenient, to compute the
volume of the resulting solid by the method we have used so far. The
problem is that there are two ``kinds'' of typical rectangles: those
that go from the line to the parabola and those that touch the
parabola on both ends, see Figure~\ref{figure:washerHARD}. To compute
the volume using this approach, we need to break the problem into two
parts and compute two integrals:
\[
  \pi\int_0^1 (1+\sqrt{y})^2-(1-\sqrt{y})^2\d y+
  \pi\int_1^4  (1+\sqrt{y})^2-(y-1)^2\d y={8\over3}\pi + {65\over6}\pi
  ={27\over2}\pi.
\]

\begin{image}
\begin{tikzpicture}
  \begin{axis}[
      xmin=0, xmax=3,domain=0:3,
      clip=false,
      axis lines =center, xlabel=$x$, ylabel=$y$,
      every axis y label/.style={at=(current axis.above origin),anchor=south},
      every axis x label/.style={at=(current axis.right of origin),anchor=west},
      axis on top,
    ] 
    \addplot [penColor2,very thick] {x+1};
    \addplot [penColor,very thick] {(x-1)^2};

    \node at (axis cs:.5,1.7) [penColor2] {$f(x)$};
    \node at (axis cs:2.5,1.7) [penColor] {$g(x)$};
    \addplot [very thick, penColor4] plot coordinates {(1.5,2.5) (1.5,.25)};
  \end{axis}
\end{tikzpicture}
%% \caption{A plot of $f(x) = x+1$ and $g(x) = (x-1)^2$ with the
%%   ``shell'' indicated.}
%% \label{figure:shellIndicated}
\end{image}


If instead we consider a typical vertical rectangle, but still rotate
around the $y$-axis, we get a thin ``shell'' instead of a thin
``washer,'' see Figure~\ref{figure:shellIndicated}. If we add up the
volume of such thin shells we will get an approximation to the true
volume.

\begin{image}
\begin{tikzpicture}
 \begin{axis}[
     view={30}{30},colormap/\surfaceColor,
     xlabel=$x$, ylabel=$z$, zlabel=$y$,
   ]

  \addplot3[surf,colormap/\surfaceColorTwo,shader=faceted,opacity=.7, %inside
  samples=10,
  samples y =20,
  domain=0:3,y domain=0:2*pi,
  z buffer=sort]
  (x* cos(deg(y)), {(x) * sin(deg(y))},{x+1});

  \addplot3[surf,colormap/\sliceColor,shader=interp,%shell
  samples=5,
  samples y = 20,
  domain=.25:2.5,y domain=0:2*pi,
  z buffer=sort
]
  ({1.5*cos(deg(y))}, {1.5*sin(deg(y))},x);


  \addplot3[surf,shader=faceted,opacity=.3, %outside
  samples=10,
  samples y =20,
  domain=0:3,y domain=0:2*pi,
  z buffer=sort]
  (x*cos(deg(y)), {(x) * sin(deg(y))},{(x-1)^2});


 \end{axis}
\end{tikzpicture}
\end{image}



What is the volume of such a shell?  Consider the shell at
$x$.  Imagine that we cut the shell vertically in one place and
``unroll'' it into a thin, flat sheet. This sheet will be $f(x)-g(x)$
tall, and $2\pi x$ wide namely, the circumference of the shell before
it was unrolled.  We may now write the integral
$$
  \int_0^3 2\pi x(f(x)-g(x))\d x=
  \int_0^3 2\pi x(x+1-(x-1)^2)\d x={27\over2}\pi.
$$
Not only does this accomplish the task with only one integral, the
integral is somewhat easier than those in the previous
calculation. Things are not always so neat, but it is often the case
that one of the two methods will be simpler than the other, so it is
worth considering both before starting to do calculations.

\begin{example} 
Suppose the area under $y=-x^2+1$ between $x=0$ and $x=1$ is rotated
around the $x$-axis.
\begin{explanation}
We'll just set up integrals for each method.

Disk method: $\int_0^1 \pi(1-x^2)^2\d x={8\over15}\pi$.


Shell method: $\int_0^1 2\pi y \sqrt{1-y}\d y={8\over15}\pi$.
\end{explanation}
\end{example}




\end{document}
