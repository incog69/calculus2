\documentclass{ximera}

%\usepackage{todonotes}

\newcommand{\todo}{}

\usepackage{tkz-euclide}
\tikzset{>=stealth} %% cool arrow head
\tikzset{shorten <>/.style={ shorten >=#1, shorten <=#1 } } %% allows shorter vectors

\usetikzlibrary{backgrounds} %% for boxes around graphs
\usetikzlibrary{shapes,positioning}  %% Clouds and stars
\usetikzlibrary{matrix} %% for matrix
\usepgfplotslibrary{polar} %% for polar plots
\usetkzobj{all}
\usepackage[makeroom]{cancel} %% for strike outs
%\usepackage{mathtools} %% for pretty underbrace % Breaks Ximera
\usepackage{multicol}





\usepackage{array}
\setlength{\extrarowheight}{+.1cm}   
\newdimen\digitwidth
\settowidth\digitwidth{9}
\def\divrule#1#2{
\noalign{\moveright#1\digitwidth
\vbox{\hrule width#2\digitwidth}}}





\newcommand{\RR}{\mathbb R}
\newcommand{\R}{\mathbb R}
\newcommand{\N}{\mathbb N}
\newcommand{\Z}{\mathbb Z}

%\renewcommand{\d}{\,d\!}
\renewcommand{\d}{\mathop{}\!d}
\newcommand{\dd}[2][]{\frac{\d #1}{\d #2}}
\renewcommand{\l}{\ell}
\newcommand{\ddx}{\frac{d}{\d x}}

\newcommand{\zeroOverZero}{\ensuremath{\boldsymbol{\tfrac{0}{0}}}}
\newcommand{\inftyOverInfty}{\ensuremath{\boldsymbol{\tfrac{\infty}{\infty}}}}
\newcommand{\zeroOverInfty}{\ensuremath{\boldsymbol{\tfrac{0}{\infty}}}}
\newcommand{\zeroTimesInfty}{\ensuremath{\small\boldsymbol{0\cdot \infty}}}
\newcommand{\inftyMinusInfty}{\ensuremath{\small\boldsymbol{\infty - \infty}}}
\newcommand{\oneToInfty}{\ensuremath{\boldsymbol{1^\infty}}}
\newcommand{\zeroToZero}{\ensuremath{\boldsymbol{0^0}}}
\newcommand{\inftyToZero}{\ensuremath{\boldsymbol{\infty^0}}}


\newcommand{\numOverZero}{\ensuremath{\boldsymbol{\tfrac{\#}{0}}}}
\newcommand{\dfn}{\textbf}
%\newcommand{\unit}{\,\mathrm}
\newcommand{\unit}{\mathop{}\!\mathrm}
\newcommand{\eval}[1]{\bigg[ #1 \bigg]}
\newcommand{\seq}[1]{\left( #1 \right)}
\renewcommand{\epsilon}{\varepsilon}
\renewcommand{\iff}{\Leftrightarrow}

\DeclareMathOperator{\arccot}{arccot}
\DeclareMathOperator{\arcsec}{arcsec}
\DeclareMathOperator{\arccsc}{arccsc}
\DeclareMathOperator{\si}{Si}

\newcommand{\tightoverset}[2]{%
  \mathop{#2}\limits^{\vbox to -.5ex{\kern-0.75ex\hbox{$#1$}\vss}}}
\newcommand{\arrowvec}[1]{\tightoverset{\scriptstyle\rightharpoonup}{#1}}
\renewcommand{\vec}{\mathbf}


\colorlet{textColor}{black} 
\colorlet{background}{white}
\colorlet{penColor}{blue!50!black} % Color of a curve in a plot
\colorlet{penColor2}{red!50!black}% Color of a curve in a plot
\colorlet{penColor3}{red!50!blue} % Color of a curve in a plot
\colorlet{penColor4}{green!50!black} % Color of a curve in a plot
\colorlet{penColor5}{orange!80!black} % Color of a curve in a plot
\colorlet{fill1}{penColor!20} % Color of fill in a plot
\colorlet{fill2}{penColor2!20} % Color of fill in a plot
\colorlet{fillp}{fill1} % Color of positive area
\colorlet{filln}{penColor2!20} % Color of negative area
\colorlet{fill3}{penColor3!20} % Fill
\colorlet{fill4}{penColor4!20} % Fill
\colorlet{fill5}{penColor5!20} % Fill
\colorlet{gridColor}{gray!50} % Color of grid in a plot

\newcommand{\surfaceColor}{violet}
\newcommand{\surfaceColorTwo}{redyellow}
\newcommand{\sliceColor}{greenyellow}




\pgfmathdeclarefunction{gauss}{2}{% gives gaussian
  \pgfmathparse{1/(#2*sqrt(2*pi))*exp(-((x-#1)^2)/(2*#2^2))}%
}


%%%%%%%%%%%%%
%% Vectors
%%%%%%%%%%%%%

%% Simple horiz vectors
\renewcommand{\vector}[1]{\left\langle #1\right\rangle}


%% %% Complex Horiz Vectors with angle brackets
%% \makeatletter
%% \renewcommand{\vector}[2][ , ]{\left\langle%
%%   \def\nextitem{\def\nextitem{#1}}%
%%   \@for \el:=#2\do{\nextitem\el}\right\rangle%
%% }
%% \makeatother

%% %% Vertical Vectors
%% \def\vector#1{\begin{bmatrix}\vecListA#1,,\end{bmatrix}}
%% \def\vecListA#1,{\if,#1,\else #1\cr \expandafter \vecListA \fi}

%%%%%%%%%%%%%
%% End of vectors
%%%%%%%%%%%%%

%\newcommand{\fullwidth}{}
%\newcommand{\normalwidth}{}



%% makes a snazzy t-chart for evaluating functions
%\newenvironment{tchart}{\rowcolors{2}{}{background!90!textColor}\array}{\endarray}

%%This is to help with formatting on future title pages.
\newenvironment{sectionOutcomes}{}{} 



%% Flowchart stuff
%\tikzstyle{startstop} = [rectangle, rounded corners, minimum width=3cm, minimum height=1cm,text centered, draw=black]
%\tikzstyle{question} = [rectangle, minimum width=3cm, minimum height=1cm, text centered, draw=black]
%\tikzstyle{decision} = [trapezium, trapezium left angle=70, trapezium right angle=110, minimum width=3cm, minimum height=1cm, text centered, draw=black]
%\tikzstyle{question} = [rectangle, rounded corners, minimum width=3cm, minimum height=1cm,text centered, draw=black]
%\tikzstyle{process} = [rectangle, minimum width=3cm, minimum height=1cm, text centered, draw=black]
%\tikzstyle{decision} = [trapezium, trapezium left angle=70, trapezium right angle=110, minimum width=3cm, minimum height=1cm, text centered, draw=black]


\title[Dig-In:]{Accumulated shells}

\begin{document}
\begin{abstract}
Some volumes of revolution are more easily computed with cylindrical shells.
\end{abstract}
\maketitle


Consider the region bounded by $f(x)=x+1$ and $g(x)=(x-1)^2$:
\begin{image}
\begin{tikzpicture}
  \begin{axis}[
      xmin=0, xmax=3,domain=0:3,
      clip=false,
      axis lines =center, xlabel=$x$, ylabel=$y$,
      every axis y label/.style={at=(current axis.above origin),anchor=south},
      every axis x label/.style={at=(current axis.right of origin),anchor=west},
      axis on top,
    ]
    \addplot [draw=none,fill=fillp] {x+1}\closedcycle;
    \addplot [draw=none,fill=white] {(x-1)^2}\closedcycle;
    \addplot [penColor,very thick] {x+1};
    \addplot [penColor2,very thick] {(x-1)^2};

    \node at (axis cs:.5,1.7) [penColor] {$f$};
    \node at (axis cs:2.5,1.7) [penColor2] {$g$};
  \end{axis}
\end{tikzpicture}
\end{image}
If this region is rotated around the $y$-axis, it is possible, but
inconvenient, to compute the volume of the resulting solid by the
methods we have used so far. The problem is that there are two
``kinds'' of cylinderical cross-sections: 
\begin{image}
\begin{tikzpicture}
  \begin{axis}[
      xmin=-3, xmax=3,
      clip=false,
      axis lines =center, xlabel=$x$, ylabel=$y$,
      every axis y label/.style={at=(current axis.above origin),anchor=south},
      every axis x label/.style={at=(current axis.right of origin),anchor=west},
      axis on top,
    ]
    \draw[penColor,very thick,fill=fillp] (axis cs:0,2) ellipse (240 and 20);
    \draw[penColor,very thick,fill=white] (axis cs:0,2) ellipse (100 and 10);

    \draw[penColor,very thick,fill=fillp] (axis cs:0,.56) ellipse (175 and 20);
    \draw[penColor,very thick,fill=white] (axis cs:0,.56) ellipse (25 and 5);

    
    \addplot [penColor,very thick,domain=0:3] {x+1};
    \addplot [penColor2,very thick,domain=0:3] {(x-1)^2};
    \addplot [penColor,very thick,domain=-3:0] {-x+1};
    \addplot [penColor2,very thick,domain=-3:0] {(-x-1)^2};

    %\node at (axis cs:.5,1.7) [penColor] {$f$};
    %\node at (axis cs:2.5,1.7) [penColor2] {$g$};
    \addplot [very thick, penColor4] plot coordinates {(.25,.56) (1.75,.56)};
    \addplot [very thick, penColor4] plot coordinates {(1,2) (2.41,2)};
    
    \end{axis}
\end{tikzpicture}
%% \caption{A plot of $f(x) = x+1$ and $g(x) = (x-1)^2$ with the two
%%   types of ``washers'' indicated.}
%% \label{figure:washerHARD}
\end{image}
As we see above, some of the cylinderical cross sections are defined
by the line that goes from the line to the parabola and others are
defined by the line that touches the parabola on both ends.  To
compute the volume using accumulated cross-sections, we need to break
the problem into two parts and compute two integrals:
\begin{align*}
  \pi\int_0^1 &(1+\sqrt{y})^2-(1-\sqrt{y})^2\d y+
  \pi\int_1^4  (1+\sqrt{y})^2-(y-1)^2\d y\\
  &={8\over3}\pi + {65\over6}\pi\\
  &={27\over2}\pi.
\end{align*}

\begin{image}
\begin{tikzpicture}
  \begin{axis}[
      xmin=0, xmax=3,domain=0:3,
      clip=false,
      axis lines =center, xlabel=$x$, ylabel=$y$,
      every axis y label/.style={at=(current axis.above origin),anchor=south},
      every axis x label/.style={at=(current axis.right of origin),anchor=west},
      axis on top,
    ] 
    \addplot [penColor2,very thick] {x+1};
    \addplot [penColor,very thick] {(x-1)^2};

    \node at (axis cs:.5,1.7) [penColor2] {$f(x)$};
    \node at (axis cs:2.5,1.7) [penColor] {$g(x)$};
    \addplot [very thick, penColor4] plot coordinates {(1.5,2.5) (1.5,.25)};
  \end{axis}
\end{tikzpicture}
%% \caption{A plot of $f(x) = x+1$ and $g(x) = (x-1)^2$ with the
%%   ``shell'' indicated.}
%% \label{figure:shellIndicated}
\end{image}


If instead we consider a typical vertical rectangle, but still rotate
around the $y$-axis, we get a thin ``shell'' instead of a thin
``washer''. If we add up the
volume of such thin shells we will get an approximation to the true
volume.

\begin{image}
\begin{tikzpicture}
  \begin{axis}[
      xmin=0, xmax=3,domain=0:3,
      clip=false,
      axis lines =center, xlabel=$x$, ylabel=$y$,
      every axis y label/.style={at=(current axis.above origin),anchor=south},
      every axis x label/.style={at=(current axis.right of origin),anchor=west},
      axis on top,
    ] 
   


   \draw[penColor,very thick,fill=fillp] (axis cs:0,2.5) ellipse (150 and 20);
    \draw[penColor,very thick,fill=white] (axis cs:0,2.5) ellipse (120 and 10);



   \draw[penColor,very thick] (axis cs:0,0.25) ellipse (150 and 20);
   \draw[white,very thick, fill=white] (axis cs:0,0.27) ellipse (150 and 20);
   \draw[penColor, dashed] (axis cs:0,0.25) ellipse (150 and 20);
   \draw[penColor, dashed] (axis cs:0,0.25) ellipse (120 and 10);


 \addplot [penColor2,very thick] {x+1};
    \addplot [penColor,very thick] {(x-1)^2};

    \node at (axis cs:.5,1.7) [penColor2] {$f(x)$};
    \node at (axis cs:2.5,1.7) [penColor] {$g(x)$};
    \addplot [very thick, penColor4] plot coordinates {(1.5,2.5) (1.5,.25)};
    \addplot [very thick, penColor4] plot coordinates {(-1.5,2.5) (-1.5,.25)};
    \addplot [very thick, penColor4, dashed] plot coordinates {(1.2,2.5) (1.2,.25)};
    \addplot [very thick, penColor4, dashed] plot coordinates {(-1.2,2.5) (-1.2,.25)};
  \end{axis}
\end{tikzpicture}
%% \caption{A plot of $f(x) = x+1$ and $g(x) = (x-1)^2$ with the
%%   ``shell'' indicated.}
%% \label{figure:shellIndicated}
\end{image}

%% \begin{image}
%% \begin{tikzpicture}
%%  \begin{axis}[
%%      view={30}{30},colormap/\surfaceColor,
%%      xlabel=$x$, ylabel=$z$, zlabel=$y$,
%%    ]

%%   \addplot3[surf,colormap/\surfaceColorTwo,shader=faceted,opacity=.7, %inside
%%   samples=10,
%%   samples y =20,
%%   domain=0:3,y domain=0:2*pi,
%%   z buffer=sort]
%%   (x* cos(deg(y)), {(x) * sin(deg(y))},{x+1});

%%   \addplot3[surf,colormap/\sliceColor,shader=interp,%shell
%%   samples=5,
%%   samples y = 20,
%%   domain=.25:2.5,y domain=0:2*pi,
%%   z buffer=sort
%% ]
%%   ({1.5*cos(deg(y))}, {1.5*sin(deg(y))},x);


%%   \addplot3[surf,shader=faceted,opacity=.3, %outside
%%   samples=10,
%%   samples y =20,
%%   domain=0:3,y domain=0:2*pi,
%%   z buffer=sort]
%%   (x*cos(deg(y)), {(x) * sin(deg(y))},{(x-1)^2});


%%  \end{axis}
%% \end{tikzpicture}
%% \end{image}



What is the volume of such a shell?  Consider the shell at
$x$.  Imagine that we cut the shell vertically in one place and
``unroll'' it into a thin, flat sheet. This sheet will be $f(x)-g(x)$
tall, and $2\pi x$ wide since this is the circumference of the shell before
it was unrolled.  We may now write the integral


\begin{image}
  \begin{tikzpicture}
    \begin{axis}[
          xmin =0,xmax=4,ymax=5,ymin=-5,
          axis lines=none, xlabel=$x$, ylabel=$y$,
          every axis y label/.style={at=(current axis.above origin),anchor=south},
          every axis x label/.style={at=(current axis.right of origin),anchor=west},
          axis on top,
          width=5in,
          xtick={0,6}, xticklabels={$0$, $20$},
          ytick={0,3},yticklabels={$0$,$20$},
            clip=false,
      ]

            
      %\addplot [draw=penColor, thick] plot coordinates {(-3,-3) (0,0)};
      %\addplot [draw=penColor, thick] plot coordinates {(6,0) (0,0)};
      %\addplot [draw=penColor, thick] plot coordinates {(1.5,3) (3,-3)};
      %\addplot [draw=penColor, thick] plot coordinates {(1.5,3) (0,0)};

      %% slab
      \addplot [draw=penColor, fill=fillp,very thick] plot coordinates {(3,2) (1,2) (0,1) (2, 1) (3,2)};
      \addplot [draw=penColor, fill=fillp,very thick] plot coordinates {(0,.8) (0,1) (2,1) (2, .8) (0,.8)};
      \addplot [draw=penColor, fill=fillp,very thick] plot coordinates {(2,1) (2, .8) (3,1.8) (3,2) (2,1)};

      %\addplot [draw=penColor, fill=fillp,very thick] plot coordinates {(3,1.8) (1,1.8) (0,.8) (2, .8) (3,1.8)};
      %\addplot [draw=penColor, fill=fillp,very thick] plot coordinates {(3,2) (1,2) (0,1) (2, 1) (3,2)};



      \draw[decoration={brace,mirror,raise=.1cm},decorate,thin] (axis cs:0,.8)--(axis cs:2,.8);
      \draw[decoration={brace,mirror,raise=.1cm},decorate,thin] (axis cs:2,.8)--(axis cs:3,1.8);
      \draw[decoration={brace,raise=.1cm},decorate,thin] (axis cs:3,2.05)--(axis cs:3,1.75);
      
     % \addplot [->] plot coordinates {(0,0) (-4,-4)};
      %\node[anchor=north east] at (axis cs:-4,-4) {$z$};

      \node at (axis cs:3.15,1.9) {$\d x$};
      \node at (axis cs:2.8,0.8) {$f(x)-g(x)$};
      \node at (axis cs:1,.4) {$2x\pi$};       
    \end{axis}
  \end{tikzpicture}
\end{image}



$$
  \int_0^3 2\pi x(f(x)-g(x))\d x=
  \int_0^3 2\pi x(x+1-(x-1)^2)\d x={27\over2}\pi.
$$
Not only does this accomplish the task with only one integral, the
integral is somewhat easier than those in the previous
calculation. Things are not always so neat, but it is often the case
that one of the two methods will be simpler than the other, so it is
worth considering both before starting to do calculations.


\begin{example} 
Suppose the area bounded by $y=\sqrt(x)$, the line $y = 2x-1$, and the $x$ axis is  rotated
around the $x$-axis.

\begin{explanation}

\begin{image}
\begin{tikzpicture}
  \begin{axis}[
      xmin=0, xmax=1.2,ymax = 3, ymin = -1,domain=0:3,
      clip=true,
      axis lines =center, xlabel=$x$, ylabel=$y$,
      every axis y label/.style={at=(current axis.above origin),anchor=south},
      every axis x label/.style={at=(current axis.right of origin),anchor=west},
      axis on top,
    ] 
  \addplot [draw=none,fill=fillp, domain=0:1, samples = 100] {sqrt(x)}\closedcycle;
    \addplot [penColor2,very thick, samples = 100] {sqrt(x)};
    \addplot [penColor2,very thick,domain=0.5:1.5] {2*x-1};
     \addplot [draw=none,fill=white,very thick,domain=0.5:1] {2*x-1.02}\closedcycle;
  \end{axis}
\end{tikzpicture}
\end{image}

While we could use vertical rectangles, we would have to break the region of integration up into two parts.

This indicates that using horizontal rectangles may be more profitable.

Solving for $x$, we have $x = y^2$ and $x= \frac{y+1}{2}$.

We obtain a cyclindrical shell with width $\d y$, length $\answer{\frac{y+1}{2} - y^2}$, and whose circumference is $\answer{2\pi y}$


\begin{hint}
	The radius is $y$, so the circumference is $2\pi y$.  The length is $\frac{y+1}{2} - y^2$.
\end{hint}

Thus the volume is

\[
\textrm{Volume} = \int_0^1 \answer{2\pi y (\frac{y+1}{2} - y^2)} \d y
\]
\end{explanation}
\end{example}

\begin{example}
	What if we had wanted to rotate the region from the last example about the line $y = -1$ instead of the $x$-axis?

\begin{explanation}
We could still use the same vertical rectangles.  Our cyclindrical shells would have the same width and height, but the circumference would change to $\answer{2\pi(y+1)}$.

\begin{hint}
	The radius is now $y+1$, so the circumference is $2\pi(y+1)$
\end{hint}

Thus the volume is 

\[
\textrm{Volume} = \int_0^1 \answer{2\pi (y+1) (\frac{y+1}{2} - y^2)}\d y
\]
\end{explanation}


\end{example}


\end{document}
