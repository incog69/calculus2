\documentclass{ximera}

%\usepackage{todonotes}

\newcommand{\todo}{}

\usepackage{tkz-euclide}
\tikzset{>=stealth} %% cool arrow head
\tikzset{shorten <>/.style={ shorten >=#1, shorten <=#1 } } %% allows shorter vectors

\usetikzlibrary{backgrounds} %% for boxes around graphs
\usetikzlibrary{shapes,positioning}  %% Clouds and stars
\usetikzlibrary{matrix} %% for matrix
\usepgfplotslibrary{polar} %% for polar plots
\usetkzobj{all}
\usepackage[makeroom]{cancel} %% for strike outs
%\usepackage{mathtools} %% for pretty underbrace % Breaks Ximera
\usepackage{multicol}





\usepackage{array}
\setlength{\extrarowheight}{+.1cm}   
\newdimen\digitwidth
\settowidth\digitwidth{9}
\def\divrule#1#2{
\noalign{\moveright#1\digitwidth
\vbox{\hrule width#2\digitwidth}}}





\newcommand{\RR}{\mathbb R}
\newcommand{\R}{\mathbb R}
\newcommand{\N}{\mathbb N}
\newcommand{\Z}{\mathbb Z}

%\renewcommand{\d}{\,d\!}
\renewcommand{\d}{\mathop{}\!d}
\newcommand{\dd}[2][]{\frac{\d #1}{\d #2}}
\renewcommand{\l}{\ell}
\newcommand{\ddx}{\frac{d}{\d x}}

\newcommand{\zeroOverZero}{\ensuremath{\boldsymbol{\tfrac{0}{0}}}}
\newcommand{\inftyOverInfty}{\ensuremath{\boldsymbol{\tfrac{\infty}{\infty}}}}
\newcommand{\zeroOverInfty}{\ensuremath{\boldsymbol{\tfrac{0}{\infty}}}}
\newcommand{\zeroTimesInfty}{\ensuremath{\small\boldsymbol{0\cdot \infty}}}
\newcommand{\inftyMinusInfty}{\ensuremath{\small\boldsymbol{\infty - \infty}}}
\newcommand{\oneToInfty}{\ensuremath{\boldsymbol{1^\infty}}}
\newcommand{\zeroToZero}{\ensuremath{\boldsymbol{0^0}}}
\newcommand{\inftyToZero}{\ensuremath{\boldsymbol{\infty^0}}}


\newcommand{\numOverZero}{\ensuremath{\boldsymbol{\tfrac{\#}{0}}}}
\newcommand{\dfn}{\textbf}
%\newcommand{\unit}{\,\mathrm}
\newcommand{\unit}{\mathop{}\!\mathrm}
\newcommand{\eval}[1]{\bigg[ #1 \bigg]}
\newcommand{\seq}[1]{\left( #1 \right)}
\renewcommand{\epsilon}{\varepsilon}
\renewcommand{\iff}{\Leftrightarrow}

\DeclareMathOperator{\arccot}{arccot}
\DeclareMathOperator{\arcsec}{arcsec}
\DeclareMathOperator{\arccsc}{arccsc}
\DeclareMathOperator{\si}{Si}

\newcommand{\tightoverset}[2]{%
  \mathop{#2}\limits^{\vbox to -.5ex{\kern-0.75ex\hbox{$#1$}\vss}}}
\newcommand{\arrowvec}[1]{\tightoverset{\scriptstyle\rightharpoonup}{#1}}
\renewcommand{\vec}{\mathbf}


\colorlet{textColor}{black} 
\colorlet{background}{white}
\colorlet{penColor}{blue!50!black} % Color of a curve in a plot
\colorlet{penColor2}{red!50!black}% Color of a curve in a plot
\colorlet{penColor3}{red!50!blue} % Color of a curve in a plot
\colorlet{penColor4}{green!50!black} % Color of a curve in a plot
\colorlet{penColor5}{orange!80!black} % Color of a curve in a plot
\colorlet{fill1}{penColor!20} % Color of fill in a plot
\colorlet{fill2}{penColor2!20} % Color of fill in a plot
\colorlet{fillp}{fill1} % Color of positive area
\colorlet{filln}{penColor2!20} % Color of negative area
\colorlet{fill3}{penColor3!20} % Fill
\colorlet{fill4}{penColor4!20} % Fill
\colorlet{fill5}{penColor5!20} % Fill
\colorlet{gridColor}{gray!50} % Color of grid in a plot

\newcommand{\surfaceColor}{violet}
\newcommand{\surfaceColorTwo}{redyellow}
\newcommand{\sliceColor}{greenyellow}




\pgfmathdeclarefunction{gauss}{2}{% gives gaussian
  \pgfmathparse{1/(#2*sqrt(2*pi))*exp(-((x-#1)^2)/(2*#2^2))}%
}


%%%%%%%%%%%%%
%% Vectors
%%%%%%%%%%%%%

%% Simple horiz vectors
\renewcommand{\vector}[1]{\left\langle #1\right\rangle}


%% %% Complex Horiz Vectors with angle brackets
%% \makeatletter
%% \renewcommand{\vector}[2][ , ]{\left\langle%
%%   \def\nextitem{\def\nextitem{#1}}%
%%   \@for \el:=#2\do{\nextitem\el}\right\rangle%
%% }
%% \makeatother

%% %% Vertical Vectors
%% \def\vector#1{\begin{bmatrix}\vecListA#1,,\end{bmatrix}}
%% \def\vecListA#1,{\if,#1,\else #1\cr \expandafter \vecListA \fi}

%%%%%%%%%%%%%
%% End of vectors
%%%%%%%%%%%%%

%\newcommand{\fullwidth}{}
%\newcommand{\normalwidth}{}



%% makes a snazzy t-chart for evaluating functions
%\newenvironment{tchart}{\rowcolors{2}{}{background!90!textColor}\array}{\endarray}

%%This is to help with formatting on future title pages.
\newenvironment{sectionOutcomes}{}{} 



%% Flowchart stuff
%\tikzstyle{startstop} = [rectangle, rounded corners, minimum width=3cm, minimum height=1cm,text centered, draw=black]
%\tikzstyle{question} = [rectangle, minimum width=3cm, minimum height=1cm, text centered, draw=black]
%\tikzstyle{decision} = [trapezium, trapezium left angle=70, trapezium right angle=110, minimum width=3cm, minimum height=1cm, text centered, draw=black]
%\tikzstyle{question} = [rectangle, rounded corners, minimum width=3cm, minimum height=1cm,text centered, draw=black]
%\tikzstyle{process} = [rectangle, minimum width=3cm, minimum height=1cm, text centered, draw=black]
%\tikzstyle{decision} = [trapezium, trapezium left angle=70, trapezium right angle=110, minimum width=3cm, minimum height=1cm, text centered, draw=black]



\outcome{Identify separable differential equation.}
\outcome{Compute general solution of a separable differential equation.}
\outcome{Find implicit solutions of a separable differential equation.}

\title[Dig-In:]{Separable differential equations}

\begin{document}
\begin{abstract}
	Separable differential equations are those in which the dependent and independent variables can be separated on opposite sides of the equation.
\end{abstract}
\maketitle

\begin{definition}
	A \dfn{separable differential equation} is a differential equation which can be written in the form 
	
	\[
		g(y) y' = h(x)
	\]
	
	In other words, the independent variable ($x$) and the dependent variable ($y$) can be placed on \textbf{separate} sides of the equals sign.
\end{definition}

\begin{question}
	Which of the following are separable differential equations?  Select all that apply.
	
\begin{selectAll}
	\choice[correct]{$y' = y/x$}
	\choice{$y'  = \sqrt{x^2+y^2}$}
	\choice[correct]{$\cos(y)y' = \frac{x}{1+y}$}
	\choice[correct]{$y' = y$}
	\choice{$y' = \sin(x+y)$}
\end{selectAll}
\end{question}

\begin{example}
	The reason we care about separable differential equations is that we have a good way to try and solve them.  
		
	\begin{align*}
		g(y)y'  &= h(x)\\
		\int g(y) y' \d x &= \int h(x) \d x\\
		\int g(y) \d y &= \int h(x) \d x
	\end{align*}
	
	Let's see this strategy in action, and try to solve the differential equation $y' = \frac{x}{y^2}$.
	
	Write with me:
	
	First we separate the variables
	
	\[
	\answer{y^2} y' = \answer{x}
	\]
	
	Then we integrate both sides with respect to $x$
	
	\begin{align*}
	\int y^2 y' \d x &= \int x \d x\\
	\int y^2 \d y &= \int x \d x\\
	\answer{\frac{1}{3} y^3} &= \answer{\frac{1}{2}x^2}+C\\
	y &= \sqrt[3]{\answer{\frac{3}{2}x^2}+3C}
	\end{align*}
	
	Since $3C$ could be any constant we like, we can rewrite the general solution to this differential equation as 
	
	\[
	y = \sqrt[3]{\frac{3}{2}x^2+C}
	\]
	
	We can check that all of these functions really are solutions to the differential equation:
	
	\begin{align*}
		\frac{\d y}{\d x} &= \ddx \sqrt[3]{\frac{3}{2}x^2+C}\\
			&=(\frac{3}{2}x^2+C)^{-\frac{2}{3}} x\\
			&= \frac{x}{(\sqrt[3]{\frac{3}{2}x^2+C})^2}\\
			&= \frac{x}{y^2}
	\end{align*}
\end{example}


In the last example, it worked out that we could solve for $y$ as a function of $x$.  This is not always the case.

\begin{example}
	Consider the separable differential equation
	
	\[
	\cos(y) y' = x
	\]
	
	Integrating both sides yields
	
	\begin{align*}
		\int \cos(y) y' \d x = \int x \d x\\
		\int \cos(y) \d y = \int x \d x\\
		\sin(y) = \frac{1}{2}x^2+C
	\end{align*}
	
	This shows that any function satisfying the differential equation must satisfy this relation.  We call such a relation that a solution to a differential equation must satisfy a \dfn{implicit solution of a differential equation}.
	
\end{example}

\begin{example}	
One really important example of a separable differential equation is the \dfn{logistic equation}:

\[
y' = ry(1-y)
\]

This equation comes up frequently in population modeling.  

\begin{question}
	Which of the following are equilibrium solutions to this differential equation?
	
\begin{multipleChoice}
\choice[correct]{$y = 0$}
\choice{$y = r$}
\choice[correct]{$y = 1$}
\end{multipleChoice}
\end{question}

Let us solve this equation together.  First we separate the differential equation:
\[
\answer{\frac{1}{y(1-y)}} y'  = r
\]

Now we integrate both sides with respect to $x$

\[
\int \frac{1}{y(1-y)} y' \d x =\int r \d x
\]

now using variable substitution we get

\[
\int \answer{\frac{1}{y(1-y)}} \d y = \int r \d x
\]

We need to use partial fraction decomposition to solve this integral.

We have

\[
\frac{1}{y(1-y)} = \frac{\answer{1}}{y}+\frac{\answer{1}}{1-y}
\]

So we get

\[
\int \frac{1}{y} + \frac{1}{1-y} \d y = \int r \d x
\]

This yields

\[
\ln(|y|) -  \ln(|1-y|)  = rx+C_1
\]


Now we can solve for $y$:

\begin{align*}
\ln(|y|) -  \ln(|1-y|)  &= rx+C_1\\
\ln(|\frac{y}{1-y}|) &= rx+C_1\\
|\frac{y}{1-y}| &= e^{rx+C_1}\\
\end{align*}

At this point, we can write $e^{rx+C_1} = e^{C_1} e^{rx}$.  We get $\frac{y}{1-y} = \pm e^{C_1} e^{rx}$.  If we rename $\pm e^{C_1}$ to be $C_2$, then we get

\begin{align*}
\frac{y}{1-y} &=  C_2e^{rx}\\
\frac{1-y}{y} &= C_3e^{-rx} \textrm{ where $C_3  =\frac{1}{C_2}$}\\
\frac{1}{y} &= 1+C_3e^{-rx}\\
y &= \frac{1}{1+C_3e^{-rx}}
\end{align*}


So this is the general solution to the logistic equation!
\end{example}





\end{document}
