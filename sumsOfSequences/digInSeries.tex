\documentclass{ximera}

%\usepackage{todonotes}

\newcommand{\todo}{}

\usepackage{tkz-euclide}
\tikzset{>=stealth} %% cool arrow head
\tikzset{shorten <>/.style={ shorten >=#1, shorten <=#1 } } %% allows shorter vectors

\usetikzlibrary{backgrounds} %% for boxes around graphs
\usetikzlibrary{shapes,positioning}  %% Clouds and stars
\usetikzlibrary{matrix} %% for matrix
\usepgfplotslibrary{polar} %% for polar plots
\usetkzobj{all}
\usepackage[makeroom]{cancel} %% for strike outs
%\usepackage{mathtools} %% for pretty underbrace % Breaks Ximera
\usepackage{multicol}





\usepackage{array}
\setlength{\extrarowheight}{+.1cm}   
\newdimen\digitwidth
\settowidth\digitwidth{9}
\def\divrule#1#2{
\noalign{\moveright#1\digitwidth
\vbox{\hrule width#2\digitwidth}}}





\newcommand{\RR}{\mathbb R}
\newcommand{\R}{\mathbb R}
\newcommand{\N}{\mathbb N}
\newcommand{\Z}{\mathbb Z}

%\renewcommand{\d}{\,d\!}
\renewcommand{\d}{\mathop{}\!d}
\newcommand{\dd}[2][]{\frac{\d #1}{\d #2}}
\renewcommand{\l}{\ell}
\newcommand{\ddx}{\frac{d}{\d x}}

\newcommand{\zeroOverZero}{\ensuremath{\boldsymbol{\tfrac{0}{0}}}}
\newcommand{\inftyOverInfty}{\ensuremath{\boldsymbol{\tfrac{\infty}{\infty}}}}
\newcommand{\zeroOverInfty}{\ensuremath{\boldsymbol{\tfrac{0}{\infty}}}}
\newcommand{\zeroTimesInfty}{\ensuremath{\small\boldsymbol{0\cdot \infty}}}
\newcommand{\inftyMinusInfty}{\ensuremath{\small\boldsymbol{\infty - \infty}}}
\newcommand{\oneToInfty}{\ensuremath{\boldsymbol{1^\infty}}}
\newcommand{\zeroToZero}{\ensuremath{\boldsymbol{0^0}}}
\newcommand{\inftyToZero}{\ensuremath{\boldsymbol{\infty^0}}}


\newcommand{\numOverZero}{\ensuremath{\boldsymbol{\tfrac{\#}{0}}}}
\newcommand{\dfn}{\textbf}
%\newcommand{\unit}{\,\mathrm}
\newcommand{\unit}{\mathop{}\!\mathrm}
\newcommand{\eval}[1]{\bigg[ #1 \bigg]}
\newcommand{\seq}[1]{\left( #1 \right)}
\renewcommand{\epsilon}{\varepsilon}
\renewcommand{\iff}{\Leftrightarrow}

\DeclareMathOperator{\arccot}{arccot}
\DeclareMathOperator{\arcsec}{arcsec}
\DeclareMathOperator{\arccsc}{arccsc}
\DeclareMathOperator{\si}{Si}

\newcommand{\tightoverset}[2]{%
  \mathop{#2}\limits^{\vbox to -.5ex{\kern-0.75ex\hbox{$#1$}\vss}}}
\newcommand{\arrowvec}[1]{\tightoverset{\scriptstyle\rightharpoonup}{#1}}
\renewcommand{\vec}{\mathbf}


\colorlet{textColor}{black} 
\colorlet{background}{white}
\colorlet{penColor}{blue!50!black} % Color of a curve in a plot
\colorlet{penColor2}{red!50!black}% Color of a curve in a plot
\colorlet{penColor3}{red!50!blue} % Color of a curve in a plot
\colorlet{penColor4}{green!50!black} % Color of a curve in a plot
\colorlet{penColor5}{orange!80!black} % Color of a curve in a plot
\colorlet{fill1}{penColor!20} % Color of fill in a plot
\colorlet{fill2}{penColor2!20} % Color of fill in a plot
\colorlet{fillp}{fill1} % Color of positive area
\colorlet{filln}{penColor2!20} % Color of negative area
\colorlet{fill3}{penColor3!20} % Fill
\colorlet{fill4}{penColor4!20} % Fill
\colorlet{fill5}{penColor5!20} % Fill
\colorlet{gridColor}{gray!50} % Color of grid in a plot

\newcommand{\surfaceColor}{violet}
\newcommand{\surfaceColorTwo}{redyellow}
\newcommand{\sliceColor}{greenyellow}




\pgfmathdeclarefunction{gauss}{2}{% gives gaussian
  \pgfmathparse{1/(#2*sqrt(2*pi))*exp(-((x-#1)^2)/(2*#2^2))}%
}


%%%%%%%%%%%%%
%% Vectors
%%%%%%%%%%%%%

%% Simple horiz vectors
\renewcommand{\vector}[1]{\left\langle #1\right\rangle}


%% %% Complex Horiz Vectors with angle brackets
%% \makeatletter
%% \renewcommand{\vector}[2][ , ]{\left\langle%
%%   \def\nextitem{\def\nextitem{#1}}%
%%   \@for \el:=#2\do{\nextitem\el}\right\rangle%
%% }
%% \makeatother

%% %% Vertical Vectors
%% \def\vector#1{\begin{bmatrix}\vecListA#1,,\end{bmatrix}}
%% \def\vecListA#1,{\if,#1,\else #1\cr \expandafter \vecListA \fi}

%%%%%%%%%%%%%
%% End of vectors
%%%%%%%%%%%%%

%\newcommand{\fullwidth}{}
%\newcommand{\normalwidth}{}



%% makes a snazzy t-chart for evaluating functions
%\newenvironment{tchart}{\rowcolors{2}{}{background!90!textColor}\array}{\endarray}

%%This is to help with formatting on future title pages.
\newenvironment{sectionOutcomes}{}{} 



%% Flowchart stuff
%\tikzstyle{startstop} = [rectangle, rounded corners, minimum width=3cm, minimum height=1cm,text centered, draw=black]
%\tikzstyle{question} = [rectangle, minimum width=3cm, minimum height=1cm, text centered, draw=black]
%\tikzstyle{decision} = [trapezium, trapezium left angle=70, trapezium right angle=110, minimum width=3cm, minimum height=1cm, text centered, draw=black]
%\tikzstyle{question} = [rectangle, rounded corners, minimum width=3cm, minimum height=1cm,text centered, draw=black]
%\tikzstyle{process} = [rectangle, minimum width=3cm, minimum height=1cm, text centered, draw=black]
%\tikzstyle{decision} = [trapezium, trapezium left angle=70, trapezium right angle=110, minimum width=3cm, minimum height=1cm, text centered, draw=black]


\outcome{Recognize a geometric series.}
\outcome{Recognize a telescoping series.}
\outcome{Compute the sum of a geometric series.}
\outcome{Compute the sum of a telescoping series.}

\title[Dig-In:]{Series}

\begin{document}
\begin{abstract}
A series is summation of a sequence.
\end{abstract}
\maketitle


Let's jump right in:


\begin{definition}
  A \dfn{series} is a sum of an infinte sequence.
\end{definition}


Let's start this off with a question (a little unfair I know!)

\begin{question}
  Can you sum an infinite number of terms to a finite value?
  \begin{prompt}
    \begin{multipleChoice}
      \choice{no}
      \choice[correct]{sometimes}
    \end{multipleChoice}
  \end{prompt}
\end{question}
As we will see the answer is ``sometimes.''  Believe it or not, you
have been working with infinite sums of numbers (also called
\textit{series}) for a long time. Consider the number
\[
\frac{1}{3} = 0.3333333333\dots
\]
this is the infinite sum of the geometric sequence
$(a_n)_{n=1}^\infty$ where $a_n = \frac{3}{10^{n}}$, as
\begin{align*}
  \sum_{n=1}^\infty 3\cdot \frac{1}{10^{n}} &= 0.3 + 0.03+0.003+ 0.0003+ 0.00003+ \cdots\\
  &= \frac{3}{10} + \frac{3}{10^2} + \frac{3}{10^3} + \frac{3}{10^4} + \frac{3}{10^5} + \cdots\\
  &=\frac{1}{3}.
\end{align*}
We can sum other geometric series to finite values as well. Consider
\[
\sum_{n=1}^\infty \left(\frac{1}{4}\right)^n =
\frac{1}{4} + \left(\frac{1}{4}\right)^2 + \left(\frac{1}{4}\right)^3 + \left(\frac{1}{4}\right)^4 + \cdots 
\]
A very clever method of summing this sequence is as follows, consider
an equilateral triangle with area $1$:
\begin{image}
  \begin{tikzpicture}[scale=3,rounded corners=.5pt]      
    \tkzDefPoint(0,1){A1} 
    \tkzDefPoint(-.58,0){A2}
    \tkzDefPoint(.58,0){A3}
    \draw[penColor,very thick] (A1)--(A2)--(A3)--cycle;
  \end{tikzpicture}
\end{image}
We can break this triangle into $4$ congruent triangles, each of area
$1/4$:
\begin{image}
  \begin{tikzpicture}[scale=3,rounded corners=.5pt]      
    \tkzDefPoint(0,1){A1} 
    \tkzDefPoint(-.58,0){A2}
    \tkzDefPoint(.58,0){A3}
    \draw[penColor,very thick] (A1)--(A2)--(A3)--cycle;

    \tkzDefPoint(0,0){B1} 
    \tkzDefPoint(-.29,.5){B2}
    \tkzDefPoint(.29,.5){B3}
    \draw[penColor,fill=fill1,very thick] (B1)--(B2)--(B3)--cycle;
  \end{tikzpicture}
\end{image}
We can break the upper triangle into $4$ more congruent triangles, each
with area $(1/4)^2$:
\begin{image}
  \begin{tikzpicture}[scale=3,rounded corners=.5pt]      
    \tkzDefPoint(0,1){A1} 
    \tkzDefPoint(-.58,0){A2}
    \tkzDefPoint(.58,0){A3}
    \draw[penColor,very thick] (A1)--(A2)--(A3)--cycle;

    \tkzDefPoint(0,0){B1} 
    \tkzDefPoint(-.29,.5){B2}
    \tkzDefPoint(.29,.5){B3}
    \draw[penColor,fill=fill1,very thick] (B1)--(B2)--(B3)--cycle;

    \tkzDefPoint(0,.5){C1} 
    \tkzDefPoint(-.14,.75){C2}
    \tkzDefPoint(.14,.75){C3}
    \draw[penColor,fill=fill1,very thick] (C1)--(C2)--(C3)--cycle;
  \end{tikzpicture}
\end{image}
Repeating this process, we find:
\begin{image}
  \begin{tikzpicture}[scale=3,rounded corners=.5pt]      
    \tkzDefPoint(0,1){A1} 
    \tkzDefPoint(-.58,0){A2}
    \tkzDefPoint(.58,0){A3}
    \draw[penColor,very thick] (A1)--(A2)--(A3)--cycle;

    \tkzDefPoint(0,0){B1} 
    \tkzDefPoint(-.29,.5){B2}
    \tkzDefPoint(.29,.5){B3}
    \draw[penColor,fill=fill1,very thick] (B1)--(B2)--(B3)--cycle;

    \tkzDefPoint(0,.5){C1} 
    \tkzDefPoint(-.14,.75){C2}
    \tkzDefPoint(.14,.75){C3}
    \draw[penColor,fill=fill1,very thick] (C1)--(C2)--(C3)--cycle;

    \tkzDefPoint(0,.75){D1} 
    \tkzDefPoint(-.07,.875){D2}
    \tkzDefPoint(.07,.875){D3}
    \draw[penColor,fill=fill1,very thick] (D1)--(D2)--(D3)--cycle;

    \tkzDefPoint(0,.875){E1} 
    \tkzDefPoint(-.04,.94){E2}
    \tkzDefPoint(.04,.94){E3}
    \draw[penColor,fill=fill1,very thick] (E1)--(E2)--(E3)--cycle;

    \tkzDefPoint(0,.94 ){F1} 
    \tkzDefPoint(-.02,.97){F2}
    \tkzDefPoint(.02,.97){F3}
    \draw[penColor,fill=fill1,very thick] (F1)--(F2)--(F3)--cycle;
  \end{tikzpicture}
\end{image}
where the area of the shaded triangles is our geometric series:
\[
\frac{1}{4} + \left(\frac{1}{4}\right)^2 + \left(\frac{1}{4}\right)^3 + \left(\frac{1}{4}\right)^4 + \cdots 
\]
The area is clearly finite (it is between $0$ and $1$!). What is the
shaded area? Well, if you look at any ``row'' of the triangle, we've
shaded in exactly $1/3$rd of the row. Hence we've shaded in $1/3$ of
the entire area, so we see
\[
\frac{1}{3}=\frac{1}{4} + \left(\frac{1}{4}\right)^2 + \left(\frac{1}{4}\right)^3 + \left(\frac{1}{4}\right)^4 + \cdots 
\]
While this is a very cool argument, it doesn't generalize well. Let me
show you an argument that will apply to more settings:

\begin{example}
  Explain why
  \[
  \frac{1}{3}=\frac{1}{4} + \left(\frac{1}{4}\right)^2 + \left(\frac{1}{4}\right)^3 + \left(\frac{1}{4}\right)^4 + \cdots 
  \]
  \begin{explanation}
    Here is the idea, ``name'' your sum $S$
    \[
    S = \frac{1}{4} + \left(\frac{1}{4}\right)^2 + \left(\frac{1}{4}\right)^3 + \left(\frac{1}{4}\right)^4 + \cdots 
    \]
    Now multiply $S$ by $\frac{1}{4}$ and write this suggestively under $S$
    \begin{align*}
      S &= \frac{1}{4} + \left(\frac{1}{4}\right)^2 + \left(\frac{1}{4}\right)^3 + \left(\frac{1}{4}\right)^4 + \cdots\\
     \left(\frac{1}{4}\right)S &=   \left(\frac{1}{4}\right)^2 + \left(\frac{1}{4}\right)^3 + \left(\frac{1}{4}\right)^4 + \left(\frac{1}{4}\right)^5+ \cdots
    \end{align*}
    subtracting the lower line from the upper line we find
    \begin{align*}
      S - \left(\frac{1}{4}\right)S &=  \frac{1}{4}\\
      S(1-\frac{1}{4}) &= \frac{1}{4}\\
      S &= \frac{\frac{1}{4}}{1-\frac{1}{4}}\\
      S &= \frac{1}{3}.
    \end{align*}
  \end{explanation}
\end{example}

This is a good method for understanding infinte sums of geometric
sequences (assuming you know the sequence sums to a finite value).

To really make this precise, we need some definitions. 
\begin{definition}
Let $(a_n)$ be a sequence.
\begin{enumerate}
\item The sum $\sum_{k=1}^\infty a_k$ is an \dfn{infinite series} (or,
  simply \dfn{series}).
\item Let $S_n = \sum_{i=1}^n a_i$; the sequence $(S_n)$ is the
  sequence of $n$th \dfn{partial sums} of $(a_n)$.
\item If $\lim_{n\to\infty} S_n = L$, we say the series
  $\sum_{n=1}^\infty a_n$ \dfn{converges} to $L$, and we write
  $\sum_{n=1}^\infty a_n = L$.
\item If $\lim_{n\to\infty} S_n$ diverges, the series
  $\sum_{k=1}^\infty a_k$ \dfn{diverges}.
\end{enumerate}
\end{definition}

\begin{question}
  Can you state what we showed above using our new terminology?
  \begin{prompt}
    The series $\sum_{n=1}^\infty \left(\frac{1}{4}\right)^n$
    \wordChoice{\choice[correct]{converges}\choice{diverges}}, and
      $\sum_{n=1}^\infty \left(\frac{1}{4}\right)^n = \answer{1/3}$.
  \end{prompt}
\end{question}


\section{Geometric series}

We started this section with two different geometric series that sum
to the same value. One reason geometric series are important is that
they have nice convergence properties.

\begin{example}
  Consider the geometric series $\sum_{k=0}^\infty r^k$. Compute the
  value of the $n$th partial sum.
  \begin{explanation}
    Let's try to get a handle of what is going on by writing out the partial sums:
    \begin{align*}
      S_1 &= r^0 \\
      S_2 &= r^0 + r^1\\
      S_3 &= r^0 + r^1 + r^2\\
      &\vdots\\
      S_n &= r^0 + r^1 + r^2 + \dots + r^{n-1}
    \end{align*}
    To compute $S_n$, use the same trick we used before, multiply $S_n$ by $r$
    \begin{align*}
      S_n   &= 1 + r^1 + r^2 + \dots + r^{n-1}\\
      r S_n &= r^1 + r^2 + \dots + r^{n-1} + r^n
    \end{align*}
    and subtract to find
    \begin{align*}
      S_n - r S_n &= 1 - r^n\\
      S_n(1-r)    &= 1 - r^n\\
      S_n &= \frac{1 - r^n}{1-r}.
    \end{align*}
    Since $S_n$ is \textbf{always} a finite sum, there is no issue
    with manipulating it the way we did.
  \end{explanation}
\end{example}

From our work above, we see that the $n$th partial sum of the
geometric sereis $a_n = r^n$ is
\[
S_n = \sum_{k=0}^{n-1} r^k= \frac{1 - r^n}{1-r}.
\]
Using this fact, we can prove a theorem:
\begin{theorem}\index{series!geometric}\index{geometric series}\index{convergence!of geometric series}\index{divergence!of geometric series}
  A geometric series
  \[
  \sum_{k= 0}^\infty r^k
  \]
  converges if and only if $|r| < 1$ and when $|r|<1$, 
  \[
  \sum_{k=0}^\infty r^k = \frac{1}{1-r}.
  \]
  \begin{explanation}
    Remember, to say that a series
    \[
    \sum_{k= 0}^\infty r^k
    \]
    converges means that the limit of the partial sums converges. We
    know already know how to compute the $n$th partial sum
    \[
    S_n = \frac{1 - r^n}{1-r},
    \]
    so write with me:
    \begin{align*}
      \sum_{k= 0}^\infty r^k &= \lim_{n\to\infty}S_n \\
      &= \lim_{n\to\infty}\frac{1 - r^n}{1-r}
    \end{align*}
    if $-1<r<1$, then we have
    \[
    \lim_{n\to\infty}\frac{1 - r^n}{1-r} = \frac{1}{1-r}
    \]
    otherwise the sequnce diverges.
  \end{explanation}
\end{theorem}

According to the theorem above the series
\[
\sum_{k=0}^\infty \left(\frac{1}{4}\right)^k = 1 + \frac{1}{4} + \left(\frac{1}{4}\right)^2 + \left(\frac{1}{4}\right)^3 + \cdots
\]
converges, and
\[
\sum_{k=0}^\infty  \left(\frac{1}{4}\right)^k = \frac{1}{1-1/4} = 4/3.
\]
This concurs with our introductory example; while there we got a sum
of $1/3$, we skipped the first term of $1$.

\begin{warning}
  You must pay close attention to how the series is indexed,
  \[
  \sum_{k=0}^n r^k \ne \sum_{k=1}^n r^k.
  \]
\end{warning}


\begin{question}
  Which of the following series converge?
  \begin{selectAll}
    \choice{$\sum_{k=0}^\infty \left(\frac{3}{2}\right)^k$}
    \choice[correct]{$\sum_{k=0}^\infty \left(\frac{-2}{3}\right)^k$}
    \choice[correct]{$\sum_{k=9}^\infty \left(\frac{1}{7}\right)^k$}
    \choice{$\sum_{k=1}^\infty (-1)^k$}
    \choice[correct]{$\sum_{k=-9}^\infty \left(\frac{1}{2}\right)^k$}    
  \end{selectAll}
  \begin{hint}
    The initial index doesn't matter as far as convergence is
    concerned, it is the ``tail'' of the sequence that determines
    convergence.
  \end{hint}
\end{question}

Let's use our new tools to compute some geometric series.

\begin{example}
If the series 
\[
\sum_{n=-2}^\infty \left(\frac{3}{4}\right)^n
\]
converges, find its sum.
\begin{explanation}
Since the common ratio between the terms of this series is
$\answer[given]{3/4}$, we see that this series
\wordChoice{\choice[correct]{converges}\choice{diverges}}. Write with me
\begin{align*}
  S &= \left(\frac{3}{4}\right)^{-2} + \left(\frac{3}{4}\right)^{-1} + \left(\frac{3}{4}\right)^{0} + \cdots\\
  \left(\frac{3}{4}\right) S &= \left(\frac{3}{4}\right)^{-1} + \left(\frac{3}{4}\right)^{0} + \left(\frac{3}{4}\right)^{1} + \cdots
\end{align*}
subtracing these two lines we find
\begin{align*}
  S -  \left(\frac{3}{4}\right) S  &= \left(\frac{3}{4}\right)^{-2}\\
  S\left(1-\frac{3}{4}\right) &= \left(\frac{3}{4}\right)^{-2}\\
  S &= \frac{(3/4)^{-2}}{1-3/4}.
\end{align*}
\end{explanation}
\end{example}




\begin{example}
If the series 
\[
\sum_{n=8}^\infty \left(\frac{-1}{2}\right)^n
\]
converges, find its sum.
\begin{explanation}
  Since the common ratio between the terms of this series is
$\answer[given]{-1/2}$, we see that this series
\wordChoice{\choice[correct]{converges}\choice{diverges}}. Write with me
\begin{align*}
  S &= \left(\frac{-1}{2}\right)^{8} + \left(\frac{-1}{2}\right)^{9} + \left(\frac{-1}{2}\right)^{10} + \cdots\\
  \left(\frac{-1}{2}\right) S &= \left(\frac{-1}{2}\right)^{9} + \left(\frac{-1}{2}\right)^{10} + \left(\frac{-1}{2}\right)^{11} + \cdots
\end{align*}
subtracing these two lines we find
\begin{align*}
  S -  \left(\frac{-1}{2}\right) S  &= \left(\frac{-1}{2}\right)^{8}\\
  S\left(1-\frac{-1}{2}\right) &= \left(\frac{-1}{2}\right)^{8}\\
  S &= \frac{(-1/2)^{8}}{1+1/2}.
\end{align*}
\end{explanation}
\end{example}


\subsection{Connections to decimals}

Remember how we pointed out that 
\[
\frac{1}{3} = 0.3333333333\dots
\]
is a geometric series? We can use our techniques for summing geometric
series to find fractions equal to given decimals.

\begin{example}
  Find a fraction equal to
  \[
  0.47474747474747\dots
  \]
  \begin{explanation}
    Do this exactly the same way as the examples we've done
    before. Write
    \begin{align*}
    N &=      0.47474747474747\dots\\
    100 N &= 47.47474747474747\dots
    \end{align*}
    Now subtract the top line from the bottom line, to find
    \begin{align*}
      100N - N &= 47\\
      N(100-1) &= 47\\
      N &= \frac{47}{99}
    \end{align*}
  \end{explanation}
\end{example}


\begin{example}
  Find a fraction equal to
  \[
  9.42764864864864864864\dots
  \]
  \begin{explanation}
    Do this exactly the same way as the examples we've done
    before. Write
    \begin{align*}
    N &=         9.42764864864864864864\dots\\
    1000 N &= 9427.64864864864864864\dots
    \end{align*}
    Now subtract the top line from the bottom line, to find
    \begin{align*}
      1000N - N &= \\
      N(1000-1) &= 9418.221\\
      N &= \frac{9418.221}{999}
    \end{align*}
    Hence our fraction will be
    \[
    \frac{9418221}{999000}.
    \]
  \end{explanation}
\end{example}



\section{Telescoping series}

Telescoping series are infinite sums that ``collapse'' to more
managable expressions.  Let's see an example

\begin{example}
  Evaluate the sum
  \[
  \sum_{k=1}^\infty\left(\frac{1}{k}-\frac{1}{k+1}\right).
  \]
  \begin{explanation}
    Let's look at the $n$th partial sum:
    \[
    \sum_{k=1}^n\left(\frac{1}{k}-\frac{1}{k+1}\right) =
    \left(\frac{1}{1}-\frac{1}{2}\right) + \left(\frac{1}{2}-\frac{1}{3}\right)+\left(\frac{1}{3}-\frac{1}{4}\right) + \dots + \left(\frac{1}{n}-\frac{1}{n+1}\right)
    \]
    
  \end{explanation}
\end{example}


Evaluate the sum $\sum_{n=1}^\infty \left(\frac1n-\frac1{n+1}\right)$.
\index{series!telescoping}\index{telescoping series}
It will help to write down some of the first few partial sums of this series.
\begin{align*}
S_1 &=	\frac11-\frac12 & & = 1-\frac12\\
S_2 &=	\left(\frac11-\frac12\right) + \left(\frac12-\frac13\right) & & = 1-\frac13\\
S_3 &=	\left(\frac11-\frac12\right) + \left(\frac12-\frac13\right)+\left(\frac13-\frac14\right) & &= 1-\frac14\\
S_4 &=	\left(\frac11-\frac12\right) + \left(\frac12-\frac13\right)+\left(\frac13-\frac14\right) +\left(\frac14-\frac15\right)& &= 1-\frac15
\end{align*}
Note how most of the terms in each partial sum are canceled out! In
general, we see that $S_n = 1-\frac{1}{n+1}$. The sequence $(S_n)$
converges, as $\lim_{n\to\infty}S_n =
\lim_{n\to\infty}\left(1-\frac1{n+1}\right) = 1$, and so we conclude
that $\sum_{n=1}^\infty \left(\frac1n-\frac1{n+1}\right) = 1$. Partial
sums of the series are plotted in Figure \ref{fig:series3}.
%\mfigure{.75}{Scatter plots relating to the series of Example \ref{ex_series3}.}{fig:series3}{figures/figseries3}


The series in Example \ref{ex_series3} is an example of a \dfn{telescoping series}. Informally, a telescoping series is one in which the partial sums reduce to just a finite number of terms. The partial sum $S_n$ did not contain $n$ terms, but rather just two: 1 and $1/(n+1)$.\index{series!telescoping}\index{telescoping series}

When possible, seek a way to write an explicit formula for the $n^\text{th}$ partial sum $S_n$. This makes evaluating the limit $\lim_{n\to\infty} S_n$ much more approachable. We do so in the next example.\\

%\noindent\textbf{Note on notation:} Most of the series we encounter will start with $n=1$. For ease of notation, we will often write $\sum a_n$ instead of writing $\sum_{n=1}^\infty a_n$.\\



\begin{example}
Evaluate each of the following infinite series.

\noindent 1. $\sum_{n=1}^\infty \frac{2}{n^2+2n}$ \qquad 2. $\sum_{n=1}^\infty \ln\left(\frac{n+1}{n}\right)$

\begin{enumerate}
\item We can decompose the fraction $2/(n^2+2n)$ as $$\frac2{n^2+2n}
  = \frac1n-\frac1{n+2}.$$ (See Section \ref{sec:partial_fraction},
  Partial Fraction Decomposition, to recall how this is done, if
  necessary.)
  
  Expressing the terms of $(S_n)$ is now more instructive:
  
  \begin{align*}
    S_1 &= 1-\frac13 &&= 1-\frac13\\
    S_2 &= \left(1-\frac13\right) + \left(\frac12-\frac14\right) &&= 1+\frac12-\frac13-\frac14\\
    S_3 &= \left(1-\frac13\right) + \left(\frac12-\frac14\right)+\left(\frac13-\frac15\right) &&= 1+\frac12-\frac14-\frac15\\
    S_4 &= \left(1-\frac13\right) + \left(\frac12-\frac14\right)+\left(\frac13-\frac15\right)+\left(\frac14-\frac16\right) &&= 1+\frac12-\frac15-\frac16\\
    S_5 &= \left(1-\frac13\right) + \left(\frac12-\frac14\right)+\left(\frac13-\frac15\right)+\left(\frac14-\frac16\right)+\left(\frac15-\frac17\right) &&= 1+\frac12-\frac16-\frac17\\
  \end{align*}
 
We again have a telescoping series. In each partial sum, most of the
terms cancel and we obtain the formula $S_n =
1+\frac12-\frac1{n+1}-\frac1{n+2}.$ Taking limits allows us to
determine the convergence of the series:
\[
\lim_{n\to\infty}S_n = \lim_{n\to\infty} \left(1+\frac12-\frac1{n+1}-\frac1{n+2}\right) = \frac32,\quad \text{so } \sum_{n=1}^\infty \frac1{n^2+2n} = \frac32.
\]
This is illustrated in Figure \ref{fig:series4}(a).
%\mfigure{.3}{Scatter plots relating to the series of Example \ref{ex_series4} part 1.}{fig:series4a}{figures/figseries4a}
%\mtable{.5}{Scatter plots relating to the series in Example \ref{ex_series4}.}{fig:series4}{%
%% \begin{tabular}{c}
%% \myincludegraphics{figures/figseries4a}\\[10pt]
%% (a)\\[15pt]
%% \myincludegraphics{figures/figseries4b}\\[10pt]
%% (b)
%% \end{tabular}
%% }
%\drawexampleline

\item We begin by writing the first few partial sums of the series:

\begin{align*}
S_1 &= \ln\left(2\right) \\
S_2 &= \ln\left(2\right)+\ln\left(\frac32\right) \\
S_3 &= \ln\left(2\right)+\ln\left(\frac32\right)+\ln\left(\frac43\right) \\
S_4 &= \ln\left(2\right)+\ln\left(\frac32\right)+\ln\left(\frac43\right)+\ln\left(\frac54\right) 
\end{align*}
At first, this does not seem helpful, but recall the logarithmic identity: $\ln x+\ln y = \ln (xy).$ Applying this to $S_4$ gives:
$$S_4 = \ln\left(2\right)+\ln\left(\frac32\right)+\ln\left(\frac43\right)+\ln\left(\frac54\right) = \ln\left(\frac21\cdot\frac32\cdot\frac43\cdot\frac54\right) = \ln\left(5\right).$$

We can conclude that $(S_n) = \big(\ln (n+1)\big)$. This sequence does
not converge, as $\lim_{n\to\infty}S_n=\infty$. Therefore
$\sum_{n=1}^\infty \ln\left(\frac{n+1}{n}\right)=\infty$; the series
diverges. Note in Figure \ref{fig:series4}(b) how the sequence of
partial sums grows slowly; after 100 terms, it is not yet over
5. Graphically we may be fooled into thinking the series converges,
but our analysis above shows that it does not.
%\mfigure{.35}{Scatter plots relating to the series of Example \ref{ex_series4} part 2.}{fig:series4b}{figures/figseries4b}
\end{enumerate}
\end{example}

%\enlargethispage{3\baselineskip}
We are learning about a new mathematical object, the series. As done
before, we apply ``old'' mathematics to this new topic.



\section{Properties of sums}

We are learning about a new mathematical object, the series. As done
before, we apply ``old'' mathematics to this new topic.

\begin{theorem}[Properties of Infinite Series]
  Let
  \[
  \sum_{n=1}^\infty a_n = L,\quad \sum_{n=1}^\infty b_n =K, 
  \]
  and let $c$ be a constant.
\begin{enumerate}
\item Constant Multiple Rule: $\sum_{n=1}^\infty c\cdot a_n =
  c\cdot\sum_{n=1}^\infty a_n = c\cdot L.$\index{Constant Multiple
    Rule!of series}
\item Sum/Difference Rule: $\sum_{n=1}^\infty \big(a_n\pm b_n\big) =
  \sum_{n=1}^\infty a_n \pm \sum_{n=1}^\infty b_n = L \pm K.$
  \index{series!properties}\index{Sum/Difference Rule!of series}
\end{enumerate} 
\end{theorem}








\end{document}
