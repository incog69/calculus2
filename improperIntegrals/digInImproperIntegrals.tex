\documentclass{ximera}

\outcome{Identify an improper integral.}
\outcome{Determine if an improper integral converges or diverges.}
\outcome{Compute integrals over infinite intervals.}
\outcome{Compute integrals of functions with vertical asymptotes.}

%\usepackage{todonotes}

\newcommand{\todo}{}

\usepackage{tkz-euclide}
\tikzset{>=stealth} %% cool arrow head
\tikzset{shorten <>/.style={ shorten >=#1, shorten <=#1 } } %% allows shorter vectors

\usetikzlibrary{backgrounds} %% for boxes around graphs
\usetikzlibrary{shapes,positioning}  %% Clouds and stars
\usetikzlibrary{matrix} %% for matrix
\usepgfplotslibrary{polar} %% for polar plots
\usetkzobj{all}
\usepackage[makeroom]{cancel} %% for strike outs
%\usepackage{mathtools} %% for pretty underbrace % Breaks Ximera
\usepackage{multicol}





\usepackage{array}
\setlength{\extrarowheight}{+.1cm}   
\newdimen\digitwidth
\settowidth\digitwidth{9}
\def\divrule#1#2{
\noalign{\moveright#1\digitwidth
\vbox{\hrule width#2\digitwidth}}}





\newcommand{\RR}{\mathbb R}
\newcommand{\R}{\mathbb R}
\newcommand{\N}{\mathbb N}
\newcommand{\Z}{\mathbb Z}

%\renewcommand{\d}{\,d\!}
\renewcommand{\d}{\mathop{}\!d}
\newcommand{\dd}[2][]{\frac{\d #1}{\d #2}}
\renewcommand{\l}{\ell}
\newcommand{\ddx}{\frac{d}{\d x}}

\newcommand{\zeroOverZero}{\ensuremath{\boldsymbol{\tfrac{0}{0}}}}
\newcommand{\inftyOverInfty}{\ensuremath{\boldsymbol{\tfrac{\infty}{\infty}}}}
\newcommand{\zeroOverInfty}{\ensuremath{\boldsymbol{\tfrac{0}{\infty}}}}
\newcommand{\zeroTimesInfty}{\ensuremath{\small\boldsymbol{0\cdot \infty}}}
\newcommand{\inftyMinusInfty}{\ensuremath{\small\boldsymbol{\infty - \infty}}}
\newcommand{\oneToInfty}{\ensuremath{\boldsymbol{1^\infty}}}
\newcommand{\zeroToZero}{\ensuremath{\boldsymbol{0^0}}}
\newcommand{\inftyToZero}{\ensuremath{\boldsymbol{\infty^0}}}


\newcommand{\numOverZero}{\ensuremath{\boldsymbol{\tfrac{\#}{0}}}}
\newcommand{\dfn}{\textbf}
%\newcommand{\unit}{\,\mathrm}
\newcommand{\unit}{\mathop{}\!\mathrm}
\newcommand{\eval}[1]{\bigg[ #1 \bigg]}
\newcommand{\seq}[1]{\left( #1 \right)}
\renewcommand{\epsilon}{\varepsilon}
\renewcommand{\iff}{\Leftrightarrow}

\DeclareMathOperator{\arccot}{arccot}
\DeclareMathOperator{\arcsec}{arcsec}
\DeclareMathOperator{\arccsc}{arccsc}
\DeclareMathOperator{\si}{Si}

\newcommand{\tightoverset}[2]{%
  \mathop{#2}\limits^{\vbox to -.5ex{\kern-0.75ex\hbox{$#1$}\vss}}}
\newcommand{\arrowvec}[1]{\tightoverset{\scriptstyle\rightharpoonup}{#1}}
\renewcommand{\vec}{\mathbf}


\colorlet{textColor}{black} 
\colorlet{background}{white}
\colorlet{penColor}{blue!50!black} % Color of a curve in a plot
\colorlet{penColor2}{red!50!black}% Color of a curve in a plot
\colorlet{penColor3}{red!50!blue} % Color of a curve in a plot
\colorlet{penColor4}{green!50!black} % Color of a curve in a plot
\colorlet{penColor5}{orange!80!black} % Color of a curve in a plot
\colorlet{fill1}{penColor!20} % Color of fill in a plot
\colorlet{fill2}{penColor2!20} % Color of fill in a plot
\colorlet{fillp}{fill1} % Color of positive area
\colorlet{filln}{penColor2!20} % Color of negative area
\colorlet{fill3}{penColor3!20} % Fill
\colorlet{fill4}{penColor4!20} % Fill
\colorlet{fill5}{penColor5!20} % Fill
\colorlet{gridColor}{gray!50} % Color of grid in a plot

\newcommand{\surfaceColor}{violet}
\newcommand{\surfaceColorTwo}{redyellow}
\newcommand{\sliceColor}{greenyellow}




\pgfmathdeclarefunction{gauss}{2}{% gives gaussian
  \pgfmathparse{1/(#2*sqrt(2*pi))*exp(-((x-#1)^2)/(2*#2^2))}%
}


%%%%%%%%%%%%%
%% Vectors
%%%%%%%%%%%%%

%% Simple horiz vectors
\renewcommand{\vector}[1]{\left\langle #1\right\rangle}


%% %% Complex Horiz Vectors with angle brackets
%% \makeatletter
%% \renewcommand{\vector}[2][ , ]{\left\langle%
%%   \def\nextitem{\def\nextitem{#1}}%
%%   \@for \el:=#2\do{\nextitem\el}\right\rangle%
%% }
%% \makeatother

%% %% Vertical Vectors
%% \def\vector#1{\begin{bmatrix}\vecListA#1,,\end{bmatrix}}
%% \def\vecListA#1,{\if,#1,\else #1\cr \expandafter \vecListA \fi}

%%%%%%%%%%%%%
%% End of vectors
%%%%%%%%%%%%%

%\newcommand{\fullwidth}{}
%\newcommand{\normalwidth}{}



%% makes a snazzy t-chart for evaluating functions
%\newenvironment{tchart}{\rowcolors{2}{}{background!90!textColor}\array}{\endarray}

%%This is to help with formatting on future title pages.
\newenvironment{sectionOutcomes}{}{} 



%% Flowchart stuff
%\tikzstyle{startstop} = [rectangle, rounded corners, minimum width=3cm, minimum height=1cm,text centered, draw=black]
%\tikzstyle{question} = [rectangle, minimum width=3cm, minimum height=1cm, text centered, draw=black]
%\tikzstyle{decision} = [trapezium, trapezium left angle=70, trapezium right angle=110, minimum width=3cm, minimum height=1cm, text centered, draw=black]
%\tikzstyle{question} = [rectangle, rounded corners, minimum width=3cm, minimum height=1cm,text centered, draw=black]
%\tikzstyle{process} = [rectangle, minimum width=3cm, minimum height=1cm, text centered, draw=black]
%\tikzstyle{decision} = [trapezium, trapezium left angle=70, trapezium right angle=110, minimum width=3cm, minimum height=1cm, text centered, draw=black]


\title[Dig-In:]{Improper Integrals}

\begin{document}
\begin{abstract}
  We can use limits to integrate functions on unbounded domains or functions with unbounded range.
\end{abstract}
\maketitle


When we defined the definite integral
\[
\int_a^b f(x) \d x,
\]
we made two stipulations:
\begin{itemize}
\item The interval over which we integrated, $[a,b]$, was a finite
  interval, and
\item The function $f$ was continuous on $[a,b]$ (ensuring that the
  range of $f$ was finite).
\end{itemize}

In this section we consider integrals where one or both of the above
conditions do not hold. Such integrals are called \textbf{improper
  integrals.} Let's state this explicitly in a definition:


\begin{definition}
  An integral
  \[
  \int_a^b f(x) \d x,
  \]
  is called an \dfn{improper integral} if one of, or both, of the conditions hold:
  \begin{itemize}
  \item The interval $[a,b]$ is not finite.
  \item The function $f$ is unbounded on the interval $[a,b]$.
  \end{itemize}
\end{definition}

\begin{question}
  Which of the following integrals are improper integrals?
  \begin{selectAll}
    \choice{$\int_{-1}^{100} \frac{1}{1+x^2} \d x$}
    \choice[correct]{$\int_{1}^{\infty}\frac{1}{1+x^2}\d x$}
    \choice[correct]{$\int_0^1 \ln(x) \d x$}
    \choice{$\int_0^1 x\ln(x) \d x$}
    \choice[correct]{$\int_{-\infty}^{\infty} \sin(x) \d x$}
    \choice[correct]{$\int_{\frac{3\pi}{2}}^{\frac{3\pi}{2}} \tan(x) \d x$}
  \end{selectAll}
\end{question}



\section{Unbounded intervals}

The easiest way to identify an improper integral is too look for
$\infty$ or $-\infty$ in the limits of integration.  How do we deal
with improper integrals of this type? Well consider the following
definite integrals:
\begin{itemize}
\item	$\int_0^{100}\frac{1}{1+x^2} \d x \approx 1.5608,$
\item	$\int_0^{1000}\frac{1}{1+x^2} \d x \approx 1.5698,$
\item	$\int_0^{10000}\frac{1}{1+x^2} \d x \approx 1.5707.$
\end{itemize}
Notice how the integrand is $1/(1+x^2)$ in each integral. As the upper
bound gets larger, one would expect the ``area under the curve'' would
also grow. While the definite integrals do increase in value as the
upper bound grows, they are not increasing by much.

\begin{question}
\[
\int_0^b \frac{1}{1+x^2} \d x =  \answer{\arctan(b)}
\]
\begin{hint}
  \[
  \int \frac{1}{1+x^2} \d x = \arctan(x)+C,
  \]
  so $\int_0^b \frac{1}{1+x^2} \d x = \arctan(b) -\arctan(0) = \arctan(b)$.
\end{hint}
\end{question}

Now we take the limit as $b$ goes to infinity.

\begin{question}
  What is the limit of $\arctan(b)$ as $b$ goes to infinity?
  \begin{hint}
    Remember $\arctan(b) = \theta$ means that $\tan(\theta) = b$ and
    that $-\pi/2 < \theta < \pi/$.
  \end{hint}
  \begin{hint}
    Remember, $\tan(\theta) = \frac{\sin(\theta)}{\cos(theta)}$.
  \end{hint}
  \begin{hint}
    If $\tan(\theta) = \frac{\sin(\theta)}{\cos(\theta)} \to \infty$
    and $-\pi/2 < \theta < \pi/$, then it must be the case that
    $\theta$ is approaching $\pi/2$.
  \end{hint}
  \begin{prompt}
    \[
    \lim_{b\to\infty} \arctan(b) = \answer{\pi/2}
    \]
  \end{prompt}
\end{question}

Let's explicitly state how to compute integrals with unbounded limits of integration:


\begin{definition}\hfil
\begin{itemize}
\item Let $f$ be a continuous function on $[a,\infty)$.
  \[
  \int_a^\infty f(x) \d x \text{ is defined to be } \lim_{b\to\infty}\int_a^b f(x) \d x
  \]
\item Let $f$ be a continuous function on $(-\infty,b]$.
  \[
  \int_{-\infty}^b f(x) \d x \text{ is defined to be } \lim_{a\to-\infty}\int_a^b f(x) \d x
  \]
\item Let $f$ be a continuous function on $(-\infty,\infty)$. Let $c$
  be any real number.
  \[
  \int_{-\infty}^\infty f(x) \d x \text{ is defined to be }
  \]
  \[
  \lim_{a\to-\infty}\int_a^c f(x) \d x + \lim_{b\to\infty}\int_c^b
  f(x) \d x
  \]
\end{itemize}
An improper integral is said to \dfn{converge} if its corresponding
limit exists and is equal to a real number. Otherwise, the improper
integral is said to \dfn{diverge}. In this case, the integral either
does not exist, or is equal to $-\infty$ or $\infty$.
\end{definition}

\begin{warning}
  The improper integral
  \[
  \int_{-\infty}^\infty f(x) \d x
  \]
  converges if and only if \textbf{both}
  \[
  \lim_{a\to-\infty}\int_a^c f(x) \d x
  \]
  and
  \[
  \lim_{b\to\infty}\int_c^b
  f(x) \d x
  \]
  converge to real numbers.
\end{warning}
A question for the young mathematician: Suppose that the improper
integral $\int_{-\infty}^\infty f(x) \d x$ exists.  Why does it not
matter which $c$ you choose to split the integral up at?



\begin{example}	
 Compute:
 \[
 \int_1^\infty \frac{1}{x^2} \d x
 \]
 \begin{explanation}
   Write with me,
   \begin{align*}
     \int_1^\infty \frac{1}{x^2} \d x &= \lim_{\answer[given]{b} \to \infty} \int_1^b \frac{1}{x^2} \d x\\
     &= \lim_{b \to \infty} \eval{\answer[given]{\frac{-1}{x}}}_1^b\\ 
     &= \lim_{b \to \infty} \left(1 - \frac{1}{b}\right)\\
     &= \answer[given]{1}.
   \end{align*}
 \end{explanation}
\end{example}

\begin{example}	
  Compute:
  \[
  \int_{-\infty}^{-1}\frac{1}{|x|\sqrt{x^2-1}} \d x
  \]
  \begin{explanation}
    Write with me,
    \begin{align*}
      \int_{-\infty}^{-1} \frac{1}{x\sqrt{x^2-1}} \d x &= \lim_{a \to -\infty} \int_{a}^{-1} \frac{1}{|x|\sqrt{x^2-1}}\d x\\
      &= \lim_{a \to -\infty} \eval{\answer[given]{\arcsec(x)}}_{a}^{-1} \\
      &= \lim_{a \to -\infty} \left(\arcsec(-1) - \arcsec(a)\right)\\
      &= \answer[given]{\pi - \frac{\pi}{2}}.
    \end{align*}
  \end{explanation}
  \end{example}

\begin{example}	
  Compute:
  \[
  \int_{-\infty}^\infty \sin(\theta) \d \theta
  \]
  \begin{explanation}
    Write with me:
    \[
    \int_{-\infty}^\infty \sin(\theta) \d \theta = \lim_{a\to -\infty} \int_a^c \sin(\theta) \d \theta + \lim_{b\to\infty}\int_c^b \sin(\theta) \d \theta.
    \]
    Let us evaluate these limits one at a time:
    \begin{align*}  
      \lim_{a\to -\infty}\eval{-\cos(\theta)}_a^c &= \lim_{a\to -\infty}\left(-\cos(c)+\cos(a)\right)\\
      &= \lim_{a\to -\infty} \cos(a) -\cos(c)\\
      &= \answer[given]{DNE}.
    \end{align*}
    We can stop here. Since one limit was infinite, the integral is
    divergent, and hence does not exist (DNE).
  \end{explanation}
\end{example}


\begin{warning}
In the last example, what would have happened if we had tried to
compute the indefinite integral with \textbf{a single limit}?
\[
\lim_{c\to \infty}\int_{-c}^c \sin(\theta)\d \theta?
\]
Since sine is an odd function on any bounded interval $[-c,c]$, we
would have found this integral to be $0$. However,
\[
\int_{-\infty}^\infty \sin(\theta) \d \theta \ne \lim_{c\to \infty}\int_{-c}^c \sin(\theta)\d \theta.
\]
We must compute \textbf{two limits} to correctly evaluate this
integral.
\end{warning}


\section{Unbounded functions}

We have just considered definite integrals where the interval of
integration was unbounded. We now consider another type of improper
integration, where the range of the integrand is unbounded.

\begin{definition}
Let $f(x)$ be a continuous function on $[a,b]$ except at $c$, $a\leq
c\leq b$, where $x=c$ is a vertical asymptote of $f$. Define
\[
\int_a^b f(x)\d x = \lim_{t\to c^-}\int_a^t f(x)\d x + \lim_{t\to c^+}\int_t^b f(x)\d x
\]
\end{definition}

Let's work a few examples:

\begin{example}	
  Compute:
  \[
  \int_{0}^1 \frac{1}{\sqrt{x}} \d x
  \]
  \begin{explanation}
    Write with me,
    \begin{align*}
      \int_{0}^1 \frac{1}{\sqrt{x}} \d x &= \lim_{\answer[given]{a} \to 0} \int_{a}^1 \frac{1}{\sqrt{x}} \d x \\
      &=  \lim_{a \to 0}  \eval{\answer[given]{2\sqrt{x}}}_a^1 \\
      &=  \lim_{a \to 0} \left(\answer[given]{2-2\sqrt{a}}\right)\\
      &= \answer[given]{2}.
    \end{align*}
  \end{explanation}
\end{example}

Isn't it quite surprising that the area bounded by a curve which has a
vertical asymptote can be finite? Consider the graph of $y=
\frac{1}{\sqrt{x}}$ on the interval $[0,1]$:
\begin{image}
  \begin{tikzpicture}
  \begin{axis}[
      xmin=-3, xmax=3,ymin=-1,ymax=3,domain=.01:1,
      axis lines =center, xlabel=$x$, ylabel=$y$,
      every axis y label/.style={at=(current axis.above origin),anchor=south},
      every axis x label/.style={at=(current axis.right of origin),anchor=west},
      axis on top,
      width=6in,
      height=3in,
    ] 
    \addplot [draw=none, fill=fillp] {1/sqrt(x)} \closedcycle;
    \addplot [penColor,very thick, smooth,domain=.01:1] {1/sqrt(x)};
    \addplot [penColor,very thick, smooth,domain=1:3] {1/sqrt(x)};
  \end{axis}
\end{tikzpicture}
\end{image}
Essentially, the function $\frac{1}{\sqrt{x}}$ goes to infinity ``fast
enough,'' as $x\to 0^+$, so that only a finite amount of area is
accumulated.


\begin{example}	
  Compute:
  \[
  \int_{-1}^1 \frac{1}{x^2} \d x
  \]
  \begin{explanation}
    The function $\frac{1}{x^2}$ has a vertical asymptote at $a=0$.
    Write with me,
  \[
  \int_{-1}^1 \frac{1}{\sqrt{x}} \d x = \lim_{t \to 0^+} \int_{t}^1 \frac{1}{x^2} \d x  + \lim_{t \to 0^-} \int_{-1}^t \frac{1}{x^2} \d x.
  \]  
  Let us evaluate these limits one at a time.
  \begin{align*}
    \lim_{t \to 0^+} \int_{t}^1 \frac{1}{x^2} \d x  &=  \lim_{\answer[given]{t} \to 0^+} \eval{\answer[given]{\frac{-1}{x}}}_t^1 \\
    &=  \lim_{t \to 0^+} \frac{1}{t}-1\\
    &= \infty.
  \end{align*}
    We can stop here. Since one limit was infinite, the integral is
    divergent, and hence does not exist (DNE).
  \end{explanation}
\end{example}

\begin{warning}
In the last example, what would have happened if had we \textbf{not
  noticed the vertical asymptote} in the integrand at $x=0$?

We probably would have blindly computed:
\begin{align*}
  \int_{-1}^1\frac1{x^2}\d x &= \eval{-\frac{1}{x}}_{-1}^1\\
  &= -1 - (1)\\
  &=-2
\end{align*}

But the integrand is always positive, so this answer of $-2$ is
complete nonsense! \textbf{Be on the lookout for vertical asymptotes!}
\end{warning}


\begin{example}
  Compute:
  \[
  \int_0^1 \ln(x)\d x
  \]
  \begin{explanation}
    Since $\lim_{x\to 0^+} \ln(x) = -\infty$, the natural logarithm
    has a vertical asymptote at $x = 0$. Thus, write with me
    \begin{align*}
    \int_0^1 \ln(x) \d x &=\lim_{\answer[given]{t}\to 0^+} \int_t^1 \ln(x) \d x \\
    &= \lim_{t\to 0^+} \eval{\answer[given]{x \ln(x)-x}}_t^1 \\
    &= \lim_{t\to 0^+} \left(\answer[given]{-1 -t \ln(t) +t}\right) \\
    &= -1.
    \end{align*}
  \end{explanation}
\end{example}

Compare our last example to
\[
\int_{-\infty}^0 e^x \d x.
\]
This is equal to
\begin{align*}
  \int_{-\infty}^0 e^x \d x &= \lim_{a\to -\infty} \int_a^0 e^x \d x\\
  &= \lim_{a\to-\infty}\eval{e^x}_{a}^0 \\
  &= \lim_{a\to-\infty}\left(e^0 - e^a\right) \\
  &=1.
\end{align*}


\begin{question}
  Is it an accident that
  \[
  \left|\int_{-\infty}^0 e^x \d x \right|= \left| \int_0^1 \ln(x)\d x \right|?
  \]
  \begin{prompt}
  \begin{multipleChoice}
    \choice{yes}
    \choice[correct]{no}
  \end{multipleChoice}
  \begin{feedback}
    Since $\ln(x)$ is the inverse function of $e^x$, these integrals
    are computing the same geometric area.
  \end{feedback}
  \end{prompt}
\end{question}


What is a practical example of an improper integral?  Sometimes in the
real world it is useful to compute using infinity because we are
interested in limiting behavior, or because we want to measure
something which is ``effectively'' infinite in some way (really, really
large).

\begin{example}
  Given two objects of mass $A$ and $B$ in kilograms, the force of
  gravity acting between them at a distance of $x$ meters is
  \[
  F(x) = G\cdot \frac{AB}{x^2},
  \]
  where $G$ is the gravitational constant.  About much work is done by
  moving these objects from a distance of $1 \unit{m}$ to a distance
  far enough away that the force of gravity acting on them can be
  disregarded?
  \begin{explanation}
    This problem Here we are accumulating large forces over
    infinitesimal distances. Hence
    \begin{align*}
      \d W &= F \d x\\
      &= \answer[given]{G\cdot \frac{AB}{x^2}} \d x
    \end{align*}
    We can approximate ``really, really far away'' with ``infinitely far away''. Write with me,
    \begin{align*}
      \textrm{Work} &= \int_1^\infty F \d x\\
      &= \int_1^\infty G\frac{AB}{x^2} \d x\\
      &= GAB \lim_{\answer[given]{b} \to \infty} \eval{\answer[given]{-\frac{1}{x}}}_1^b\\
      &=GAB \lim_{b \to \infty} \left(\answer[given]{1 - \frac{1}{b}}\right)\\
      &=\answer[given]{GAB}.
    \end{align*}
  \end{explanation}
\end{example}

\end{document}
