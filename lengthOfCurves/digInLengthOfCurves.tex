\documentclass{ximera}

%\usepackage{todonotes}

\newcommand{\todo}{}

\usepackage{tkz-euclide}
\tikzset{>=stealth} %% cool arrow head
\tikzset{shorten <>/.style={ shorten >=#1, shorten <=#1 } } %% allows shorter vectors

\usetikzlibrary{backgrounds} %% for boxes around graphs
\usetikzlibrary{shapes,positioning}  %% Clouds and stars
\usetikzlibrary{matrix} %% for matrix
\usepgfplotslibrary{polar} %% for polar plots
\usetkzobj{all}
\usepackage[makeroom]{cancel} %% for strike outs
%\usepackage{mathtools} %% for pretty underbrace % Breaks Ximera
\usepackage{multicol}





\usepackage{array}
\setlength{\extrarowheight}{+.1cm}   
\newdimen\digitwidth
\settowidth\digitwidth{9}
\def\divrule#1#2{
\noalign{\moveright#1\digitwidth
\vbox{\hrule width#2\digitwidth}}}





\newcommand{\RR}{\mathbb R}
\newcommand{\R}{\mathbb R}
\newcommand{\N}{\mathbb N}
\newcommand{\Z}{\mathbb Z}

%\renewcommand{\d}{\,d\!}
\renewcommand{\d}{\mathop{}\!d}
\newcommand{\dd}[2][]{\frac{\d #1}{\d #2}}
\renewcommand{\l}{\ell}
\newcommand{\ddx}{\frac{d}{\d x}}

\newcommand{\zeroOverZero}{\ensuremath{\boldsymbol{\tfrac{0}{0}}}}
\newcommand{\inftyOverInfty}{\ensuremath{\boldsymbol{\tfrac{\infty}{\infty}}}}
\newcommand{\zeroOverInfty}{\ensuremath{\boldsymbol{\tfrac{0}{\infty}}}}
\newcommand{\zeroTimesInfty}{\ensuremath{\small\boldsymbol{0\cdot \infty}}}
\newcommand{\inftyMinusInfty}{\ensuremath{\small\boldsymbol{\infty - \infty}}}
\newcommand{\oneToInfty}{\ensuremath{\boldsymbol{1^\infty}}}
\newcommand{\zeroToZero}{\ensuremath{\boldsymbol{0^0}}}
\newcommand{\inftyToZero}{\ensuremath{\boldsymbol{\infty^0}}}


\newcommand{\numOverZero}{\ensuremath{\boldsymbol{\tfrac{\#}{0}}}}
\newcommand{\dfn}{\textbf}
%\newcommand{\unit}{\,\mathrm}
\newcommand{\unit}{\mathop{}\!\mathrm}
\newcommand{\eval}[1]{\bigg[ #1 \bigg]}
\newcommand{\seq}[1]{\left( #1 \right)}
\renewcommand{\epsilon}{\varepsilon}
\renewcommand{\iff}{\Leftrightarrow}

\DeclareMathOperator{\arccot}{arccot}
\DeclareMathOperator{\arcsec}{arcsec}
\DeclareMathOperator{\arccsc}{arccsc}
\DeclareMathOperator{\si}{Si}

\newcommand{\tightoverset}[2]{%
  \mathop{#2}\limits^{\vbox to -.5ex{\kern-0.75ex\hbox{$#1$}\vss}}}
\newcommand{\arrowvec}[1]{\tightoverset{\scriptstyle\rightharpoonup}{#1}}
\renewcommand{\vec}{\mathbf}


\colorlet{textColor}{black} 
\colorlet{background}{white}
\colorlet{penColor}{blue!50!black} % Color of a curve in a plot
\colorlet{penColor2}{red!50!black}% Color of a curve in a plot
\colorlet{penColor3}{red!50!blue} % Color of a curve in a plot
\colorlet{penColor4}{green!50!black} % Color of a curve in a plot
\colorlet{penColor5}{orange!80!black} % Color of a curve in a plot
\colorlet{fill1}{penColor!20} % Color of fill in a plot
\colorlet{fill2}{penColor2!20} % Color of fill in a plot
\colorlet{fillp}{fill1} % Color of positive area
\colorlet{filln}{penColor2!20} % Color of negative area
\colorlet{fill3}{penColor3!20} % Fill
\colorlet{fill4}{penColor4!20} % Fill
\colorlet{fill5}{penColor5!20} % Fill
\colorlet{gridColor}{gray!50} % Color of grid in a plot

\newcommand{\surfaceColor}{violet}
\newcommand{\surfaceColorTwo}{redyellow}
\newcommand{\sliceColor}{greenyellow}




\pgfmathdeclarefunction{gauss}{2}{% gives gaussian
  \pgfmathparse{1/(#2*sqrt(2*pi))*exp(-((x-#1)^2)/(2*#2^2))}%
}


%%%%%%%%%%%%%
%% Vectors
%%%%%%%%%%%%%

%% Simple horiz vectors
\renewcommand{\vector}[1]{\left\langle #1\right\rangle}


%% %% Complex Horiz Vectors with angle brackets
%% \makeatletter
%% \renewcommand{\vector}[2][ , ]{\left\langle%
%%   \def\nextitem{\def\nextitem{#1}}%
%%   \@for \el:=#2\do{\nextitem\el}\right\rangle%
%% }
%% \makeatother

%% %% Vertical Vectors
%% \def\vector#1{\begin{bmatrix}\vecListA#1,,\end{bmatrix}}
%% \def\vecListA#1,{\if,#1,\else #1\cr \expandafter \vecListA \fi}

%%%%%%%%%%%%%
%% End of vectors
%%%%%%%%%%%%%

%\newcommand{\fullwidth}{}
%\newcommand{\normalwidth}{}



%% makes a snazzy t-chart for evaluating functions
%\newenvironment{tchart}{\rowcolors{2}{}{background!90!textColor}\array}{\endarray}

%%This is to help with formatting on future title pages.
\newenvironment{sectionOutcomes}{}{} 



%% Flowchart stuff
%\tikzstyle{startstop} = [rectangle, rounded corners, minimum width=3cm, minimum height=1cm,text centered, draw=black]
%\tikzstyle{question} = [rectangle, minimum width=3cm, minimum height=1cm, text centered, draw=black]
%\tikzstyle{decision} = [trapezium, trapezium left angle=70, trapezium right angle=110, minimum width=3cm, minimum height=1cm, text centered, draw=black]
%\tikzstyle{question} = [rectangle, rounded corners, minimum width=3cm, minimum height=1cm,text centered, draw=black]
%\tikzstyle{process} = [rectangle, minimum width=3cm, minimum height=1cm, text centered, draw=black]
%\tikzstyle{decision} = [trapezium, trapezium left angle=70, trapezium right angle=110, minimum width=3cm, minimum height=1cm, text centered, draw=black]


\title[Dig-In:]{Length of curves}

\begin{document}
\begin{abstract}
SOMETHING
\end{abstract}
\maketitle




Here is another geometric application of the integral, finding the
length of a portion of a curve. As usual, we need to think about how
we might approximate the length, and turn the approximation into an
integral.

\begin{image}
\begin{tikzpicture}
  \begin{axis}[
      xmin=0, xmax=7, ymin=0, ymax=5,
      clip=false,
      ticks=none,
      axis lines =center, xlabel=$x$, ylabel=$y$,
      every axis y label/.style={at=(current axis.above origin),anchor=south},
      every axis x label/.style={at=(current axis.right of origin),anchor=west},
      axis on top,
    ] 
    \addplot [very thick, penColor] plot coordinates {(3,2) (6,4)};
    \addplot [dashed, textColor] plot coordinates {(3,2) (6,2)};
    \addplot [dashed, textColor] plot coordinates {(6,4) (6,2)};
    \addplot[color=penColor,fill=penColor,only marks,mark=*] coordinates{(3,2)};  %% closed hole
    \addplot[color=penColor,fill=penColor,only marks,mark=*] coordinates{(6,4)};  %% closed hole
    \node at (axis cs:3,2) [anchor=east] {$(x_0,y_0)$};
    \node at (axis cs:6,4) [anchor=west] {$(x_1,y_1)$};
    \node at (axis cs:4.5,3.2) [penColor,anchor=east] {$\sqrt{(x_1-x_0)^2+(y_1-y_0)^2}$};
  \end{axis}
\end{tikzpicture}
%\caption{The length of a line segment.}
%\label{fig:length of a line segment}
\end{image}


We already know how to compute one simple arc length, that of a line
segment. If the endpoints are $(x_0,y_0)$ and $(x_1,y_1)$ then the
length of the segment is the distance between the points,
$\sqrt{(x_1-x_0)^2+(y_1-y_0)^2}$, see Figure~\ref{fig:length of a line segment}.

Now if $f(x)$ is ``nice'' (say, differentiable) it appears that we can
approximate the length of a portion of the curve with line segments,
and that as the number of segments increases, and their lengths
decrease, the sum of the lengths of the line segments will approach
the true arc length, see Figure~\ref{fig:approximating arc length}.

\begin{image}
\begin{tikzpicture}
	\begin{axis}[
            domain=-3:2,
            ymax=5,
            ymin=-5,
            %samples=100,
            axis lines =middle, xlabel=$x$, ylabel=$y$,
            every axis y label/.style={at=(current axis.above origin),anchor=south},
            every axis x label/.style={at=(current axis.right of origin),anchor=west}
          ]
          \addplot [dashed, textColor, smooth] {-(x^4)/4 - (x^3)/3 +x^2};
          \addplot [very thick, penColor] plot coordinates {
            (-3,-9/4) (-2,8/3) (-1,13/12) (0,0) (1,5/12) (2,-8/3)
          };
          \addplot[color=penColor,fill=penColor,only marks,mark=*] coordinates{(-3,-9/4) (-2,8/3) (-1,13/12) (0,0) (1,5/12) (2,-8/3)};  %% closed hole

          \node at (axis cs:-1.3,2) [anchor=west,textColor] {$f(x)$};  
        \end{axis}
\end{tikzpicture}
%\caption{Approximating the arc length of the curve defined by $f(x)$.}
%\label{fig:approximating arc length}
\end{image}

Now we need to write a formula for the sum of the lengths of the line
segments, in a form that we know becomes an integral in the limit.  So
we suppose we have divided the interval $[a,b]$ into $n$ subintervals
as usual, each with length $h=(b-a)/n$, and endpoints 
\[
a=x_0, x_1, x_2, \dots, x_n=b.
\]
The length of a typical line segment, joining $(x_i,f(x_i))$ to $
(x_{i+1},f(x_{i+1}))$, is 
\[
\sqrt{h^2 +(f(x_{i+1})-f(x_i))^2}.
\]
By the Mean Value Theorem, Theorem~\ref{thm:mvt}, there is a number $c_i$
in $(x_i,x_{i+1})$ such that 
\[
f'(c_i)=\frac{f(x_{i+1})-f(x_i)}{x_{i+1}-x_i} = \frac{f(x_{i+1})-f(x_i)}{h}.
\]
so $f'(c_i)h=f(x_{i+1})-f(x_i).$ Hence, the length of the line segment
can be written as
$$
  \sqrt{h^2 + (f'(c_i))^2h^2}=
  h\sqrt{1+(f'(c_i))^2}.
$$ 
The arc length is then
\[
\lim_{n\to\infty}\sum_{i=0}^{n-1} h\sqrt{1+(f'(c_i))^2} 
\]
This is a Riemann sum! Now we may take the limit as the number of
$x_i$'s chosen goes to infinity and $h$ goes to zero to obtain the
integral
\[
  \int_a^b \sqrt{1+(f'(x))^2}\d x.
\]
Note that the sum looks a bit different than others we have
encountered, because the approximation contains a $c_i$ instead of an
$x_i$. In the past we have always used left endpoints (namely, $x_i$)
to get a representative value of $f$ on $[x_i,x_{i+1}]$; now we are
using a different point, but the principle is the same.

To summarize, to compute the length of a curve on the interval
$[a,b]$, we compute the integral
$$\int_a^b \sqrt{1+(f'(x))^2}\d x.$$ 
Unfortunately, integrals of this form are typically difficult or
impossible to compute exactly, because usually none of our methods for
finding antiderivatives will work. In practice this means that the
integral will usually have to be approximated.


\begin{example} Let $f(x) = \sqrt{r^2-x^2}$, the upper half circle of radius
$r$. The length of this curve is half the circumference, namely $\pi
r$. Let's compute this with the arc length formula.
The derivative $f'$ is $-x/\sqrt{r^2-x^2}$ so the integral is
$$
  \int_{-r}^r \sqrt{1+{x^2\over r^2-x^2}}\d x
  =\int_{-r}^r \sqrt{r^2\over r^2-x^2}\d x
  =r\int_{-r}^r \sqrt{1\over r^2-x^2}\d x.
$$
Using a trigonometric substitution, we find the antiderivative, namely
$\arcsin(x/r)$. Notice that the integral is improper at both
endpoints, as the function $\sqrt{1/(r^2-x^2)}$ is undefined when
$x=\pm r$. So we need to compute
$$
  \lim_{D\to-r^+}\int_D^0  \sqrt{1\over r^2-x^2}\d x +
  \lim_{D\to r^-}\int_0^D  \sqrt{1\over r^2-x^2}\d x.
$$
This is not difficult, and has value $\pi$, so the original integral,
with the extra $r$ in front, has value $\pi r$ as expected.
\end{example}


\end{document}
