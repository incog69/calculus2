\documentclass{ximera}

%\usepackage{todonotes}

\newcommand{\todo}{}

\usepackage{tkz-euclide}
\tikzset{>=stealth} %% cool arrow head
\tikzset{shorten <>/.style={ shorten >=#1, shorten <=#1 } } %% allows shorter vectors

\usetikzlibrary{backgrounds} %% for boxes around graphs
\usetikzlibrary{shapes,positioning}  %% Clouds and stars
\usetikzlibrary{matrix} %% for matrix
\usepgfplotslibrary{polar} %% for polar plots
\usetkzobj{all}
\usepackage[makeroom]{cancel} %% for strike outs
%\usepackage{mathtools} %% for pretty underbrace % Breaks Ximera
\usepackage{multicol}





\usepackage{array}
\setlength{\extrarowheight}{+.1cm}   
\newdimen\digitwidth
\settowidth\digitwidth{9}
\def\divrule#1#2{
\noalign{\moveright#1\digitwidth
\vbox{\hrule width#2\digitwidth}}}





\newcommand{\RR}{\mathbb R}
\newcommand{\R}{\mathbb R}
\newcommand{\N}{\mathbb N}
\newcommand{\Z}{\mathbb Z}

%\renewcommand{\d}{\,d\!}
\renewcommand{\d}{\mathop{}\!d}
\newcommand{\dd}[2][]{\frac{\d #1}{\d #2}}
\renewcommand{\l}{\ell}
\newcommand{\ddx}{\frac{d}{\d x}}

\newcommand{\zeroOverZero}{\ensuremath{\boldsymbol{\tfrac{0}{0}}}}
\newcommand{\inftyOverInfty}{\ensuremath{\boldsymbol{\tfrac{\infty}{\infty}}}}
\newcommand{\zeroOverInfty}{\ensuremath{\boldsymbol{\tfrac{0}{\infty}}}}
\newcommand{\zeroTimesInfty}{\ensuremath{\small\boldsymbol{0\cdot \infty}}}
\newcommand{\inftyMinusInfty}{\ensuremath{\small\boldsymbol{\infty - \infty}}}
\newcommand{\oneToInfty}{\ensuremath{\boldsymbol{1^\infty}}}
\newcommand{\zeroToZero}{\ensuremath{\boldsymbol{0^0}}}
\newcommand{\inftyToZero}{\ensuremath{\boldsymbol{\infty^0}}}


\newcommand{\numOverZero}{\ensuremath{\boldsymbol{\tfrac{\#}{0}}}}
\newcommand{\dfn}{\textbf}
%\newcommand{\unit}{\,\mathrm}
\newcommand{\unit}{\mathop{}\!\mathrm}
\newcommand{\eval}[1]{\bigg[ #1 \bigg]}
\newcommand{\seq}[1]{\left( #1 \right)}
\renewcommand{\epsilon}{\varepsilon}
\renewcommand{\iff}{\Leftrightarrow}

\DeclareMathOperator{\arccot}{arccot}
\DeclareMathOperator{\arcsec}{arcsec}
\DeclareMathOperator{\arccsc}{arccsc}
\DeclareMathOperator{\si}{Si}

\newcommand{\tightoverset}[2]{%
  \mathop{#2}\limits^{\vbox to -.5ex{\kern-0.75ex\hbox{$#1$}\vss}}}
\newcommand{\arrowvec}[1]{\tightoverset{\scriptstyle\rightharpoonup}{#1}}
\renewcommand{\vec}{\mathbf}


\colorlet{textColor}{black} 
\colorlet{background}{white}
\colorlet{penColor}{blue!50!black} % Color of a curve in a plot
\colorlet{penColor2}{red!50!black}% Color of a curve in a plot
\colorlet{penColor3}{red!50!blue} % Color of a curve in a plot
\colorlet{penColor4}{green!50!black} % Color of a curve in a plot
\colorlet{penColor5}{orange!80!black} % Color of a curve in a plot
\colorlet{fill1}{penColor!20} % Color of fill in a plot
\colorlet{fill2}{penColor2!20} % Color of fill in a plot
\colorlet{fillp}{fill1} % Color of positive area
\colorlet{filln}{penColor2!20} % Color of negative area
\colorlet{fill3}{penColor3!20} % Fill
\colorlet{fill4}{penColor4!20} % Fill
\colorlet{fill5}{penColor5!20} % Fill
\colorlet{gridColor}{gray!50} % Color of grid in a plot

\newcommand{\surfaceColor}{violet}
\newcommand{\surfaceColorTwo}{redyellow}
\newcommand{\sliceColor}{greenyellow}




\pgfmathdeclarefunction{gauss}{2}{% gives gaussian
  \pgfmathparse{1/(#2*sqrt(2*pi))*exp(-((x-#1)^2)/(2*#2^2))}%
}


%%%%%%%%%%%%%
%% Vectors
%%%%%%%%%%%%%

%% Simple horiz vectors
\renewcommand{\vector}[1]{\left\langle #1\right\rangle}


%% %% Complex Horiz Vectors with angle brackets
%% \makeatletter
%% \renewcommand{\vector}[2][ , ]{\left\langle%
%%   \def\nextitem{\def\nextitem{#1}}%
%%   \@for \el:=#2\do{\nextitem\el}\right\rangle%
%% }
%% \makeatother

%% %% Vertical Vectors
%% \def\vector#1{\begin{bmatrix}\vecListA#1,,\end{bmatrix}}
%% \def\vecListA#1,{\if,#1,\else #1\cr \expandafter \vecListA \fi}

%%%%%%%%%%%%%
%% End of vectors
%%%%%%%%%%%%%

%\newcommand{\fullwidth}{}
%\newcommand{\normalwidth}{}



%% makes a snazzy t-chart for evaluating functions
%\newenvironment{tchart}{\rowcolors{2}{}{background!90!textColor}\array}{\endarray}

%%This is to help with formatting on future title pages.
\newenvironment{sectionOutcomes}{}{} 



%% Flowchart stuff
%\tikzstyle{startstop} = [rectangle, rounded corners, minimum width=3cm, minimum height=1cm,text centered, draw=black]
%\tikzstyle{question} = [rectangle, minimum width=3cm, minimum height=1cm, text centered, draw=black]
%\tikzstyle{decision} = [trapezium, trapezium left angle=70, trapezium right angle=110, minimum width=3cm, minimum height=1cm, text centered, draw=black]
%\tikzstyle{question} = [rectangle, rounded corners, minimum width=3cm, minimum height=1cm,text centered, draw=black]
%\tikzstyle{process} = [rectangle, minimum width=3cm, minimum height=1cm, text centered, draw=black]
%\tikzstyle{decision} = [trapezium, trapezium left angle=70, trapezium right angle=110, minimum width=3cm, minimum height=1cm, text centered, draw=black]


\outcome{Use the integral test to  determine that a series diverges.}
\outcome{Use integrals to estimate the value of a series within an error.}

\title[Dig-In:]{The integral test}

\begin{document}
\begin{abstract}
Infinite sums can be studied using improper integrals.
\end{abstract}
\maketitle

We have seen that we can graph a sequence as a collection of points in
the plane.  For instance, conside the \dfn{harmonic sequence}, this is the sequence where $a_n = 1/n$:
\[
1, \, \frac{1}{2}, \, \frac{1}{3}, \, \frac{1}{4}, \, \frac{1}{5}, \,\dots  
\]
If we plot the harmonic sequence it looks like this:
\begin{image}
\begin{tikzpicture}
	\begin{axis}[
            domain=0:6,xmin=0,xmax=6,ymin=0,ymax=1.5,
            width=4in,
            height=2in,
            xtick={1,2,...,5},
            ytick={1,.5,.333,.25,.2},
            yticklabels={},%$a_1 = 10$,$a_2=30$,$a_3=90$,$a_4=270$,$a_5=810$},
            axis lines =middle, xlabel=$n$, ylabel=$a$,
            every axis y label/.style={at=(current axis.above origin),anchor=south},
            every axis x label/.style={at=(current axis.right of origin),anchor=west},
            clip=false,
            axis on top,
          ]
          \addplot[color=penColor,fill=penColor,only marks,mark=*] coordinates{(1,1)};  %% closed hole          
          \addplot[color=penColor,fill=penColor,only marks,mark=*] coordinates{(2,1/2)};  %% closed hole          
          \addplot[color=penColor,fill=penColor,only marks,mark=*] coordinates{(3,1/3)};  %% closed hole          
          \addplot[color=penColor,fill=penColor,only marks,mark=*] coordinates{(4,1/4)};  %% closed hole          
          \addplot[color=penColor,fill=penColor,only marks,mark=*] coordinates{(5,1/5)};  %% closed hole  
        \end{axis}
\end{tikzpicture}
\end{image}

Is there a nice way to visualize the sum
\[
\sum_{k=1}^\infty \frac{1}{k}?
\]
One way to visualize the sum is to make rectangles!
\begin{image}
\begin{tikzpicture}
	\begin{axis}[
            domain=0:6,xmin=0,xmax=6,ymin=0,ymax=1.5,
            width=4in,
            height=2in,
            xtick={1,2,...,5},
            ytick={1,.5,.333,.25,.2},
            yticklabels={},%$a_1 = 10$,$a_2=30$,$a_3=90$,$a_4=270$,$a_5=810$},
            axis lines =middle, xlabel=$n$, ylabel=$a$,
            every axis y label/.style={at=(current axis.above origin),anchor=south},
            every axis x label/.style={at=(current axis.right of origin),anchor=west},
            axis on top,
          ]
          \addplot[color=penColor,fill=penColor,only marks,mark=*] coordinates{(1,1)};  %% closed hole
		\addplot [draw=penColor, fill = fillp] plot coordinates {(1,0) (2,0) (2, 1) (1,1) (1, 0)};          
          
	\addplot[color=penColor,fill=penColor,only marks,mark=*] coordinates{(2,.5)};  %% closed hole
		\addplot [draw=penColor, fill = fillp] plot coordinates {(2,0) (3,0) (3, .5) (2,.5) (2, 0)};          
          
          \addplot[color=penColor,fill=penColor,only marks,mark=*] coordinates{(3,.333)};  %% closed hole
          	\addplot [draw=penColor, fill = fillp] plot coordinates {(3,0) (4,0) (4, .333) (3,.333) (3, 0)};          

          
          \addplot[color=penColor,fill=penColor,only marks,mark=*] coordinates{(4,.25)};  %% closed hole        
          	\addplot [draw=penColor, fill = fillp] plot coordinates {(4,0) (5,0) (5, .25) (4,.25) (4, 0)};          

		  
          \addplot[color=penColor,fill=penColor,only marks,mark=*] coordinates{(5,.2)};  %% closed hole 
          	\addplot [draw=penColor, fill = fillp] plot coordinates {(5,0) (6,0) (6, .2) (5,.2) (5, 0)};          
           %\addplot [draw=penColor,very thick,smooth, domain=.5:6] {1/x};
        \end{axis}
\end{tikzpicture}
\end{image}
The area above is \textbf{exactly} equal to the sum:
\[
\sum_{k=1}^\infty\frac{1}{k}
\]
This lets us visually compare the sum of an infinite series to the
value of an improper integral.  For instance, if we add a plot of
$1/x$ to our picture above
\begin{image}
\begin{tikzpicture}
	\begin{axis}[
            domain=0:6,xmin=0,xmax=6,ymin=0,ymax=1.5,
            width=4in,
            height=2in,
            xtick={1,2,...,5},
            ytick={1,.5,.333,.25,.2},
            yticklabels={},%$a_1 = 10$,$a_2=30$,$a_3=90$,$a_4=270$,$a_5=810$},
            axis lines =middle, xlabel={}, ylabel={},
            every axis y label/.style={at=(current axis.above origin),anchor=south},
            every axis x label/.style={at=(current axis.right of origin),anchor=west},
            axis on top,
          ]
          \addplot[color=penColor,fill=penColor,only marks,mark=*] coordinates{(1,1)};  %% closed hole
		\addplot [draw=penColor, fill = fillp] plot coordinates {(1,0) (2,0) (2, 1) (1,1) (1, 0)};          
          
	\addplot[color=penColor,fill=penColor,only marks,mark=*] coordinates{(2,.5)};  %% closed hole
		\addplot [draw=penColor, fill = fillp] plot coordinates {(2,0) (3,0) (3, .5) (2,.5) (2, 0)};          
          
          \addplot[color=penColor,fill=penColor,only marks,mark=*] coordinates{(3,.333)};  %% closed hole
          	\addplot [draw=penColor, fill = fillp] plot coordinates {(3,0) (4,0) (4, .333) (3,.333) (3, 0)};          

          
          \addplot[color=penColor,fill=penColor,only marks,mark=*] coordinates{(4,.25)};  %% closed hole        
          	\addplot [draw=penColor, fill = fillp] plot coordinates {(4,0) (5,0) (5, .25) (4,.25) (4, 0)};          

		  
          \addplot[color=penColor,fill=penColor,only marks,mark=*] coordinates{(5,.2)};  %% closed hole 
          	\addplot [draw=penColor, fill = fillp] plot coordinates {(5,0) (6,0) (6, .2) (5,.2) (5, 0)};          
           \addplot [draw=penColor,very thick,smooth, domain=.5:6] {1/x};
        \end{axis}
\end{tikzpicture}
\end{image}
we see that
\[
\int_1^\infty \frac{1}{x} \d x < \sum_{k=1}^\infty\frac{1}{k}.
\]
This leads us to an interesting observation:
\begin{quote}
  Let $f$ be continuous, positive, and decreasing on $[1,\infty)$ with
    $a_k = f(k)$.  If $\int_1^\infty f(x) \d x$ diverges, so does
    $\sum_{k=1}^\infty a_k$.
\end{quote}

That's a pretty good observation, but we can do better. Consider the
sequence $a_n = 1/n^2$:
\[
1, \, \frac{1}{4}, \, \frac{1}{9}, \, \frac{1}{16}, \, \frac{1}{25}, \,\dots  
\]
Think about it If
\[
\sum_{k=1}^\infty a_k
\]
diverges, then so does
\[
\sum_{k=1}^\infty a_{k+1}
\]
since is simply the first sum missing $a_1$ and no single number can
be responsible for this sum diverging to infinity. This fact can be summed up int he following theorem:

\begin{theorem}[Infinite Nature of Series]
The convergence or divergence remains unchanged by the addition or
subtraction of any finite number of terms. That is:
\begin{itemize}
\item A divergent series will remain divergent with the addition or
  subtraction of any finite number of terms.
\item A convergent series will remain convergent with the addition or
  subtraction of any finite number of terms. (Of course, the
  \emph{sum} will likely change.)
\end{itemize}
\end{theorem}
Back to the task at hand, let's graph $a_{k+1}$ with rectangles:
\begin{image}
\begin{tikzpicture}
	\begin{axis}[
            domain=0:6,xmin=0,xmax=6,ymin=0,ymax=1.5,
            width=4in,
            height=2in,
            xtick={1,2,...,5},
            ytick={1,.5,.333,.25,.2},
            yticklabels={},%$a_1 = 10$,$a_2=30$,$a_3=90$,$a_4=270$,$a_5=810$},
            axis lines =middle, xlabel=$n$, ylabel=$a$,
            every axis y label/.style={at=(current axis.above origin),anchor=south},
            every axis x label/.style={at=(current axis.right of origin),anchor=west},
            axis on top,
          ]
          \addplot[color=penColor,fill=penColor,only marks,mark=*] coordinates{(1,1/4)};  %% closed hole
		\addplot [draw=penColor, fill = fillp] plot coordinates {(1,0) (2,0) (2, 1/4) (1,1/4) (1, 0)};          
          
	\addplot[color=penColor,fill=penColor,only marks,mark=*] coordinates{(2,1/9)};  %% closed hole
		\addplot [draw=penColor, fill = fillp] plot coordinates {(2,0) (3,0) (3, 1/9) (2,1/9) (2, 0)};          
          
          \addplot[color=penColor,fill=penColor,only marks,mark=*] coordinates{(3,1/16)};  %% closed hole
          	\addplot [draw=penColor, fill = fillp] plot coordinates {(3,0) (4,0) (4, 1/16) (3,1/16) (3, 0)};          

          
          \addplot[color=penColor,fill=penColor,only marks,mark=*] coordinates{(4,1/25)};  %% closed hole        
          	\addplot [draw=penColor, fill = fillp] plot coordinates {(4,0) (5,0) (5, 1/25) (4,1/25) (4, 0)};          

		  
          \addplot[color=penColor,fill=penColor,only marks,mark=*] coordinates{(5,1/36)};  %% closed hole 
          	\addplot [draw=penColor, fill = fillp] plot coordinates {(5,0) (6,0) (6, 1/36) (5,1/36) (5, 0)};          
           %\addplot [draw=penColor,very thick,smooth, domain=.5:6] {1/x};
        \end{axis}
\end{tikzpicture}
\end{image}
The area above is \textbf{exactly} equal to the sum:
\[
\sum_{k=1}^\infty\frac{1}{(k+1)^2}
\]
This lets us visually compare the sum of an infinite series to the
value of an improper integral.  For instance, if we add a plot of
$1/x^2$ to our picture above
\begin{image}
\begin{tikzpicture}
	\begin{axis}[
            domain=0:6,xmin=0,xmax=6,ymin=0,ymax=1.5,
            width=4in,
            height=2in,
            xtick={1,2,...,5},
            ytick={1,.5,.333,.25,.2},
            yticklabels={},%$a_1 = 10$,$a_2=30$,$a_3=90$,$a_4=270$,$a_5=810$},
            axis lines =middle, xlabel={}, ylabel={},
            every axis y label/.style={at=(current axis.above origin),anchor=south},
            every axis x label/.style={at=(current axis.right of origin),anchor=west},
            axis on top,
          ]
          \addplot[color=penColor,fill=penColor,only marks,mark=*] coordinates{(1,1/4)};  %% closed hole
		\addplot [draw=penColor, fill = fillp] plot coordinates {(1,0) (2,0) (2, 1/4) (1,1/4) (1, 0)};          
          
	\addplot[color=penColor,fill=penColor,only marks,mark=*] coordinates{(2,1/9)};  %% closed hole
		\addplot [draw=penColor, fill = fillp] plot coordinates {(2,0) (3,0) (3, 1/9) (2,1/9) (2, 0)};          
          
          \addplot[color=penColor,fill=penColor,only marks,mark=*] coordinates{(3,1/16)};  %% closed hole
          	\addplot [draw=penColor, fill = fillp] plot coordinates {(3,0) (4,0) (4, 1/16) (3,1/16) (3, 0)};          

          
          \addplot[color=penColor,fill=penColor,only marks,mark=*] coordinates{(4,1/25)};  %% closed hole        
          	\addplot [draw=penColor, fill = fillp] plot coordinates {(4,0) (5,0) (5, 1/25) (4,1/25) (4, 0)};          

		  
          \addplot[color=penColor,fill=penColor,only marks,mark=*] coordinates{(5,1/36)};  %% closed hole 
          	\addplot [draw=penColor, fill = fillp] plot coordinates {(5,0) (6,0) (6, 1/36) (5,1/36) (5, 0)};          
           \addplot [draw=penColor,very thick,smooth, domain=.5:6] {1/x^2};
        \end{axis}
\end{tikzpicture}
\end{image}
we see that
\[
\sum_{k=1}^\infty\frac{1}{k}.
<
\int_1^\infty \frac{1}{x} \d x
\]
This leads us to another interesting observation:
\begin{quote}
  Let $f$ be continuous, positive, and decreasing on $[1,\infty)$ with
    $a_k = f(k)$.  If $\sum_{k=1}^\infty a_k$ diverges, so does
    $\int_1^\infty f(x) \d x$.
\end{quote}
This leads us to the \textit{integral test}:
\begin{theorem}[Integral Test]
  Let $f$ be continuous, positive, and decreasing on $[1,\infty)$.
    with $a_k = f(k)$ Then either
    \[
    \sum_{k=1}^\infty a_k\text{ and }\int_1^\infty f(x) \d x
    \]
    \textbf{both converge, or they both diverge}.  It is impossible
    for one of them to diverge, and the other to converge.
\end{theorem}

\begin{question}
  Does the harmonic series $\sum_{k=1}^\infty \frac{1}{k} = 1 + \frac{1}{2} + \frac{1}{3}+ \dots$ converge or diverge?
  \begin{multipleChoice}
    \choice{converge}
    \choice[correct]{diverge}
  \end{multipleChoice}
  \begin{hint}
    By the integral test, $\sum_{k=1}^\infty \frac{1}{k}$ converges if
    and only if $\int_1^\infty \frac{1}{x} \d x$ converges
  \end{hint}
  \begin{hint}
    \begin{align*}
      \int_1^\infty \frac{1}{x} \d x &= \lim_{b \to \infty} \int_1^b \frac{1}{x} \d x\\
      &=\lim_{b \to \infty} \ln(b) \\
      &=\infty
    \end{align*}
    Thus the Harmonic series must diverge.
  \end{hint}
\end{question}

Our last example is a ``theorem'' in some textbooks. We think it is
just an application of the integral test.

\begin{example}[$p$-test]\index{p-test@$p$-test}
  Explain why the sum of the ``$p$-series''
  \[
  \sum_{k=1}^\infty \frac{1}{n^p}
  \]
  converges if and only if
  \[
  p \geq 1.
  \]
  \begin{explanation}
    By the integral test,
    \[
    \sum_1^\infty \frac{1}{n^p}
    \]
    converges if and only if
    \[
    \int_1^\infty \answer[given]{\frac{1}{x^p}} \d x
    \]
    converges. Let's check this, write with me:
    \begin{align*}
      \int_1^\infty \frac{1}{x^p} \d x &= \lim_{b\to\infty} \int_1^b x^{-p} \d x\\
      &= \lim_{b\to\infty} \eval{\answer[given]{\frac{x^{1-p}}{1-p}}}_1^b\\
      &= \lim_{b\to\infty} \frac{b^{1-p}}{1-p} - \frac{1^{1-p}}{1-p}\\
      &= \lim_{b\to\infty} \frac{b^{1-p}}{1-p} - \frac{1}{1-p}
    \end{align*}
    and this only converges when $p>\answer[given]{1}$.
  \end{explanation}
\end{example}


\section{Estimating Series Using Improper Integrals}

Another cool application of this graphical viewpoint on series
involves estimating series.

We cannot yet compute $\sum_{k=1}^\infty \frac{1}{k^2}$ exactly, and
we will not learn how to do so in this course. 

Say we wanted to \textbf{approximate} this sum within an error of
$\frac{1}{100}$.  How many terms should we sum?  Is it enough to sum
the first ten terms?  The first hundred?  We want to find a number $N$
where we can be sure that
\[
\sum_{k=1}^\infty \frac{1}{k^2}-\sum_{k=1}^N \frac{1}{k^2} < \frac{1}{100}.
\]
Now,
\[
\sum_{k=1}^\infty \frac{1}{k^2}- \sum_{k=1}^N \frac{1}{k^2} = \sum_{N+1}^\infty \frac{1}{k^2}.
\]
Set $R_N$ to be 
\[
R_N = \sum_{N+1}^\infty \frac{1}{k^2}.
\]
We can see from the graph where the red rectangles correspond to $R_N$
\begin{image}
\begin{tikzpicture}
	\begin{axis}[
            domain=0:6,xmin=0,xmax=6,ymin=0,ymax=2,
            width=4in,
            height=2in,
            xtick={1,2,...,5},
            xticklabels={$N$,$N+1$,$N+2$, $N+3$, $N+4$, $N+5$},
            ytick = {},
            yticklabels = {,,},
            axis lines =middle, xlabel=$n$, ylabel=$a$,
            every axis y label/.style={at=(current axis.above origin),anchor=south},
            every axis x label/.style={at=(current axis.right of origin),anchor=west},
            clip=false,
            axis on top,
          ]          
	  \addplot[color=penColor,fill=penColor,only marks,mark=*] coordinates{(2,7/10)};  %% closed hole
	  \addplot [draw=penColor, fill = fill1] plot coordinates {(2,0) (3,0) (3, 7/10) (2,7/10) (2, 0)};          
                    
          \addplot[color=penColor,fill=penColor,only marks,mark=*] coordinates{(3,7/20)};  %% closed hole
          \addplot [draw=penColor, fill = fill1] plot coordinates {(3,0) (4,0) (4, 7/20) (3,7/20) (3, 0)};          
                    
          \addplot[color=penColor,fill=penColor,only marks,mark=*] coordinates{(4,7/40)};  %% closed hole        
          \addplot [draw=penColor, fill = fill1] plot coordinates {(4,0) (5,0) (5, 7/40) (4,7/40) (4, 0)};          
          
          \addplot[color=penColor,fill=penColor,only marks,mark=*] coordinates{(5,7/80)};  %% closed hole
          \addplot [draw=penColor, fill = fill1] plot coordinates {(5,0) (6,0) (6, 7/80) (5,7/80) (5, 0)};          

	  \addplot [draw=penColor,very thick, domain=0.5:6] {(14/5)*2^(-x)};
        \end{axis}
\end{tikzpicture}
\end{image}
shifting these rectangles over one unit, we see
\begin{image}
\begin{tikzpicture}
	\begin{axis}[
            domain=0:6,xmin=0,xmax=6,ymin=0,ymax=2,
            width=4in,
            height=2in,
            xtick={1,2,...,5},
            xticklabels={$N$,$N+1$,$N+2$, $N+3$, $N+4$, $N+5$},
            ytick = {},
            yticklabels = {,,},
            axis lines =middle, xlabel=$n$, ylabel=$a$,
            every axis y label/.style={at=(current axis.above origin),anchor=south},
            every axis x label/.style={at=(current axis.right of origin),anchor=west},
            clip=false,
            axis on top,
          ]
          \addplot [draw=penColor, fill = fill1] plot coordinates {(1,0) (2,0) (2, 7/10) (1,7/10) (1, 0)};          
          
	  \addplot[color=penColor,fill=penColor,only marks,mark=*] coordinates{(2,7/10)};  %% closed hole
	  \addplot [draw=penColor, fill = fill1] plot coordinates {(2,0) (3,0) (3, 7/20) (2,7/20) (2, 0)};          
          
          \addplot[color=penColor,fill=penColor,only marks,mark=*] coordinates{(3,7/20)};  %% closed hole
          \addplot [draw=penColor, fill = fill1] plot coordinates {(3,0) (4,0) (4, 7/40) (3,7/40) (3, 0)};          
          
          
          \addplot[color=penColor,fill=penColor,only marks,mark=*] coordinates{(4,7/40)};  %% closed hole        
          \addplot [draw=penColor, fill = fill1] plot coordinates {(4,0) (5,0) (5, 7/80) (4,7/80) (4, 0)};          

          \addplot[color=penColor,fill=penColor,only marks,mark=*] coordinates{(5,7/80)};  %% closed hole

	  \addplot [draw=penColor,very thick, domain=0.5:6] {(14/5)*2^(-x)};
        \end{axis}
\end{tikzpicture}
\end{image}
and so we say
\[
R_N < \int_N^\infty \frac{1}{x^2} \d x 
\]
So if we can find a whole number $N$ for which
\[
\int_N^\infty \frac{1}{x^2} \d x < \frac{1}{100},
\]
we will be sure that we have summed enough terms in the series to get
to within $\frac{1}{100}$.


\begin{example}
  What is the least natural number $N$ that will bring 
  \[
  \sum_{k=1}^N \frac{1}{k^2}
  \]
  such that
  \[
  \sum_{k=1}^\infty \frac{1}{k^2} - \sum_{k=1}^N \frac{1}{k^2} <
  \frac{1}{100}
  \]
  \begin{explanation}
    From our work above, we see that we need to find $N$ such that
    \[
    \int_N^\infty \frac{1}{x^2} \d x < \answer[given]{\frac{1}{100}}
    \]
    write with me,
    \begin{align*}
      \int_N^\infty \frac{1}{x^2} \d x &< \frac{1}{100}\\
      \lim_{b \to \infty} \int_N^b \frac{1}{x^2} \d x &< \frac{1}{100}\\
      \lim_{b \to \infty} \eval{\answer[given]{\frac{-1}{x}}}_N^b& < \frac{1}{100}\\
      \lim_{b \to \infty} \frac{1}{N} - \frac{1}{b}& < \frac{1}{100}\\
      \frac{1}{N} &< \frac{1}{100}\\
      N&>\answer[given]{100}.
    \end{align*}
    So we need $N$ to be at least $\answer[given]{101}$.
  \end{explanation}
\end{example}

This shows that if we sum $101$ terms of the series $\sum_{k=1}^\infty
\frac{1}{k^2}$, we will get within $\frac{1}{100}$ of the true answer.

It turns out this series actually converges to $\frac{\pi^2}{6}$
(finding this uses advanced concepts like Fourier Analysis).  Using a
computer to sum the first $101$ terms we get
\[
\sum_{k=1}^{101} \frac{1}{k^2} \approx 1.6350,
\]
while $\frac{\pi^2}{6} \approx 1.6449$.  Their difference is just
barely less than $\frac{1}{100}$.

\end{document}


