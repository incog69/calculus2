\documentclass{ximera}

%\usepackage{todonotes}

\newcommand{\todo}{}

\usepackage{tkz-euclide}
\tikzset{>=stealth} %% cool arrow head
\tikzset{shorten <>/.style={ shorten >=#1, shorten <=#1 } } %% allows shorter vectors

\usetikzlibrary{backgrounds} %% for boxes around graphs
\usetikzlibrary{shapes,positioning}  %% Clouds and stars
\usetikzlibrary{matrix} %% for matrix
\usepgfplotslibrary{polar} %% for polar plots
\usetkzobj{all}
\usepackage[makeroom]{cancel} %% for strike outs
%\usepackage{mathtools} %% for pretty underbrace % Breaks Ximera
\usepackage{multicol}





\usepackage{array}
\setlength{\extrarowheight}{+.1cm}   
\newdimen\digitwidth
\settowidth\digitwidth{9}
\def\divrule#1#2{
\noalign{\moveright#1\digitwidth
\vbox{\hrule width#2\digitwidth}}}





\newcommand{\RR}{\mathbb R}
\newcommand{\R}{\mathbb R}
\newcommand{\N}{\mathbb N}
\newcommand{\Z}{\mathbb Z}

%\renewcommand{\d}{\,d\!}
\renewcommand{\d}{\mathop{}\!d}
\newcommand{\dd}[2][]{\frac{\d #1}{\d #2}}
\renewcommand{\l}{\ell}
\newcommand{\ddx}{\frac{d}{\d x}}

\newcommand{\zeroOverZero}{\ensuremath{\boldsymbol{\tfrac{0}{0}}}}
\newcommand{\inftyOverInfty}{\ensuremath{\boldsymbol{\tfrac{\infty}{\infty}}}}
\newcommand{\zeroOverInfty}{\ensuremath{\boldsymbol{\tfrac{0}{\infty}}}}
\newcommand{\zeroTimesInfty}{\ensuremath{\small\boldsymbol{0\cdot \infty}}}
\newcommand{\inftyMinusInfty}{\ensuremath{\small\boldsymbol{\infty - \infty}}}
\newcommand{\oneToInfty}{\ensuremath{\boldsymbol{1^\infty}}}
\newcommand{\zeroToZero}{\ensuremath{\boldsymbol{0^0}}}
\newcommand{\inftyToZero}{\ensuremath{\boldsymbol{\infty^0}}}


\newcommand{\numOverZero}{\ensuremath{\boldsymbol{\tfrac{\#}{0}}}}
\newcommand{\dfn}{\textbf}
%\newcommand{\unit}{\,\mathrm}
\newcommand{\unit}{\mathop{}\!\mathrm}
\newcommand{\eval}[1]{\bigg[ #1 \bigg]}
\newcommand{\seq}[1]{\left( #1 \right)}
\renewcommand{\epsilon}{\varepsilon}
\renewcommand{\iff}{\Leftrightarrow}

\DeclareMathOperator{\arccot}{arccot}
\DeclareMathOperator{\arcsec}{arcsec}
\DeclareMathOperator{\arccsc}{arccsc}
\DeclareMathOperator{\si}{Si}

\newcommand{\tightoverset}[2]{%
  \mathop{#2}\limits^{\vbox to -.5ex{\kern-0.75ex\hbox{$#1$}\vss}}}
\newcommand{\arrowvec}[1]{\tightoverset{\scriptstyle\rightharpoonup}{#1}}
\renewcommand{\vec}{\mathbf}


\colorlet{textColor}{black} 
\colorlet{background}{white}
\colorlet{penColor}{blue!50!black} % Color of a curve in a plot
\colorlet{penColor2}{red!50!black}% Color of a curve in a plot
\colorlet{penColor3}{red!50!blue} % Color of a curve in a plot
\colorlet{penColor4}{green!50!black} % Color of a curve in a plot
\colorlet{penColor5}{orange!80!black} % Color of a curve in a plot
\colorlet{fill1}{penColor!20} % Color of fill in a plot
\colorlet{fill2}{penColor2!20} % Color of fill in a plot
\colorlet{fillp}{fill1} % Color of positive area
\colorlet{filln}{penColor2!20} % Color of negative area
\colorlet{fill3}{penColor3!20} % Fill
\colorlet{fill4}{penColor4!20} % Fill
\colorlet{fill5}{penColor5!20} % Fill
\colorlet{gridColor}{gray!50} % Color of grid in a plot

\newcommand{\surfaceColor}{violet}
\newcommand{\surfaceColorTwo}{redyellow}
\newcommand{\sliceColor}{greenyellow}




\pgfmathdeclarefunction{gauss}{2}{% gives gaussian
  \pgfmathparse{1/(#2*sqrt(2*pi))*exp(-((x-#1)^2)/(2*#2^2))}%
}


%%%%%%%%%%%%%
%% Vectors
%%%%%%%%%%%%%

%% Simple horiz vectors
\renewcommand{\vector}[1]{\left\langle #1\right\rangle}


%% %% Complex Horiz Vectors with angle brackets
%% \makeatletter
%% \renewcommand{\vector}[2][ , ]{\left\langle%
%%   \def\nextitem{\def\nextitem{#1}}%
%%   \@for \el:=#2\do{\nextitem\el}\right\rangle%
%% }
%% \makeatother

%% %% Vertical Vectors
%% \def\vector#1{\begin{bmatrix}\vecListA#1,,\end{bmatrix}}
%% \def\vecListA#1,{\if,#1,\else #1\cr \expandafter \vecListA \fi}

%%%%%%%%%%%%%
%% End of vectors
%%%%%%%%%%%%%

%\newcommand{\fullwidth}{}
%\newcommand{\normalwidth}{}



%% makes a snazzy t-chart for evaluating functions
%\newenvironment{tchart}{\rowcolors{2}{}{background!90!textColor}\array}{\endarray}

%%This is to help with formatting on future title pages.
\newenvironment{sectionOutcomes}{}{} 



%% Flowchart stuff
%\tikzstyle{startstop} = [rectangle, rounded corners, minimum width=3cm, minimum height=1cm,text centered, draw=black]
%\tikzstyle{question} = [rectangle, minimum width=3cm, minimum height=1cm, text centered, draw=black]
%\tikzstyle{decision} = [trapezium, trapezium left angle=70, trapezium right angle=110, minimum width=3cm, minimum height=1cm, text centered, draw=black]
%\tikzstyle{question} = [rectangle, rounded corners, minimum width=3cm, minimum height=1cm,text centered, draw=black]
%\tikzstyle{process} = [rectangle, minimum width=3cm, minimum height=1cm, text centered, draw=black]
%\tikzstyle{decision} = [trapezium, trapezium left angle=70, trapezium right angle=110, minimum width=3cm, minimum height=1cm, text centered, draw=black]


\outcome{Use the divergence test to determine that a series diverges.}
\outcome{Recognize known convergent or divergent series.}

\title[Dig-In:]{The divergence test}

\begin{document}
\begin{abstract}
If an infinite sum converges, then its terms must tend to zero.
\end{abstract}
\maketitle


As one contemplates the behavior of series, a few facts become clear.
In order to add an infinite list of nonzero numbers and get a finite
result, ``most'' of those numbers must be ``very near'' $0$.  Think of 
this in the opposite sense: what happens if you try to sum $\sum_{n=1}^\infty 2$?

If a series diverges, it means that the sum of an infinite list of
numbers is not finite (it may approach $\pm \infty$ or it may
oscillate), and:
\begin{itemize}
\item The series will still diverge if the first term is removed.
\item The series will still diverge if the first $10$ terms are
  removed.
\item The series will still diverge if the first $1000000$ terms
  are removed.
\item The series will still diverge if \textbf{any finite number} of terms
  from anywhere in the series are removed.
\end{itemize}

These concepts are very important and lie at the heart of the next
theorems.

\begin{theorem}[Divergence test]
  Consider the series
  \[
  \sum_{n=1}^\infty a_n.
  \]
\begin{enumerate}
\item If $\sum_{n=1}^\infty a_n$ converges, then $\lim_{n\to\infty}a_n
  =0$.
\item If $\lim_{n\to\infty}a_n \neq 0$, then $\sum_{n=1}^\infty a_n$
  diverges.
\end{enumerate}
\end{theorem}
Note that the two statements above are really the same. In order to
converge, the limit of the terms of the sequence must approach $0$; if
they do not, the series will not converge.
\begin{warning}
  This theorem \emph{does not state} that if $\lim_{n\to\infty} a_n =
  0$ then $\sum_{n=1}^\infty a_n$ converges.
\end{warning}

The standard example of a sequence whose terms go to zero, and yet does
not converge, is the harmonic series. The Harmonic sequence,
$(1/n)$, converges to $0$ while the Harmonic Series,
\[
\sum_{n=1}^\infty\frac{1}{n}\qquad\text{diverges.}
\]

Let's see if you've digested what we've been saying:

\begin{question}
Which of the following statements are true?  Mark all that apply.
\begin{selectAll}
  \choice[correct]{If $\sum_{k=0}^\infty a_k$ is convergent, then $\lim_{k \to \infty} a_k = 0$ }
  \choice{If $a_k \to 0$ as $k \to \infty$, then $\sum_{k=0}^\infty a_k$ is convergent}
  \choice{If $\sum_{k=0}^\infty a_k$ is divergent, then $\lim_{k \to \infty} a_k \neq 0$ }
  \choice[correct]{If $\lim_{k \to \infty} a_k \neq 0$, then $\sum_{k=0}^\infty a_k$ is divergent}
\end{selectAll}
\end{question}


\begin{question}
  We say that a series ``passes the divergence test'' if its sequence
  of terms tends to zero.  Which of the following series pass the
  divergence test?
\begin{selectAll}
	\choice[correct]{$\sum_{k=3}^\infty \frac{1}{\ln{ k }}$}
	\choice{$\sum_{k=0}^\infty \sin(k)$}
	\choice[correct]{$\sum_{k=0}^\infty \frac{\sin(k)}{k^2}$}
	\choice{$\sum_{k=5}^\infty \frac{k+7}{k+6}$}
	\choice{$\sum_{k=0}^\infty \frac{2k}{k - 5}$}
\end{selectAll}
\end{question}

Restating this point again (because it is very important): passing the divergence test 
means that a series has a chance to converge.  The divergence test cannot tell us 
whether a series converges.


\section{Some questions}
%% Jim Talamo's questions

\begin{question}
  Suppose $(a_n)_{n \geq 1}$ is a sequence and $\sum^{\infty}_{n= 1}
  a_n$ converges to $L>0$.  Let $S_n = \sum^n_{k=1} a_k$. Select all
  statements that must be true:
  \begin{selectAll}
    \choice{$\lim_{n \to \infty} a_n = L$}
    \choice[correct]{$\lim_{n \to \infty} a_n = 0$}
    \choice{$\lim_{n \to \infty} S_n = 0$}
    \choice[correct]{$\lim_{n \to \infty} S_n = L$}
    \choice[correct]{$\sum^{\infty}_{n=1} S_n$ must diverge.}
    \choice{$\sum^{\infty}_{n=1} (a_n+1) = L+1$}
    \choice{The divergence test tells us $\sum^{\infty}_{n= 1} a_n$ converges to $L$.}
  \end{selectAll}
\end{question}

\begin{question}
  Suppose that $\seq{a_n}_{n \geq 1}$ is a \emph{decreasing} sequence.
  Let $S_n = \sum^n_{k=1} a_k$ and suppose $\lim_{n \to \infty} S_n$
  does not exist. Select all statements that must be true:
  \begin{selectAll}
    \choice{$\lim_{n \to \infty} a_n$ does not exist.}
    \choice{$\sum^{\infty}_{k=1} a_k$ could converge.}
    \choice[correct]{$\sum^{\infty}_{n=1} S_n$ must diverge.} 
    \choice{$\seq{S_n}$ must be monotonic.}
    \choice{$\seq{S_n}$ must be bounded.}
    \choice{$\lim_{n \to \infty} S_n = -\infty$}
    \choice{The divergence test applied to $\sum^{\infty}_{k= 1} a_k$ would
      guarantee that $\sum^{\infty}_{k=1} a_k$ diverges.}
  \end{selectAll}
\end{question}


\begin{question}
  Suppose that $\seq{a_n}_{n \geq 1}$ is a sequence with $a_n > 0$ for all $n \geq 1$.
  Let $S_n = \sum^n_{k=1} a_k$ and suppose $\lim_{n \to \infty} S_n =
  L$. Select all statements that must be true:
  \begin{selectAll}
    \choice[correct]{$\sum^{\infty}_{k=1} a_k = L$}
    \choice[correct]{$\lim_{n \to \infty} a_n = 0$}
    \choice[correct]{$\seq{S_n}$ must be monotonic}
    \choice[correct]{$\seq{S_n}$ must be bounded}
    \choice{$\sum^{\infty}_{n=1} (a_n-L) = 0$}
    \choice[correct]{$\sum^{\infty}_{n=1} S_n$ must diverge}
    \choice{The divergence test applied to $\sum^{\infty}_{k= 1} a_k$ would
      guarantee that $\sum^{\infty}_{k=1} a_k$ converges.}
  \end{selectAll}
\end{question}

It's a great idea at this point to stop and compare the previous two questions.









\end{document}


