\documentclass{ximera}

%\usepackage{todonotes}

\newcommand{\todo}{}

\usepackage{tkz-euclide}
\tikzset{>=stealth} %% cool arrow head
\tikzset{shorten <>/.style={ shorten >=#1, shorten <=#1 } } %% allows shorter vectors

\usetikzlibrary{backgrounds} %% for boxes around graphs
\usetikzlibrary{shapes,positioning}  %% Clouds and stars
\usetikzlibrary{matrix} %% for matrix
\usepgfplotslibrary{polar} %% for polar plots
\usetkzobj{all}
\usepackage[makeroom]{cancel} %% for strike outs
%\usepackage{mathtools} %% for pretty underbrace % Breaks Ximera
\usepackage{multicol}





\usepackage{array}
\setlength{\extrarowheight}{+.1cm}   
\newdimen\digitwidth
\settowidth\digitwidth{9}
\def\divrule#1#2{
\noalign{\moveright#1\digitwidth
\vbox{\hrule width#2\digitwidth}}}





\newcommand{\RR}{\mathbb R}
\newcommand{\R}{\mathbb R}
\newcommand{\N}{\mathbb N}
\newcommand{\Z}{\mathbb Z}

%\renewcommand{\d}{\,d\!}
\renewcommand{\d}{\mathop{}\!d}
\newcommand{\dd}[2][]{\frac{\d #1}{\d #2}}
\renewcommand{\l}{\ell}
\newcommand{\ddx}{\frac{d}{\d x}}

\newcommand{\zeroOverZero}{\ensuremath{\boldsymbol{\tfrac{0}{0}}}}
\newcommand{\inftyOverInfty}{\ensuremath{\boldsymbol{\tfrac{\infty}{\infty}}}}
\newcommand{\zeroOverInfty}{\ensuremath{\boldsymbol{\tfrac{0}{\infty}}}}
\newcommand{\zeroTimesInfty}{\ensuremath{\small\boldsymbol{0\cdot \infty}}}
\newcommand{\inftyMinusInfty}{\ensuremath{\small\boldsymbol{\infty - \infty}}}
\newcommand{\oneToInfty}{\ensuremath{\boldsymbol{1^\infty}}}
\newcommand{\zeroToZero}{\ensuremath{\boldsymbol{0^0}}}
\newcommand{\inftyToZero}{\ensuremath{\boldsymbol{\infty^0}}}


\newcommand{\numOverZero}{\ensuremath{\boldsymbol{\tfrac{\#}{0}}}}
\newcommand{\dfn}{\textbf}
%\newcommand{\unit}{\,\mathrm}
\newcommand{\unit}{\mathop{}\!\mathrm}
\newcommand{\eval}[1]{\bigg[ #1 \bigg]}
\newcommand{\seq}[1]{\left( #1 \right)}
\renewcommand{\epsilon}{\varepsilon}
\renewcommand{\iff}{\Leftrightarrow}

\DeclareMathOperator{\arccot}{arccot}
\DeclareMathOperator{\arcsec}{arcsec}
\DeclareMathOperator{\arccsc}{arccsc}
\DeclareMathOperator{\si}{Si}

\newcommand{\tightoverset}[2]{%
  \mathop{#2}\limits^{\vbox to -.5ex{\kern-0.75ex\hbox{$#1$}\vss}}}
\newcommand{\arrowvec}[1]{\tightoverset{\scriptstyle\rightharpoonup}{#1}}
\renewcommand{\vec}{\mathbf}


\colorlet{textColor}{black} 
\colorlet{background}{white}
\colorlet{penColor}{blue!50!black} % Color of a curve in a plot
\colorlet{penColor2}{red!50!black}% Color of a curve in a plot
\colorlet{penColor3}{red!50!blue} % Color of a curve in a plot
\colorlet{penColor4}{green!50!black} % Color of a curve in a plot
\colorlet{penColor5}{orange!80!black} % Color of a curve in a plot
\colorlet{fill1}{penColor!20} % Color of fill in a plot
\colorlet{fill2}{penColor2!20} % Color of fill in a plot
\colorlet{fillp}{fill1} % Color of positive area
\colorlet{filln}{penColor2!20} % Color of negative area
\colorlet{fill3}{penColor3!20} % Fill
\colorlet{fill4}{penColor4!20} % Fill
\colorlet{fill5}{penColor5!20} % Fill
\colorlet{gridColor}{gray!50} % Color of grid in a plot

\newcommand{\surfaceColor}{violet}
\newcommand{\surfaceColorTwo}{redyellow}
\newcommand{\sliceColor}{greenyellow}




\pgfmathdeclarefunction{gauss}{2}{% gives gaussian
  \pgfmathparse{1/(#2*sqrt(2*pi))*exp(-((x-#1)^2)/(2*#2^2))}%
}


%%%%%%%%%%%%%
%% Vectors
%%%%%%%%%%%%%

%% Simple horiz vectors
\renewcommand{\vector}[1]{\left\langle #1\right\rangle}


%% %% Complex Horiz Vectors with angle brackets
%% \makeatletter
%% \renewcommand{\vector}[2][ , ]{\left\langle%
%%   \def\nextitem{\def\nextitem{#1}}%
%%   \@for \el:=#2\do{\nextitem\el}\right\rangle%
%% }
%% \makeatother

%% %% Vertical Vectors
%% \def\vector#1{\begin{bmatrix}\vecListA#1,,\end{bmatrix}}
%% \def\vecListA#1,{\if,#1,\else #1\cr \expandafter \vecListA \fi}

%%%%%%%%%%%%%
%% End of vectors
%%%%%%%%%%%%%

%\newcommand{\fullwidth}{}
%\newcommand{\normalwidth}{}



%% makes a snazzy t-chart for evaluating functions
%\newenvironment{tchart}{\rowcolors{2}{}{background!90!textColor}\array}{\endarray}

%%This is to help with formatting on future title pages.
\newenvironment{sectionOutcomes}{}{} 



%% Flowchart stuff
%\tikzstyle{startstop} = [rectangle, rounded corners, minimum width=3cm, minimum height=1cm,text centered, draw=black]
%\tikzstyle{question} = [rectangle, minimum width=3cm, minimum height=1cm, text centered, draw=black]
%\tikzstyle{decision} = [trapezium, trapezium left angle=70, trapezium right angle=110, minimum width=3cm, minimum height=1cm, text centered, draw=black]
%\tikzstyle{question} = [rectangle, rounded corners, minimum width=3cm, minimum height=1cm,text centered, draw=black]
%\tikzstyle{process} = [rectangle, minimum width=3cm, minimum height=1cm, text centered, draw=black]
%\tikzstyle{decision} = [trapezium, trapezium left angle=70, trapezium right angle=110, minimum width=3cm, minimum height=1cm, text centered, draw=black]


\title[Dig-In:]{Trigonometric integrals}

\begin{document}
\begin{abstract}
	We can use substitution and trig to integrate trigonometric polynomials
\end{abstract}
\maketitle

Here are some strategies for dealing with expressions of the following form:

\[
\int \sin^n(x)\cos^m(x) \d x
\]

Functions like these, consisting of products of the sine and cosine function, can be
integrated by using substitution and trigonometric identities. These
can sometimes be tedious, but the technique is straightforward. The
basic idea in each case is to somehow take advantage of a
trigonometric identity, usually:
\[
\cos^2(x) + \sin^2(x) = 1, \qquad \sin^2(x) = \frac{1-\cos(2x)}{2}, \qquad \cos^2(x)= \frac{1+\cos(2x)}{2}.
\]

\section{If either power is odd}

\begin{example}
Consider an integral like 

\[
\int \sin^5(x) \cos^2(x) \d x
\]

Since the power of sine is odd, we can rewrite this suggestively as

\[
\int \sin^4(x) \cos^2(x) (\sin(x) \d x) = \int (\sin^2)^2 \cos^2(x) (\sin(x) \d x)
\]

Now using the Pythagorean identity we can rewrite everything in terms of $\cos(x)$, 
but reserve the one power of $\sin(x)$ to help with the differential in an eventual variable substitution.

\[
\int (1-\cos^2(x))^2 \cos^2(x) (\sin(x) \d x)
\]


Making the substitution $u = \cos(x)$ yields

\[
\int \answer{-(1-u^2)^2 u^2} \d u
\]

This is a polynomial in $u$, and so we can easily finish the integration by just expanding the polynomial out and integrating term by term.
\end{example}

We can generalize this thinking:

\begin{explanation}[Odd power of sine]
If we are integrating a product of powers of sine and cosine functions, and the power of sine is odd, then the function looks like

\[
\int \sin^{2k+1}(x) \cos^m(x) \d x
\]

where $k$ is some integer. 

Then

\[
\int \sin^{2k+1}(x) \cos^m(x) \d x = \int (1-\cos^2(x))^k \cos^m(x) \sin(x)\d x 
\]

So letting $u = \cos(x)$, so $\d u =-\sin(x) \d x$, we have

\[
\int -(1-u^2)^k u^m \d u
\]

which is a polynomial, and so can be integrated term by term after expansion.
\end{explanation}

\begin{explanation}[Odd power of cosine]
We can follow exactly the same line of thinking in this case.

\begin{question}
	After making the substitution $u = \sin(x)$, the integral
	
	\[
	\int \sin^n(x) \cos^{2k+1}(x) \d x
	\]
	
	becomes
	
	\[
	\int \answer{ u^n(1-u^2)^k} \d u
	\]
	
	\begin{hint}
		\begin{align*}
			\int \sin^n(x) \cos^{2k+1}(x) \d x &= \int \sin^n(x) \cos^{2k}(x) \cos(x) \d x\\
				&=\int \sin^n(x) (\cos^2(x))^k \cos(x)\d x\\
				&=\int \sin^n(x)(1-\sin^2(x))^k \cos(x) \d x\\
				&=\int u^n(1-u^2)^k \d u
		\end{align*}
	\end{hint}
\end{question}
\end{explanation}

\begin{warning}
Instead of memorizing this general result, you should focus on internalizing the technique by which the result was obtained.  In each case, you take off one power from an odd power of the trig function, and you are left with an even power of that trig function.  You can then use a Pythagorean identity to rewrite that even power in terms of the other trig function.  Making a variable substitution finishes the conversion to a polynomial function in $u$.
\end{warning}

\begin{example}
Compute

\[
\int \sin^5 x \d x = \answer{-\cos x+\frac{2}{3}\cos^3 x-\frac{1}{5}\cos^5x} + C
\]

\begin{hint}
Rewrite the function:
\[
  \int \sin^5 x\d x=\int \sin x \sin^4 x\d x=
  \int \sin x (\sin^2 x)^2\d x=
  \int \sin x (1-\cos^2 x)^2\d x.
\]
\end{hint}
\begin{hint}
Now use $u=\cos x$, $du=-\sin x\d x$:
\begin{align*}
  \int \sin x (1-\cos^2 x)^2\d x&=\int -(1-u^2)^2\d u \\
  &=\int -(1-2u^2+u^4)\d u \\
  &=-u+{2\over3}u^3-{1\over5}u^5+C \\
  &=-\cos x+{2\over3}\cos^3 x-{1\over5}\cos^5x+C. 
\end{align*}
\end{hint}

\end{example}

\section{When both powers are even}

What about when both powers are even?  Our strategy above is sunk, since peeling off one power leaves an odd power, and we cannot use the Pythagorean identity to rewrite this in a nice way.  Instead, we will use the half angle reduction formulae.

\begin{example}
Compute 
\[
\int \sin^2x\cos^2x\d x.
\]
\begin{explanation} 
Use the formulas
$\sin^2x =(1-\cos(2x))/2$ and $\cos^2x =(1+\cos(2x))/2$ to get:
\[
  \int \sin^2x\cos^2x\d x=\int \frac{1-\cos(2x)}{2}\cdot
  \frac{1+\cos(2x)}{2}\d x
\]

From here, we can try to simplify further.  We may end up having to use either of these strategies multiple times.  In this case

\begin{align*}
	\int \frac{1-\cos(2x)}{2} \cdot \frac{1+\cos(2x)}{2}\d x &= \frac{1}{4} \int 1-\cos^2(2x) \d x
		&= \frac{1}{4} \int \sin^2(2x) \d x
\end{align*}

Since this expression has only even powers of sine, we must use the same strategy again, and employ a half angle identity:

\[
\sin^2(2x) = \frac{1-\cos(\answer{4x})}{2}
\]

so we get

\[
\frac{1}{8}\int 1-\cos(4x) \d x = \answer{\frac{1}{8}(x - \frac{1}{4}\sin(4x))} + C
\]
\end{explanation}

So when confronted with even powers of both sine and cosine, we employ the half angle identities, simplify, and see what we end up with.  If we get some odd powered terms, we can use that strategy for those parts.  If we end up with both even powers again, we might have to use this strategy multiple times.  This can get tricky algebraically, but if you are careful in your algebra you will eventually emerge victorious.
\end{example}

\end{document}

