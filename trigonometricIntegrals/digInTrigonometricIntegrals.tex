\documentclass{ximera}

%\usepackage{todonotes}

\newcommand{\todo}{}

\usepackage{tkz-euclide}
\tikzset{>=stealth} %% cool arrow head
\tikzset{shorten <>/.style={ shorten >=#1, shorten <=#1 } } %% allows shorter vectors

\usetikzlibrary{backgrounds} %% for boxes around graphs
\usetikzlibrary{shapes,positioning}  %% Clouds and stars
\usetikzlibrary{matrix} %% for matrix
\usepgfplotslibrary{polar} %% for polar plots
\usetkzobj{all}
\usepackage[makeroom]{cancel} %% for strike outs
%\usepackage{mathtools} %% for pretty underbrace % Breaks Ximera
\usepackage{multicol}





\usepackage{array}
\setlength{\extrarowheight}{+.1cm}   
\newdimen\digitwidth
\settowidth\digitwidth{9}
\def\divrule#1#2{
\noalign{\moveright#1\digitwidth
\vbox{\hrule width#2\digitwidth}}}





\newcommand{\RR}{\mathbb R}
\newcommand{\R}{\mathbb R}
\newcommand{\N}{\mathbb N}
\newcommand{\Z}{\mathbb Z}

%\renewcommand{\d}{\,d\!}
\renewcommand{\d}{\mathop{}\!d}
\newcommand{\dd}[2][]{\frac{\d #1}{\d #2}}
\renewcommand{\l}{\ell}
\newcommand{\ddx}{\frac{d}{\d x}}

\newcommand{\zeroOverZero}{\ensuremath{\boldsymbol{\tfrac{0}{0}}}}
\newcommand{\inftyOverInfty}{\ensuremath{\boldsymbol{\tfrac{\infty}{\infty}}}}
\newcommand{\zeroOverInfty}{\ensuremath{\boldsymbol{\tfrac{0}{\infty}}}}
\newcommand{\zeroTimesInfty}{\ensuremath{\small\boldsymbol{0\cdot \infty}}}
\newcommand{\inftyMinusInfty}{\ensuremath{\small\boldsymbol{\infty - \infty}}}
\newcommand{\oneToInfty}{\ensuremath{\boldsymbol{1^\infty}}}
\newcommand{\zeroToZero}{\ensuremath{\boldsymbol{0^0}}}
\newcommand{\inftyToZero}{\ensuremath{\boldsymbol{\infty^0}}}


\newcommand{\numOverZero}{\ensuremath{\boldsymbol{\tfrac{\#}{0}}}}
\newcommand{\dfn}{\textbf}
%\newcommand{\unit}{\,\mathrm}
\newcommand{\unit}{\mathop{}\!\mathrm}
\newcommand{\eval}[1]{\bigg[ #1 \bigg]}
\newcommand{\seq}[1]{\left( #1 \right)}
\renewcommand{\epsilon}{\varepsilon}
\renewcommand{\iff}{\Leftrightarrow}

\DeclareMathOperator{\arccot}{arccot}
\DeclareMathOperator{\arcsec}{arcsec}
\DeclareMathOperator{\arccsc}{arccsc}
\DeclareMathOperator{\si}{Si}

\newcommand{\tightoverset}[2]{%
  \mathop{#2}\limits^{\vbox to -.5ex{\kern-0.75ex\hbox{$#1$}\vss}}}
\newcommand{\arrowvec}[1]{\tightoverset{\scriptstyle\rightharpoonup}{#1}}
\renewcommand{\vec}{\mathbf}


\colorlet{textColor}{black} 
\colorlet{background}{white}
\colorlet{penColor}{blue!50!black} % Color of a curve in a plot
\colorlet{penColor2}{red!50!black}% Color of a curve in a plot
\colorlet{penColor3}{red!50!blue} % Color of a curve in a plot
\colorlet{penColor4}{green!50!black} % Color of a curve in a plot
\colorlet{penColor5}{orange!80!black} % Color of a curve in a plot
\colorlet{fill1}{penColor!20} % Color of fill in a plot
\colorlet{fill2}{penColor2!20} % Color of fill in a plot
\colorlet{fillp}{fill1} % Color of positive area
\colorlet{filln}{penColor2!20} % Color of negative area
\colorlet{fill3}{penColor3!20} % Fill
\colorlet{fill4}{penColor4!20} % Fill
\colorlet{fill5}{penColor5!20} % Fill
\colorlet{gridColor}{gray!50} % Color of grid in a plot

\newcommand{\surfaceColor}{violet}
\newcommand{\surfaceColorTwo}{redyellow}
\newcommand{\sliceColor}{greenyellow}




\pgfmathdeclarefunction{gauss}{2}{% gives gaussian
  \pgfmathparse{1/(#2*sqrt(2*pi))*exp(-((x-#1)^2)/(2*#2^2))}%
}


%%%%%%%%%%%%%
%% Vectors
%%%%%%%%%%%%%

%% Simple horiz vectors
\renewcommand{\vector}[1]{\left\langle #1\right\rangle}


%% %% Complex Horiz Vectors with angle brackets
%% \makeatletter
%% \renewcommand{\vector}[2][ , ]{\left\langle%
%%   \def\nextitem{\def\nextitem{#1}}%
%%   \@for \el:=#2\do{\nextitem\el}\right\rangle%
%% }
%% \makeatother

%% %% Vertical Vectors
%% \def\vector#1{\begin{bmatrix}\vecListA#1,,\end{bmatrix}}
%% \def\vecListA#1,{\if,#1,\else #1\cr \expandafter \vecListA \fi}

%%%%%%%%%%%%%
%% End of vectors
%%%%%%%%%%%%%

%\newcommand{\fullwidth}{}
%\newcommand{\normalwidth}{}



%% makes a snazzy t-chart for evaluating functions
%\newenvironment{tchart}{\rowcolors{2}{}{background!90!textColor}\array}{\endarray}

%%This is to help with formatting on future title pages.
\newenvironment{sectionOutcomes}{}{} 



%% Flowchart stuff
%\tikzstyle{startstop} = [rectangle, rounded corners, minimum width=3cm, minimum height=1cm,text centered, draw=black]
%\tikzstyle{question} = [rectangle, minimum width=3cm, minimum height=1cm, text centered, draw=black]
%\tikzstyle{decision} = [trapezium, trapezium left angle=70, trapezium right angle=110, minimum width=3cm, minimum height=1cm, text centered, draw=black]
%\tikzstyle{question} = [rectangle, rounded corners, minimum width=3cm, minimum height=1cm,text centered, draw=black]
%\tikzstyle{process} = [rectangle, minimum width=3cm, minimum height=1cm, text centered, draw=black]
%\tikzstyle{decision} = [trapezium, trapezium left angle=70, trapezium right angle=110, minimum width=3cm, minimum height=1cm, text centered, draw=black]


\title[Dig-In:]{Trigonometric integrals}

\begin{document}
\begin{abstract}
  We can use substitution and trig to integrate trigonometric polynomials
\end{abstract}
\maketitle

Here are some strategies for dealing with expressions of the following form:
\[
\int \sin^n(x)\cos^m(x) \d x
\]
Functions like these, consisting of products of the sine and cosine
function, can be integrated by using substitution and trigonometric
identities. These can sometimes be tedious, but the technique is
straightforward. The basic idea in each case is to somehow take
advantage of a trigonometric identity, usually:
\begin{description}
\item[Pythagorean Identity] $\cos^2(x) + \sin^2(x) = 1$
\item[Sine Power-Reduction] $\sin^2(x) = \frac{1-\cos(2x)}{2}$,
\item[Cosine Power-Reduction] $\cos^2(x)= \frac{1+\cos(2x)}{2}$.
\end{description}

\section{If either power is odd}

\begin{example}
  Compute:
  \[
  \int \sin^5(x) \cos^2(x) \d x
  \]
  \begin{explanation}
    Since the power of sine is odd, we can rewrite this suggestively as
    \[
    \int \sin^4(x) \cos^2(x) (\sin(x) \d x) = \int (\sin^2x)^2 \cos^2(x) (\sin(x) \d x)
    \]
    Now using the Pythagorean identity we can rewrite everything in
    terms of $\cos(x)$, but reserve the one power of $\sin(x)$ to help
    with the differential in an eventual variable substitution.
    \[
    \int (1-\cos^2(x))^2 \cos^2(x) (\sin(x) \d x)
    \]
    Making the substitution
    \begin{align*}
      g &= \cos(x)\\
      \d g &=\answer[given]{-\sin(x)} \d x
    \end{align*}
    yields
    \[
    \int \answer{-(1-g^2)^2 g^2} \d g
    \]
    This is a polynomial in $g$, and so we can easily finish the
    integration by just expanding the polynomial out and integrating
    term by term.
  \end{explanation}
\end{example}

We can generalize this thinking:

\subsection{Odd powers of sine}

If we are integrating a product of powers of sine and cosine
functions, and the power of sine is odd, then the integral looks like
\[
\int \sin^{2k+1}(x) \cos^m(x) \d x
\]
where $k$ is some integer.  Then
\[
\int \sin^{2k+1}(x) \cos^m(x) \d x = \int (1-\cos^2(x))^k \cos^m(x) \sin(x)\d x.
\]
So letting $g = \cos(x)$, $\d g =-\sin(x) \d x$, and we have
\[
\int -(1-g^2)^k g^m \d g
\]
which is a polynomial, and can be integrated term by term after
expansion.


\begin{question}
  Consider the following integral:
  \[
  SOME INTEGRAL!!!
  \]
  After making the substitution, $g = \cos(x)$ we find
  \begin{multipleChoice}
    \choice{SOME CHOICE}
  \end{multipleChoice}
\end{question}



\subsection{Odd powers of cosine}

We can follow exactly the same line of thinking in this case. If we
are integrating a product of powers of sine and cosine, and the power
of cosine is odd, then the integral looks like
\[
\int \sin^n(x) \cos^{2k+1}(x) \d x
\]
where $k$ is some integer. Then
\[
\int \sin^n(x) \cos^{2k+1}(x) \d x = \int \sin^n(x) \cos^{2k}(x) \cos(x) \d x.
\]
So letting $g = \sin(x)$, $\d g =\cos(x) \d x$, and we have 
\[
\int \answer{ g^n(1-g^2)^k} \d g
\]
which is a polynomial, and can be integrated term by term after
expansion.

\begin{question}
  Consider the following integral:
  \[
  SOME INTEGRAL!!!
  \]
  After making the substitution, $g = \sin(x)$ we find
  \begin{multipleChoice}
    \choice{SOME CHOICE}
  \end{multipleChoice}
\end{question}


\begin{warning}
Instead of memorizing this general result, you should focus on
internalizing the technique by which the result was obtained.  In each
case, you take off one power from an odd power of the trig function,
and you are left with an even power of that trig function.  You can
then use a Pythagorean identity to rewrite that even power in terms of
the other trig function.  Making a variable substitution finishes the
conversion to a polynomial function in $g$.
\end{warning}



Sometimes we need to massage our function into the correct form:

\begin{example}
  Compute:
  \[
  \int \sin^5 x \d x
  \]
  \begin{explanation}
    Write with me
    \begin{align*}
      \int \sin^5 x\d x &=\int \sin x \sin^4 x\d x\\
      &= \int \sin x (\sin^2 x)^2\d x\\
      &= \int \sin x (1-\cos^2 x)^2\d x.
    \end{align*}
    Now set $g=\cos x$, $\d g=-\sin x\d x$:
    \begin{align*}
      \int \sin x (1-\cos^2 x)^2\d x&=\int \answer[given]{-(1-g^2)^2} \d g \\
      &=\int -(1-2g^2+g^4)\d g \\
      &=\answer[given]{-g+\frac{2}{3}g^3-\frac{1}{5}g^5}+C \\
      &=\answer[given]{-\cos x+\frac{2}{3}\cos^3 x-\frac{1}{5}\cos^5x}+C 
    \end{align*}
  \end{explanation}
\end{example}
  
\section{When both powers are even}

What about when both powers are even?  Our strategy above is sunk,
since peeling off one power leaves an odd power, and we cannot use the
Pythagorean identity to rewrite this in a nice way.  Instead, we will
use the power-reduction formulas.

\begin{example}
  Compute:
  \[
  \int \sin^2x\cos^2x\d x
  \]
  \begin{explanation} 
    Use both power-reduction formulas
    \begin{itemize}
    \item $\sin^2x =(1-\cos(2x))/2$ and
    \item $\cos^2x =(1+\cos(2x))/2$,
    \end{itemize}
    to get:
    \[
    \int \sin^2x\cos^2x\d x=\int \frac{1-\cos(2x)}{2}\cdot
    \frac{1+\cos(2x)}{2}\d x
    \]
    From here, we can try to simplify further.  We may end up having
    to use either of these strategies multiple times.  In this case
    \begin{align*}
      \int \frac{1-\cos(2x)}{2} \cdot \frac{1+\cos(2x)}{2}\d x &= \frac{1}{4} \int 1-\cos^2(2x) \d x
      &= \frac{1}{4} \int \sin^2(2x) \d x
    \end{align*}
    
    Since this expression has only even powers of sine, we must use
    the same strategy again, and employ a power-reduction formula:
    \[
    \sin^2(2x) = \frac{1-\cos(\answer{4x})}{2}
    \]
    so we get
    \[
    \frac{1}{8}\int 1-\cos(4x) \d x = \answer{\frac{1}{8}(x - \frac{1}{4}\sin(4x))} + C
    \]
  \end{explanation}
\end{example}

\subsection{Summary}

So when confronted with even powers of both sine and cosine, we employ
the half angle identities, simplify, and see what we end up with.  If
we get some odd powered terms, we can use that strategy for those
parts.  If we end up with both even powers again, we might have to use
this strategy multiple times.  This can get tricky algebraically, but
if you are careful in your algebra you will eventually emerge
victorious.


\end{document}

