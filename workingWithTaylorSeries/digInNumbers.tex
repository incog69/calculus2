\documentclass{ximera}

%\usepackage{todonotes}

\newcommand{\todo}{}

\usepackage{tkz-euclide}
\tikzset{>=stealth} %% cool arrow head
\tikzset{shorten <>/.style={ shorten >=#1, shorten <=#1 } } %% allows shorter vectors

\usetikzlibrary{backgrounds} %% for boxes around graphs
\usetikzlibrary{shapes,positioning}  %% Clouds and stars
\usetikzlibrary{matrix} %% for matrix
\usepgfplotslibrary{polar} %% for polar plots
\usetkzobj{all}
\usepackage[makeroom]{cancel} %% for strike outs
%\usepackage{mathtools} %% for pretty underbrace % Breaks Ximera
\usepackage{multicol}





\usepackage{array}
\setlength{\extrarowheight}{+.1cm}   
\newdimen\digitwidth
\settowidth\digitwidth{9}
\def\divrule#1#2{
\noalign{\moveright#1\digitwidth
\vbox{\hrule width#2\digitwidth}}}





\newcommand{\RR}{\mathbb R}
\newcommand{\R}{\mathbb R}
\newcommand{\N}{\mathbb N}
\newcommand{\Z}{\mathbb Z}

%\renewcommand{\d}{\,d\!}
\renewcommand{\d}{\mathop{}\!d}
\newcommand{\dd}[2][]{\frac{\d #1}{\d #2}}
\renewcommand{\l}{\ell}
\newcommand{\ddx}{\frac{d}{\d x}}

\newcommand{\zeroOverZero}{\ensuremath{\boldsymbol{\tfrac{0}{0}}}}
\newcommand{\inftyOverInfty}{\ensuremath{\boldsymbol{\tfrac{\infty}{\infty}}}}
\newcommand{\zeroOverInfty}{\ensuremath{\boldsymbol{\tfrac{0}{\infty}}}}
\newcommand{\zeroTimesInfty}{\ensuremath{\small\boldsymbol{0\cdot \infty}}}
\newcommand{\inftyMinusInfty}{\ensuremath{\small\boldsymbol{\infty - \infty}}}
\newcommand{\oneToInfty}{\ensuremath{\boldsymbol{1^\infty}}}
\newcommand{\zeroToZero}{\ensuremath{\boldsymbol{0^0}}}
\newcommand{\inftyToZero}{\ensuremath{\boldsymbol{\infty^0}}}


\newcommand{\numOverZero}{\ensuremath{\boldsymbol{\tfrac{\#}{0}}}}
\newcommand{\dfn}{\textbf}
%\newcommand{\unit}{\,\mathrm}
\newcommand{\unit}{\mathop{}\!\mathrm}
\newcommand{\eval}[1]{\bigg[ #1 \bigg]}
\newcommand{\seq}[1]{\left( #1 \right)}
\renewcommand{\epsilon}{\varepsilon}
\renewcommand{\iff}{\Leftrightarrow}

\DeclareMathOperator{\arccot}{arccot}
\DeclareMathOperator{\arcsec}{arcsec}
\DeclareMathOperator{\arccsc}{arccsc}
\DeclareMathOperator{\si}{Si}

\newcommand{\tightoverset}[2]{%
  \mathop{#2}\limits^{\vbox to -.5ex{\kern-0.75ex\hbox{$#1$}\vss}}}
\newcommand{\arrowvec}[1]{\tightoverset{\scriptstyle\rightharpoonup}{#1}}
\renewcommand{\vec}{\mathbf}


\colorlet{textColor}{black} 
\colorlet{background}{white}
\colorlet{penColor}{blue!50!black} % Color of a curve in a plot
\colorlet{penColor2}{red!50!black}% Color of a curve in a plot
\colorlet{penColor3}{red!50!blue} % Color of a curve in a plot
\colorlet{penColor4}{green!50!black} % Color of a curve in a plot
\colorlet{penColor5}{orange!80!black} % Color of a curve in a plot
\colorlet{fill1}{penColor!20} % Color of fill in a plot
\colorlet{fill2}{penColor2!20} % Color of fill in a plot
\colorlet{fillp}{fill1} % Color of positive area
\colorlet{filln}{penColor2!20} % Color of negative area
\colorlet{fill3}{penColor3!20} % Fill
\colorlet{fill4}{penColor4!20} % Fill
\colorlet{fill5}{penColor5!20} % Fill
\colorlet{gridColor}{gray!50} % Color of grid in a plot

\newcommand{\surfaceColor}{violet}
\newcommand{\surfaceColorTwo}{redyellow}
\newcommand{\sliceColor}{greenyellow}




\pgfmathdeclarefunction{gauss}{2}{% gives gaussian
  \pgfmathparse{1/(#2*sqrt(2*pi))*exp(-((x-#1)^2)/(2*#2^2))}%
}


%%%%%%%%%%%%%
%% Vectors
%%%%%%%%%%%%%

%% Simple horiz vectors
\renewcommand{\vector}[1]{\left\langle #1\right\rangle}


%% %% Complex Horiz Vectors with angle brackets
%% \makeatletter
%% \renewcommand{\vector}[2][ , ]{\left\langle%
%%   \def\nextitem{\def\nextitem{#1}}%
%%   \@for \el:=#2\do{\nextitem\el}\right\rangle%
%% }
%% \makeatother

%% %% Vertical Vectors
%% \def\vector#1{\begin{bmatrix}\vecListA#1,,\end{bmatrix}}
%% \def\vecListA#1,{\if,#1,\else #1\cr \expandafter \vecListA \fi}

%%%%%%%%%%%%%
%% End of vectors
%%%%%%%%%%%%%

%\newcommand{\fullwidth}{}
%\newcommand{\normalwidth}{}



%% makes a snazzy t-chart for evaluating functions
%\newenvironment{tchart}{\rowcolors{2}{}{background!90!textColor}\array}{\endarray}

%%This is to help with formatting on future title pages.
\newenvironment{sectionOutcomes}{}{} 



%% Flowchart stuff
%\tikzstyle{startstop} = [rectangle, rounded corners, minimum width=3cm, minimum height=1cm,text centered, draw=black]
%\tikzstyle{question} = [rectangle, minimum width=3cm, minimum height=1cm, text centered, draw=black]
%\tikzstyle{decision} = [trapezium, trapezium left angle=70, trapezium right angle=110, minimum width=3cm, minimum height=1cm, text centered, draw=black]
%\tikzstyle{question} = [rectangle, rounded corners, minimum width=3cm, minimum height=1cm,text centered, draw=black]
%\tikzstyle{process} = [rectangle, minimum width=3cm, minimum height=1cm, text centered, draw=black]
%\tikzstyle{decision} = [trapezium, trapezium left angle=70, trapezium right angle=110, minimum width=3cm, minimum height=1cm, text centered, draw=black]


\outcome{Use Taylor series to approximate constants (like e and pi).}
\outcome{Use Taylor series to compute limits.}
\outcome{Use Taylor series to sum series.}

\title[Dig-In:]{Numbers and Taylor Series}

\begin{document}
\begin{abstract}
  Taylor series are a new computational tool
\end{abstract}
\maketitle

\section{Using series to approximate special constants}

You probably know that $\pi  \approx 3.14159$.  Have you ever wondered how this kind of approximation was obtained?  There are many ways to do it, but one exciting way is to use Taylor series!

\begin{example}
	Here is a plan for approximating $\pi$ using a series:
		\begin{itemize}
			\item Find a function which takes a nice number (like $0$, or $1$, or $\frac{1}{2}$) as an input and returns something involving $\pi$ (like $\pi$, or $\frac{\pi}{2}$, or $\pi^2$) as an output.
			\item Make sure that this function has a Taylor series which we can compute easily.
			\item Plug the nice number into the Taylor series:  we now have an algorithm for approximating $\pi$
		\end{itemize} 
		
The most convenient functions which return multiples of $\pi$ are the inverse trig functions.  For instance, $\arcsin(1) = \frac{\pi}{2}$, and $\arctan(1) = \frac{\pi}{4}$.  

If you recall, we found that Taylor series for $\arctan$ already, by using substitution of the geometric series $\frac{1}{1-u}$ to find the series of $\frac{1}{1+x^2}$, and then integrating this series to find the series for $\arctan(x)$:


\[
\arctan(x) = x-\frac{x^3}{3}+\frac{x^5}{5}-\frac{x^7}{7}+\dots
\]

And so 

\[
\frac{\pi}{4} = \arctan(1) = 1-\frac{1}{3}+\frac{1}{5}-\frac{1}{7}+\dots
\]

Thus

\[
\pi =  4-\frac{4}{3}+\frac{4}{5}-\frac{4}{7}+\dots
\]

Really cool!

Say we wanted to get an approximation of  $\pi$ using this series which was accurate to within $\frac{1}{100}$.  How many terms would we need?

Since this is an alternating series, we can use the alternating series estimation test.

We need

\[
\frac{1}{2k+1} \leq \frac{1}{100}
\]

Which means we need

\[
k \geq 50
\]

So we would need to sum $50$ terms just to get the $3.14$ approximation! (Even then, you have to be careful not to accumulate too many rounding errors when performing the sum).

Part of the reason this converges so slowly is that the series is not absolutely convergent.  Also $1$ is the right endpoint of the interval of convergence for $\arctan(x)$:  it just barely makes the cut between convergent and not convergent!

There are more advanced series for $\pi$ which converge much more quickly, for example Ramanujan's formula

\[
\frac{1}{\pi} = \frac{2 \sqrt 2}{99^2} \sum_{k=0}^\infty \frac{(4k)!}{k!^4} \frac{26390k+1103}{396^{4k}}
\]
		
\end{example}

\begin{example}

	You already know one way to compute the number $e$:  as the $\lim_{n \to \infty} \left( 1+\frac{1}{n}\right)^n$.
	
	Now we can also approximate $e$ using series as follows:
		
	Since 
	
	\[
	e^x = 1+x+\frac{x^2}{2!}+\frac{x^3}{3!}+\dots
	\]
	
	we have that
	
	\[
	e = 1+1+\frac{1}{2!}+\frac{1}{3!}+\dots
	\]
\end{example}

\section{Taylor series as a tool to evaluate limits}

Taylor series can be used like l'H\^{o}pital's rule on steroids when evaluating limits.

First lets see why Taylor's series subsumes l'H\^{o}pital's rule:

Say $f(a) = g(a) = 0$, and we are interested in $\lim_{x \to a} \frac{f(x)}{g(x)}$.

Then using Taylor series

\begin{align*}
	\lim_{x \to a} \frac{f(x)}{g(x)} &= \lim_{x \to a} \frac{f(a)+f'(a)(x-a)+\frac{f''(a)}{2!}(x-a)^2+...}{g(a)+g'(a)(x-a)+\frac{g''(a)}{2!}(x-a)^2+...}\\
		&=  \lim_{x \to a} \frac{0+f'(a)(x-a)+\frac{f''(a)}{2!}(x-a)^2+...}{0+g'(a)(x-a)+\frac{g''(a)}{2!}(x-a)^2+...}\\
		&=  \lim_{x \to a} \frac{f'(a)+\frac{f''(a)}{2!}(x-a)+\frac{f^{(3)}(a)}{3!}(x-a)^2+...}{g'(a)+\frac{g''(a)}{2!}(x-a)+\frac{g^{(3)}(a)}{3!}(x-a)^2+...}\\
		&=\frac{f'(a)}{g'(a)}
\end{align*}

As long as $g'(a) \neq 0$.   This is exactly l'H\^{o}pital's rule!

Let's use this in a l'H\^{o}pital's rule situation, without invoking l'H\^{o}pital's rule directly:

\begin{example}
	
	\begin{align*}
	\lim_{x \to 0} \frac{e^x - 1}{x} &= \lim_{x \to 0} \frac{(1+x+\frac{x^2}{2!}+\frac{x^3}{3!}+\dots) - 1}{x}\\
		&= \lim_{x \to 0} \frac{x+\frac{x^2}{2!} +\frac{x^3}{3!}+\dots}{x}\\
		&=\lim_{x \to 0} 1+\frac{x}{2!}+\frac{x^2}{3!}+\dots\\
		&=\answer{1}
	\end{align*}
		
\end{example}

We can also use this approach to limit evaluation in cases where l'H\^{o}pital's rule would need to be applied multiple times:

\begin{question}
	Use Taylor series to evaluate the limit:
	
	\[
	\lim_{x \to 0} \frac{\sin(x)-x}{x^3\cos(x)} = \answer{-\frac{1}{6}}
	\]
	
	\begin{hint}
		\begin{align*}
			\lim_{x \to 0} \frac{\sin(x)-x}{x\cos(x)} &= \lim_{x \to 0} \frac{(x-\frac{x^3}{3!}+\frac{x^5}{5!}-\dots)-x}{x^3(1-\frac{x^2}{2!}+\frac{x^4}{4!}-\dots)}\\
				&= \lim_{x \to \infty} \frac{-\frac{x^3}{3!}+\frac{x^5}{5!}-\dots}{x^3 - \frac{x^5}{2!}+\frac{x^7}{4!}-\dots}\\
				&=\lim_{x \to \infty} \frac{-\frac{1}{6}+\frac{x^2}{3!}-\dots}{1-\frac{x^2}{2!}+\frac{x^4}{4!}-\dots}\\
				&=-\frac{1}{6}
		\end{align*}
	\end{hint}
\end{question}


It doesn't seem like Taylor series would be much help evaluating limits at infinity, since Taylor series are all about approximating a function close to some given point.  It turns out that we can still use Taylor series to study function behavior at infinity by transforming the function:  composing with $\frac{1}{u}$ ``moves infinity to zero'', and we can then use Maclaurin series to study the behavior.  Let's see that in action:

\begin{example}
	We want to evaluate
	
	\[
	\lim_{x \to \infty} 2x^8\cos(\frac{1}{x^2}) - x^4-2x^8
	\]
	
	Letting $t  =\frac{1}{x}$ we have
	
	\[
	\lim_{t \to 0} \frac{2\cos(t^2) - t^4-2}{t^8}
	\]
	
	You should be able to handle it from here using the Taylor series method:
	
	
	\[
	\lim_{t \to 0} \frac{2\cos(t^2) - t^4-2}{t^8} = \answer{\frac{1}{12}}
	\]
	
	\begin{hint}
		\begin{align*}
		\lim_{t \to 0} \frac{2\cos(t^2) - t^4-2}{t^8} &= \lim_{t \to 0} \frac{2(1-\frac{t^4}{2!}+\frac{t^8}{4!}-\frac{t^12}{6!}+\dots) - t^4-2}{t^8}\\
			&=  \lim_{t \to 0} \frac{\frac{2t^8}{4!}-\frac{2t^12}{12!}+\dots}{t^8}\\
			&=\lim_{t \to 0} \frac{2}{4!}-\frac{2t^4}{6!} +\dots\\
			&=\frac{2}{4!}\\
			&=\frac{1}{12}
		\end{align*}
	\end{hint}
	
\end{example}

\section{Evaluating series}

Sometimes we get a series as an answer to some problem (For instance, in the next section we will find series solutions to differential equations), but we would really like a closed form solution.

This is not always possible, but sometimes if we are insightful we can manipulate a given series into form where we can recognize it.

\begin{example}
	
	What is a closed form expression for the following series?
		
	\[
	f(x) = \sum_1^\infty nx^{n-1} = \answer{\frac{1}{(1-x^2)}}
	\]
	
	\begin{hint}
		What happens if you integrate?
	\end{hint}
	
	\begin{hint}
		\[
		\int f(x) \d x = C+  \sum_1^\infty x^n
		\]
		
		by integrating term by term, for some $C$.
	\end{hint}
	
	\begin{hint}
		Choosing $C = 1$, the RHS becomes the geometric series with starting value $1$ and ratio $x$, and so sums to $\frac{1}{1-x}$ 
	\end{hint}
	
	\begin{hint}
		Thus $f(x) = \frac{d}{dx} \frac{1}{1-x}$
	\end{hint}
	
	\begin{hint}
		So $f(x) = \frac{1}{(1-x^2)}$
	\end{hint}
	

\end{example}



\end{document}
