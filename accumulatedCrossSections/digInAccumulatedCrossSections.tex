\documentclass{ximera}

%\usepackage{todonotes}

\newcommand{\todo}{}

\usepackage{tkz-euclide}
\tikzset{>=stealth} %% cool arrow head
\tikzset{shorten <>/.style={ shorten >=#1, shorten <=#1 } } %% allows shorter vectors

\usetikzlibrary{backgrounds} %% for boxes around graphs
\usetikzlibrary{shapes,positioning}  %% Clouds and stars
\usetikzlibrary{matrix} %% for matrix
\usepgfplotslibrary{polar} %% for polar plots
\usetkzobj{all}
\usepackage[makeroom]{cancel} %% for strike outs
%\usepackage{mathtools} %% for pretty underbrace % Breaks Ximera
\usepackage{multicol}





\usepackage{array}
\setlength{\extrarowheight}{+.1cm}   
\newdimen\digitwidth
\settowidth\digitwidth{9}
\def\divrule#1#2{
\noalign{\moveright#1\digitwidth
\vbox{\hrule width#2\digitwidth}}}





\newcommand{\RR}{\mathbb R}
\newcommand{\R}{\mathbb R}
\newcommand{\N}{\mathbb N}
\newcommand{\Z}{\mathbb Z}

%\renewcommand{\d}{\,d\!}
\renewcommand{\d}{\mathop{}\!d}
\newcommand{\dd}[2][]{\frac{\d #1}{\d #2}}
\renewcommand{\l}{\ell}
\newcommand{\ddx}{\frac{d}{\d x}}

\newcommand{\zeroOverZero}{\ensuremath{\boldsymbol{\tfrac{0}{0}}}}
\newcommand{\inftyOverInfty}{\ensuremath{\boldsymbol{\tfrac{\infty}{\infty}}}}
\newcommand{\zeroOverInfty}{\ensuremath{\boldsymbol{\tfrac{0}{\infty}}}}
\newcommand{\zeroTimesInfty}{\ensuremath{\small\boldsymbol{0\cdot \infty}}}
\newcommand{\inftyMinusInfty}{\ensuremath{\small\boldsymbol{\infty - \infty}}}
\newcommand{\oneToInfty}{\ensuremath{\boldsymbol{1^\infty}}}
\newcommand{\zeroToZero}{\ensuremath{\boldsymbol{0^0}}}
\newcommand{\inftyToZero}{\ensuremath{\boldsymbol{\infty^0}}}


\newcommand{\numOverZero}{\ensuremath{\boldsymbol{\tfrac{\#}{0}}}}
\newcommand{\dfn}{\textbf}
%\newcommand{\unit}{\,\mathrm}
\newcommand{\unit}{\mathop{}\!\mathrm}
\newcommand{\eval}[1]{\bigg[ #1 \bigg]}
\newcommand{\seq}[1]{\left( #1 \right)}
\renewcommand{\epsilon}{\varepsilon}
\renewcommand{\iff}{\Leftrightarrow}

\DeclareMathOperator{\arccot}{arccot}
\DeclareMathOperator{\arcsec}{arcsec}
\DeclareMathOperator{\arccsc}{arccsc}
\DeclareMathOperator{\si}{Si}

\newcommand{\tightoverset}[2]{%
  \mathop{#2}\limits^{\vbox to -.5ex{\kern-0.75ex\hbox{$#1$}\vss}}}
\newcommand{\arrowvec}[1]{\tightoverset{\scriptstyle\rightharpoonup}{#1}}
\renewcommand{\vec}{\mathbf}


\colorlet{textColor}{black} 
\colorlet{background}{white}
\colorlet{penColor}{blue!50!black} % Color of a curve in a plot
\colorlet{penColor2}{red!50!black}% Color of a curve in a plot
\colorlet{penColor3}{red!50!blue} % Color of a curve in a plot
\colorlet{penColor4}{green!50!black} % Color of a curve in a plot
\colorlet{penColor5}{orange!80!black} % Color of a curve in a plot
\colorlet{fill1}{penColor!20} % Color of fill in a plot
\colorlet{fill2}{penColor2!20} % Color of fill in a plot
\colorlet{fillp}{fill1} % Color of positive area
\colorlet{filln}{penColor2!20} % Color of negative area
\colorlet{fill3}{penColor3!20} % Fill
\colorlet{fill4}{penColor4!20} % Fill
\colorlet{fill5}{penColor5!20} % Fill
\colorlet{gridColor}{gray!50} % Color of grid in a plot

\newcommand{\surfaceColor}{violet}
\newcommand{\surfaceColorTwo}{redyellow}
\newcommand{\sliceColor}{greenyellow}




\pgfmathdeclarefunction{gauss}{2}{% gives gaussian
  \pgfmathparse{1/(#2*sqrt(2*pi))*exp(-((x-#1)^2)/(2*#2^2))}%
}


%%%%%%%%%%%%%
%% Vectors
%%%%%%%%%%%%%

%% Simple horiz vectors
\renewcommand{\vector}[1]{\left\langle #1\right\rangle}


%% %% Complex Horiz Vectors with angle brackets
%% \makeatletter
%% \renewcommand{\vector}[2][ , ]{\left\langle%
%%   \def\nextitem{\def\nextitem{#1}}%
%%   \@for \el:=#2\do{\nextitem\el}\right\rangle%
%% }
%% \makeatother

%% %% Vertical Vectors
%% \def\vector#1{\begin{bmatrix}\vecListA#1,,\end{bmatrix}}
%% \def\vecListA#1,{\if,#1,\else #1\cr \expandafter \vecListA \fi}

%%%%%%%%%%%%%
%% End of vectors
%%%%%%%%%%%%%

%\newcommand{\fullwidth}{}
%\newcommand{\normalwidth}{}



%% makes a snazzy t-chart for evaluating functions
%\newenvironment{tchart}{\rowcolors{2}{}{background!90!textColor}\array}{\endarray}

%%This is to help with formatting on future title pages.
\newenvironment{sectionOutcomes}{}{} 



%% Flowchart stuff
%\tikzstyle{startstop} = [rectangle, rounded corners, minimum width=3cm, minimum height=1cm,text centered, draw=black]
%\tikzstyle{question} = [rectangle, minimum width=3cm, minimum height=1cm, text centered, draw=black]
%\tikzstyle{decision} = [trapezium, trapezium left angle=70, trapezium right angle=110, minimum width=3cm, minimum height=1cm, text centered, draw=black]
%\tikzstyle{question} = [rectangle, rounded corners, minimum width=3cm, minimum height=1cm,text centered, draw=black]
%\tikzstyle{process} = [rectangle, minimum width=3cm, minimum height=1cm, text centered, draw=black]
%\tikzstyle{decision} = [trapezium, trapezium left angle=70, trapezium right angle=110, minimum width=3cm, minimum height=1cm, text centered, draw=black]


\title[Dig-In:]{Accumulated cross-sections}

\outcome{}

\begin{document}
\begin{abstract}
\end{abstract}
\maketitle

\section{Volume}

We have seen how to compute certain areas by using integration. Now we
will see how to compute some volumes.  The volumes that we can compute
will have cross-sections that are easy to describe. Sometimes we think
of these cross-sections as being ``slabs'' that we are layering to
create a volume.


\begin{example}
Find the volume of a pyramid with a square base that is $20$ meters tall
and $20$ meters on a side at the base. 
\begin{image}
%pyramid
\begin{tikzpicture}
 \begin{axis}[view={30}{30},colormap/\surfaceColor,
   xlabel=$x$, ylabel=$y$, zlabel=$z$,]

  \addplot3[surf,shader=faceted,opacity=.3,
  samples=12,
  samples y = 6,
  domain=-10:10,y domain=0:.5,
  z buffer=sort]
  ({10*(1-y)},{(1-y)*(x)}, {y*20});

  \addplot3[surf,shader=faceted,opacity=.3,
  samples=12,
  samples y = 6,
  domain=-10:10,y domain=0:.5,
  z buffer=sort]
  ({x*(1-y)},{(1-y)*(10)}, {y*20}); 

  \addplot3[surf,shader=faceted,opacity=.7, %front
  samples=12,
  samples y = 6,
  domain=-10:10,y domain=0:.5,
  z buffer=sort]
  ({x*(1-y)},{(1-y)*(-10)}, {y*20});
 
 \addplot3[surf,shader=faceted,opacity=.7, %front
  samples=12,
  samples y = 6,
  domain=-10:10,y domain=0:.5,
  z buffer=sort]
  ({10*(1-y)},{(1-y)*(x)}, {y*20});



  %square
  \addplot3[surf,colormap/\sliceColor,shader=flat,
  samples=4,
  samples y = 4,
  domain=-1:1,y domain=-1:1,
  z buffer=sort]
  ({10*(.5)*y},{(.5)*(x)*10}, {.5*20});


  \addplot3[surf,shader=faceted,opacity=.3,
  samples=12,
  samples y = 6,
  domain=-10:10,y domain=.5:1,
  z buffer=sort]
  ({10*(1-y)},{(1-y)*(x)}, {y*20});

  \addplot3[surf,shader=faceted,opacity=.3,
  samples=12,
  samples y = 6,
  domain=-10:10,y domain=.5:1,
  z buffer=sort]
  ({x*(1-y)},{(1-y)*(10)}, {y*20}); 

  \addplot3[surf,shader=faceted,opacity=.7, %front
  samples=12,
  samples y = 6,
  domain=-10:10,y domain=.5:1,
  z buffer=sort]
  ({x*(1-y)},{(1-y)*(-10)}, {y*20});
 
 \addplot3[surf,shader=faceted,opacity=.7, %front
  samples=12,
  samples y = 6,
  domain=-10:10,y domain=.5:1,
  z buffer=sort]
  ({10*(1-y)},{(1-y)*(x)}, {y*20});
 \end{axis}
\end{tikzpicture}

\end{image}


\begin{explanation}
We will  ``sum up" (integrate) infinitely many ``infinitesmal'' slabs which are parallel to the base to obtain
the volume of the pyramid.

Centering our pyramid at the origin, and placing the base of the box at $z = 0$,
the infinitesmal volume of each slab at height $z$ is 
\[
\d V = (L(z))^2 \d z
\]
where $\d z$ is the infinitesmal height of the slab, and $L(z)$ is the side length of the slab at height $z$. 

\begin{question}
\[
L(z) = \answer{20-z}
\]

\begin{hint}
	$L(z)$ must is a linear function of $z$, $L(z) = 20$, and $L(20) = 0$.  So $L(z) = 20-z$
\end{hint}
\end{question}

We want to sum these infinitesmal volumes from $z=0$ to $z = 20$, so we obtain the integral

\begin{question}
\[
\textrm{Volume} = \int_0^{20} (L(z))^2 \d z = \answer{\frac{20^3}{3}}
\]

\begin{hint}
	Making the substitution $u = z-20$, we have $\d u = - \d z$, $u$ going from $20$ to $0$, and 
	\begin{align*}
	\int_0^{20} (20-z)^2 \d z &= -\int_{20}^{0} u^2 \d u\\
		&= \int_0^{20} u^2 \d u\\
		&= \eval{\frac{1}{3} u^3}_0^{20}\\
		&= \frac{1}{3}(20^3)
	\end{align*}
\end{hint}
\end{question}

As you may know, the volume of a pyramid is 
$(1/3)(\hbox{height})(\hbox{area of base})=(1/3)(20)(400)$, which
agrees with our answer.
\end{explanation}
\end{example}



\begin{example}
The base of a solid is the region between $f(x)=x^2-1$ and
$g(x)=-x^2+1$, as pictured below. Its cross-sections
perpendicular to the $x$-axis are equilateral triangles.  Find the volume of the
solid.
\begin{image}
\begin{tikzpicture}
  \begin{axis}[
      xmin=-1, xmax=1,domain=-1:1,
      clip=false,
      axis lines =center, xlabel=$x$, ylabel=$z$,
      every axis y label/.style={at=(current axis.above origin),anchor=south},
      every axis x label/.style={at=(current axis.right of origin),anchor=west},
      axis on top,
    ] 
    \addplot [penColor2,very thick] {x^2-1};
    \addplot [penColor,very thick] {-x^2+1};
    \node at (axis cs:1,0.8) [penColor] {$y = -x^2+1$};
    \node at (axis cs:1,-0.8) [penColor2] {$y = x^2-1$};
  \end{axis}
\end{tikzpicture}

\end{image}

\begin{explanation}




\begin{image}
% triangleRegion
\begin{tikzpicture}
 \begin{axis}[view={30}{30},colormap/\surfaceColor,
   xlabel=$x$, ylabel=$z$, zlabel=$y$,]

  \addplot3[surf,shader=faceted,opacity=.3,
  samples=12,
  samples y = 6,
  domain=-1:.5,y domain=0:1,
  z buffer=sort]
  (x,{(1-y)*(-x^2+1)}, {y*sqrt(3)*(1-x^2)});

  \addplot3[surf,shader=faceted,opacity=.7,%front
  samples=12,
  samples y = 6,
  domain=-1:.5,y domain=0:1,
  z buffer=sort]
  (x,{(1-y)*(x^2-1)}, {y*sqrt(3)*(1-x^2)});

  %tri
  \addplot3[surf,colormap/\sliceColor,shader=flat,
  samples=6,
  samples y = 6,
  domain=-1:1,y domain=0:1,
  z buffer=sort]
  (.5,{x*(1-y)*(.5^2-1)}, {abs(x)*y*sqrt(3)*(1-.5^2)});

  \addplot3[surf,shader=faceted,opacity=.3, 
  samples=4,
  samples y = 6,
  domain=.5:1,y domain=0:1,
  z buffer=sort]
  (x,{(1-y)*(-x^2+1)}, {y*sqrt(3)*(1-x^2)});

  \addplot3[surf,shader=faceted,opacity=.7,%front
  samples=4,
  samples y = 6,
  domain=.5:1,y domain=0:1,
  z buffer=sort]
  (x,{(1-y)*(x^2-1)}, {y*sqrt(3)*(1-x^2)});

 \end{axis}
\end{tikzpicture}
%% \caption{A solid with equilateral triangle cross-sections bounded by
%%   the region between $f(x)=x^2-1$ and $g(x)=-x^2+1$.}
%% \label{fig:triangular cross-sections}
\end{image}

We want to find the volume of the triangular slab at $x$. We know its width is $\d x$.  What is the area,  $A(x)$, of the triangular face of this slab?



\begin{question}
	$A(x) = \answer{\sqrt{3}(1-x^2)^2}$
	
	\begin{hint}
		The base of the slab has length $2(1-x^2)$ since $(1-x^2)- (x^2-1) = 2(1-x^2)$. By geometry, the height is $\sqrt{3}(1-x^2)$.  So the area of this triangle is $\frac{1}{2} \textrm{base} \cdot \textrm{height} = \sqrt{3}(1-x^2)^2$.
	\end{hint}	
\end{question}


Thus the total volume is
\[
\int_{-1}^1 \sqrt3(1-x^2)^2\d x={16\over15}\sqrt3.
\]
\end{explanation}
\end{example}

One easy way to get ``nice'' cross-sections is by rotating a plane
figure around a line. Below we see $f$ bounded by two vertical lines:
\begin{image}
\begin{tikzpicture}
  \begin{axis}[
      xmin=0, xmax=5,domain=0:5, clip=false, axis lines =center,
      xlabel=$x$, ylabel=$z$, every axis y label/.style={at=(current
        axis.above origin),anchor=south}, every axis x
      label/.style={at=(current axis.right of origin),anchor=west},
      axis on top, ]
    \addplot [penColor,very thick,smooth]{16-19*x+8*x^2-x^3};
    \addplot [textColor,dashed] plot coordinates {(1,0) (1,4)};
    \addplot [textColor,dashed] plot coordinates {(4,0) (4,4)};
  \end{axis}
\end{tikzpicture}
%% \caption{A plot of $f(x)$.}
%% \label{figure:regionCurve}
\end{image}


Rotating $f$ around the $x$-axis will generate a figure whose
volume we can compute.
%disk
\begin{image}
\begin{tikzpicture}
 \begin{axis}[view={30}{30},colormap/\surfaceColor,
   xlabel=$x$, ylabel=$z$, zlabel=$y$,
   ]

  \addplot3[surf,shader=faceted,opacity=.3,
  samples=10,
  samples y =20,
  domain=1:2.7,y domain=0:2*pi,
  z buffer=sort]
  (x,{(16-19*x+8*x^2-x^3) * cos(deg(y))}, {(16-19*x+8*x^2-x^3) * sin(deg(y))});

  \addplot3[surf,colormap/\sliceColor,shader=flat,%disk
  samples=5,
  samples y = 20,
  domain=0:1,y domain=0:2*pi,
  z buffer=sort
]
  (2.7,{3.337*x*cos(deg(y))}, {3.337*x*sin(deg(y))});


  \addplot3[surf,shader=faceted,opacity=.3,
  samples=10,
  samples y=20,
  domain=2.7:4,y domain=0:2*pi,
  z buffer=sort]
  (x,{(16-19*x+8*x^2-x^3) * cos(deg(y))}, {(16-19*x+8*x^2-x^3) * sin(deg(y))});


  \addplot3[surf,shader=faceted,opacity=.7,
  samples=5,
  samples y = 20,
  domain=0:1,y domain=0:2*pi,
  z buffer=sort
]
  (4,{4*x*cos(deg(y))}, {4*x*sin(deg(y))});

 \end{axis}
\end{tikzpicture}
\end{image}

The volume of each disk will have the form $\pi r^2 h$. As long as we
can write $r$ in terms of $x$ we can compute the volume by an
integral.




\begin{example}
Find the volume of a right circular cone with base radius 10 and
height 20. Here, a right circular cone is one with a circular base and
with the tip of the cone directly over the center of the base.

\begin{image}
%cone
\begin{tikzpicture}
 \begin{axis}[
     view={30}{30},colormap/\surfaceColor,
     xlabel=$x$, ylabel=$z$, zlabel=$y$,
   ]

  \addplot3[surf,shader=faceted,opacity=.3,
  samples=10,
  samples y =20,
  domain=0:12,y domain=0:2*pi,
  z buffer=sort]
  (x,{(x/2) * cos(deg(y))}, {(x/2) * sin(deg(y))});

  \addplot3[surf,colormap/\sliceColor,shader=flat,%disk
  samples=5,
  samples y = 20,
  domain=0:1,y domain=0:2*pi,
  z buffer=sort
]
  (12,{6*x*cos(deg(y))}, {6*x*sin(deg(y))});


  \addplot3[surf,shader=faceted,opacity=.3,
  samples=10,
  samples y=20,
  domain=12:20,y domain=0:2*pi,
  z buffer=sort]
  (x,{(x/2) * cos(deg(y))}, {(x/2) * sin(deg(y))});


  \addplot3[surf,shader=faceted,opacity=.7,
  samples=5,
  samples y = 20,
  domain=0:1,y domain=0:2*pi,
  z buffer=sort
]
  (20,{10*x*cos(deg(y))}, {10*x*sin(deg(y))});

 \end{axis}
\end{tikzpicture}
%% \caption{A right circular cone with base radius 10 and height 20.}
%% \label{fig:line to cone}
\end{image}

We can view this cone as produced by the rotation of the line
$y=x/2$ rotated about the $x$-axis, as indicated in
Figure~\ref{fig:line to cone}.

At a particular point on the $x$-axis, the radius of the resulting
cone is the $y$-coordinate of the corresponding point on the line
$y=x/2$. The area of the cross section is given by
\[
\pi \cdot \text{radius}^2 = \pi \left(\frac{x}{2}\right)^2
\]
so the volume is given by
$$
  \int_0^{20} \pi
  {x^2\over4}\d x={\pi\over4}{20^3\over3}={2000\pi\over3}.
$$ 
Note that we can instead do the calculation with a generic height and
radius: 
\[
  \int_0^{h} \pi{r^2\over h^2}x^2\d x
  ={\pi r^2\over h^2}{h^3\over3}={\pi r^2h\over3},
\]
giving us the usual formula for the volume of a cone.
\end{example}


\end{document}
