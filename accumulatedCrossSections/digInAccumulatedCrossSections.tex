\documentclass{ximera}

%\usepackage{todonotes}

\newcommand{\todo}{}

\usepackage{tkz-euclide}
\tikzset{>=stealth} %% cool arrow head
\tikzset{shorten <>/.style={ shorten >=#1, shorten <=#1 } } %% allows shorter vectors

\usetikzlibrary{backgrounds} %% for boxes around graphs
\usetikzlibrary{shapes,positioning}  %% Clouds and stars
\usetikzlibrary{matrix} %% for matrix
\usepgfplotslibrary{polar} %% for polar plots
\usetkzobj{all}
\usepackage[makeroom]{cancel} %% for strike outs
%\usepackage{mathtools} %% for pretty underbrace % Breaks Ximera
\usepackage{multicol}





\usepackage{array}
\setlength{\extrarowheight}{+.1cm}   
\newdimen\digitwidth
\settowidth\digitwidth{9}
\def\divrule#1#2{
\noalign{\moveright#1\digitwidth
\vbox{\hrule width#2\digitwidth}}}





\newcommand{\RR}{\mathbb R}
\newcommand{\R}{\mathbb R}
\newcommand{\N}{\mathbb N}
\newcommand{\Z}{\mathbb Z}

%\renewcommand{\d}{\,d\!}
\renewcommand{\d}{\mathop{}\!d}
\newcommand{\dd}[2][]{\frac{\d #1}{\d #2}}
\renewcommand{\l}{\ell}
\newcommand{\ddx}{\frac{d}{\d x}}

\newcommand{\zeroOverZero}{\ensuremath{\boldsymbol{\tfrac{0}{0}}}}
\newcommand{\inftyOverInfty}{\ensuremath{\boldsymbol{\tfrac{\infty}{\infty}}}}
\newcommand{\zeroOverInfty}{\ensuremath{\boldsymbol{\tfrac{0}{\infty}}}}
\newcommand{\zeroTimesInfty}{\ensuremath{\small\boldsymbol{0\cdot \infty}}}
\newcommand{\inftyMinusInfty}{\ensuremath{\small\boldsymbol{\infty - \infty}}}
\newcommand{\oneToInfty}{\ensuremath{\boldsymbol{1^\infty}}}
\newcommand{\zeroToZero}{\ensuremath{\boldsymbol{0^0}}}
\newcommand{\inftyToZero}{\ensuremath{\boldsymbol{\infty^0}}}


\newcommand{\numOverZero}{\ensuremath{\boldsymbol{\tfrac{\#}{0}}}}
\newcommand{\dfn}{\textbf}
%\newcommand{\unit}{\,\mathrm}
\newcommand{\unit}{\mathop{}\!\mathrm}
\newcommand{\eval}[1]{\bigg[ #1 \bigg]}
\newcommand{\seq}[1]{\left( #1 \right)}
\renewcommand{\epsilon}{\varepsilon}
\renewcommand{\iff}{\Leftrightarrow}

\DeclareMathOperator{\arccot}{arccot}
\DeclareMathOperator{\arcsec}{arcsec}
\DeclareMathOperator{\arccsc}{arccsc}
\DeclareMathOperator{\si}{Si}

\newcommand{\tightoverset}[2]{%
  \mathop{#2}\limits^{\vbox to -.5ex{\kern-0.75ex\hbox{$#1$}\vss}}}
\newcommand{\arrowvec}[1]{\tightoverset{\scriptstyle\rightharpoonup}{#1}}
\renewcommand{\vec}{\mathbf}


\colorlet{textColor}{black} 
\colorlet{background}{white}
\colorlet{penColor}{blue!50!black} % Color of a curve in a plot
\colorlet{penColor2}{red!50!black}% Color of a curve in a plot
\colorlet{penColor3}{red!50!blue} % Color of a curve in a plot
\colorlet{penColor4}{green!50!black} % Color of a curve in a plot
\colorlet{penColor5}{orange!80!black} % Color of a curve in a plot
\colorlet{fill1}{penColor!20} % Color of fill in a plot
\colorlet{fill2}{penColor2!20} % Color of fill in a plot
\colorlet{fillp}{fill1} % Color of positive area
\colorlet{filln}{penColor2!20} % Color of negative area
\colorlet{fill3}{penColor3!20} % Fill
\colorlet{fill4}{penColor4!20} % Fill
\colorlet{fill5}{penColor5!20} % Fill
\colorlet{gridColor}{gray!50} % Color of grid in a plot

\newcommand{\surfaceColor}{violet}
\newcommand{\surfaceColorTwo}{redyellow}
\newcommand{\sliceColor}{greenyellow}




\pgfmathdeclarefunction{gauss}{2}{% gives gaussian
  \pgfmathparse{1/(#2*sqrt(2*pi))*exp(-((x-#1)^2)/(2*#2^2))}%
}


%%%%%%%%%%%%%
%% Vectors
%%%%%%%%%%%%%

%% Simple horiz vectors
\renewcommand{\vector}[1]{\left\langle #1\right\rangle}


%% %% Complex Horiz Vectors with angle brackets
%% \makeatletter
%% \renewcommand{\vector}[2][ , ]{\left\langle%
%%   \def\nextitem{\def\nextitem{#1}}%
%%   \@for \el:=#2\do{\nextitem\el}\right\rangle%
%% }
%% \makeatother

%% %% Vertical Vectors
%% \def\vector#1{\begin{bmatrix}\vecListA#1,,\end{bmatrix}}
%% \def\vecListA#1,{\if,#1,\else #1\cr \expandafter \vecListA \fi}

%%%%%%%%%%%%%
%% End of vectors
%%%%%%%%%%%%%

%\newcommand{\fullwidth}{}
%\newcommand{\normalwidth}{}



%% makes a snazzy t-chart for evaluating functions
%\newenvironment{tchart}{\rowcolors{2}{}{background!90!textColor}\array}{\endarray}

%%This is to help with formatting on future title pages.
\newenvironment{sectionOutcomes}{}{} 



%% Flowchart stuff
%\tikzstyle{startstop} = [rectangle, rounded corners, minimum width=3cm, minimum height=1cm,text centered, draw=black]
%\tikzstyle{question} = [rectangle, minimum width=3cm, minimum height=1cm, text centered, draw=black]
%\tikzstyle{decision} = [trapezium, trapezium left angle=70, trapezium right angle=110, minimum width=3cm, minimum height=1cm, text centered, draw=black]
%\tikzstyle{question} = [rectangle, rounded corners, minimum width=3cm, minimum height=1cm,text centered, draw=black]
%\tikzstyle{process} = [rectangle, minimum width=3cm, minimum height=1cm, text centered, draw=black]
%\tikzstyle{decision} = [trapezium, trapezium left angle=70, trapezium right angle=110, minimum width=3cm, minimum height=1cm, text centered, draw=black]


\title[Dig-In:]{Accumulated cross-sections}

\outcome{Identify cross sections.}
\outcome{Use the disk method to compute volumes.}
\outcome{Use the washer method to compute volumes.}
\outcome{Compute volumes of revolution around the $x$-axis.}
\outcome{Compute volumes of revolution around the $y$-axis.}
\outcome{Compute volumes of revolution around arbitrary lines.}

\begin{document}
\begin{abstract}
  We can also use integrals to compute volume.
\end{abstract}
\maketitle

\section{Accumulation of cross-sections}

We have seen how to compute certain areas by using integration. Now we
will see how to compute some volumes.  The volumes that we can compute
will have cross-sections that are easy to describe. Sometimes we think
of these cross-sections as being ``slabs'' that we are layering to
create a volume.\index{slabs}\index{cross-sections}


\begin{example}
Find the volume of a pyramid with a square base that is $20$ meters tall
and $20$ meters on a side at the base.
\begin{image}
  \begin{tikzpicture}
    \begin{axis}[
          xmin =-4,xmax=7,ymax=4,ymin=-4,
          axis lines=center, xlabel=$x$, ylabel=$y$,
          every axis y label/.style={at=(current axis.above origin),anchor=south},
          every axis x label/.style={at=(current axis.right of origin),anchor=west},
          width=5in,
          axis on top,
          xtick={0,6}, xticklabels={$0$, $20$},
          ytick={0,3},yticklabels={$0$,$20$},
            clip=false,
      ]
      \addplot [draw=penColor, fill = fill1, very thick] plot coordinates {(-3,-3) (3,-3) (6,0) (1.5, 3) (-3,-3)};
      \addplot [draw=penColor, very thick] plot coordinates {(-3,-3) (0,0)};
      \addplot [draw=penColor, very thick] plot coordinates {(6,0) (0,0)};
      \addplot [draw=penColor, very thick] plot coordinates {(1.5,3) (3,-3)};
      \addplot [draw=penColor, very thick] plot coordinates {(1.5,3) (0,0)};

      \addplot [->] plot coordinates {(0,0) (-4,-4)};
      \node[anchor=north east] at (axis cs:-4,-4) {$z$};       
    \end{axis}
  \end{tikzpicture}
\end{image}

\begin{explanation}
To solve this problem, we will ``sum up'' (integrate) infinitely many
``infinitesimal'' slabs which are parallel to the base of the pyramid
to obtain the volume.
\begin{image}
  \begin{tikzpicture}
    \begin{axis}[
          xmin =-4,xmax=7,ymax=4,ymin=-4,
          axis lines=center, xlabel=$x$, ylabel=$y$,
          every axis y label/.style={at=(current axis.above origin),anchor=south},
          every axis x label/.style={at=(current axis.right of origin),anchor=west},
          axis on top,
          width=5in,
          xtick={0,6}, xticklabels={$0$, $20$},
          ytick={0,3},yticklabels={$0$,$20$},
            clip=false,
      ]
      \addplot [draw=penColor, thick] plot coordinates {(-3,-3) (3,-3) (6,0) (1.5, 3) (-3,-3)};
            
      \addplot [draw=penColor, thick] plot coordinates {(-3,-3) (0,0)};
      \addplot [draw=penColor, thick] plot coordinates {(6,0) (0,0)};
      \addplot [draw=penColor, thick] plot coordinates {(1.5,3) (3,-3)};
      \addplot [draw=penColor, thick] plot coordinates {(1.5,3) (0,0)};

      %% slab
      \addplot [draw=penColor, fill=fill1,very thick] plot coordinates {(3,2) (1,2) (0,1) (2, 1) (3,2)};
      \addplot [draw=penColor, fill=fill1,very thick] plot coordinates {(0,.8) (0,1) (2,1) (2, .8) (0,.8)};
      \addplot [draw=penColor, fill=fill1,very thick] plot coordinates {(2,1) (2, .8) (3,1.8) (3,2) (2,1)};

      %\addplot [draw=penColor, fill=fill1,very thick] plot coordinates {(3,1.8) (1,1.8) (0,.8) (2, .8) (3,1.8)};
      %\addplot [draw=penColor, fill=fill1,very thick] plot coordinates {(3,2) (1,2) (0,1) (2, 1) (3,2)};

      \addplot [draw=penColor, thick] plot coordinates {(1.5,3) (3,-3)};

      \draw[decoration={brace,mirror,raise=.1cm},decorate,thin] (axis cs:0,.8)--(axis cs:2,.8);
      \draw[decoration={brace,raise=.1cm},decorate,thin] (axis cs:3.1,2.05)--(axis cs:3.1,1.75);
      
      \addplot [->] plot coordinates {(0,0) (-4,-4)};
      \node[anchor=north east] at (axis cs:-4,-4) {$z$};

      \node at (axis cs:3.6,1.9) {$\d y$};
      \node at (axis cs:1,.4) {$L(y)$};       
    \end{axis}
  \end{tikzpicture}
\end{image}
The ``height'' of each slab will be $\d y$, and we'll let $L(y)$ be the width:
\[
L(y) = \answer[given]{20-y}
\]
\begin{hint}
  Since $L(y)$ is a linear function of $y$, and $L(0) = 20$, and
  $L(20) = 0$ we see $L(y) = 20-y$ by the point-slope formula.
\end{hint}
For each slab, the infinitesimal volume at height $y$ is
\begin{align*}
  \d V &= \mathrm{length} \cdot \mathrm{width}\cdot \mathrm{height}\\
  &= L(y)\cdot L(y)\cdot  \d y
\end{align*}
so the total volume is given by
\begin{align*}
  \mathrm{Volume} &= \int_0^{20} L(y)^2 \d y \\
  &= \int_0^{20} (\answer[given]{20-y})^2 \d y
\end{align*}
Making the substitution $g = 20-y $, we have $\d g = - \d y$, $g$ going from $20$ to $0$, and 
	\begin{align*}
	\int_0^{20} (20-y)^2 \d y &= -\int_{\answer[given]{20}}^{\answer[given]{0}} g^2 \d g\\
		&= \int_0^{20} g^2 \d g\\
		&= \eval{\answer[given]{g^3/3}}_0^{20}\\
		&= \frac{20^3}{3}.
	\end{align*}
As you may know, the volume of a pyramid is
$(1/3)(\text{height})(\text{area of base})=(1/3)(20)(400)$, which
agrees with our answer.
\end{explanation}
\end{example}



\begin{example}
The base of a solid is the region bounded by $f(x)=x^2-1$ and
$g(x)=-x^2+1$:
\begin{image}
\begin{tikzpicture}
  \begin{axis}[
      xmin=-1, xmax=1,domain=-1:1,
      clip=false,
      axis lines =center, xlabel=$x$, ylabel=$y$,
      every axis y label/.style={at=(current axis.above origin),anchor=south},
      every axis x label/.style={at=(current axis.right of origin),anchor=west},
      axis on top,
    ] 
    \addplot [penColor2,very thick] {x^2-1};
    \addplot [penColor,very thick] {-x^2+1};
    \node at (axis cs:1,0.8) [penColor] {$y = -x^2+1$};
    \node at (axis cs:1,-0.8) [penColor2] {$y = x^2-1$};
  \end{axis}
\end{tikzpicture}
\end{image}
The cross-section of this solid consists of equilateral triangles
that are perpendicular to the $x$-axis:
\begin{image}
\begin{tikzpicture}
  \begin{axis}[
      xmin=-1.2, xmax=1.2,domain=-1:1,
      ymin=-.4,ymax=1,
      clip=false,
      ytick={0,.8},yticklabels={$0$,$1$},
      axis lines =center, xlabel=$x$, ylabel=$z$,
      every axis y label/.style={at=(current axis.above origin),anchor=south},
      every axis x label/.style={at=(current axis.right of origin),anchor=west},
      axis on top,
    ] 
    \addplot [penColor2,very thick] {(0.3)*x^2-(0.3)};
    \addplot [penColor,very thick] {(-0.3)*x^2+(0.3)};

    \addplot [draw=penColor, fill=fill1,very thick] plot coordinates {(-.11,-.3) (.1,.3) (0,.8) (-.11,-.3)};

    \addplot [draw=penColor, fill=fill1,very thick] plot coordinates {(-.67,-.17) (-.53,.21) (-.6,.6) (-.67,-.17)};

    \addplot [draw=penColor, fill=fill1,very thick] plot coordinates {(.53,-.22) (.67,.16) (.6,.6) (.53,-.22)};


    \addplot [black,very thick] {-.28 + .0086*sqrt(16031-14948*x^2)};

    \addplot [->] plot coordinates {(-.6,-.8) (.6,0.8)};
    \node[anchor=south west] at (axis cs:.6,0.8) {$y$};
  \end{axis}
\end{tikzpicture}
\end{image}
Find the volume of this solid.
\begin{explanation}
We want to find the volume of the triangular slab at a given $x$
value. We know the width of each slab is $\d x$:
\begin{image}
\begin{tikzpicture}
  \begin{axis}[
      xmin=-1.2, xmax=1.2,domain=-1:1,
      ymin=-.4,ymax=1,
      clip=false,
      ytick={0,.8},yticklabels={$0$,$1$},
      axis lines =center, xlabel=$x$, ylabel=$z$,
      every axis y label/.style={at=(current axis.above origin),anchor=south},
      every axis x label/.style={at=(current axis.right of origin),anchor=west},
      axis on top,
    ] 
    \addplot [penColor2,very thick] {(0.3)*x^2-(0.3)};
    \addplot [penColor,very thick] {(-0.3)*x^2+(0.3)};

    \addplot [draw=penColor, fill=fill1,very thick] plot coordinates {(-.11,-.3) (.1,.3) (0,.8) (-.11,-.3)};

    \addplot [draw=penColor, fill=fill1,very thick] plot coordinates {(-.11,-.3) (-.19,-.3) (-.08,.8) (0,.8) (-.11,-.3)};

    \draw[decoration={brace,raise=.1cm},decorate,thin] (axis cs:-.11,-.3)--(axis cs:-.19,-.3);  
    \node[anchor=north] at (axis cs:.-.15,-.35) {$\d x$};
    
    \addplot [black,very thick] {-.28 + .0086*sqrt(16031-14948*x^2)};

    \addplot [->] plot coordinates {(-.6,-.8) (.6,0.8)};
    \node[anchor=south west] at (axis cs:.6,0.8) {$y$};
  \end{axis}
\end{tikzpicture}
\end{image}
The area of each triangular cross-section is given by.
\[
A(x) = \answer[given]{\sqrt{3}(1-x^2)^2}
\]
\begin{hint}
  The base of the slab has length $\answer[given]{2(1-x^2)}$ since
  \[
  (1-x^2)- (x^2-1) = 2(1-x^2).
  \]
  By geometry, the height is $\answer[given]{\sqrt{3}(1-x^2)}$.  So
  the area of this triangle is
  \[
  \frac{1}{2} \text{base} \cdot \text{height} =
  \sqrt{3}(1-x^2)^2.
  \]
\end{hint}

We want to ``sum'' (integrate) all of the infinitesimal volumes
\[
\d V = A(x) \d x
\]
from $x=-1$ to $x=1$.  Thus the total volume is
\begin{align*}
  \text{Volume} &= \int_{-1}^1 A(x) \d x  \\
  &= \int_{-1}^1 \sqrt{3} (1-x^2)^2 \d x\\
  &=\sqrt{3} \int_{-1}^1 1-2x^2+x^4 \d x \\
  &=\sqrt{3} \eval{\answer[given]{x-\frac{2}{3}x^3+\frac{x^5}{5}}}_{-1}^1\\
  &=2\sqrt{3}\left(1-\frac{2}{3}+\frac{1}{5}\right)\\
  &=\frac{16\sqrt{3}}{15}.
\end{align*}
\end{explanation}
\end{example}



\section{Solids of revolution}\index{solid of revolution}

\index{volume of a solid of revolution}

One easy way to get ``nice'' cross-sections is by rotating a plane
figure around a line. Here the essential skill for the young
mathmatician is to turn
\begin{quote}
  \textbf{radial cross-sections into cylindical-cross sections.}
\end{quote}
Below we see a function $f$ bounded by two vertical lines:
\begin{image}
\begin{tikzpicture}
  \begin{axis}[
      xmin=0, xmax=5,domain=0:5, clip=false, axis lines =center,
      xlabel=$x$, ylabel=$y$, every axis y label/.style={at=(current
        axis.above origin),anchor=south}, every axis x
      label/.style={at=(current axis.right of origin),anchor=west},
      axis on top, ]
    \addplot [penColor,very thick,smooth]{16-19*x+8*x^2-x^3};
    \addplot [textColor,dashed] plot coordinates {(1,0) (1,4)};
    \addplot [textColor,dashed] plot coordinates {(4,0) (4,4)};
  \end{axis}
\end{tikzpicture}
\end{image}
Rotating $f$ around the $x$-axis will generate a figure whose volume
we can compute:
\begin{image}
\begin{tikzpicture}
  \begin{axis}[
      xmin=0, xmax=5,domain=0:5, clip=false, axis lines =center,
      xlabel=$x$, ylabel=$z$, every axis y label/.style={at=(current
        axis.above origin),anchor=south}, every axis x
      label/.style={at=(current axis.right of origin),anchor=west},
      axis on top, ]
    \addplot [penColor,very thick,smooth,domain=1:4,fill=fill1,draw=none] {16-19*x+8*x^2-x^3} \closedcycle;
    \addplot [penColor,very thick,smooth,domain=1:4,fill=fill1,draw=none] {-16+19*x-8*x^2+x^3} \closedcycle;

    \draw[penColor,very thick,fill=fill1] (axis cs:1,0) ellipse (20 and 400);

    \addplot [penColor,very thick,smooth,domain=1:4] {16-19*x+8*x^2-x^3};
    \addplot [penColor,very thick,smooth,domain=1:4] {-16+19*x-8*x^2+x^3};
    
    \draw[penColor,very thick,fill=fill1!50!white] (axis cs:4,0) ellipse (20 and 400);
  \end{axis}
\end{tikzpicture}
\end{image}
To compute the total volume, we should look at the cross-section
\begin{image}
\begin{tikzpicture}
  \begin{axis}[
      xmin=0, xmax=5,domain=0:5, clip=false, axis lines =center,
      xlabel=$x$, ylabel=$z$, every axis y label/.style={at=(current
        axis.above origin),anchor=south}, every axis x
      label/.style={at=(current axis.right of origin),anchor=west},
      axis on top, ]
    \draw[penColor,very thick] (axis cs:1,0) ellipse (20 and 400);

    \addplot [penColor,very thick,smooth,domain=1:4] {16-19*x+8*x^2-x^3};
    \addplot [penColor,very thick,smooth,domain=1:4] {-16+19*x-8*x^2+x^3};

    \draw[penColor,very thick,fill=fill1] (axis cs:2.9,0) ellipse (20 and 400);
    \draw[penColor,very thick,fill=fill1] (axis cs:3,0) ellipse (20 and 400);

    \draw[decoration={brace,mirror,raise=.1cm},decorate,thin] (axis cs:2.8,-4.1)--(axis cs:3.1,-4.1);  
    \node[anchor=north] at (axis cs:2.95,-4.3) {$\d x$};
    
    \draw[penColor,very thick] (axis cs:4,0) ellipse (20 and 400);
  \end{axis}
\end{tikzpicture}
\end{image}
The volume of each disk will be:
\[
\d V = \pi \left( f(x)\right)^2 \d x
\]
To compute the total volume we must accumulate these infinitesimal
volumes. Let's see an example.

\begin{example}
Find the volume of a right circular cone with base radius $10$ and
height $20$.

\begin{explanation}
We can view this cone as produced by the line $y=x/2$ rotated about
the $x$-axis:
\begin{image}
\begin{tikzpicture}
  \begin{axis}[
      xmin=0, xmax=20,domain=0:20,
      clip=false, axis lines =center,
      xlabel=$x$, ylabel=$y$,
      every axis y label/.style={at=(current axis.above origin),anchor=south},
      every axis x label/.style={at=(current axis.right of origin),anchor=west},
      axis on top,
    ]
    \addplot [penColor,very thick,smooth,domain=0:20] {x/2};
    \addplot [penColor,very thick,smooth,domain=0:20] {-x/2};

    \draw[penColor,very thick,fill=fill1] (axis cs:11.5,0) ellipse (20 and 600);
    \draw[penColor,very thick,fill=fill1] (axis cs:12,0) ellipse (20 and 600);
    
    \draw[penColor,very thick] (axis cs:20,0) ellipse (20 and 1000);
    
    \draw[decoration={brace,mirror,raise=.1cm},decorate,thin] (axis cs:11,-6)--(axis cs:12.5,-6);  
    \node[anchor=north] at (axis cs:11.75,-6.4) {$\d x$};
  \end{axis}
\end{tikzpicture}
\end{image}

At a particular point on the $x$-axis, the radius of the resulting
cone is the $y$-coordinate of the corresponding point on the line
$y=x/2$. The area of the cross-section is given by
\[
A(x) = \answer[given]{\pi \left(\frac{x}{2}\right)^2}
\]
The infinitesimal volume of each disc is then $A(x) \d x$, so the
total volume is the integral of these infinitesimal volumes from $x =
0$ to $x = 20$.
\[
\text{Volume} = 
  \int_0^{20} A(x) \d x=\answer[given]{\frac{2000\pi}{3}}.
\]
\end{explanation}
\end{example}

Now let's try a more involved example.

\begin{example}
  The cross-section of a glass can be modeled by the function $r(x) =
  \frac{x^4}{3}$, with units in centimeters:
  \begin{image}
    \begin{tikzpicture}[
        declare function = {f(\x) = (1/3)* pow(\x,4);} ]
      \begin{axis}[
          xmin =-4,xmax=4,ymax=23,ymin=-.2,
          axis lines=center, xlabel=$x$, ylabel=$y$,
          every axis y label/.style={at=(current axis.above origin),anchor=south},
          every axis x label/.style={at=(current axis.right of origin),anchor=west},
          axis on top,
        ]
        \addplot [draw=none,fill=fill1!50!white,domain=-2.65:2.65, smooth] {.8*sqrt(2.65^2-x^2)+16.8} \closedcycle;
        \addplot [draw=none,fill=fill1,domain=-2.65:2.65, smooth] {-.8*sqrt(2.65^2-x^2)+16.8} \closedcycle; 
        \addplot [draw=none,fill=white,domain=-2.7:2.7, smooth] {f(x)} \closedcycle;
        \addplot [ultra thick,penColor, smooth,domain=-2.75:2.75] {f(x)};

        \draw[penColor,very thick] (axis cs:0,16.8) ellipse (265 and 20);
        \draw[penColor,very thick] (axis cs:0,19) ellipse (275 and 20);

        \node[black] at (axis cs:2.5,3) {$y=\frac{x^4}{3}$};       
      \end{axis}
    \end{tikzpicture}
  \end{image}
  At what height would one need to place a mark indicating
  $250\unit{ml}$ of fluid?
  \begin{explanation}
    To start, we should be looking at the following cross-section:
    \begin{image}
    \begin{tikzpicture}[
        declare function = {f(\x) = (1/3)* pow(\x,4);} ]
      \begin{axis}[
          xmin =-4,xmax=4,ymax=23,ymin=-.2,
          axis lines=center, xlabel=$x$, ylabel=$y$,
          every axis y label/.style={at=(current axis.above origin),anchor=south},
          every axis x label/.style={at=(current axis.right of origin),anchor=west},
          axis on top,
        ]
        
        
        \addplot [ultra thick,penColor, smooth,domain=-2.75:2.75] {f(x)};

        \draw[penColor,very thick,fill=fill1] (axis cs:0,16.2) ellipse (265 and 20);
        \draw[penColor,very thick,fill=fill1] (axis cs:0,16.8) ellipse (265 and 20);
        
        \draw[penColor,very thick] (axis cs:0,19) ellipse (275 and 20);
        
        \draw[decoration={brace,mirror,raise=.1cm},decorate,thin] (axis cs:2.7,15.9)--(axis cs:2.7,16.9);  
        \node[anchor=west] at (axis cs:2.9,16.5) {$\d y$};

        \node[black] at (axis cs:2.5,3) {$y=\frac{x^4}{3}$};       
      \end{axis}
    \end{tikzpicture}
    \end{image}
    The radius of the disk, given any value of $y$ is given by
    $\answer[given]{(3y)^{1/4}}$. Hence the volume of the
    infinitesimal disk is
    \[
    \d V = \pi (\answer[given]{(3y)^{1/4}})^2 \d y
    \]
    summing these all together, via integration we find
    \begin{align*}
      \int_0^h \pi (\answer[given]{(3y)^{1/4}})^2 \d y &= \int_0^h \pi\answer[given]{3^{1/2} y^{1/2}} \d y\\
      &= \eval{\answer[given]{\frac{2\pi y^{3/2}}{\sqrt{3}}}}_0^h\\
      &= \answer[given]{\frac{2\pi h^{3/2}}{\sqrt{3}}}.
    \end{align*}
    Now that we have a formula for volume, we need to see when it is
    equal to $250$. Write with me
    \begin{align*}
      250 &= \frac{2\pi h^{3/2}}{\sqrt{3}},\\
      \frac{\sqrt{3} \cdot 125}{\pi} &=  h^{3/2},\\
      \left(\frac{\sqrt{3} \cdot 125}{\pi}\right)^{2/3} &=  h,
    \end{align*}
    So we see that we should put our mark at approximately $16.8$
    centimeters.
  \end{explanation}
\end{example}

\section{Washers as cross-sections}


Sometimes the ``slabs'' look like disks with holes in them, or
``washers.'' Let's see an example of this.\index{washers}

\begin{example}
Find the volume of the object generated when the area between $f(x) =
x$ and $g(x)=x^2$ is rotated around the $x$-axis:
\begin{image}
\begin{tikzpicture}
  \begin{axis}[
      xmin=0, xmax=1,domain=0:1,
      clip=false,
      axis lines =center, xlabel=$x$, ylabel=$y$,
      every axis y label/.style={at=(current axis.above origin),anchor=south},
      every axis x label/.style={at=(current axis.right of origin),anchor=west},
      axis on top,
    ] 
    \addplot [draw=none,fill=fill1,very thick] {x} \closedcycle;
    \addplot [draw=none, fill=background,very thick] {x^2}\closedcycle;

    \addplot [penColor,very thick] {x};
    \addplot [penColor2,very thick] {x^2};

    \node at (axis cs:.5,.58) [penColor] {$f$};
    \node at (axis cs:.5,.2) [penColor2] {$g$};
    \end{axis}
\end{tikzpicture}
\end{image}

\begin{explanation}
This solid has a ``hole'' in the middle. We can compute the volume by
subtracting the volume of the hole from the volume enclosed by the
outer surface of the solid. To do this, let's look at one of the
``washer'' shaped cross-sections:
\begin{image}
\begin{tikzpicture}
  \begin{axis}[
      xmin=0, xmax=1,domain=0:1,
      clip=false,
      axis lines =center, xlabel=$x$, ylabel=$y$,
      every axis y label/.style={at=(current axis.above origin),anchor=south},
      every axis x label/.style={at=(current axis.right of origin),anchor=west},
      axis on top,
    ] 
    \addplot [very thick, penColor] {x};
    \addplot [very thick,penColor2] {x^2};

    \addplot [very thick,penColor] {-x};
    \addplot [very thick,penColor2] {-x^2};

    \draw[penColor,very thick,fill=fill1] (axis cs:.48,0) ellipse (5 and 500);
    \draw[penColor2,very thick,fill=white] (axis cs:.48,0) ellipse (2.5 and 250);
    
    \draw[penColor,very thick,fill=fill1] (axis cs:.5,0) ellipse (5 and 500);
    \draw[penColor2,very thick,fill=white] (axis cs:.5,0) ellipse (2.5 and 250);

    \addplot [very thick,penColor2,domain=.5:1] {x^2};
    \addplot [very thick,penColor2,domain=.5:1] {-x^2};
    
    \draw[penColor,very thick] (axis cs:1,0) ellipse (5 and 1000);
    
    
    \draw[decoration={brace,mirror,raise=.1cm},decorate,thin] (axis cs:.47,-.5)--(axis cs:.51,-.5);  
    \node[anchor=north] at (axis cs:.49,-.6) {$\d x$};
   
    \end{axis}
\end{tikzpicture}
\end{image}
The outer radius is given by $f(x) = x$ and the inner radius is given
by $g(x)= x^2$. Hence we see that
\[
\d V = \left(\answer[given]{\pi x^2 - \pi x^4} \right)\d x. 
\]
Thus, the total volume is
\begin{align*}
  \int_0^1 \pi x^2-\pi x^4\d x &= \eval{\answer[given]{\pi\left(\frac{x^3}{3}-\frac{x^5}{5}\right)}}_0^1\\
  &=\pi\left(\frac{1}{3}-\frac{1}{5}\right)\\
  &=\frac{2\pi}{15}.
\end{align*}
\end{explanation}
\end{example}


\section{Revolving around other lines}

You might want to investigate solids of revolution that are produced
by rotating curves around lines that are not the $x$-axis or
$y$-axis. The basic concept is the same as what we have done before,
and the intrepid young mathematician could discover the technique for
themselves, if pressed. Let's see an example.

\begin{example}
  Find the volume of the object generated by rotating the area bounded
  by $x^2$, the line $x= 1$, and the $x$-axis around the line $x=3$.
  \begin{image}
    \begin{tikzpicture}[
        declare function = {f(\x) = pow(\x,2);} ]
      \begin{axis}[
          xmin =0,xmax=6,ymax=10,ymin=-1,
          axis lines=center, xlabel=$x$, ylabel=$y$,
          every axis y label/.style={at=(current axis.above origin),anchor=south},
          every axis x label/.style={at=(current axis.right of origin),anchor=west},
          axis on top,
          clip=false,
        ]
        \addplot [fill1,fill=fill1,domain=1:3, smooth] {f(x)}\closedcycle;
        \addplot [fill1,fill=fill1,domain=3:5, smooth] {f(-x+6)}\closedcycle;

        \addplot [penColor,very thick,domain=1:3, smooth] {f(x)};
        \addplot [penColor,very thick,domain=3:5, smooth] {f(-x+6)};

        \draw[penColor,very thick,fill=fill1] (axis cs:3,0) ellipse (200 and 10);

        \draw[penColor,very thick] (axis cs:3,1) ellipse (200 and 10);

        \addplot [penColor,very thick] plot coordinates {(1,0) (1,1)};

        \addplot [penColor,very thick] plot coordinates {(5,0) (5,1)};
        
        \draw[penColor,very thick] (axis cs:3,4) ellipse (100 and 5);
      \end{axis}
    \end{tikzpicture}
  \end{image}
  \begin{explanation}
    To start, note that our solid consists of \textit{two} pieces, a
    cylinder:
  \begin{image}
    \begin{tikzpicture}[
        declare function = {f(\x) = pow(\x,2);} ]
      \begin{axis}[
          xmin =0,xmax=6,ymax=10,ymin=-1,
          axis lines=center, xlabel=$x$, ylabel=$y$,
          every axis y label/.style={at=(current axis.above origin),anchor=south},
          every axis x label/.style={at=(current axis.right of origin),anchor=west},
          axis on top,
          clip=false,
        ]
        \addplot [draw=none, fill=fill1,very thick] plot coordinates {(1,0) (1,1) (5,1) (5,0)}\closedcycle;
        \draw[penColor,very thick,fill=fill1] (axis cs:3,0) ellipse (200 and 10);

        \draw[penColor,very thick,fill=fill1] (axis cs:3,1) ellipse (200 and 10);

        \addplot [penColor,very thick] plot coordinates {(1,0) (1,1)};

        \addplot [penColor,very thick] plot coordinates {(5,0) (5,1)};
        
      \end{axis}
    \end{tikzpicture}
  \end{image}
  and a pointy-shape:
  \begin{image}
    \begin{tikzpicture}[
        declare function = {f(\x) = pow(\x,2);} ]
      \begin{axis}[
          xmin =0,xmax=6,ymax=10,ymin=-1,
          axis lines=center, xlabel=$x$, ylabel=$y$,
          every axis y label/.style={at=(current axis.above origin),anchor=south},
          every axis x label/.style={at=(current axis.right of origin),anchor=west},
          axis on top,
          clip=false,
        ]
        \addplot [fill1,fill=fill1,domain=1:3, smooth] {f(x)}\closedcycle;
        \addplot [fill1,fill=fill1,domain=3:5, smooth] {f(-x+6)}\closedcycle;

        

        \addplot [penColor,very thick,domain=1:3, smooth] {f(x)};
        \addplot [penColor,very thick,domain=3:5, smooth] {f(-x+6)};


        \draw[draw=none,very thick,fill=fill1] (axis cs:3,1) ellipse (200 and 10);
        
        \draw[penColor,very thick] (axis cs:3,4) ellipse (100 and 5);

        \addplot [draw=white,fill=white,domain=1:5, smooth] {(-sqrt(4-(x-3)^2)+2)/2}\closedcycle;

        \draw[penColor,very thick] (axis cs:3,1) ellipse (200 and 10);
      \end{axis}
    \end{tikzpicture}
  \end{image}
  The volume of the cylinder is
  \[
  \text{Volume of cylinder}=\answer[given]{4\pi}
  \]
  since the radius is $2$ and the height is $1$.  On the other hand,
  to compute the volume of the pointy-shape, let's consider the
  following cross-sections:
      \begin{image}
    \begin{tikzpicture}[
        declare function = {f(\x) = pow(\x,2);} ]
      \begin{axis}[
          xmin =0,xmax=6,ymax=10,ymin=-1,
          axis lines=center, xlabel=$x$, ylabel=$y$,
          every axis y label/.style={at=(current axis.above origin),anchor=south},
          every axis x label/.style={at=(current axis.right of origin),anchor=west},
          axis on top,
          clip=false,
        ]
%        \addplot [fill1,fill=fill1,domain=1:3, smooth] {f(x)}\closedcycle;
%        \addplot [fill1,fill=fill1,domain=3:5, smooth] {f(-x+6)}\closedcycle;

        \addplot [penColor,very thick,domain=1:3, smooth] {f(x)};
        \addplot [penColor,very thick,domain=3:5, smooth] {f(-x+6)};
        
        \draw[penColor,very thick] (axis cs:3,1) ellipse (200 and 10);
         
        \draw[penColor,very thick,fill=fill1] (axis cs:3,3.6) ellipse (100 and 5);
        \draw[penColor,very thick,fill=fill1] (axis cs:3,4) ellipse (100 and 5);
         
        \draw[decoration={brace,mirror,raise=.1cm},decorate,thin] (axis cs:4,3.4)--(axis cs:4,4.2);  
        \node[anchor=west] at (axis cs:4.2,3.8) {$\d y$};
      \end{axis}
    \end{tikzpicture}
      \end{image}
      Where each radius of the cross-section is given (in terms of $y$) by
      \[
      \text{radius} = \answer[given]{3-\sqrt{y}}.
      \]
      Hence
      \[
      \d V = \pi (3-\sqrt{y})^2 \d y
      \]
      and so the volume of the pointy-shape is
      \begin{align*}
        \int_{\answer[given]{1}}^{\answer[given]{9}} \pi (3-\sqrt{y})^2 \d y
        &= \pi \int_{\answer[given]{1}}^{\answer[given]{9}} 9-6\sqrt{y}+y \d y\\
        &=\pi\cdot \eval{\answer[given]{9y-4y^{3/2} + y^2/2}}_{\answer[given]{1}}^{\answer[given]{9}}\\
        &=\pi\cdot \answer[given]{8}.
      \end{align*}
      Thus the volume of the complete object is $\answer[given]{12\pi}$.
  \end{explanation}
\end{example}


\end{document}
