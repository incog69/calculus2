\documentclass{ximera}

%\usepackage{todonotes}

\newcommand{\todo}{}

\usepackage{tkz-euclide}
\tikzset{>=stealth} %% cool arrow head
\tikzset{shorten <>/.style={ shorten >=#1, shorten <=#1 } } %% allows shorter vectors

\usetikzlibrary{backgrounds} %% for boxes around graphs
\usetikzlibrary{shapes,positioning}  %% Clouds and stars
\usetikzlibrary{matrix} %% for matrix
\usepgfplotslibrary{polar} %% for polar plots
\usetkzobj{all}
\usepackage[makeroom]{cancel} %% for strike outs
%\usepackage{mathtools} %% for pretty underbrace % Breaks Ximera
\usepackage{multicol}





\usepackage{array}
\setlength{\extrarowheight}{+.1cm}   
\newdimen\digitwidth
\settowidth\digitwidth{9}
\def\divrule#1#2{
\noalign{\moveright#1\digitwidth
\vbox{\hrule width#2\digitwidth}}}





\newcommand{\RR}{\mathbb R}
\newcommand{\R}{\mathbb R}
\newcommand{\N}{\mathbb N}
\newcommand{\Z}{\mathbb Z}

%\renewcommand{\d}{\,d\!}
\renewcommand{\d}{\mathop{}\!d}
\newcommand{\dd}[2][]{\frac{\d #1}{\d #2}}
\renewcommand{\l}{\ell}
\newcommand{\ddx}{\frac{d}{\d x}}

\newcommand{\zeroOverZero}{\ensuremath{\boldsymbol{\tfrac{0}{0}}}}
\newcommand{\inftyOverInfty}{\ensuremath{\boldsymbol{\tfrac{\infty}{\infty}}}}
\newcommand{\zeroOverInfty}{\ensuremath{\boldsymbol{\tfrac{0}{\infty}}}}
\newcommand{\zeroTimesInfty}{\ensuremath{\small\boldsymbol{0\cdot \infty}}}
\newcommand{\inftyMinusInfty}{\ensuremath{\small\boldsymbol{\infty - \infty}}}
\newcommand{\oneToInfty}{\ensuremath{\boldsymbol{1^\infty}}}
\newcommand{\zeroToZero}{\ensuremath{\boldsymbol{0^0}}}
\newcommand{\inftyToZero}{\ensuremath{\boldsymbol{\infty^0}}}


\newcommand{\numOverZero}{\ensuremath{\boldsymbol{\tfrac{\#}{0}}}}
\newcommand{\dfn}{\textbf}
%\newcommand{\unit}{\,\mathrm}
\newcommand{\unit}{\mathop{}\!\mathrm}
\newcommand{\eval}[1]{\bigg[ #1 \bigg]}
\newcommand{\seq}[1]{\left( #1 \right)}
\renewcommand{\epsilon}{\varepsilon}
\renewcommand{\iff}{\Leftrightarrow}

\DeclareMathOperator{\arccot}{arccot}
\DeclareMathOperator{\arcsec}{arcsec}
\DeclareMathOperator{\arccsc}{arccsc}
\DeclareMathOperator{\si}{Si}

\newcommand{\tightoverset}[2]{%
  \mathop{#2}\limits^{\vbox to -.5ex{\kern-0.75ex\hbox{$#1$}\vss}}}
\newcommand{\arrowvec}[1]{\tightoverset{\scriptstyle\rightharpoonup}{#1}}
\renewcommand{\vec}{\mathbf}


\colorlet{textColor}{black} 
\colorlet{background}{white}
\colorlet{penColor}{blue!50!black} % Color of a curve in a plot
\colorlet{penColor2}{red!50!black}% Color of a curve in a plot
\colorlet{penColor3}{red!50!blue} % Color of a curve in a plot
\colorlet{penColor4}{green!50!black} % Color of a curve in a plot
\colorlet{penColor5}{orange!80!black} % Color of a curve in a plot
\colorlet{fill1}{penColor!20} % Color of fill in a plot
\colorlet{fill2}{penColor2!20} % Color of fill in a plot
\colorlet{fillp}{fill1} % Color of positive area
\colorlet{filln}{penColor2!20} % Color of negative area
\colorlet{fill3}{penColor3!20} % Fill
\colorlet{fill4}{penColor4!20} % Fill
\colorlet{fill5}{penColor5!20} % Fill
\colorlet{gridColor}{gray!50} % Color of grid in a plot

\newcommand{\surfaceColor}{violet}
\newcommand{\surfaceColorTwo}{redyellow}
\newcommand{\sliceColor}{greenyellow}




\pgfmathdeclarefunction{gauss}{2}{% gives gaussian
  \pgfmathparse{1/(#2*sqrt(2*pi))*exp(-((x-#1)^2)/(2*#2^2))}%
}


%%%%%%%%%%%%%
%% Vectors
%%%%%%%%%%%%%

%% Simple horiz vectors
\renewcommand{\vector}[1]{\left\langle #1\right\rangle}


%% %% Complex Horiz Vectors with angle brackets
%% \makeatletter
%% \renewcommand{\vector}[2][ , ]{\left\langle%
%%   \def\nextitem{\def\nextitem{#1}}%
%%   \@for \el:=#2\do{\nextitem\el}\right\rangle%
%% }
%% \makeatother

%% %% Vertical Vectors
%% \def\vector#1{\begin{bmatrix}\vecListA#1,,\end{bmatrix}}
%% \def\vecListA#1,{\if,#1,\else #1\cr \expandafter \vecListA \fi}

%%%%%%%%%%%%%
%% End of vectors
%%%%%%%%%%%%%

%\newcommand{\fullwidth}{}
%\newcommand{\normalwidth}{}



%% makes a snazzy t-chart for evaluating functions
%\newenvironment{tchart}{\rowcolors{2}{}{background!90!textColor}\array}{\endarray}

%%This is to help with formatting on future title pages.
\newenvironment{sectionOutcomes}{}{} 



%% Flowchart stuff
%\tikzstyle{startstop} = [rectangle, rounded corners, minimum width=3cm, minimum height=1cm,text centered, draw=black]
%\tikzstyle{question} = [rectangle, minimum width=3cm, minimum height=1cm, text centered, draw=black]
%\tikzstyle{decision} = [trapezium, trapezium left angle=70, trapezium right angle=110, minimum width=3cm, minimum height=1cm, text centered, draw=black]
%\tikzstyle{question} = [rectangle, rounded corners, minimum width=3cm, minimum height=1cm,text centered, draw=black]
%\tikzstyle{process} = [rectangle, minimum width=3cm, minimum height=1cm, text centered, draw=black]
%\tikzstyle{decision} = [trapezium, trapezium left angle=70, trapezium right angle=110, minimum width=3cm, minimum height=1cm, text centered, draw=black]



\outcome{State the definition of a vector.}
\outcome{Work with vectors in two or three dimensions. }
\outcome{Multiply vectors by scalars.}
\outcome{Add and subtract vectors.}
\outcome{Calculate the magnitude of a vector.}
\outcome{Find unit vectors.}
\outcome{Use vectors in applied settings.}
\outcome{Find the equations for planes.}
\outcome{Give the equation for a sphere or ball.}

\title[Dig-In:]{Vectors}

\begin{document}
\begin{abstract}
  Vectors are lists of numbers
\end{abstract}
\maketitle

So far, we have mostly studied functions which take one input and
return one output.  \textbf{Multivariable Calculus} is the study of
functions with multiple inputs and multiple outputs.  In these final
sections, we set the stage for this study.  When we want to keep track
of more than one number at a time, we use a \textbf{vector}.

\begin{definition}
  A \dfn{vector} of length $n$ is an ordered list of $n$ real numbers.
  The collection of all vectors of length $n$ is called $\mathbb{R}^n$
  (pronounced ``Are enn'').  The $n$ in $\mathbb{R}^n$ is called the
  \dfn{dimension}.
\end{definition}

We write vectors typographically as either boldfaced variables (like
$\mathbf{v}$ or $\mathbf{w}$) or decorated with an arrow hat (like
$\vec{v}$ or $\vec{w}$).  We also generally write vectors as vertical
lists like this:

%\[
%\mathbf{v} = \vecList{3;4;1}
%\]
but we sometimes will write the same vector horizontally like this

\[
\mathbf{v} = \langle 3, 4, 1 \rangle
\]

\begin{question}
  What is the dimension of the vector 
  \[
  \begin{bmatrix} 3 \\ 4 \\1 \\ -4 \end{bmatrix}
  \]
  \[
  \textrm{Dimension} = \answer{4}
  \]
\end{question}

\section{Operations on vectors}

We can add vectors of the same dimension together by summing component by component:

\begin{question}
  \[
  \begin{bmatrix} 1 \\ 2 \\3  \end{bmatrix}+ \begin{bmatrix} -1\\ 2 \\ 2  \end{bmatrix} = \begin{bmatrix} \answer{0} \\ \answer{4} \\ \answer{5}  \end{bmatrix}
  \]
  \begin{hint}
    The answer is  
    \[
    \begin{bmatrix} 0 \\ 4 \\ 5 \end{bmatrix}
    \]
  \end{hint}
\end{question}

We can also multiply vectors by a constant, by multiplying each component by by the constant:

\begin{question}

\[
4\begin{bmatrix} 2 \\ 4 \\ 0  \\ 1 \end{bmatrix} = \begin{bmatrix} \answer{8}\\ \answer{16} \\ \answer{0}  \\ \answer{4} \end{bmatrix} 
\]	

\begin{hint}
\[
 \begin{bmatrix} 8 \\ 16 \\ 0  \\ 4 \end{bmatrix} 
 \]
\end{hint}
\end{question}

In this context, we call a number to which we multiply a vector a \dfn{scalar} for reasons that will become apparent in the next section.

\section{The geometry of vectors}

We can visualize vectors in dimension $1$, $2$, or $3$ as directed arrows.


\begin{question}
	Which vector is depicted by the following picture?
	
\begin{image}
  \begin{tikzpicture}
	\begin{axis}[
            domain=(-1:6),
            clip=false,
            axis lines=center,
            %ticks=none,
            unit vector ratio*=1 1 1,
            xlabel=$x$, ylabel=$y$,
            ytick={-2,-1,...,7},
	    %yticklabels={$0.5$,$1$,$1.5$,$2$},
	    xtick={-2,-1,...,10},
	    %xticklabels={$0.5$,$1$,$1.5$,$2$},
	    grid = major,
            every axis y label/.style={at=(current axis.above origin),anchor=south},
            every axis x label/.style={at=(current axis.right of origin),anchor=west},
          ]
         % \addplot[very thick,penColor2!50!white,->,>=stealth'] plot coordinates {(2,1) (6,4)};
          \addplot[very thick,penColor2,->,>=stealth'] plot coordinates {(0,0) (2,3)};
          \addplot[color=penColor,fill=penColor] coordinates{(6,6)};  %% closed hole
         % \addplot[color=penColor,dashed] ({2+4*x},{1+3*x});
        \end{axis}
\end{tikzpicture}
\end{image}
	
	\[
	\begin{bmatrix} 2  \\ 3  \end{bmatrix}
	\]
\end{question}

\begin{question}
	What is the length of the vector 
	\[
		\begin{bmatrix} 2  \\ 3  \end{bmatrix}
	\]
	
	\[
	\textrm{Length}  = \answer{\sqrt{13}}
	\]
	
	\begin{hint}
		The length is $\sqrt{2^2+3^2} = \sqrt{13}$
	\end{hint}
\end{question}

You were able to find the answer to the question above because you are used to working with $2$ and $3$ dimensional objects.  We make the following definition in $n$ dimensions.

\begin{definition}
	Let $\mathbf{v} = \langle v_1, v_2, v_3, \dots, v_n \rangle \in \mathbb{R}^n$ is an $n$ dimensional vector.  Then the \dfn{length} or \dfn{magnitude} of $\mathbf{v}$ is denoted by $|\mathbf{v}|$ and is defined by
	
	\[
	|\mathbf{v}| = \sqrt{v_1^2+v_2^2+v_3^2+\dots+v_n^2}
	\]
\end{definition}

We can understand the operations of scalar multiplication and vector addition from this graphical perspective as well:

Let

\[
\mathbf{v} = \begin{bmatrix} 1  \\ 2  \end{bmatrix}
\]

Then 

\[
2\mathbf{v} =  \begin{bmatrix} 2  \\ 4  \end{bmatrix}
\]

If we graph both of these, we see that $2\mathbf{v}$ points in the same direction as $\mathbf{v}$, but is twice as long.  In other words it \textit{scales} the vector by a factor of $2$ (which explains why we call numbers ``scalars'' in this context). 


\begin{image}
  \begin{tikzpicture}
	\begin{axis}[
            domain=(-1:2),
            clip=false,
            axis lines=center,
            %ticks=none,
            unit vector ratio*=1 1 1,
            xlabel=$x$, ylabel=$y$,
            ytick={-2,-1,...,7},
	    %yticklabels={$0.5$,$1$,$1.5$,$2$},
	    xtick={-2,-1,...,10},
	    %xticklabels={$0.5$,$1$,$1.5$,$2$},
	    grid = major,
            every axis y label/.style={at=(current axis.above origin),anchor=south},
            every axis x label/.style={at=(current axis.right of origin),anchor=west},
          ]
          \addplot[very  thick,penColor,->,>=stealth'] plot coordinates {(0,0) (2,4)};
          \addplot[ thick,penColor2,->,>=stealth'] plot coordinates {(0,0) (1,2)};
          \addplot[color=penColor,fill=penColor] coordinates{(6,6)};  %% closed hole
        \end{axis}
\end{tikzpicture}
\end{image}


\begin{observation}
	If $c$ is a positive constant, and $\mathbf{v}$ is a vector, then vector $c\mathbf{v}$ points in the same direction as $\mathbf{v}$, but its length is scaled by a factor of $c$.  If $c$ is negative, then $c\mathbf{v}$ points in the opposite direction of $\mathbf{v}$, and its length is scaled by a factor of $|c|$.
\end{observation}

\begin{definition}
	A \dfn{unit vector} is a vector of length $1$.
\end{definition}

\begin{question}
	Find a unit vector $\mathbf{u}$ which points in the same direction as the vector $\mathbf{v} = \langle 2,1,3,7,1\rangle$ .
	
	\[
	\begin{bmatrix}
	\answer{2/8}\\
	\answer{1/8}\\
	\answer{3/8}\\
	\answer{7/8}\\
	\answer{1/8}
	\end{bmatrix}
	\]
	
	\begin{hint}
		Scaling the vector $\mathbf{v}$ by the reciprocal of its length should result in a length $1$ vector which points in the same direction
	\end{hint}
	
	\begin{hint}
		$|\mathbf{v}| = \sqrt{2^2+1^2+3^2+7^2+1^2} = \sqrt{64} = 8$
	\end{hint}
	
	\begin{hint}
		Thus the vector $ \mathbf{u} = \langle \frac{2}{8},  \frac{1}{8},\frac{3}{8},\frac{7}{8},\frac{1}{8}\rangle$ points in the same direction of $\mathbf{v}$, but is of unit length.
	\end{hint}
\end{question}

\begin{observation}
	If $\mathbf{v}$ is a nonzero vector, then the unit vector which points in the same direction as $\mathbf{v}$ is $\frac{1}{|\mathbf{v}|} \mathbf{v}$
\end{observation}

Now let us investigate the geometry of addition of vectors:

\[
\mathbf{v} = \begin{bmatrix} 1  \\ 2  \end{bmatrix} \hphantom{fdsfd} \mathbf{w} = \begin{bmatrix} 3  \\ 1  \end{bmatrix}
\]

Then

\[
\mathbf{v}+\mathbf{w} = \begin{bmatrix} 4 \\ 3 \end{bmatrix}
\]

\begin{image}
  \begin{tikzpicture}
	\begin{axis}[
            domain=(-1:2),
            clip=false,
            axis lines=center,
            %ticks=none,
            unit vector ratio*=1 1 1,
            xlabel=$x$, ylabel=$y$,
            ytick={-2,-1,...,7},
	    %yticklabels={$0.5$,$1$,$1.5$,$2$},
	    xtick={-2,-1,...,10},
	    %xticklabels={$0.5$,$1$,$1.5$,$2$},
	    grid = major,
            every axis y label/.style={at=(current axis.above origin),anchor=south},
            every axis x label/.style={at=(current axis.right of origin),anchor=west},
          ]
          \addplot[very  thick,penColor,->,>=stealth'] plot coordinates {(1,2) (4,3)};
          \addplot[ thick,penColor2,->,>=stealth'] plot coordinates {(0,0) (1,2)};
          \addplot[ thick,penColor3,->,>=stealth'] plot coordinates {(0,0) (4,3)};

           \node at (axis cs:0, 1.2 ) [penColor2,anchor=west] {$\mathbf{v}$};
            \node at (axis cs:2, 2.7 ) [penColor,anchor=west] {$\mathbf{w}$};
            \node at (axis cs:1.9, 1.1 ) [penColor3,anchor=west] {$\mathbf{v}+\mathbf{w}$};

          \addplot[color=penColor,fill=penColor] coordinates{(6,6)};  %% closed hole
        \end{axis}
\end{tikzpicture}
\end{image}


\begin{observation}
	If we place the base of the vector $\mathbf{w}$ at the head of the vector $\mathbf{v}$, then the sum $\mathbf{v}+\mathbf{w}$ connects the base of $\mathbf{v}$ to the head of $\mathbf{w}$
\end{observation}

\section{Some three dimensional geometry}

In three dimensions we have three coordinates axes, the $x$, $y$, and $z$ axis:

\begin{image}
\begin{tikzpicture}
\begin{axis}[
  view={35}{15},
  axis lines=center,
  width=15cm,height=15cm,
  xtick={-10,-5,5,10},ytick={-10,-5,5,10},ztick={-10,-5,5,10},
  minor tick={-12,-11,...,12},
  xmin=-11,xmax=11,ymin=-11,ymax=11,zmin=-11,zmax=11,
 % xlabel={$x$},ylabel={$y$},zlabel={$z$},
]


\node [above right] at (axis cs:10,0,0) {$x$};
\node [above right] at (axis cs:0,10,0) {$y$};
\node [above right] at (axis cs:0,0,10) {$z$};


\end{axis}
\end{tikzpicture}
\end{image}

Together, these form three coordinate planes (The $xy$-plane, the $xz$-plane, and the $yz$-plane).

\begin{question}
	The $yz$-plane corresponds to which of the following equations?
	
	\begin{multipleChoice}
		\choice[correct]{$x=0$}
		\choice{$y=0$}
		\choice{$z=0$}
	\end{multipleChoice}
	
	\begin{hint}
		Every point on the $yz$ axis has $x=0$, so this is the answer.
	\end{hint}
\end{question}

\begin{question}
	Which of the following most accurately describes the solution set of $y=2$ in $\mathbb{R}^3$?
	
	\begin{multipleChoice}
		\choice{A plane parallel to the $xy$ plane}
		\choice[correct]{A plane parallel to the $xz$ plane}
		\choice{A plane parallel to the $yz$ plane}
	\end{multipleChoice}

\begin{hint}
	$y=2$ consists of all those points where $y=2$, but $x$ and $z$ are allowed to be anything.  This corresponds to a plane which is parallel to the $xz$ plane ($y=0$).
\end{hint}
	
\end{question}

\begin{question}
	The equation $(x-1)^2+y^2+(z+2)^2 = 4$ has a solution set in $\mathbb{R}^3$ which is a sphere.  What is the center and radius of this sphere?
	
\[
\textrm{Radius} = \answer{2}
\]

\[
\textrm{Center} = \left(\answer{1},\answer{0}, \answer{-2} \right)
\]

\begin{hint}
	$(x-1)^2+y^2+(z+2)^2$ is the square of the distance from $(1,0,-2)$ to $(x,y,z)$.  If the square of the distance is $4$, then the distance is $2$.  Since the solution set of this equation is all points which are a distance of $2$ away from $(1,0,-2)$, then this is a sphere of radius $2$ centered at $(1,0,-2)$
\end{hint}
\end{question}

\end{document}
