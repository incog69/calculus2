\documentclass{ximera}

%\usepackage{todonotes}

\newcommand{\todo}{}

\usepackage{tkz-euclide}
\tikzset{>=stealth} %% cool arrow head
\tikzset{shorten <>/.style={ shorten >=#1, shorten <=#1 } } %% allows shorter vectors

\usetikzlibrary{backgrounds} %% for boxes around graphs
\usetikzlibrary{shapes,positioning}  %% Clouds and stars
\usetikzlibrary{matrix} %% for matrix
\usepgfplotslibrary{polar} %% for polar plots
\usetkzobj{all}
\usepackage[makeroom]{cancel} %% for strike outs
%\usepackage{mathtools} %% for pretty underbrace % Breaks Ximera
\usepackage{multicol}





\usepackage{array}
\setlength{\extrarowheight}{+.1cm}   
\newdimen\digitwidth
\settowidth\digitwidth{9}
\def\divrule#1#2{
\noalign{\moveright#1\digitwidth
\vbox{\hrule width#2\digitwidth}}}





\newcommand{\RR}{\mathbb R}
\newcommand{\R}{\mathbb R}
\newcommand{\N}{\mathbb N}
\newcommand{\Z}{\mathbb Z}

%\renewcommand{\d}{\,d\!}
\renewcommand{\d}{\mathop{}\!d}
\newcommand{\dd}[2][]{\frac{\d #1}{\d #2}}
\renewcommand{\l}{\ell}
\newcommand{\ddx}{\frac{d}{\d x}}

\newcommand{\zeroOverZero}{\ensuremath{\boldsymbol{\tfrac{0}{0}}}}
\newcommand{\inftyOverInfty}{\ensuremath{\boldsymbol{\tfrac{\infty}{\infty}}}}
\newcommand{\zeroOverInfty}{\ensuremath{\boldsymbol{\tfrac{0}{\infty}}}}
\newcommand{\zeroTimesInfty}{\ensuremath{\small\boldsymbol{0\cdot \infty}}}
\newcommand{\inftyMinusInfty}{\ensuremath{\small\boldsymbol{\infty - \infty}}}
\newcommand{\oneToInfty}{\ensuremath{\boldsymbol{1^\infty}}}
\newcommand{\zeroToZero}{\ensuremath{\boldsymbol{0^0}}}
\newcommand{\inftyToZero}{\ensuremath{\boldsymbol{\infty^0}}}


\newcommand{\numOverZero}{\ensuremath{\boldsymbol{\tfrac{\#}{0}}}}
\newcommand{\dfn}{\textbf}
%\newcommand{\unit}{\,\mathrm}
\newcommand{\unit}{\mathop{}\!\mathrm}
\newcommand{\eval}[1]{\bigg[ #1 \bigg]}
\newcommand{\seq}[1]{\left( #1 \right)}
\renewcommand{\epsilon}{\varepsilon}
\renewcommand{\iff}{\Leftrightarrow}

\DeclareMathOperator{\arccot}{arccot}
\DeclareMathOperator{\arcsec}{arcsec}
\DeclareMathOperator{\arccsc}{arccsc}
\DeclareMathOperator{\si}{Si}

\newcommand{\tightoverset}[2]{%
  \mathop{#2}\limits^{\vbox to -.5ex{\kern-0.75ex\hbox{$#1$}\vss}}}
\newcommand{\arrowvec}[1]{\tightoverset{\scriptstyle\rightharpoonup}{#1}}
\renewcommand{\vec}{\mathbf}


\colorlet{textColor}{black} 
\colorlet{background}{white}
\colorlet{penColor}{blue!50!black} % Color of a curve in a plot
\colorlet{penColor2}{red!50!black}% Color of a curve in a plot
\colorlet{penColor3}{red!50!blue} % Color of a curve in a plot
\colorlet{penColor4}{green!50!black} % Color of a curve in a plot
\colorlet{penColor5}{orange!80!black} % Color of a curve in a plot
\colorlet{fill1}{penColor!20} % Color of fill in a plot
\colorlet{fill2}{penColor2!20} % Color of fill in a plot
\colorlet{fillp}{fill1} % Color of positive area
\colorlet{filln}{penColor2!20} % Color of negative area
\colorlet{fill3}{penColor3!20} % Fill
\colorlet{fill4}{penColor4!20} % Fill
\colorlet{fill5}{penColor5!20} % Fill
\colorlet{gridColor}{gray!50} % Color of grid in a plot

\newcommand{\surfaceColor}{violet}
\newcommand{\surfaceColorTwo}{redyellow}
\newcommand{\sliceColor}{greenyellow}




\pgfmathdeclarefunction{gauss}{2}{% gives gaussian
  \pgfmathparse{1/(#2*sqrt(2*pi))*exp(-((x-#1)^2)/(2*#2^2))}%
}


%%%%%%%%%%%%%
%% Vectors
%%%%%%%%%%%%%

%% Simple horiz vectors
\renewcommand{\vector}[1]{\left\langle #1\right\rangle}


%% %% Complex Horiz Vectors with angle brackets
%% \makeatletter
%% \renewcommand{\vector}[2][ , ]{\left\langle%
%%   \def\nextitem{\def\nextitem{#1}}%
%%   \@for \el:=#2\do{\nextitem\el}\right\rangle%
%% }
%% \makeatother

%% %% Vertical Vectors
%% \def\vector#1{\begin{bmatrix}\vecListA#1,,\end{bmatrix}}
%% \def\vecListA#1,{\if,#1,\else #1\cr \expandafter \vecListA \fi}

%%%%%%%%%%%%%
%% End of vectors
%%%%%%%%%%%%%

%\newcommand{\fullwidth}{}
%\newcommand{\normalwidth}{}



%% makes a snazzy t-chart for evaluating functions
%\newenvironment{tchart}{\rowcolors{2}{}{background!90!textColor}\array}{\endarray}

%%This is to help with formatting on future title pages.
\newenvironment{sectionOutcomes}{}{} 



%% Flowchart stuff
%\tikzstyle{startstop} = [rectangle, rounded corners, minimum width=3cm, minimum height=1cm,text centered, draw=black]
%\tikzstyle{question} = [rectangle, minimum width=3cm, minimum height=1cm, text centered, draw=black]
%\tikzstyle{decision} = [trapezium, trapezium left angle=70, trapezium right angle=110, minimum width=3cm, minimum height=1cm, text centered, draw=black]
%\tikzstyle{question} = [rectangle, rounded corners, minimum width=3cm, minimum height=1cm,text centered, draw=black]
%\tikzstyle{process} = [rectangle, minimum width=3cm, minimum height=1cm, text centered, draw=black]
%\tikzstyle{decision} = [trapezium, trapezium left angle=70, trapezium right angle=110, minimum width=3cm, minimum height=1cm, text centered, draw=black]


\title[Dig-In:]{The Divergence Test}

\begin{document}
\begin{abstract}
If an infinite sum converges, then its terms must tend to zero.
\end{abstract}
\maketitle


\begin{question}
If we have a sequence $a_k$, $k=0,1,2, \dots $ we know how to generate the sequence of partial sum $S_n = \sum_0^n a_k$.

What about going backwards?

For instance, if I know that $S_n = 3-\frac{4}{n+1}$ is the sequence of partial sums for some series $a_k$, can you find a formula for $a_k$?

\[
a_0 = \answer{-1}
\]

\[
a_k = \answer{\frac{4}{k+1} - \frac{4}{k}} \text{when $k>0$}
\]

\[
\sum_0^\infty a_k = \answer{3}
\]


\begin{hint}
$S_0 = a_0$, so $a_0 = 3-\frac{4}{0+1} = -1$.
\end{hint}

\begin{hint}
$S_k = a_0+a_1+ \dots +a_{k-1}+a_k = S_{k-1}+a_k$.  So $a_k = S_k - S_{k-1}$
\end{hint}

\begin{hint}
Thus $a_k = (3-\frac{4}{k+1})  - (3-\frac{4}{k}) = \frac{4}{k+1} - \frac{4}{k}$
\end{hint}


\begin{hint}
 $\sum_0^\infty a_k $ is defined as the limit of its sequence of partial sums.  Thus it is equal to $\lim_{n \to \infty} 3-\frac{4}{n+1} = 3$
\end{hint}

\end{question}

\begin{explanation}
In the last question, we used the key idea that if $a_k$ is a sequence, and $S_n = \sum_0^n a_k $ is its sequence of partial sums, then $a_k = S_k-S_{k-1}$.

Suppose that $\sum_0^\infty a_k$ converges to $L$.  Then we can use this equation to learn something about the sequence $a_k$.

\begin{align*}
	a_k &= S_k-S_{k-1}\\
	\lim_{k \to \infty} a_k &= \lim_{k \to \infty} S_k -  \lim_{k \to \infty} S_{k-1}\\
	\lim_{k \to \infty} a_k &= \answer{L} - \answer{L}\\
	\lim_{k \to \infty} a_k &= \answer{0}
\end{align*}

\begin{hint}
	$\lim_{k \to \infty} S_{k}$ is the definition of $\sum_0^\infty a_k$, so it is equal to $L$.  $S_{k-1}$ is the same sequence, just shifted by one slot, so it has the same limit of $L$.  Thus their difference is $0$.  So $\lim_{k \to \infty} a_k = 0$.
\end{hint}
\end{explanation}

We have just proven the following theorem:

\begin{theorem}[Divergence Test]
	If $\sum_0^\infty a_k$ is a convergent series, then $\lim_{k \to \infty} a_k = 0$.  In words ``The sequence of terms of any convergent series must tend to zero''.
\end{theorem}

\begin{question}
Which of the following statements are true?  Mark all that apply.

\begin{multipleChoice}
	\choice[correct]{If $\sum_0^\infty a_k$ is convergent, then $\lim_{k \to \infty} a_k = 0$ }
	\choice{If $a_k \to 0$ as $k \to \infty$, then $\sum_0^\infty a_k$ is convergent}
	\choice{If $\sum_0^\infty a_k$ is divergent, then $\lim_{k \to \infty} a_k \neq 0$ }
	\choice[correct]{If  $\lim_{k \to \infty} a_k \neq 0$, then $\sum_0^\infty a_k$ is divergent}
\end{multipleChoice}


\begin{warning}
This question is important!  Mistakes in this logic are among the most common mistakes made by calculus students answering questions about sequences and series.
\end{warning}

\end{question}

\begin{question}
	We say that a series ``passes the divergence test'' if its sequence of terms tends to zero.  Which of the following series pass the divergence test?

\begin{multipleChoice}
	\choice[correct]{$\sum_3^\infty \frac{1}{\ln{ n }}$}
	\choice{$\sum_0^\infty \sin(n)$}
	\choice[correct]{$\sum_0^\infty \frac{\sin(n)}{n^2}$}
	\choice[correct]{$\sum_5^\infty \frac{n+7}{n+6} - \frac{\sin(n)}{n}$}
	\choice{$\sum_0^\infty \frac{2n}{n - 5}$}
\end{multipleChoice}

\end{question}



\end{document}


