\documentclass{ximera}

%\usepackage{todonotes}

\newcommand{\todo}{}

\usepackage{tkz-euclide}
\tikzset{>=stealth} %% cool arrow head
\tikzset{shorten <>/.style={ shorten >=#1, shorten <=#1 } } %% allows shorter vectors

\usetikzlibrary{backgrounds} %% for boxes around graphs
\usetikzlibrary{shapes,positioning}  %% Clouds and stars
\usetikzlibrary{matrix} %% for matrix
\usepgfplotslibrary{polar} %% for polar plots
\usetkzobj{all}
\usepackage[makeroom]{cancel} %% for strike outs
%\usepackage{mathtools} %% for pretty underbrace % Breaks Ximera
\usepackage{multicol}





\usepackage{array}
\setlength{\extrarowheight}{+.1cm}   
\newdimen\digitwidth
\settowidth\digitwidth{9}
\def\divrule#1#2{
\noalign{\moveright#1\digitwidth
\vbox{\hrule width#2\digitwidth}}}





\newcommand{\RR}{\mathbb R}
\newcommand{\R}{\mathbb R}
\newcommand{\N}{\mathbb N}
\newcommand{\Z}{\mathbb Z}

%\renewcommand{\d}{\,d\!}
\renewcommand{\d}{\mathop{}\!d}
\newcommand{\dd}[2][]{\frac{\d #1}{\d #2}}
\renewcommand{\l}{\ell}
\newcommand{\ddx}{\frac{d}{\d x}}

\newcommand{\zeroOverZero}{\ensuremath{\boldsymbol{\tfrac{0}{0}}}}
\newcommand{\inftyOverInfty}{\ensuremath{\boldsymbol{\tfrac{\infty}{\infty}}}}
\newcommand{\zeroOverInfty}{\ensuremath{\boldsymbol{\tfrac{0}{\infty}}}}
\newcommand{\zeroTimesInfty}{\ensuremath{\small\boldsymbol{0\cdot \infty}}}
\newcommand{\inftyMinusInfty}{\ensuremath{\small\boldsymbol{\infty - \infty}}}
\newcommand{\oneToInfty}{\ensuremath{\boldsymbol{1^\infty}}}
\newcommand{\zeroToZero}{\ensuremath{\boldsymbol{0^0}}}
\newcommand{\inftyToZero}{\ensuremath{\boldsymbol{\infty^0}}}


\newcommand{\numOverZero}{\ensuremath{\boldsymbol{\tfrac{\#}{0}}}}
\newcommand{\dfn}{\textbf}
%\newcommand{\unit}{\,\mathrm}
\newcommand{\unit}{\mathop{}\!\mathrm}
\newcommand{\eval}[1]{\bigg[ #1 \bigg]}
\newcommand{\seq}[1]{\left( #1 \right)}
\renewcommand{\epsilon}{\varepsilon}
\renewcommand{\iff}{\Leftrightarrow}

\DeclareMathOperator{\arccot}{arccot}
\DeclareMathOperator{\arcsec}{arcsec}
\DeclareMathOperator{\arccsc}{arccsc}
\DeclareMathOperator{\si}{Si}

\newcommand{\tightoverset}[2]{%
  \mathop{#2}\limits^{\vbox to -.5ex{\kern-0.75ex\hbox{$#1$}\vss}}}
\newcommand{\arrowvec}[1]{\tightoverset{\scriptstyle\rightharpoonup}{#1}}
\renewcommand{\vec}{\mathbf}


\colorlet{textColor}{black} 
\colorlet{background}{white}
\colorlet{penColor}{blue!50!black} % Color of a curve in a plot
\colorlet{penColor2}{red!50!black}% Color of a curve in a plot
\colorlet{penColor3}{red!50!blue} % Color of a curve in a plot
\colorlet{penColor4}{green!50!black} % Color of a curve in a plot
\colorlet{penColor5}{orange!80!black} % Color of a curve in a plot
\colorlet{fill1}{penColor!20} % Color of fill in a plot
\colorlet{fill2}{penColor2!20} % Color of fill in a plot
\colorlet{fillp}{fill1} % Color of positive area
\colorlet{filln}{penColor2!20} % Color of negative area
\colorlet{fill3}{penColor3!20} % Fill
\colorlet{fill4}{penColor4!20} % Fill
\colorlet{fill5}{penColor5!20} % Fill
\colorlet{gridColor}{gray!50} % Color of grid in a plot

\newcommand{\surfaceColor}{violet}
\newcommand{\surfaceColorTwo}{redyellow}
\newcommand{\sliceColor}{greenyellow}




\pgfmathdeclarefunction{gauss}{2}{% gives gaussian
  \pgfmathparse{1/(#2*sqrt(2*pi))*exp(-((x-#1)^2)/(2*#2^2))}%
}


%%%%%%%%%%%%%
%% Vectors
%%%%%%%%%%%%%

%% Simple horiz vectors
\renewcommand{\vector}[1]{\left\langle #1\right\rangle}


%% %% Complex Horiz Vectors with angle brackets
%% \makeatletter
%% \renewcommand{\vector}[2][ , ]{\left\langle%
%%   \def\nextitem{\def\nextitem{#1}}%
%%   \@for \el:=#2\do{\nextitem\el}\right\rangle%
%% }
%% \makeatother

%% %% Vertical Vectors
%% \def\vector#1{\begin{bmatrix}\vecListA#1,,\end{bmatrix}}
%% \def\vecListA#1,{\if,#1,\else #1\cr \expandafter \vecListA \fi}

%%%%%%%%%%%%%
%% End of vectors
%%%%%%%%%%%%%

%\newcommand{\fullwidth}{}
%\newcommand{\normalwidth}{}



%% makes a snazzy t-chart for evaluating functions
%\newenvironment{tchart}{\rowcolors{2}{}{background!90!textColor}\array}{\endarray}

%%This is to help with formatting on future title pages.
\newenvironment{sectionOutcomes}{}{} 



%% Flowchart stuff
%\tikzstyle{startstop} = [rectangle, rounded corners, minimum width=3cm, minimum height=1cm,text centered, draw=black]
%\tikzstyle{question} = [rectangle, minimum width=3cm, minimum height=1cm, text centered, draw=black]
%\tikzstyle{decision} = [trapezium, trapezium left angle=70, trapezium right angle=110, minimum width=3cm, minimum height=1cm, text centered, draw=black]
%\tikzstyle{question} = [rectangle, rounded corners, minimum width=3cm, minimum height=1cm,text centered, draw=black]
%\tikzstyle{process} = [rectangle, minimum width=3cm, minimum height=1cm, text centered, draw=black]
%\tikzstyle{decision} = [trapezium, trapezium left angle=70, trapezium right angle=110, minimum width=3cm, minimum height=1cm, text centered, draw=black]


\title[Dig-In:]{The Integral Test}

\begin{document}
\begin{abstract}
Infinite sums can be viewed as improper integrals
\end{abstract}
\maketitle

We have seen that we can graph a sequence as a collection of points in the plane.

For instance, the  a sequence $a_k$ has a graph which looks like this: 
 
\begin{image}
\begin{tikzpicture}
	\begin{axis}[
            domain=0:6,xmin=0,xmax=6,ymin=0,ymax=2,
            width=4in,
            height=2in,
            xtick={1,2,...,5},
            ytick={1.4,.7,.35,.175,.0875},
            yticklabels={},%$a_1 = 10$,$a_2=30$,$a_3=90$,$a_4=270$,$a_5=810$},
            axis lines =middle, xlabel=$n$, ylabel=$a$,
            every axis y label/.style={at=(current axis.above origin),anchor=south},
            every axis x label/.style={at=(current axis.right of origin),anchor=west},
            clip=false,
            %axis on top,
          ]
          \addplot[color=penColor,fill=penColor,only marks,mark=*] coordinates{(1,7/5)};  %% closed hole          
          \addplot[color=penColor,fill=penColor,only marks,mark=*] coordinates{(2,7/10)};  %% closed hole          
          \addplot[color=penColor,fill=penColor,only marks,mark=*] coordinates{(3,7/20)};  %% closed hole          
          \addplot[color=penColor,fill=penColor,only marks,mark=*] coordinates{(4,7/40)};  %% closed hole          
          \addplot[color=penColor,fill=penColor,only marks,mark=*] coordinates{(5,7/80)};  %% closed hole  
        \end{axis}
\end{tikzpicture}
\end{image}

Is there a nice way to visualize the sum $\sum_1^n a_k$?

One answer is just that it is the sum of all of the heights of these points.  That is a bit unsatisfying.

Another way to visualize the sum is to make rectangles!

\begin{image}
\begin{tikzpicture}
	\begin{axis}[
            domain=0:6,xmin=0,xmax=6,ymin=0,ymax=2,
            width=4in,
            height=2in,
            xtick={1,2,...,5},
            ytick={1.4,.7,.35,.175,.0875},
            yticklabels={},%$a_1 = 10$,$a_2=30$,$a_3=90$,$a_4=270$,$a_5=810$},
            axis lines =middle, xlabel=$n$, ylabel=$a$,
            every axis y label/.style={at=(current axis.above origin),anchor=south},
            every axis x label/.style={at=(current axis.right of origin),anchor=west},
            clip=false,
            %axis on top,
          ]
          \addplot[color=penColor,fill=penColor,only marks,mark=*] coordinates{(1,7/5)};  %% closed hole
		\addplot [draw=penColor, fill = fillp] plot coordinates {(0,0) (1,0) (1, 7/5) (0,7/5) (0, 0)};          
          
	\addplot[color=penColor,fill=penColor,only marks,mark=*] coordinates{(2,7/10)};  %% closed hole
		\addplot [draw=penColor, fill = fillp] plot coordinates {(1,0) (2,0) (2, 7/10) (1,7/10) (1, 0)};          
          
          \addplot[color=penColor,fill=penColor,only marks,mark=*] coordinates{(3,7/20)};  %% closed hole
          	\addplot [draw=penColor, fill = fillp] plot coordinates {(2,0) (3,0) (3, 7/20) (2,7/20) (2, 0)};          

          
          \addplot[color=penColor,fill=penColor,only marks,mark=*] coordinates{(4,7/40)};  %% closed hole        
          	\addplot [draw=penColor, fill = fillp] plot coordinates {(3,0) (4,0) (4, 7/40) (3,7/40) (3, 0)};          

		  
          \addplot[color=penColor,fill=penColor,only marks,mark=*] coordinates{(5,7/80)};  %% closed hole 
          	\addplot [draw=penColor, fill = fillp] plot coordinates {(4,0) (5,0) (5, 7/80) (4,7/80) (4, 0)};          
 
        \end{axis}
\end{tikzpicture}
\end{image}

This lets us visually compare the sum of an infinite series to the value of an improper integral.

For instance, we can see that this infinite series must have a value which is less than $a_0+\int_1^\infty f(x) \d x$:

\begin{image}
\begin{tikzpicture}
	\begin{axis}[
            domain=0:6,xmin=0,xmax=6,ymin=0,ymax=2,
            width=4in,
            height=2in,
            xtick={1,2,...,5},
            ytick={1.4,.7,.35,.175,.0875},
            yticklabels={},%$a_1 = 10$,$a_2=30$,$a_3=90$,$a_4=270$,$a_5=810$},
            axis lines =middle, xlabel=$n$, ylabel=$a$,
            every axis y label/.style={at=(current axis.above origin),anchor=south},
            every axis x label/.style={at=(current axis.right of origin),anchor=west},
            clip=false,
            %axis on top,
          ]
          \addplot [draw=penColor,very thick] {(14/5)*2^(-x)};
          \addplot[color=penColor,fill=penColor,only marks,mark=*] coordinates{(1,7/5)};  %% closed hole
		\addplot [draw=penColor, fill = fillp] plot coordinates {(0,0) (1,0) (1, 7/5) (0,7/5) (0, 0)};          
          
	\addplot[color=penColor,fill=penColor,only marks,mark=*] coordinates{(2,7/10)};  %% closed hole
		\addplot [draw=penColor, fill = fillp] plot coordinates {(1,0) (2,0) (2, 7/10) (1,7/10) (1, 0)};          
          
          \addplot[color=penColor,fill=penColor,only marks,mark=*] coordinates{(3,7/20)};  %% closed hole
          	\addplot [draw=penColor, fill = fillp] plot coordinates {(2,0) (3,0) (3, 7/20) (2,7/20) (2, 0)};          

          
          \addplot[color=penColor,fill=penColor,only marks,mark=*] coordinates{(4,7/40)};  %% closed hole        
          	\addplot [draw=penColor, fill = fillp] plot coordinates {(3,0) (4,0) (4, 7/40) (3,7/40) (3, 0)};          

		  
          \addplot[color=penColor,fill=penColor,only marks,mark=*] coordinates{(5,7/80)};  %% closed hole 
          	\addplot [draw=penColor, fill = fillp] plot coordinates {(4,0) (5,0) (5, 7/80) (4,7/80) (4, 0)};          
 
        \end{axis}
\end{tikzpicture}
\end{image}

So if $\int_1^\infty f(x) \d x$ converged, we would have to have that the infinite series $ \sum_1^\infty a_k$ converged.

On the other hand, just by shifting all of the rectangles over one, we can see that if $\int_1^\infty f(x) \d x$ diverged, we would have to have that the infinite series $ \sum_1^\infty a_k$ also diverged.

\begin{image}
\begin{tikzpicture}
	\begin{axis}[
            domain=0:6,xmin=0,xmax=6,ymin=0,ymax=2,
            width=4in,
            height=2in,
            xtick={1,2,...,5},
            ytick={1.4,.7,.35,.175,.0875},
            yticklabels={},%$a_1 = 10$,$a_2=30$,$a_3=90$,$a_4=270$,$a_5=810$},
            axis lines =middle, xlabel=$n$, ylabel=$a$,
            every axis y label/.style={at=(current axis.above origin),anchor=south},
            every axis x label/.style={at=(current axis.right of origin),anchor=west},
            clip=false,
            %axis on top,
          ]
          \addplot[color=penColor,fill=penColor,only marks,mark=*] coordinates{(1,7/5)};  %% closed hole
		\addplot [draw=penColor, fill = fillp] plot coordinates {(1,0) (2,0) (2, 7/5) (1,7/5) (1, 0)};          
          
	\addplot[color=penColor,fill=penColor,only marks,mark=*] coordinates{(2,7/10)};  %% closed hole
		\addplot [draw=penColor, fill = fillp] plot coordinates {(2,0) (3,0) (3, 7/10) (2,7/10) (2, 0)};          
          
          \addplot[color=penColor,fill=penColor,only marks,mark=*] coordinates{(3,7/20)};  %% closed hole
          	\addplot [draw=penColor, fill = fillp] plot coordinates {(3,0) (4,0) (4, 7/20) (3,7/20) (3, 0)};          

          
          \addplot[color=penColor,fill=penColor,only marks,mark=*] coordinates{(4,7/40)};  %% closed hole        
          	\addplot [draw=penColor, fill = fillp] plot coordinates {(4,0) (5,0) (5, 7/40) (4,7/40) (4, 0)};          

		  
          \addplot[color=penColor,fill=penColor,only marks,mark=*] coordinates{(5,7/80)};  %% closed hole 
          	\addplot [draw=penColor, fill = fillp] plot coordinates {(5,0) (6,0) (6, 7/80) (5,7/80) (5, 0)};          
           \addplot [draw=penColor,very thick] {(14/5)*2^(-x)};

 
        \end{axis}
\end{tikzpicture}
\end{image}

This leads us to the \textbf{Integral test}

\begin{theorem}[Integral Test]
	Let $f$ is continuous, positive, and decreasing on $[1,\infty)$.
	
	Let $a_k = f(k)$.
	
	Then either $\sum_1^\infty a_k$ and $\int_1^\infty f(x) \d x$ both converge, or they both diverge.  It is impossible for one of them to diverge, and the other to converge. 
\end{theorem}

\begin{question}
	Does the harmonic series $\sum_1^\infty \frac{1}{n} = 1 + \frac{1}{2} + \frac{1}{3}+ \dots$ converge or diverge?
	
	\begin{multipleChoice}
		\choice{converge}
		\choice[correct]{diverge}
	\end{multipleChoice}
	
	\begin{hint}
		By the integral test, $\sum_1^\infty \frac{1}{n}$ converges if and only if $\int_1^\infty \frac{1}{x} \d x$ converges 
	\end{hint}
	
	\begin{hint}
		\begin{align*}
			\int_1^\infty \frac{1}{x} \d x &= \lim_{b \to \infty} \int_1^b \frac{1}{x} \d x\\
				&=\lim_{b \to \infty} \ln(b) \\
				&=\infty
		\end{align*}
		
		Thus the Harmonic series must diverge.
	\end{hint}
\end{question}

\begin{question}
Generalizing the last question, we can see that the sum of the ``$p$-series' '$\sum_1^\infty \frac{1}{n^p}$ converges iff and only if
\[
p \geq \answer{1}
\]

	\begin{hint}
		By the integral test, $\sum_1^\infty \frac{1}{n^p}$ converges if and only if $\int_1^\infty \frac{1}{x^p} \d x$ converges.  But we already know that this only converges when $p>1$.
	\end{hint}
\end{question}


\section{Estimating Series Using Improper Integrals}

Another cool application of this graphical viewpoint on series involves estimating series.

We cannot yet compute $\sum_1^\infty \frac{1}{k^2}$ exactly, and we will not learn how to do so in this course.

Say we wanted to \textbf{approximate} this sum within an error of $\frac{1}{100}$.  How many terms should we sum?  Is it enough to sum the first ten terms?  The first hundred?

Let $L = \sum_1^\infty \frac{1}{k^2}$.  We want to find a number $N$ where we can be sure that $L - \sum_1^N \frac{1}{k^2} < \frac{1}{100}$.  Now, $L - \sum_1^N \frac{1}{k^2}$  can be rewritten as $R_n = \sum_{N+1}^\infty \frac{1}{k^2}$.  We can see from a picture that  $R_n < \int_N^\infty \frac{1}{x^2} \d x$ (note that we need the red rectangles here:  there are two different ways to have rectangles correspond to a series!).  So if we can choose an $N$ for which $\int_N^\infty \frac{1}{x^2} \d x < \frac{1}{100}$, we will be sure that we have summed enough terms in the series to get to within $\frac{1}{100}$.

\begin{image}
\begin{tikzpicture}
	\begin{axis}[
            domain=0:6,xmin=0,xmax=6,ymin=0,ymax=2,
            width=4in,
            height=2in,
            xtick={1,2,...,5},
            xticklabels={$N$,$N+1$,$N+2$, $N+3$, $N+4$, $N+5$},
            ytick = {},
            yticklabels = {,,},
            axis lines =middle, xlabel=$n$, ylabel=$a$,
            every axis y label/.style={at=(current axis.above origin),anchor=south},
            every axis x label/.style={at=(current axis.right of origin),anchor=west},
            clip=false,
            %axis on top,
          ]
          \addplot[color=penColor,fill=penColor,only marks,mark=*] coordinates{(1,7/5)};  %% closed hole
		\addplot [draw=penColor, fill = fillp] plot coordinates {(1,0) (2,0) (2, 7/5) (1,7/5) (1, 0)};          
          		\addplot [draw=penColor, fill = filln] plot coordinates {(1,0) (2,0) (2, 7/10) (1,7/10) (1, 0)};          

	\addplot[color=penColor,fill=penColor,only marks,mark=*] coordinates{(2,7/10)};  %% closed hole
		\addplot [draw=penColor, fill = fillp] plot coordinates {(2,0) (3,0) (3, 7/10) (2,7/10) (2, 0)};          
                    	\addplot [draw=penColor, fill = filln] plot coordinates {(2,0) (3,0) (3, 7/20) (2,7/20) (2, 0)};          

          \addplot[color=penColor,fill=penColor,only marks,mark=*] coordinates{(3,7/20)};  %% closed hole
          	\addplot [draw=penColor, fill = fillp] plot coordinates {(3,0) (4,0) (4, 7/20) (3,7/20) (3, 0)};          
          		\addplot [draw=penColor, fill = filln] plot coordinates {(3,0) (4,0) (4, 7/40) (3,7/40) (3, 0)};          

          
          \addplot[color=penColor,fill=penColor,only marks,mark=*] coordinates{(4,7/40)};  %% closed hole        
          	\addplot [draw=penColor, fill = fillp] plot coordinates {(4,0) (5,0) (5, 7/40) (4,7/40) (4, 0)};          
          		\addplot [draw=penColor, fill = filln] plot coordinates {(4,0) (5,0) (5, 7/80) (4,7/80) (4, 0)};          

	          \addplot [draw=penColor,very thick, domain=0.5:6] {(14/5)*2^(-x)};

 
        \end{axis}
\end{tikzpicture}
\end{image}

\begin{question}
	What is the least natural number $N$ with $\int_N^\infty \frac{1}{x^2} \d x < \frac{1}{100}$?
	
		\[
			N = \answer{101}
		\]
		
		\begin{hint}
			\begin{align*}
				\int_N^\infty \frac{1}{x^2} \d x &< \frac{1}{100}\\
				\lim_{b \to \infty} \int_N^b \frac{1}{x^2} \d x &< \frac{1}{100}\\
				\lim_{b \to \infty} \eval{-\frac{1}{x}}_N^b& < \frac{1}{100}\\
				\lim_{b \to \infty} \frac{1}{N} - \frac{1}{b}& < \frac{1}{100}\\
				\frac{1}{N} &< \frac{1}{100}\\
				N&>100
			\end{align*}
			
			
			So we need $N$ to be at least $101$.
		\end{hint}
\end{question}

This shows that if we sum $101$ terms of the series $\sum_1^\infty \frac{1}{k^2}$, we will get within $\frac{1}{100}$ of the true answer.

It turns out this series actually converges to $\frac{\pi^2}{6}$ (finding this uses advanced concepts like Fourier Analysis).  Using a computer to sum the first $101$ terms we get $\sum_1^101 \frac{1}{k^2} \approx 1.6350$, while $\frac{\pi^2}{6} \approx 1.6449 $.  Their difference is just barely less than $\frac{1}{100}$.

\end{document}


