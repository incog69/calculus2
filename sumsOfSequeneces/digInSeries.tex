\documentclass{ximera}

%\usepackage{todonotes}

\newcommand{\todo}{}

\usepackage{tkz-euclide}
\tikzset{>=stealth} %% cool arrow head
\tikzset{shorten <>/.style={ shorten >=#1, shorten <=#1 } } %% allows shorter vectors

\usetikzlibrary{backgrounds} %% for boxes around graphs
\usetikzlibrary{shapes,positioning}  %% Clouds and stars
\usetikzlibrary{matrix} %% for matrix
\usepgfplotslibrary{polar} %% for polar plots
\usetkzobj{all}
\usepackage[makeroom]{cancel} %% for strike outs
%\usepackage{mathtools} %% for pretty underbrace % Breaks Ximera
\usepackage{multicol}





\usepackage{array}
\setlength{\extrarowheight}{+.1cm}   
\newdimen\digitwidth
\settowidth\digitwidth{9}
\def\divrule#1#2{
\noalign{\moveright#1\digitwidth
\vbox{\hrule width#2\digitwidth}}}





\newcommand{\RR}{\mathbb R}
\newcommand{\R}{\mathbb R}
\newcommand{\N}{\mathbb N}
\newcommand{\Z}{\mathbb Z}

%\renewcommand{\d}{\,d\!}
\renewcommand{\d}{\mathop{}\!d}
\newcommand{\dd}[2][]{\frac{\d #1}{\d #2}}
\renewcommand{\l}{\ell}
\newcommand{\ddx}{\frac{d}{\d x}}

\newcommand{\zeroOverZero}{\ensuremath{\boldsymbol{\tfrac{0}{0}}}}
\newcommand{\inftyOverInfty}{\ensuremath{\boldsymbol{\tfrac{\infty}{\infty}}}}
\newcommand{\zeroOverInfty}{\ensuremath{\boldsymbol{\tfrac{0}{\infty}}}}
\newcommand{\zeroTimesInfty}{\ensuremath{\small\boldsymbol{0\cdot \infty}}}
\newcommand{\inftyMinusInfty}{\ensuremath{\small\boldsymbol{\infty - \infty}}}
\newcommand{\oneToInfty}{\ensuremath{\boldsymbol{1^\infty}}}
\newcommand{\zeroToZero}{\ensuremath{\boldsymbol{0^0}}}
\newcommand{\inftyToZero}{\ensuremath{\boldsymbol{\infty^0}}}


\newcommand{\numOverZero}{\ensuremath{\boldsymbol{\tfrac{\#}{0}}}}
\newcommand{\dfn}{\textbf}
%\newcommand{\unit}{\,\mathrm}
\newcommand{\unit}{\mathop{}\!\mathrm}
\newcommand{\eval}[1]{\bigg[ #1 \bigg]}
\newcommand{\seq}[1]{\left( #1 \right)}
\renewcommand{\epsilon}{\varepsilon}
\renewcommand{\iff}{\Leftrightarrow}

\DeclareMathOperator{\arccot}{arccot}
\DeclareMathOperator{\arcsec}{arcsec}
\DeclareMathOperator{\arccsc}{arccsc}
\DeclareMathOperator{\si}{Si}

\newcommand{\tightoverset}[2]{%
  \mathop{#2}\limits^{\vbox to -.5ex{\kern-0.75ex\hbox{$#1$}\vss}}}
\newcommand{\arrowvec}[1]{\tightoverset{\scriptstyle\rightharpoonup}{#1}}
\renewcommand{\vec}{\mathbf}


\colorlet{textColor}{black} 
\colorlet{background}{white}
\colorlet{penColor}{blue!50!black} % Color of a curve in a plot
\colorlet{penColor2}{red!50!black}% Color of a curve in a plot
\colorlet{penColor3}{red!50!blue} % Color of a curve in a plot
\colorlet{penColor4}{green!50!black} % Color of a curve in a plot
\colorlet{penColor5}{orange!80!black} % Color of a curve in a plot
\colorlet{fill1}{penColor!20} % Color of fill in a plot
\colorlet{fill2}{penColor2!20} % Color of fill in a plot
\colorlet{fillp}{fill1} % Color of positive area
\colorlet{filln}{penColor2!20} % Color of negative area
\colorlet{fill3}{penColor3!20} % Fill
\colorlet{fill4}{penColor4!20} % Fill
\colorlet{fill5}{penColor5!20} % Fill
\colorlet{gridColor}{gray!50} % Color of grid in a plot

\newcommand{\surfaceColor}{violet}
\newcommand{\surfaceColorTwo}{redyellow}
\newcommand{\sliceColor}{greenyellow}




\pgfmathdeclarefunction{gauss}{2}{% gives gaussian
  \pgfmathparse{1/(#2*sqrt(2*pi))*exp(-((x-#1)^2)/(2*#2^2))}%
}


%%%%%%%%%%%%%
%% Vectors
%%%%%%%%%%%%%

%% Simple horiz vectors
\renewcommand{\vector}[1]{\left\langle #1\right\rangle}


%% %% Complex Horiz Vectors with angle brackets
%% \makeatletter
%% \renewcommand{\vector}[2][ , ]{\left\langle%
%%   \def\nextitem{\def\nextitem{#1}}%
%%   \@for \el:=#2\do{\nextitem\el}\right\rangle%
%% }
%% \makeatother

%% %% Vertical Vectors
%% \def\vector#1{\begin{bmatrix}\vecListA#1,,\end{bmatrix}}
%% \def\vecListA#1,{\if,#1,\else #1\cr \expandafter \vecListA \fi}

%%%%%%%%%%%%%
%% End of vectors
%%%%%%%%%%%%%

%\newcommand{\fullwidth}{}
%\newcommand{\normalwidth}{}



%% makes a snazzy t-chart for evaluating functions
%\newenvironment{tchart}{\rowcolors{2}{}{background!90!textColor}\array}{\endarray}

%%This is to help with formatting on future title pages.
\newenvironment{sectionOutcomes}{}{} 



%% Flowchart stuff
%\tikzstyle{startstop} = [rectangle, rounded corners, minimum width=3cm, minimum height=1cm,text centered, draw=black]
%\tikzstyle{question} = [rectangle, minimum width=3cm, minimum height=1cm, text centered, draw=black]
%\tikzstyle{decision} = [trapezium, trapezium left angle=70, trapezium right angle=110, minimum width=3cm, minimum height=1cm, text centered, draw=black]
%\tikzstyle{question} = [rectangle, rounded corners, minimum width=3cm, minimum height=1cm,text centered, draw=black]
%\tikzstyle{process} = [rectangle, minimum width=3cm, minimum height=1cm, text centered, draw=black]
%\tikzstyle{decision} = [trapezium, trapezium left angle=70, trapezium right angle=110, minimum width=3cm, minimum height=1cm, text centered, draw=black]


\outcome{Recognize a geometric series.}
\outcome{Recognize a telescoping series.}
\outcome{Compute the sum of a geometric series.}
\outcome{Compute the sum of a telescoping series.}

\title[Dig-In:]{Series}

\begin{document}
\begin{abstract}
A series is summation of a sequence.
\end{abstract}
\maketitle


Let's jump right in:


\begin{definition}
  A \dfn{series} is a sum of an infinte sequence.
\end{definition}

Believe it or not, you have been working with series for a long
time. Consider the number
\[
\frac{1}{3} = 0.3333333333\dots
\]
this is the infinite sum of the geometric sequence
$(a_n)_{n=1}^\infty$ where $a_n = 3\cdot 10^{n}$, as
\begin{align*}
  \sum_{n=1}^\infty 3\cdot \frac{1}{10^{n}} &= 0.3 + 0.03+0.003+ 0.0003+ 0.00003+ \dots\\
  &=\frac{1}{3}.
\end{align*}
We can sum other geometric series as well. Consider
\[
\sum_{n=1}^\infty \left(\frac{1}{4}\right)^n =
\frac{1}{4} + \left(\frac{1}{4}\right)^2 + \left(\frac{1}{4}\right)^3 + \left(\frac{1}{4}\right)^4 + \cdots 
\]
A very clever method of summing this sequence is as follows, consider
an equilateral triangle with area $1$
\begin{image}
  \begin{tikzpicture}[scale=3,rounded corners=.5pt]      
    \tkzDefPoint(0,1){A1} 
    \tkzDefPoint(-.58,0){A2}
    \tkzDefPoint(.58,0){A3}
    \draw[penColor,very thick] (A1)--(A2)--(A3)--cycle;
  \end{tikzpicture}
\end{image}
We can break this triangle into $4$ congruent triangles, each of area
$1/4$:
\begin{image}
  \begin{tikzpicture}[scale=3,rounded corners=.5pt]      
    \tkzDefPoint(0,1){A1} 
    \tkzDefPoint(-.58,0){A2}
    \tkzDefPoint(.58,0){A3}
    \draw[penColor,very thick] (A1)--(A2)--(A3)--cycle;

    \tkzDefPoint(0,0){B1} 
    \tkzDefPoint(-.29,.5){B2}
    \tkzDefPoint(.29,.5){B3}
    \draw[penColor,fill=fill1,very thick] (B1)--(B2)--(B3)--cycle;
  \end{tikzpicture}
\end{image}

\begin{image}
  \begin{tikzpicture}[scale=3,rounded corners=.5pt]      
    \tkzDefPoint(0,1){A1} 
    \tkzDefPoint(-.58,0){A2}
    \tkzDefPoint(.58,0){A3}
    \draw[penColor,very thick] (A1)--(A2)--(A3)--cycle;

    \tkzDefPoint(0,0){B1} 
    \tkzDefPoint(-.29,.5){B2}
    \tkzDefPoint(.29,.5){B3}
    \draw[penColor,fill=fill1,very thick] (B1)--(B2)--(B3)--cycle;

    \tkzDefPoint(0,.5){C1} 
    \tkzDefPoint(-.14,.75){C2}
    \tkzDefPoint(.14,.75){C3}
    \draw[penColor,fill=fill1,very thick] (C1)--(C2)--(C3)--cycle;
  \end{tikzpicture}
\end{image}


\begin{image}
  \begin{tikzpicture}[scale=3,rounded corners=.5pt]      
    \tkzDefPoint(0,1){A1} 
    \tkzDefPoint(-.58,0){A2}
    \tkzDefPoint(.58,0){A3}
    \draw[penColor,very thick] (A1)--(A2)--(A3)--cycle;

    \tkzDefPoint(0,0){B1} 
    \tkzDefPoint(-.29,.5){B2}
    \tkzDefPoint(.29,.5){B3}
    \draw[penColor,fill=fill1,very thick] (B1)--(B2)--(B3)--cycle;

    \tkzDefPoint(0,.5){C1} 
    \tkzDefPoint(-.14,.75){C2}
    \tkzDefPoint(.14,.75){C3}
    \draw[penColor,fill=fill1,very thick] (C1)--(C2)--(C3)--cycle;

    \tkzDefPoint(0,.75){D1} 
    \tkzDefPoint(-.07,.875){D2}
    \tkzDefPoint(.07,.875){D3}
    \draw[penColor,fill=fill1,very thick] (D1)--(D2)--(D3)--cycle;

    \tkzDefPoint(0,.875){E1} 
    \tkzDefPoint(-.04,.94){E2}
    \tkzDefPoint(.04,.94){E3}
    \draw[penColor,fill=fill1,very thick] (E1)--(E2)--(E3)--cycle;

    \tkzDefPoint(0,.94 ){F1} 
    \tkzDefPoint(-.02,.97){F2}
    \tkzDefPoint(.02,.97){F3}
    \draw[penColor,fill=fill1,very thick] (F1)--(F2)--(F3)--cycle;
  \end{tikzpicture}
\end{image}




Given the sequence $(a_n) = (1/2^n) = 1/2, 1/4, 1/8, \ldots$, consider
the following sums:
\[
\begin{array}{ccccc}
a_1				&=& 1/2					 &=& 1/2\\
a_1+a_2		&=& 1/2+1/4			 &=& 3/4\\
a_1+a_2+a_3 &=& 1/2+1/4+1/8  &=& 7/8\\
a_1+a_2+a_3+a_4 &=& 1/2+1/4+1/8+1/16 & =& 15/16
\end{array}
\]
In general, we can show that
\[
a_1+a_2+a_3+\cdots +a_n = \frac{2^n-1}{2^n} = 1-\frac{1}{2^n}.
\]
Let $S_n$ be the sum of the first $n$ terms of the sequence
$(1/2^n)$. From the above, we see that $S_1=1/2$, $S_2 = 3/4$,
etc. Our formula at the end shows that $S_n = 1-1/2^n$.

Now consider the following limit: $\lim_{n\to\infty}S_n =
\lim_{n\to\infty}\big(1-1/2^n\big) = 1$. This limit can be interpreted
as saying something amazing: \emph{the sum of \emph{all} the terms of
  the sequence $(1/2^n)$ is 1.}

This example illustrates some interesting concepts that we explore in
this section. We begin this exploration with some definitions.

\begin{definition}
Let $(a_n)$ be a sequence.
\begin{enumerate}
\item The sum $\sum_{n=1}^\infty a_n$ is an \dfn{infinite series} (or,
  simply \dfn{series}).
\item Let $S_n = \sum_{i=1}^n a_i$; the sequence $(S_n)$ is the
  sequence of $n^\text{th}$ \dfn{partial sums} of $(a_n)$.
\item If the sequence $(S_n)$ converges to $L$, we say the series
  $\sum_{n=1}^\infty a_n$ \dfn{converges} to $L$, and we write
  $\sum_{n=1}^\infty a_n = L$.
\item If the sequence $(S_n)$ diverges, the series $\sum_{n=1}^\infty
  a_n$ \dfn{diverges}.
\end{enumerate}
\end{definition}

Using our new terminology, we can state that the series
$\sum_{n=1}^\infty 1/2^n$ converges, and $\sum_{n=1}^\infty 1/2^n =
1.$

We will explore a variety of series in this section. We start with two
series that diverge, showing how we might discern divergence.

\begin{example}
\begin{enumerate}
\item Let $(a_n) = (n^2)$. Show $\sum_{n=1}^\infty a_n$ diverges.
\item Let $(b_n) = ((-1)^{n+1})$. Show $\sum_{n=1}^\infty b_n$
  diverges.
\end{enumerate}
\begin{explanation}
\begin{enumerate}
\item Consider $S_n$, the $n^\text{th}$ partial sum.
  \begin{align*} S_n &= a_1+a_2+a_3+\cdots+a_n \\		
    &= 1^2+2^2+3^2\cdots +
    n^2.  \intertext{By Theorem \ref{thm:summation}, this is} &=
    \frac{n(n+1)(2n+1)}{6}.
  \end{align*}
  Since $\lim_{n\to\infty}S_n = \infty$, we conclude that the series
  $\sum_{n=1}^\infty n^2$ diverges. It is instructive to write
  $\sum_{n=1}^\infty n^2=\infty$ for this tells us \emph{how} the series
  diverges: it grows without bound.
  A scatter plot of the sequences $(a_n)$ and $(S_n)$ is given in Figure
  \ref{fig:series1}(a). The terms of $(a_n)$ are growing, so the terms
  of the partial sums $(S_n)$ are growing even faster, illustrating that
  the series diverges.
  
\item Consider some of the partial sums $S_n$ of $(b_n)$:
  \begin{align*}
    S_1 &= 1\\
    S_2 &= 0\\
    S_3 &= 1\\
    S_4 &= 0
  \end{align*}
  This pattern repeats; we find that $S_n = \left(\begin{array}{cc} 1  & n\ \text{ is odd}\\
    0  & n\  \text{ is even}
  \end{array}\right..$
  As $(S_n)$ oscillates, repeating 1, 0, 1, 0, $\ldots$, we conclude
  that $\lim_{n\to\infty}S_n$ does not exist, hence $\sum_{n=1}^\infty
  (-1)^{n+1}$ diverges.
  
  A scatter plot of the sequence $(b_n)$ and the partial sums $(S_n)$
  is given in Figure \ref{fig:series1}(b). When $n$ is odd, $b_n =
  S_n$ so the marks for $b_n$ are drawn oversized to show they
  coincide.

%% \mtable{.6}{Scatter plots relating to Example \ref{ex_series1}.}{fig:series1}{%
%% \begin{tabular}{c}
%% \myincludegraphics{figures/figseries1a}\\[10pt]
%% (a)\\[15pt]
%% \myincludegraphics{figures/figseries1b}\\[10pt]
%% (b)
%% \end{tabular}
%% }	

\end{enumerate}
\end{explanation}
\end{example}

While it is important to recognize when a series diverges, we are
generally more interested in the series that converge. In this section
we will demonstrate a few general techniques for determining
convergence; later sections will delve deeper into this topic.

\section{Geometric series}

One important type of series is a \textit{geometric series}.

\begin{definition}
Let $(a_n)$ be a geometric sequence. A \dfn{geometric series} is a
series of the form
\[
\sum_{n=1}^\infty a_n
\]
\end{definition}

We started this section with a geometric series, although we dropped
the first term of $1$. One reason geometric series are important is
that they have nice convergence properties.

\begin{theorem}
Consider the geometric series $\sum_{n=0}^\infty r^n$.
\begin{enumerate}
\item The $n^\text{th}$ partial sum is: $S_n = \frac{1-r^{n+1}}{1-r}$.
\item The series converges if, and only if, $|r| < 1$. When $|r|<1$, 
\index{series!geometric}\index{geometric series}\index{convergence!of geometric series}\index{divergence!of geometric series}
\[
\sum_{n=0}^\infty r^n = \frac{1}{1-r}.
\]
\end{enumerate}
\end{theorem}

According to Theorem \ref{thm:geom_series}, the series
$\sum_{n=0}^\infty \frac{1}{2^n} = 1+\frac12+\frac14+\cdots$
converges, and $\sum_{n=0}^\infty \frac{1}{2^n} = \frac{1}{1-1/2} =
2.$ This concurs with our introductory example; while there we got a
sum of 1, we skipped the first term of 1.



\begin{example}
Check the convergence of the following series. If the series
converges, find its sum.
$\sum_{n=2}^\infty \left(\frac34\right)^n$

$\sum_{n=0}^\infty \left(\frac{-1}{2}\right)^n$


$\sum_{n=0}^\infty 3^n$ 

\begin{enumerate}
\item Since $r=3/4<1$, this series converges. By Theorem
  \ref{thm:geom_series}, we have that
  \[
  \sum_{n=0}^\infty \left(\frac34\right)^n = \frac{1}{1-3/4} = 4.\]
  However, note the subscript of the summation in the given series: we
  are to start with $n=2$. Therefore we subtract off the first two
  terms, giving:
  \[
  \sum_{n=2}^\infty \left(\frac34\right)^n = 4 - 1 - \frac34 =
  \frac94.
  \]
  This is illustrated in Figure \ref{fig:series2a}.
  %% \mfigure{.35}{Scatter plots relating to the series in Example
  %%   \ref{ex_series2}.}{fig:series2a}{figures/figseries2a}

%\mfigure{.8}{Scatter plots relating to the series of Example \ref{ex_series2} part 1.}{fig:series2a}{figures/figseries2a}
  
\item Since $|r| = 1/2 < 1$, this series converges, and by Theorem
  \ref{thm:geom_series},
  $$\sum_{n=0}^\infty \left(\frac{-1}{2}\right)^n = \frac{1}{1-(-1/2)}
  = \frac23.$$ The partial sums of this series are plotted in Figure
  \ref{fig:series2}(a). Note how the partial sums are not purely
  increasing as some of the terms of the sequence $((-1/2)^n)$ are
  negative.
  
  %\mfigure{.55}{Scatter plots relating to the series of Example \ref{ex_series2} part 2.}{fig:series2b}{figures/figseries2b}
  
\item Since $r>1$, the series diverges. (This makes ``common sense'';
  we expect the sum $$1+3+9+27 + 81+243+\cdots$$ to diverge.) This is
  illustrated in Figure \ref{fig:series2}(b).
  
  %\mfigure{.3}{Scatter plots relating to the series of Example \ref{ex_series2} part 3.}{fig:series2c}{figures/figseries2c}
%\mtable{.67}{Scatter plots relating to the series in Example \ref{ex_series2}.}{fig:series2}{%
%\begin{tabular}{c}
%\myincludegraphics{figures/figseries2a}\\[10pt]
%(a)\\[15pt]
%\myincludegraphics{figures/figseries2b}\\[10pt]
%(a)\\[15pt]
%\myincludegraphics{figures/figseries2c}\\[10pt]
%(b)
%\end{tabular}
%}
\end{enumerate}
\end{example}

\begin{example}
Evaluate the sum $\sum_{n=1}^\infty \left(\frac1n-\frac1{n+1}\right)$.
\index{series!telescoping}\index{telescoping series}
It will help to write down some of the first few partial sums of this series.
\begin{align*}
S_1 &=	\frac11-\frac12 & & = 1-\frac12\\
S_2 &=	\left(\frac11-\frac12\right) + \left(\frac12-\frac13\right) & & = 1-\frac13\\
S_3 &=	\left(\frac11-\frac12\right) + \left(\frac12-\frac13\right)+\left(\frac13-\frac14\right) & &= 1-\frac14\\
S_4 &=	\left(\frac11-\frac12\right) + \left(\frac12-\frac13\right)+\left(\frac13-\frac14\right) +\left(\frac14-\frac15\right)& &= 1-\frac15
\end{align*}
Note how most of the terms in each partial sum are canceled out! In
general, we see that $S_n = 1-\frac{1}{n+1}$. The sequence $(S_n)$
converges, as $\lim_{n\to\infty}S_n =
\lim_{n\to\infty}\left(1-\frac1{n+1}\right) = 1$, and so we conclude
that $\sum_{n=1}^\infty \left(\frac1n-\frac1{n+1}\right) = 1$. Partial
sums of the series are plotted in Figure \ref{fig:series3}.
%\mfigure{.75}{Scatter plots relating to the series of Example \ref{ex_series3}.}{fig:series3}{figures/figseries3}
\end{example}

The series in Example \ref{ex_series3} is an example of a \dfn{telescoping series}. Informally, a telescoping series is one in which the partial sums reduce to just a finite number of terms. The partial sum $S_n$ did not contain $n$ terms, but rather just two: 1 and $1/(n+1)$.\index{series!telescoping}\index{telescoping series}

When possible, seek a way to write an explicit formula for the $n^\text{th}$ partial sum $S_n$. This makes evaluating the limit $\lim_{n\to\infty} S_n$ much more approachable. We do so in the next example.\\

%\noindent\textbf{Note on notation:} Most of the series we encounter will start with $n=1$. For ease of notation, we will often write $\sum a_n$ instead of writing $\sum_{n=1}^\infty a_n$.\\



\begin{example}
Evaluate each of the following infinite series.

\noindent 1. $\sum_{n=1}^\infty \frac{2}{n^2+2n}$ \qquad 2. $\sum_{n=1}^\infty \ln\left(\frac{n+1}{n}\right)$

\begin{enumerate}
\item We can decompose the fraction $2/(n^2+2n)$ as $$\frac2{n^2+2n}
  = \frac1n-\frac1{n+2}.$$ (See Section \ref{sec:partial_fraction},
  Partial Fraction Decomposition, to recall how this is done, if
  necessary.)
  
  Expressing the terms of $(S_n)$ is now more instructive:
  
  \begin{align*}
    S_1 &= 1-\frac13 &&= 1-\frac13\\
    S_2 &= \left(1-\frac13\right) + \left(\frac12-\frac14\right) &&= 1+\frac12-\frac13-\frac14\\
    S_3 &= \left(1-\frac13\right) + \left(\frac12-\frac14\right)+\left(\frac13-\frac15\right) &&= 1+\frac12-\frac14-\frac15\\
    S_4 &= \left(1-\frac13\right) + \left(\frac12-\frac14\right)+\left(\frac13-\frac15\right)+\left(\frac14-\frac16\right) &&= 1+\frac12-\frac15-\frac16\\
    S_5 &= \left(1-\frac13\right) + \left(\frac12-\frac14\right)+\left(\frac13-\frac15\right)+\left(\frac14-\frac16\right)+\left(\frac15-\frac17\right) &&= 1+\frac12-\frac16-\frac17\\
  \end{align*}
 
We again have a telescoping series. In each partial sum, most of the
terms cancel and we obtain the formula $S_n =
1+\frac12-\frac1{n+1}-\frac1{n+2}.$ Taking limits allows us to
determine the convergence of the series:
\[
\lim_{n\to\infty}S_n = \lim_{n\to\infty} \left(1+\frac12-\frac1{n+1}-\frac1{n+2}\right) = \frac32,\quad \text{so } \sum_{n=1}^\infty \frac1{n^2+2n} = \frac32.
\]
This is illustrated in Figure \ref{fig:series4}(a).
%\mfigure{.3}{Scatter plots relating to the series of Example \ref{ex_series4} part 1.}{fig:series4a}{figures/figseries4a}
%\mtable{.5}{Scatter plots relating to the series in Example \ref{ex_series4}.}{fig:series4}{%
%% \begin{tabular}{c}
%% \myincludegraphics{figures/figseries4a}\\[10pt]
%% (a)\\[15pt]
%% \myincludegraphics{figures/figseries4b}\\[10pt]
%% (b)
%% \end{tabular}
%% }
%\drawexampleline

\item We begin by writing the first few partial sums of the series:

\begin{align*}
S_1 &= \ln\left(2\right) \\
S_2 &= \ln\left(2\right)+\ln\left(\frac32\right) \\
S_3 &= \ln\left(2\right)+\ln\left(\frac32\right)+\ln\left(\frac43\right) \\
S_4 &= \ln\left(2\right)+\ln\left(\frac32\right)+\ln\left(\frac43\right)+\ln\left(\frac54\right) 
\end{align*}
At first, this does not seem helpful, but recall the logarithmic identity: $\ln x+\ln y = \ln (xy).$ Applying this to $S_4$ gives:
$$S_4 = \ln\left(2\right)+\ln\left(\frac32\right)+\ln\left(\frac43\right)+\ln\left(\frac54\right) = \ln\left(\frac21\cdot\frac32\cdot\frac43\cdot\frac54\right) = \ln\left(5\right).$$

We can conclude that $(S_n) = \big(\ln (n+1)\big)$. This sequence does
not converge, as $\lim_{n\to\infty}S_n=\infty$. Therefore
$\sum_{n=1}^\infty \ln\left(\frac{n+1}{n}\right)=\infty$; the series
diverges. Note in Figure \ref{fig:series4}(b) how the sequence of
partial sums grows slowly; after 100 terms, it is not yet over
5. Graphically we may be fooled into thinking the series converges,
but our analysis above shows that it does not.
%\mfigure{.35}{Scatter plots relating to the series of Example \ref{ex_series4} part 2.}{fig:series4b}{figures/figseries4b}
\end{enumerate}
\end{example}

%\enlargethispage{3\baselineskip}
We are learning about a new mathematical object, the series. As done
before, we apply ``old'' mathematics to this new topic.

\begin{theorem}[Properties of Infinite Series]
  Let
  \[
  \sum_{n=1}^\infty a_n = L,\quad \sum_{n=1}^\infty b_n =K, 
  \]
  and let $c$ be a constant.
\begin{enumerate}
\item Constant Multiple Rule: $\sum_{n=1}^\infty c\cdot a_n =
  c\cdot\sum_{n=1}^\infty a_n = c\cdot L.$\index{Constant Multiple
    Rule!of series}
\item Sum/Difference Rule: $\sum_{n=1}^\infty \big(a_n\pm b_n\big) =
  \sum_{n=1}^\infty a_n \pm \sum_{n=1}^\infty b_n = L \pm K.$
  \index{series!properties}\index{Sum/Difference Rule!of series}
\end{enumerate} 
\end{theorem}



















\end{document}
