\documentclass{ximera}

%\usepackage{todonotes}

\newcommand{\todo}{}

\usepackage{tkz-euclide}
\tikzset{>=stealth} %% cool arrow head
\tikzset{shorten <>/.style={ shorten >=#1, shorten <=#1 } } %% allows shorter vectors

\usetikzlibrary{backgrounds} %% for boxes around graphs
\usetikzlibrary{shapes,positioning}  %% Clouds and stars
\usetikzlibrary{matrix} %% for matrix
\usepgfplotslibrary{polar} %% for polar plots
\usetkzobj{all}
\usepackage[makeroom]{cancel} %% for strike outs
%\usepackage{mathtools} %% for pretty underbrace % Breaks Ximera
\usepackage{multicol}





\usepackage{array}
\setlength{\extrarowheight}{+.1cm}   
\newdimen\digitwidth
\settowidth\digitwidth{9}
\def\divrule#1#2{
\noalign{\moveright#1\digitwidth
\vbox{\hrule width#2\digitwidth}}}





\newcommand{\RR}{\mathbb R}
\newcommand{\R}{\mathbb R}
\newcommand{\N}{\mathbb N}
\newcommand{\Z}{\mathbb Z}

%\renewcommand{\d}{\,d\!}
\renewcommand{\d}{\mathop{}\!d}
\newcommand{\dd}[2][]{\frac{\d #1}{\d #2}}
\renewcommand{\l}{\ell}
\newcommand{\ddx}{\frac{d}{\d x}}

\newcommand{\zeroOverZero}{\ensuremath{\boldsymbol{\tfrac{0}{0}}}}
\newcommand{\inftyOverInfty}{\ensuremath{\boldsymbol{\tfrac{\infty}{\infty}}}}
\newcommand{\zeroOverInfty}{\ensuremath{\boldsymbol{\tfrac{0}{\infty}}}}
\newcommand{\zeroTimesInfty}{\ensuremath{\small\boldsymbol{0\cdot \infty}}}
\newcommand{\inftyMinusInfty}{\ensuremath{\small\boldsymbol{\infty - \infty}}}
\newcommand{\oneToInfty}{\ensuremath{\boldsymbol{1^\infty}}}
\newcommand{\zeroToZero}{\ensuremath{\boldsymbol{0^0}}}
\newcommand{\inftyToZero}{\ensuremath{\boldsymbol{\infty^0}}}


\newcommand{\numOverZero}{\ensuremath{\boldsymbol{\tfrac{\#}{0}}}}
\newcommand{\dfn}{\textbf}
%\newcommand{\unit}{\,\mathrm}
\newcommand{\unit}{\mathop{}\!\mathrm}
\newcommand{\eval}[1]{\bigg[ #1 \bigg]}
\newcommand{\seq}[1]{\left( #1 \right)}
\renewcommand{\epsilon}{\varepsilon}
\renewcommand{\iff}{\Leftrightarrow}

\DeclareMathOperator{\arccot}{arccot}
\DeclareMathOperator{\arcsec}{arcsec}
\DeclareMathOperator{\arccsc}{arccsc}
\DeclareMathOperator{\si}{Si}

\newcommand{\tightoverset}[2]{%
  \mathop{#2}\limits^{\vbox to -.5ex{\kern-0.75ex\hbox{$#1$}\vss}}}
\newcommand{\arrowvec}[1]{\tightoverset{\scriptstyle\rightharpoonup}{#1}}
\renewcommand{\vec}{\mathbf}


\colorlet{textColor}{black} 
\colorlet{background}{white}
\colorlet{penColor}{blue!50!black} % Color of a curve in a plot
\colorlet{penColor2}{red!50!black}% Color of a curve in a plot
\colorlet{penColor3}{red!50!blue} % Color of a curve in a plot
\colorlet{penColor4}{green!50!black} % Color of a curve in a plot
\colorlet{penColor5}{orange!80!black} % Color of a curve in a plot
\colorlet{fill1}{penColor!20} % Color of fill in a plot
\colorlet{fill2}{penColor2!20} % Color of fill in a plot
\colorlet{fillp}{fill1} % Color of positive area
\colorlet{filln}{penColor2!20} % Color of negative area
\colorlet{fill3}{penColor3!20} % Fill
\colorlet{fill4}{penColor4!20} % Fill
\colorlet{fill5}{penColor5!20} % Fill
\colorlet{gridColor}{gray!50} % Color of grid in a plot

\newcommand{\surfaceColor}{violet}
\newcommand{\surfaceColorTwo}{redyellow}
\newcommand{\sliceColor}{greenyellow}




\pgfmathdeclarefunction{gauss}{2}{% gives gaussian
  \pgfmathparse{1/(#2*sqrt(2*pi))*exp(-((x-#1)^2)/(2*#2^2))}%
}


%%%%%%%%%%%%%
%% Vectors
%%%%%%%%%%%%%

%% Simple horiz vectors
\renewcommand{\vector}[1]{\left\langle #1\right\rangle}


%% %% Complex Horiz Vectors with angle brackets
%% \makeatletter
%% \renewcommand{\vector}[2][ , ]{\left\langle%
%%   \def\nextitem{\def\nextitem{#1}}%
%%   \@for \el:=#2\do{\nextitem\el}\right\rangle%
%% }
%% \makeatother

%% %% Vertical Vectors
%% \def\vector#1{\begin{bmatrix}\vecListA#1,,\end{bmatrix}}
%% \def\vecListA#1,{\if,#1,\else #1\cr \expandafter \vecListA \fi}

%%%%%%%%%%%%%
%% End of vectors
%%%%%%%%%%%%%

%\newcommand{\fullwidth}{}
%\newcommand{\normalwidth}{}



%% makes a snazzy t-chart for evaluating functions
%\newenvironment{tchart}{\rowcolors{2}{}{background!90!textColor}\array}{\endarray}

%%This is to help with formatting on future title pages.
\newenvironment{sectionOutcomes}{}{} 



%% Flowchart stuff
%\tikzstyle{startstop} = [rectangle, rounded corners, minimum width=3cm, minimum height=1cm,text centered, draw=black]
%\tikzstyle{question} = [rectangle, minimum width=3cm, minimum height=1cm, text centered, draw=black]
%\tikzstyle{decision} = [trapezium, trapezium left angle=70, trapezium right angle=110, minimum width=3cm, minimum height=1cm, text centered, draw=black]
%\tikzstyle{question} = [rectangle, rounded corners, minimum width=3cm, minimum height=1cm,text centered, draw=black]
%\tikzstyle{process} = [rectangle, minimum width=3cm, minimum height=1cm, text centered, draw=black]
%\tikzstyle{decision} = [trapezium, trapezium left angle=70, trapezium right angle=110, minimum width=3cm, minimum height=1cm, text centered, draw=black]


\outcome{Understand what a slope field is.}
\outcome{Understand what a slope field tells us about a differential equation.}
\outcome{Define an autonomous differential equation.}
\outcome{Understand and use Euler's method.}
\outcome{Sketch slope fields.}
\outcome{Find equilibrium solutions.}

\title[Dig-In:]{Euler's method}

\begin{document}
\begin{abstract}
Something
\end{abstract}
\maketitle


\section{Slope Fields}

We cannot (yet) solve the differential equation

\[
y' = x+y
\]

We do know some things about the solutions however.

\begin{question}
	Consider the solution $f$ to the differential equation $y'=x+y$ which passes through the point $(1,2)$.   Is $f$ increasing or decreasing at $x=1$?
	
	\begin{multipleChoice}
		\choice[correct]{Increasing}
		\choice{Decreasing}
	\end{multipleChoice}
	
	\begin{hint}
		By definition, a solution to this differential equation which passes through $(1,2)$ must have $f(1)=2$, and $f'(1) = 1+f(1) = 1+2=3$.  
		This is positive, so the function is increasing at $x=1$
	\end{hint}
\end{question}

Here is an idea for getting a graphical understanding of the solutions to this differential equation:

\begin{itemize}
	\item Pick a regular grid of points.
	\item At each point, compute the slope given by the differential equation
	\item At each point, make a short line segment with that slope
\end{itemize}

\begin{definition}
	A \dfn{slope field} is a graphical aid to understanding a differential equation, formed by following the three steps above.
\end{definition}

The slope field for the differential equation $y' = x+y$ looks like this:

\begin{image}
{\def\length{sqrt(1+(x+y)^2)}
\begin{tikzpicture}
  \begin{axis}[
      xmin=-3, xmax=3,ymin=-3,ymax=3,domain=-3:3,view={0}{90},
      axis lines =center, xlabel=$x$, ylabel=$y$,
      every axis y label/.style={at=(current axis.above origin),anchor=south},
      every axis x label/.style={at=(current axis.right of origin),anchor=west},
      axis on top,
    ] 
    \addplot3 [penColor, quiver={u={1/\length}, v={(x+y)/(\length)},scale arrows=.2},samples=20] {0};
]  \end{axis}
\end{tikzpicture}}
\end{image}

Here is the same slope field, with a few solutions to the differential equation also graphed.

\begin{image}
{\def\length{sqrt(1+(x+y)^2)}
\begin{tikzpicture}
  \begin{axis}[
      xmin=-3, xmax=3,ymin=-3,ymax=3,domain=-3:3,view={0}{90},
      axis lines =center, xlabel=$x$, ylabel=$y$,
      every axis y label/.style={at=(current axis.above origin),anchor=south},
      every axis x label/.style={at=(current axis.right of origin),anchor=west},
      axis on top,
    ] 
    \addplot3 [penColor, quiver={u={1/\length}, v={(x+y)/(\length)},scale arrows=.2},samples=20] {0};
    \addplot[penColor,very thick]{e^x-x-1};
     \addplot[penColor,very thick]{-e^x-x-1};
     \addplot[penColor,very thick]{2*e^x-x-1};

]  \end{axis}
\end{tikzpicture}}
\end{image}

\begin{question}
Notice that the slope field clearly suggests one solution to this differential equation, which is a straight line.

This solution is 

\[
y = \answer{-1-x}
\]

\begin{hint}
The equation of this line is $y = -x-1$
\end{hint}
\end{question}

\begin{question}
Consider the differential equation $y' = (y+1)(2-y)$, whose slope field is given below.  Which of the following statements appear to be true?

\begin{image}
{\def\length{sqrt(1+((y+1)*(2-y))^2)}
\begin{tikzpicture}
  \begin{axis}[
      xmin=-3, xmax=3,ymin=-3,ymax=3,domain=-3:3,view={0}{90},
      axis lines =center, xlabel=$x$, ylabel=$y$,
      every axis y label/.style={at=(current axis.above origin),anchor=south},
      every axis x label/.style={at=(current axis.right of origin),anchor=west},
      axis on top,
    ] 
    \addplot3 [penColor, quiver={u={1/\length}, v={((y+1)*(2-y))/(\length)},scale arrows=.2},samples=20] {0};
]  \end{axis}
\end{tikzpicture}}
\end{image}

\begin{multipleChoice}
	\choice[correct]{The solution passing through the origin has $\lim_{x \to \infty} f(x) = 2$}
	\choice{The solution passing through $(0,3)$ has $\lim_{x \to \infty} f(x) = \infty$}
	\choice[correct]{The solution passing through $(0,-3)$ is always decreasing}
	\choice[correct]{There are two solutions which are constant functions}
	\choice{Every solution has a vertical tangent line at $x=0$}
\end{multipleChoice}
\end{question}

The differential equation $y'=(y+1)(y-2)$ does not involve the independent variable at all.  Such differential equations are called \dfn{autonomous} differential equations.

The constant functions $f(x) = -1$ and $f(x)  =2$ are solutions to this differential equation.  In fact, for any autonomous differential equation $y'  = A(y)$, 
where $A$ is a function of $y$, if $A(C) = 0$, then $y = C$ will be a constant solution to the differential equation.  
These constant solutions are also known as \dfn{equilibrium} solutions.

\begin{question}
Consider the autonomous differential equation $y'=\cos(y)$.  Which of the following are equilibrium solutions?

\begin{multipleChoice}
\choice{$0$}
\choice[correct]{$\frac{\pi}{2}$}
\choice{$\pi$}
\end{multipleChoice}
\end{question}

\section{Euler's Method}

In science and mathematics, finding exact solutions to differential equations is not always possible.
We have already seen that slope fields give us a powerful way to understand the qualitative features of solutions.
Sometimes we need a more precise quantitative understanding, i.e. we would like numerical approximations of the solutions.

\begin{example}
Suppose you have set up the following differential equation
\[
y' = x+y
\]

While one can solve this differential equation, we cannot solve it
\textit{yet}. 

If we know that $f$ solves this differential equation, and $f(0) = 1$, how might we go about approximating $f(1)$?

One idea is to repeatedly use linear approximation.

Let us approximate just using the two subintervals $[0,\frac{1}{2}]$ and $[\frac{1}{2},1]$.

Since $f'(0) = \answer{1}$, we know that $f(\frac{1}{2}) \approx 1+\answer{1} \cdot \frac{1}{2} = \answer{\frac{3}{2}}$ by linear approximation.

But now, we have $f'(\frac{1}{2}) \approx \frac{1}{2}+\frac{3}{2} = 2$ so $f(1) \approx  f(\frac{1}{2})+\frac{1}{2} \cdot 2 \approx \answer{\frac{5}{2}}$.

This approximation could be improved by using more subintervals
\end{example}

We will now formalize the method of using repeated linear approximation to approximate solutions to differential equations, and call it \dfn{Euler's Method}.

\begin{definition}[Euler's Method]
Consider a differential equation of the form $y' = R(x,y)$, where $R$ is a function of two variables.

If $f$ satisfies the initial value problem $f(a) = b$, and we want to approximate $f(T)$, then we follow this procedure:

\begin{itemize}
	\item Decide how many subintervals you want to divide the interval $[a,T]$ into.  Call this number $n$.
	\item Let $x_k = a+\frac{T-a}{n}k$ for $k=0,1,2,...,n-1$ be the left endpoints of the subintervals.
	\item Let $y_0 = b$
	\item Iteratively define $y_{k+1} = y_k+A(x_k,y_k)\frac{T-a}{n}$ 
\end{itemize}

Then $f(x_k) \approx y_k$, and so $f(T) \approx y_{n}$.

\end{definition}

%BADBAD Picture
	
\begin{question}
$f$ is a solution to 

\[
y'=x+y
\]

with initial condition $f(0)=1$.  We want to approximate $f(2)$.  

Using the notation from the definition, we have 

\[
A(x,y) = \answer{x+y}
\]
Fill out the following table for Euler's method using $4$ subintervals.

	\[
\begin{array}{l|l|l}
k & x_k & y_k \\ \hline
0 & 0   & 1 \\
1 & 0.5 & \answer{1.5} \\
2 & \answer{1} & \answer{2.5}  \\
3 & 1.5 & 4.25 \\
4 & 2 & \answer{7.125} \\
\end{array}
\]

	Thus $f(2) \approx \answer{7.125}$

\end{question}	


\end{document}