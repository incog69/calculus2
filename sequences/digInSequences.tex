\documentclass{ximera}

%\usepackage{todonotes}

\newcommand{\todo}{}

\usepackage{tkz-euclide}
\tikzset{>=stealth} %% cool arrow head
\tikzset{shorten <>/.style={ shorten >=#1, shorten <=#1 } } %% allows shorter vectors

\usetikzlibrary{backgrounds} %% for boxes around graphs
\usetikzlibrary{shapes,positioning}  %% Clouds and stars
\usetikzlibrary{matrix} %% for matrix
\usepgfplotslibrary{polar} %% for polar plots
\usetkzobj{all}
\usepackage[makeroom]{cancel} %% for strike outs
%\usepackage{mathtools} %% for pretty underbrace % Breaks Ximera
\usepackage{multicol}





\usepackage{array}
\setlength{\extrarowheight}{+.1cm}   
\newdimen\digitwidth
\settowidth\digitwidth{9}
\def\divrule#1#2{
\noalign{\moveright#1\digitwidth
\vbox{\hrule width#2\digitwidth}}}





\newcommand{\RR}{\mathbb R}
\newcommand{\R}{\mathbb R}
\newcommand{\N}{\mathbb N}
\newcommand{\Z}{\mathbb Z}

%\renewcommand{\d}{\,d\!}
\renewcommand{\d}{\mathop{}\!d}
\newcommand{\dd}[2][]{\frac{\d #1}{\d #2}}
\renewcommand{\l}{\ell}
\newcommand{\ddx}{\frac{d}{\d x}}

\newcommand{\zeroOverZero}{\ensuremath{\boldsymbol{\tfrac{0}{0}}}}
\newcommand{\inftyOverInfty}{\ensuremath{\boldsymbol{\tfrac{\infty}{\infty}}}}
\newcommand{\zeroOverInfty}{\ensuremath{\boldsymbol{\tfrac{0}{\infty}}}}
\newcommand{\zeroTimesInfty}{\ensuremath{\small\boldsymbol{0\cdot \infty}}}
\newcommand{\inftyMinusInfty}{\ensuremath{\small\boldsymbol{\infty - \infty}}}
\newcommand{\oneToInfty}{\ensuremath{\boldsymbol{1^\infty}}}
\newcommand{\zeroToZero}{\ensuremath{\boldsymbol{0^0}}}
\newcommand{\inftyToZero}{\ensuremath{\boldsymbol{\infty^0}}}


\newcommand{\numOverZero}{\ensuremath{\boldsymbol{\tfrac{\#}{0}}}}
\newcommand{\dfn}{\textbf}
%\newcommand{\unit}{\,\mathrm}
\newcommand{\unit}{\mathop{}\!\mathrm}
\newcommand{\eval}[1]{\bigg[ #1 \bigg]}
\newcommand{\seq}[1]{\left( #1 \right)}
\renewcommand{\epsilon}{\varepsilon}
\renewcommand{\iff}{\Leftrightarrow}

\DeclareMathOperator{\arccot}{arccot}
\DeclareMathOperator{\arcsec}{arcsec}
\DeclareMathOperator{\arccsc}{arccsc}
\DeclareMathOperator{\si}{Si}

\newcommand{\tightoverset}[2]{%
  \mathop{#2}\limits^{\vbox to -.5ex{\kern-0.75ex\hbox{$#1$}\vss}}}
\newcommand{\arrowvec}[1]{\tightoverset{\scriptstyle\rightharpoonup}{#1}}
\renewcommand{\vec}{\mathbf}


\colorlet{textColor}{black} 
\colorlet{background}{white}
\colorlet{penColor}{blue!50!black} % Color of a curve in a plot
\colorlet{penColor2}{red!50!black}% Color of a curve in a plot
\colorlet{penColor3}{red!50!blue} % Color of a curve in a plot
\colorlet{penColor4}{green!50!black} % Color of a curve in a plot
\colorlet{penColor5}{orange!80!black} % Color of a curve in a plot
\colorlet{fill1}{penColor!20} % Color of fill in a plot
\colorlet{fill2}{penColor2!20} % Color of fill in a plot
\colorlet{fillp}{fill1} % Color of positive area
\colorlet{filln}{penColor2!20} % Color of negative area
\colorlet{fill3}{penColor3!20} % Fill
\colorlet{fill4}{penColor4!20} % Fill
\colorlet{fill5}{penColor5!20} % Fill
\colorlet{gridColor}{gray!50} % Color of grid in a plot

\newcommand{\surfaceColor}{violet}
\newcommand{\surfaceColorTwo}{redyellow}
\newcommand{\sliceColor}{greenyellow}




\pgfmathdeclarefunction{gauss}{2}{% gives gaussian
  \pgfmathparse{1/(#2*sqrt(2*pi))*exp(-((x-#1)^2)/(2*#2^2))}%
}


%%%%%%%%%%%%%
%% Vectors
%%%%%%%%%%%%%

%% Simple horiz vectors
\renewcommand{\vector}[1]{\left\langle #1\right\rangle}


%% %% Complex Horiz Vectors with angle brackets
%% \makeatletter
%% \renewcommand{\vector}[2][ , ]{\left\langle%
%%   \def\nextitem{\def\nextitem{#1}}%
%%   \@for \el:=#2\do{\nextitem\el}\right\rangle%
%% }
%% \makeatother

%% %% Vertical Vectors
%% \def\vector#1{\begin{bmatrix}\vecListA#1,,\end{bmatrix}}
%% \def\vecListA#1,{\if,#1,\else #1\cr \expandafter \vecListA \fi}

%%%%%%%%%%%%%
%% End of vectors
%%%%%%%%%%%%%

%\newcommand{\fullwidth}{}
%\newcommand{\normalwidth}{}



%% makes a snazzy t-chart for evaluating functions
%\newenvironment{tchart}{\rowcolors{2}{}{background!90!textColor}\array}{\endarray}

%%This is to help with formatting on future title pages.
\newenvironment{sectionOutcomes}{}{} 



%% Flowchart stuff
%\tikzstyle{startstop} = [rectangle, rounded corners, minimum width=3cm, minimum height=1cm,text centered, draw=black]
%\tikzstyle{question} = [rectangle, minimum width=3cm, minimum height=1cm, text centered, draw=black]
%\tikzstyle{decision} = [trapezium, trapezium left angle=70, trapezium right angle=110, minimum width=3cm, minimum height=1cm, text centered, draw=black]
%\tikzstyle{question} = [rectangle, rounded corners, minimum width=3cm, minimum height=1cm,text centered, draw=black]
%\tikzstyle{process} = [rectangle, minimum width=3cm, minimum height=1cm, text centered, draw=black]
%\tikzstyle{decision} = [trapezium, trapezium left angle=70, trapezium right angle=110, minimum width=3cm, minimum height=1cm, text centered, draw=black]


\outcome{Identify a sequence.}
\outcome{Write the first several terms of a sequence given an explicit formula.}
\outcome{Write the first several terms of a sequence given a recurrance
  relation.}
\outcome{Given a sequence find an explicit formula.}
\outcome{Given a sequence find a recurrance relation.}

\title[Dig-In:]{Sequences}

\begin{document}
\begin{abstract}
  A function from positive integers to the real numbers is a sequence.
\end{abstract}
\maketitle


Let's get to the heart of the matter:

\begin{definition}
  A \dfn{sequence} (of things) is an ordered list of things.
\end{definition}

In our class, we're mostly talking about a sequence of real numbers.
For example, here is a sequence, meaning an ordered list of numbers:
\[
1,1, 2, 3, 5, 8, 13, 21, \ldots
\]
Note that numbers in the list can repeat.  And consider those little
dots at the end!  The dots ``\ldots'' signify that the list keeps
going, and going, and going, forever.

Sequences can be represented in a few different, but equivalent, ways;
you might see a sequence written as
\begin{align*}
  & a_1, a_2,  a_3, \ldots, \\
  & a_n \\
  & \left(a_n\right)_{n \in \N}, \\
  & \left\{a_n\right\}_{n=1}^\infty, \\
  & \left\{f(n)\right\}_{n=1}^\infty, \quad \text{or} \\
  & \left(f(n)\right)_{n \in \N},
\end{align*}
depending on which author you read, here $\N=\{1,2,3,\ldots\}$.  Worse, depending on the
situation, the same author (and this author) might use various
notations for a sequence!  We will usually write $(a_n)$ if I want to
speak of the sequence as a whole and write $a_n$ if we are speaking of
a specific element in the sequence.


\begin{question}
  Considering the sequence
  \[
  1, 2, 4, 8, 16, \dots
  \]
  what number comes next?
  \begin{multipleChoice}
    \choice{$32$}
    \choice{$31$}
    \choice{$18$}
    \choice[correct]{there is no way to know}
  \end{multipleChoice}
  \begin{feedback}
    Sequences need to be defined by a ``rule.'' Without such a rule,
    it is difficult, if not impossible, to say which element comes
    next.
  \end{feedback}
\end{question}

\begin{example}
  Let $a_n = 2^{n-1}$.  Write down $a_1$, $a_2$, $a_3$, $a_4$, $a_5$,
  $a_6$.
  \begin{explanation}
    Just use the formula to see:
    \begin{align*}
      a_1 &= \answer[given]{1}\\
      a_2 &= \answer[given]{2}\\
      a_3 &= \answer[given]{4}\\
      a_4 &= \answer[given]{8}\\
      a_5 &= \answer[given]{16}\\
      a_6 &= \answer[given]{32}
    \end{align*}
  \end{explanation}
\end{example}


\begin{example}
  Consider a circle with $n$ points on it. Let $b_n$ be the maximum
  number of regions produced by connecting these points with
  chords. Write down $b_1$, $b_2$, $b_3$, $b_4$, $b_5$, $b_6$.
  \begin{explanation}
    There is only one way to solve this problem: Start drawing
    pictures and counting regions.
    \begin{itemize}
      \item \begin{tikzpicture}[framed,scale=1,baseline=-1ex]      
            \tkzDefPoint(0,0){O} 
            \tkzDefPoint(1,0){A} 
            \tkzDrawCircle[color=penColor,very thick](O,A)
            \tkzDrawPoint[color=penColor,fill=penColor](A)
      \end{tikzpicture} We see that $b_1 = \answer[given]{1}$
      \item \begin{tikzpicture}[framed,scale=1,baseline=-1ex]      
            \tkzDefPoint(0,0){O} 
            \tkzDefPoint(1,0){A} 
            \tkzDefPoint(-1,0){B}
            \tkzDrawCircle[color=penColor,very thick](O,A)
            \tkzDrawPoint[color=penColor,fill=penColor](A)
            \tkzDrawPoint[color=penColor,fill=penColor](B)
            \tkzDrawSegment[color=penColor](A,B)
      \end{tikzpicture} We see that $b_2 = \answer[given]{2}$
      \item \begin{tikzpicture}[framed,scale=1,baseline=-1ex]      
            \tkzDefPoint(0,0){O} 
            \tkzDefPoint(1,0){A} 
            \tkzDefPoint(-.6,.8){B}
            \tkzDefPoint(-.6,-.8){C}
            \tkzDrawCircle[color=penColor,very thick](O,A)
            \tkzDrawPoint[color=penColor,fill=penColor](A)
            \tkzDrawPoint[color=penColor,fill=penColor](B)
            \tkzDrawPoint[color=penColor,fill=penColor](C)
            \tkzDrawSegment[color=penColor](A,B)
            \tkzDrawSegment[color=penColor](A,C)
            \tkzDrawSegment[color=penColor](B,C)
      \end{tikzpicture} We see that $b_3 = \answer[given]{4}$
      \item \begin{tikzpicture}[framed,scale=1,baseline=-1ex]      
            \tkzDefPoint(0,0){O} 
            \tkzDefPoint(.6,.8){A} 
            \tkzDefPoint(-.6,.8){B}
            \tkzDefPoint(-.6,-.8){C}
            \tkzDefPoint(.6,-.8){D}
            \tkzDrawCircle[color=penColor,very thick](O,A)
            \tkzDrawPoint[color=penColor,fill=penColor](A)
            \tkzDrawPoint[color=penColor,fill=penColor](B)
            \tkzDrawPoint[color=penColor,fill=penColor](C)
            \tkzDrawPoint[color=penColor,fill=penColor](D)
            \tkzDrawSegment[color=penColor](A,B)
            \tkzDrawSegment[color=penColor](A,C)
            \tkzDrawSegment[color=penColor](A,D)
            \tkzDrawSegment[color=penColor](B,C)
            \tkzDrawSegment[color=penColor](B,D)
            \tkzDrawSegment[color=penColor](C,D)
      \end{tikzpicture} We see that $b_4 = \answer[given]{8}$
        \item \begin{tikzpicture}[framed,scale=1,baseline=-1ex]      
            \tkzDefPoint(0,0){O} 
            \tkzDefPoint(1,0){A} 
            \tkzDefPoint(.27,.96){B}
            \tkzDefPoint(-.8,.6){C}
            \tkzDefPoint(-.8,-.6){D}
            \tkzDefPoint(.33,-.95){E}
            \tkzDrawCircle[color=penColor,very thick](O,A)
            \tkzDrawPoint[color=penColor,fill=penColor](A)
            \tkzDrawPoint[color=penColor,fill=penColor](B)
            \tkzDrawPoint[color=penColor,fill=penColor](C)
            \tkzDrawPoint[color=penColor,fill=penColor](D)
            \tkzDrawPoint[color=penColor,fill=penColor](E)
            \tkzDrawSegment[color=penColor](A,B)
            \tkzDrawSegment[color=penColor](A,C)
            \tkzDrawSegment[color=penColor](A,D)
            \tkzDrawSegment[color=penColor](A,E)
            \tkzDrawSegment[color=penColor](B,C)
            \tkzDrawSegment[color=penColor](B,D)
            \tkzDrawSegment[color=penColor](B,E)
            \tkzDrawSegment[color=penColor](C,D)
            \tkzDrawSegment[color=penColor](C,E)
            \tkzDrawSegment[color=penColor](D,E)
        \end{tikzpicture} We see that $b_5 = \answer[given]{16}$
        \item \begin{tikzpicture}[framed,scale=1,baseline=-1ex]      
            \tkzDefPoint(0,0){O} 
            \tkzDefPoint(1,0){A} 
            \tkzDefPoint(0,1){B}
            \tkzDefPoint(-.8,.6){C}
            \tkzDefPoint(-.8,-.6){D}
            \tkzDefPoint(0,-1){E}
            \tkzDefPoint(.7,-.71){F}
            \tkzDrawCircle[color=penColor,very thick](O,A)
            \tkzDrawPoint[color=penColor,fill=penColor](A)
            \tkzDrawPoint[color=penColor,fill=penColor](B)
            \tkzDrawPoint[color=penColor,fill=penColor](C)
            \tkzDrawPoint[color=penColor,fill=penColor](D)
            \tkzDrawPoint[color=penColor,fill=penColor](E)
            \tkzDrawPoint[color=penColor,fill=penColor](F)
            \tkzDrawSegment[color=penColor](A,B)
            \tkzDrawSegment[color=penColor](A,C)
            \tkzDrawSegment[color=penColor](A,D)
            \tkzDrawSegment[color=penColor](A,E)
            \tkzDrawSegment[color=penColor](A,F)
            \tkzDrawSegment[color=penColor](B,C)
            \tkzDrawSegment[color=penColor](B,D)
            \tkzDrawSegment[color=penColor](B,E)
            \tkzDrawSegment[color=penColor](B,F)
            \tkzDrawSegment[color=penColor](C,D)
            \tkzDrawSegment[color=penColor](C,E)
            \tkzDrawSegment[color=penColor](C,F)
            \tkzDrawSegment[color=penColor](D,E)
            \tkzDrawSegment[color=penColor](D,F)
            \tkzDrawSegment[color=penColor](E,F)
            \end{tikzpicture} We see that $b_6 = \answer[given]{31}$
    \end{itemize}
    While we have made no real argument that these are the maximum
    number of regions, we belive that if the young mathematician draws
    more pictures they will be convinced.
  \end{explanation}
\end{example}


From the two examples above, it is hoped that you see that ``finding a
pattern'' is \textbf{not enough} when dealing with sequences. You need
to understand how \textbf{exactly} the sequence was produced.




\section{Defining sequences by a function}

Just as real-valued functions were usually expressed by a formula, we
will most often encounter sequences that can be expressed by a
formula.  Examples are easy to cook up, as now \textit{many} functions
can define sequences
\begin{align*}
  a_i &=\frac{i}{i+1}, \\
  b_n &=\frac{1}{2^n}, \\
  c_j &=\sin(j\pi/6), \text{ or} \\
  d_k &=\frac{(k-1)(k+2)}{2^k}. 
\end{align*}
Frequently these formulas will make sense if thought of either as
functions with domain $\R$ or $\N$, though occasionally the given
formula will make sense only for integers (positive, zero, and
negative whole numbers).

\begin{warning}
  A common misconception is to confuse the sequence with the rule for
  generating the sequence.  The sequences $(a_n)$ and $(b_n)$ given by
  the rules $a_n = (-1)^n$ and $b_n = \cos (\pi \cdot n)$ are, despite
  appearances, different rules which give rise to the \textit{same}
  sequence.  These are just different names for the same object.
\end{warning}

\begin{definition}
  Suppose $(a_n)$ and $(b_n)$ are sequences starting at $1$.  These
  sequences are \dfn{equal}\index{sequence!equality} if for all
  natural numbers $n$, we have $a_n = b_n$.

  More generally, two sequences $(a_n)$ and $(b_n)$ are
  \dfn{equal} if they have the same initial index $N$, and for
  every integer $n \geq N$, the $n$th terms have the same value, that is,
  \[
  a_n = b_n \quad \text{for all $n \geq N$.}
  \]
\end{definition}
In other words, sequences are the same if they have the same set of
valid indexes, and produce the same real numbers for each of those
indexes---regardless of whether the given ``rules'' or procedures for
computing those sequences resemble each other in any way.





\section{Defining sequences by previous elements}


Another way to define a sequence is \index{recursive|see {recursive}}\textit{recursively},
that is, by defining the later outputs in terms of previous outputs.
We start by defining the first few elements of the sequence, and then
describe how later elements are computed in terms of previous
elements.

\begin{example}
Define a sequence recursively by
\[
a_1 = 1, \quad a_2 = 3, \quad a_3 = 10,
\]
and the rule that $a_n = a_{n-1} - a_{n-3}$.  Compute $a_5$.
\begin{explanation}
  First we compute $a_4$.  Substituting $4$ for $n$ in the recursive
  rule $a_n = a_{n-1} - a_{n-3}$, we find
  \begin{align*}
  a_4 &= a_{\answer[given]{4}-1} - a_{\answer[given]{4}-3} \\
  &= a_3 - a_1.
  \end{align*}
  But we have values for $a_3$ and $a_1$, namely $10$ and $1$,
  respectively.  Therefore $a_4 = 10 - 1 = \answer[given]{9}$.  Now we
  are in a position to compute $a_5$.  Substituting $5$ for $n$ in the
  rule $a_n = a_{n-1} - a_{n-3}$, we find
  \[
  a_5 = a_{5-1} - a_{5-3} = a_4 - a_2.
  \]
  We just computed $a_4 = 9$; we were given $a_2 = 3$.  Therefore
  \begin{align*}
    a_5&= \answer[given]{9} - 3
    &= \answer[given]{6}.
  \end{align*}
\end{explanation}
\end{example}


\begin{question}
  Consider the sequence $a_{n}$ defined recursively by the
  rule
  \[
  a_n = {a_{n-1}} {a_{n-2}} + 3 \, {a_{n-1}} - {a_{n-2}}
  \]
  and the facts that $a_0 = -3$ and $a_1 = 5$.  What is $a_4$?
  
    \begin{hint}
      We have been told the first two terms of the sequence, namely $a_0 = -3$ and $a_1 = 5$.
    \end{hint}
    \begin{hint}
      We also have a rule $a_n = {a_{n-1}} {a_{n-2}} + 3 \, {a_{n-1}} - {a_{n-2}}$ which lets us compute the third term $a_2$ using these first two terms $a_0$ and $a_1$.
    \end{hint}
    \begin{hint}
      To compute $a_2$, we set $n = 2$ in the recursive rule $a_n = {a_{n-1}} {a_{n-2}} + 3 \, {a_{n-1}} - {a_{n-2}}$.  This gives us $a_2 = {a_{2-1}} {a_{2-2}} + 3 \, {a_{2-1}} - {a_{2-2}}$.
    \end{hint}
    \begin{hint}
      Plugging $a_{2-1} = a_{1} = 5$ and $a_{2-2} = a_{0} = -3$ into the rule, we learn $a_2 = 3$.
    \end{hint}
    \begin{hint}
      To compute $a_3$, we set $n = 3$ in the recursive rule $a_n = {a_{n-1}} {a_{n-2}} + 3 \, {a_{n-1}} - {a_{n-2}}$, giving $a_3 = {a_{3-1}} {a_{3-2}} + 3 \, {a_{3-1}} - {a_{3-2}}$.
    \end{hint}
    \begin{hint}
      To evaluate that, we will have to know $a_{3-1} = a_{2}$, but we just found out that $a_{2} = 3$.
    \end{hint}
    \begin{hint}
      Plugging $a_{3-1} = 3$ and $a_{3-2} = a_{1} = 5$ into the rule, we find $a_3 = 19$.
    \end{hint}
    \begin{hint}
      To compute $a_4$, we set $n = 4$ in the recursive rule $a_n = {a_{n-1}} {a_{n-2}} + 3 \, {a_{n-1}} - {a_{n-2}}$, giving $a_4 = {a_{4-1}} {a_{4-2}} + 3 \, {a_{4-1}} - {a_{4-2}}$.
    \end{hint}
    \begin{hint}
      To evaluate that, we will have to know $a_{4-1} = a_{3}$, but we just found out that $a_{3} = 19$.
    \end{hint}
    \begin{hint}
      Plugging $a_{4-1} = 19$ and $a_{4-2} = a_{2} = 3$ into the rule, we learn $a_4 = 111$.
    \end{hint}
    \begin{hint}
      So we conclude $a_4 = 111$.
    \end{hint}
    \begin{prompt}
      \[
      a_4 = \answer{111}.
      \]
    \end{prompt}
\end{question}






\section{Arithmetic sequences}


The first family we consider are the ``arithmetic'' sequences.  Here
is a definition.

%{Mathematically, the word \dfn{family}
%  does not have an entirely precise definition; a family of things is
%  a \dfn{collection} or a \dfn{set} of things, but family
%  also has a connotation of some sort of relatedness.} 

\begin{definition}
  An \dfn{arithmetic sequence} (sometimes called an arithmetic
  progression)\index{arithmetic progression} is a sequence where the
  difference between subsequent elements is constant.
\end{definition}


\begin{example}
  Here is an example of an arithmetic sequence:
  \begin{image}
  \begin{tikzpicture}[node distance=1.5cm]
    \node (a1) {$-5$,};
    \node (a2) [right of=a1] {$-1$,};
    \node (a3) [right of=a2] {$3$,};
    \node (a4) [right of=a3] {$7$,};
    \node (a5) [right of=a4] {$11$,};
    \node (a6) [right of=a5] {$\ldots$};

    \path[->] (a1) edge [bend left=20] node[above]{$+4$} (a2);
    \path[->] (a2) edge [bend left=20] node[above]{$+4$} (a3);
    \path[->] (a3) edge [bend left=20] node[above]{$+4$} (a4);
    \path[->] (a4) edge [bend left=20] node[above]{$+4$} (a5);
    \path[->] (a5) edge [bend left=20] node[above]{$+4$} (a6);
  \end{tikzpicture}
  \end{image}
  This sequence is given by the function $a_n=\answer[given]{4n-9}$,
  or the recursive rule $a_1 = \answer[given]{-5}$ and $a_{i+1} = a_i
  + \answer[given]{4}$. Each term differs from the previous by four.
\end{example}

An arithmetic sequence can also decrease as it progresses.

\begin{example}
  Here is an example of an arithmetic sequence:
  \begin{image}
  \begin{tikzpicture}[node distance=1.5cm]
    \node (a1) {$17$,};
    \node (a2) [right of=a1] {$15$,};
    \node (a3) [right of=a2] {$13$,};
    \node (a4) [right of=a3] {$11$,};
    \node (a5) [right of=a4] {$9$,};
    \node (a6) [right of=a5] {$\ldots$};

    \path[->] (a1) edge [bend left=20] node[above]{$-2$} (a2);
    \path[->] (a2) edge [bend left=20] node[above]{$-2$} (a3);
    \path[->] (a3) edge [bend left=20] node[above]{$-2$} (a4);
    \path[->] (a4) edge [bend left=20] node[above]{$-2$} (a5);
    \path[->] (a5) edge [bend left=20] node[above]{$-2$} (a6);
  \end{tikzpicture}
  \end{image}
  This sequence is given by the function $a_n=\answer[given]{-2n + 19}$, or
  the recursive rule $a_1 = \answer[given]{17}$ and $a_{i+1} = a_i +
  \answer[given]{(-2)}$. Each term differs from the previous by negative
  two.
\end{example}



In general, an arithmetic sequence in which subsequent terms differ
by $m$ can be written as
\[
a_n = m (n-1) + a_1.
\]
Alternatively, we could describe an arithmetic sequence recursively,
by giving a starting value $a_1$, and using the rule that $a_{n} =
a_{n-1} + m$.


\begin{remark}
Why are arithmetic sequences called \textit{arithmetic}?  Note that
every term is the \dfn{arithmetic mean} or \dfn{average} of its two
neighbors.  The aritmetic mean of two numbers $a$ and $b$ is defined
to be $\frac{a+b}{2}$.
\end{remark}



\begin{question}
  Which of the following sequences are arithmetic sequences for
  $n=1,2,3,\dots$?
  \begin{selectAll}
    \choice[correct]{$a_n = -3n + 5$}
    \choice{$a_n = -5\cdot 7^{n+4}$}
    \choice[correct]{$a_n = 5$}
    \choice{$a_n = \left(\frac{1}{3}\right)\cdot a_{n-1}$ where $a_1 = 10$}
    \choice[correct]{$a_n = \frac{4n^2+n-14}{n+2}$}
    \choice{$a_n = 2^{n\cdot \cos(2\pi n + pi))}$} 
    \choice{$a_n = \cos(\pi/2^n)$}
    \choice{$a_n = \cos(\pi n -5)$}
    \choice[correct]{$a_n = n\cdot \cos(2\pi n) +4$}
  \end{selectAll}
\end{question}


\subsection{Arithmetic sequences in the wild}


BADBAD HERE!!!!


\section{Geometric sequences}


The second family of sequences we consider are ``geometric''
sequences.

\begin{definition}
  A \dfn{geometric sequence} (sometimes called a geometric
  progression)\index{geometric progression} is a sequence where the
  ratio between subsequent elements is constant.
\end{definition}

\begin{example}
  Here is an example of a geometric sequence:
  \begin{image}
    \begin{tikzpicture}[node distance=1.5cm]
    \node (a1) {$10$,};
    \node (a2) [right of=a1] {$30$,};
    \node (a3) [right of=a2] {$90$,};
    \node (a4) [right of=a3] {$270$,};
    \node (a5) [right of=a4] {$810$,};
    \node (a6) [right of=a5] {$\ldots$};

    \path[->] (a1) edge [bend left=20] node[above] {$\times 3$} (a2);
    \path[->] (a2) edge [bend left=20] node[above] {$\times 3$} (a3);
    \path[->] (a3) edge [bend left=20] node[above] {$\times 3$} (a4);
    \path[->] (a4) edge [bend left=20] node[above] {$\times 3$} (a5);
    \path[->] (a5) edge [bend left=20] node[above] {$\times 3$} (a6);
  \end{tikzpicture}
  \end{image}
  This sequence is given by the function $a_n=\answer[given]{10\cdot
    3^{n-1}}$, or the recursive rule $a_1 = \answer[given]{10}$ and
  $a_{i+1} = \answer[given]{3}\cdot a_i$. Each term differs from the
  previous by a multiple of three.
\end{example}

A geometric sequence can also decrease as it progresses.

\begin{example}
  Here is an example of a geometric sequence:
  \begin{image}
    \begin{tikzpicture}[node distance=1.5cm]
    \node (a1) {$\frac{7}{5}$,};
    \node (a2) [right of=a1] {$\frac{7}{10}$,};
    \node (a3) [right of=a2] {$\frac{7}{20}$,};
    \node (a4) [right of=a3] {$\frac{7}{40}$,};
    \node (a5) [right of=a4] {$\frac{7}{80}$,};
    \node (a6) [right of=a5] {$\ldots$};

    \path[->] (a1) edge [bend left=20] node[above] {$\times\frac{1}{2}$} (a2);
    \path[->] (a2) edge [bend left=20] node[above] {$\times\frac{1}{2}$} (a3);
    \path[->] (a3) edge [bend left=20] node[above] {$\times\frac{1}{2}$} (a4);
    \path[->] (a4) edge [bend left=20] node[above] {$\times\frac{1}{2}$} (a5);
    \path[->] (a5) edge [bend left=20] node[above] {$\times\frac{1}{2}$} (a6);
  \end{tikzpicture}
  \end{image}
  This sequence is given by the function $a_n=\answer[given]{\left(\frac{7}{5}\right)\cdot
    \left(\frac{1}{2}\right)^{n-1}}$, or the recursive rule $a_1 = \answer[given]{\frac{7}{5}}$ and
  $a_{i+1} = \answer[given]{\left(\frac{1}{2}\right)}\cdot a_i$. Each term differs from the
  previous by a multiple of one-half.
\end{example}



In general, a geometric sequence in which the ratio between
subsequent terms is $r$ can be written as
\[
a_n = a_1 \cdot r^{n-1}.
\]
Alternatively, we could describe a geometric sequence
recursively, by giving a starting value $a_1$, and using the rule that
$a_{n} = r \cdot a_{n-1}$.

\begin{remark}
Why are geometric sequences called \textit{geometric}?  Note that
every term is the \dfn{geometric mean} of its two neighbors.  The
geometric mean of two numbers $a$ and $b$ is defined to be
$\sqrt{ab}$.

Of course, that raises another question: why is the geometric mean
called \textit{geometric?}  One geometric interpretation of the
geometric mean of $a$ and $b$ is this: the geometric mean is the side
length of a square whose area is equal to that of the rectangle having
side lengths $a$ and $b$.
\end{remark}

\begin{question}
  Which of the following sequences are arithmetic sequences for
  $n=1,2,3,\dots$?
  \begin{selectAll}
    \choice{$a_n = -3n + 5$}
    \choice[correct]{$a_n = -5\cdot 7^{n+4}$}
    \choice[correct]{$a_n = 5$}
    \choice[correct]{$a_n = \left(\frac{1}{3}\right)\cdot a_{n-1}$ where $a_1 = 10$}
    \choice{$a_n = \frac{4n^2+n-14}{n+2}$}
    \choice[correct]{$a_n = 2^{n\cdot \cos(2\pi n + pi))}$} 
    \choice{$a_n = \cos(\pi/2^n)$}
    \choice{$a_n = \cos(\pi n -5)$}
    \choice{$a_n = n\cdot \cos(2\pi n) +4$}
  \end{selectAll}
\end{question}

\subsection{Geometric sequences in the wild}




\section{Unsolved mysteries}

\subsection{Perfect numbers}


\subsection{Collatz}

Here is a fun sequence with which to amuse your friends---or distract
your enemies.  Let's start our sequence with $a_1 = 6$.  Subsequent
terms are defined using the rule
$$
a_n = \begin{cases} a_{n-1} / 2 & \text{ if $a_{n-1}$ is even, and } \\
3 \, a_{n-1} + 1 & \text{ if $a_{n-1}$ is odd.}
\end{cases}
$$
Let's compute $a_2$.  Since $a_1$ is even, we follow the instructions
in the first line, to find that $a_2 = a_1/2 = 3$. To compute $a_3$,
note that $a_2$ is odd so we follow the instruction in the second
line, and $a_3 = 3 \, a_2 + 1 = 3 \cdot 3 + 1 = 10$.  Since $a_3$ is
even, the first line applies, and $a_4 = a_3 / 2 = 10 / 2 = 5$.  But
$a_4$ is odd, so the second line applies, and we find $a_5 = 3 \cdot 5
+ 1 = 16$.  And $a_5$ is even, so $a_6 = 16 / 2 = 8$.  And $a_6$ is
even, so $a_7 = 8/4 = 4$.  And $a_7$ is even, so $a_8 = 4 / 2 = 2$,
and then $a_9 = 2/2 = 1$.  Oh, but $a_9$ is odd, so $a_{10} = 3 \cdot
1 + 1 = 4$.  And it repeats.  Let's write down the start of this sequence:
$$
6,\quad %1 
3,\quad %2
10,\quad  %3
5,\quad  %4
16,\quad  %5
8,\quad  %6
4,\quad  %7
2,\quad  %8
1,\quad  %9
4,\quad %10
2,\quad %11
1,\quad %12
\overbrace{4,\quad %10
2,\quad %11
1,}^{\text{repeats}}\quad %12
4,\quad %10
\ldots
$$
What if we had started with a number other than six?  What if we set
$a_1 = 25$ but then we used the same rule?  In that case, since $a_1$
is odd, we compute $a_2$ by finding $3 \, a_1 + 1 = 3 \cdot 25 + 1 =
76$.  Since $76$ is even, the next term is half that, meaning $a_3 =
38$.  If we keep this up, we find that our sequence begins
\begin{align*}
&25,\quad 76,\quad 38,\quad 19,\quad 58,\quad 29,\quad 88,\quad 44,\quad 22,\quad 11,\quad 34,\quad 17,\quad 52,\quad 26, \\
&13,\quad 40,\quad 20,\quad 10,\quad 5,\quad 16,\quad 8,\quad 4,\quad 2, \quad 1, \quad \ldots
\end{align*}
and then it repeats ``4, 2, 1, 4, 2, 1, \ldots'' just like before.

If you think you have an argument that answers the Collatz conjecture, I challenge you to try your hand at the $5x+1$ conjecture, that is, use the rule
\[
a_n = \displaystyle\begin{cases} a_{n-1} / 2 & \text{ if $a_{n-1}$ is even, and } \\
5 \, a_{n-1} + 1 & \text{ if $a_{n-1}$ is odd.}
\end{cases}
\]

Does this always happen?  Is it true that no matter which positive
integer you start with, if you apply the half-if-even, $3x+1$-if-odd
rule, you end up getting stuck in the ``4, 2, 1, \ldots'' loop?  That
this is true is the \dfn{Collatz conjecture}; it has been
verified for all starting values below $5 \times 2^{60}$.  Nobody has
found a value which doesn't return to one, but for all anybody knows
there \textit{might} well be a very large initial value which doesn't
return to one; nobody knows either way.  It is an unsolved
problem


This is not the last unsolved problem we will
  encounter in this course.  There are many things which humans do not
  understand. in mathematics.



\end{document}
