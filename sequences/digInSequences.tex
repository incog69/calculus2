\documentclass{ximera}

%\usepackage{todonotes}

\newcommand{\todo}{}

\usepackage{tkz-euclide}
\tikzset{>=stealth} %% cool arrow head
\tikzset{shorten <>/.style={ shorten >=#1, shorten <=#1 } } %% allows shorter vectors

\usetikzlibrary{backgrounds} %% for boxes around graphs
\usetikzlibrary{shapes,positioning}  %% Clouds and stars
\usetikzlibrary{matrix} %% for matrix
\usepgfplotslibrary{polar} %% for polar plots
\usetkzobj{all}
\usepackage[makeroom]{cancel} %% for strike outs
%\usepackage{mathtools} %% for pretty underbrace % Breaks Ximera
\usepackage{multicol}





\usepackage{array}
\setlength{\extrarowheight}{+.1cm}   
\newdimen\digitwidth
\settowidth\digitwidth{9}
\def\divrule#1#2{
\noalign{\moveright#1\digitwidth
\vbox{\hrule width#2\digitwidth}}}





\newcommand{\RR}{\mathbb R}
\newcommand{\R}{\mathbb R}
\newcommand{\N}{\mathbb N}
\newcommand{\Z}{\mathbb Z}

%\renewcommand{\d}{\,d\!}
\renewcommand{\d}{\mathop{}\!d}
\newcommand{\dd}[2][]{\frac{\d #1}{\d #2}}
\renewcommand{\l}{\ell}
\newcommand{\ddx}{\frac{d}{\d x}}

\newcommand{\zeroOverZero}{\ensuremath{\boldsymbol{\tfrac{0}{0}}}}
\newcommand{\inftyOverInfty}{\ensuremath{\boldsymbol{\tfrac{\infty}{\infty}}}}
\newcommand{\zeroOverInfty}{\ensuremath{\boldsymbol{\tfrac{0}{\infty}}}}
\newcommand{\zeroTimesInfty}{\ensuremath{\small\boldsymbol{0\cdot \infty}}}
\newcommand{\inftyMinusInfty}{\ensuremath{\small\boldsymbol{\infty - \infty}}}
\newcommand{\oneToInfty}{\ensuremath{\boldsymbol{1^\infty}}}
\newcommand{\zeroToZero}{\ensuremath{\boldsymbol{0^0}}}
\newcommand{\inftyToZero}{\ensuremath{\boldsymbol{\infty^0}}}


\newcommand{\numOverZero}{\ensuremath{\boldsymbol{\tfrac{\#}{0}}}}
\newcommand{\dfn}{\textbf}
%\newcommand{\unit}{\,\mathrm}
\newcommand{\unit}{\mathop{}\!\mathrm}
\newcommand{\eval}[1]{\bigg[ #1 \bigg]}
\newcommand{\seq}[1]{\left( #1 \right)}
\renewcommand{\epsilon}{\varepsilon}
\renewcommand{\iff}{\Leftrightarrow}

\DeclareMathOperator{\arccot}{arccot}
\DeclareMathOperator{\arcsec}{arcsec}
\DeclareMathOperator{\arccsc}{arccsc}
\DeclareMathOperator{\si}{Si}

\newcommand{\tightoverset}[2]{%
  \mathop{#2}\limits^{\vbox to -.5ex{\kern-0.75ex\hbox{$#1$}\vss}}}
\newcommand{\arrowvec}[1]{\tightoverset{\scriptstyle\rightharpoonup}{#1}}
\renewcommand{\vec}{\mathbf}


\colorlet{textColor}{black} 
\colorlet{background}{white}
\colorlet{penColor}{blue!50!black} % Color of a curve in a plot
\colorlet{penColor2}{red!50!black}% Color of a curve in a plot
\colorlet{penColor3}{red!50!blue} % Color of a curve in a plot
\colorlet{penColor4}{green!50!black} % Color of a curve in a plot
\colorlet{penColor5}{orange!80!black} % Color of a curve in a plot
\colorlet{fill1}{penColor!20} % Color of fill in a plot
\colorlet{fill2}{penColor2!20} % Color of fill in a plot
\colorlet{fillp}{fill1} % Color of positive area
\colorlet{filln}{penColor2!20} % Color of negative area
\colorlet{fill3}{penColor3!20} % Fill
\colorlet{fill4}{penColor4!20} % Fill
\colorlet{fill5}{penColor5!20} % Fill
\colorlet{gridColor}{gray!50} % Color of grid in a plot

\newcommand{\surfaceColor}{violet}
\newcommand{\surfaceColorTwo}{redyellow}
\newcommand{\sliceColor}{greenyellow}




\pgfmathdeclarefunction{gauss}{2}{% gives gaussian
  \pgfmathparse{1/(#2*sqrt(2*pi))*exp(-((x-#1)^2)/(2*#2^2))}%
}


%%%%%%%%%%%%%
%% Vectors
%%%%%%%%%%%%%

%% Simple horiz vectors
\renewcommand{\vector}[1]{\left\langle #1\right\rangle}


%% %% Complex Horiz Vectors with angle brackets
%% \makeatletter
%% \renewcommand{\vector}[2][ , ]{\left\langle%
%%   \def\nextitem{\def\nextitem{#1}}%
%%   \@for \el:=#2\do{\nextitem\el}\right\rangle%
%% }
%% \makeatother

%% %% Vertical Vectors
%% \def\vector#1{\begin{bmatrix}\vecListA#1,,\end{bmatrix}}
%% \def\vecListA#1,{\if,#1,\else #1\cr \expandafter \vecListA \fi}

%%%%%%%%%%%%%
%% End of vectors
%%%%%%%%%%%%%

%\newcommand{\fullwidth}{}
%\newcommand{\normalwidth}{}



%% makes a snazzy t-chart for evaluating functions
%\newenvironment{tchart}{\rowcolors{2}{}{background!90!textColor}\array}{\endarray}

%%This is to help with formatting on future title pages.
\newenvironment{sectionOutcomes}{}{} 



%% Flowchart stuff
%\tikzstyle{startstop} = [rectangle, rounded corners, minimum width=3cm, minimum height=1cm,text centered, draw=black]
%\tikzstyle{question} = [rectangle, minimum width=3cm, minimum height=1cm, text centered, draw=black]
%\tikzstyle{decision} = [trapezium, trapezium left angle=70, trapezium right angle=110, minimum width=3cm, minimum height=1cm, text centered, draw=black]
%\tikzstyle{question} = [rectangle, rounded corners, minimum width=3cm, minimum height=1cm,text centered, draw=black]
%\tikzstyle{process} = [rectangle, minimum width=3cm, minimum height=1cm, text centered, draw=black]
%\tikzstyle{decision} = [trapezium, trapezium left angle=70, trapezium right angle=110, minimum width=3cm, minimum height=1cm, text centered, draw=black]


\title[Dig-In:]{Sequences}

\begin{document}
\begin{abstract}
  A function from positive integers to the real numbers is a sequence.
\end{abstract}
\maketitle

A ``sequence'' of numbers is just a list of numbers.  For example,
here is a list of numbers:
\[
1,1, 2, 3, 5, 8, 13, 21, \ldots
\]
Note that numbers in the list can repeat.  And consider those little
dots at the end!  The dots ``\ldots'' signify that the list keeps
going, and going, and going, forever.  Presumably the sequence
continues by following the pattern that the first few ``terms''
suggest.  But what's that pattern?

\begin{warning}
  Maybe you are feeling that this formality is unnecessary, or even
  ridiculous; why can't we just list off a few terms and pick up on
  the pattern intuitively?  As we'll see later, that might be very
  hard to do!  There might be very different, but equally reasonable,
  patterns that start the same way.

  To resolve this ambiguity, it is perhaps not so ridiculous to
  introduce the formalism of ``functions.''  Functions provide a nice
  language for associating numbers (terms) to other numbers
  (indices).
\end{warning}

%\youtube{http://www.youtube.com/watch?v=-o5GtOQD1GY}

To make this talk of ``patterns'' less ambiguous, it is useful to
think of a sequence as a function. We have up until now dealt with
functions whose domains are the real numbers, or a subset of the real
numbers, like $f(x)=\sin (1/x)$.

A real-valued function with domain the natural numbers
$\N=\{1,2,3,\ldots\}$ is a \dfn{sequence}.

Other functions will also be regarded as sequences: the domain might
include $0$ alongside the positive integers, meaning that the
domain is the non-negative integers, $\Z^{\ge0}=\{0,1,2,3,\ldots\}$.  The range of the function is still
allowed to be the real numbers; in symbols, the function $f\colon
\N\to\R$ is a sequence.

Sequences are written down in a few different, but equivalent,
ways; you might see a sequence written as
\begin{align*}
  & a_1, a_2,  a_3, \ldots, \\
  & a_n \\
  & \left(a_n\right)_{n \in \N}, \\
  & \left\{a_n\right\}_{n=1}^\infty, \\
  & \left\{f(n)\right\}_{n=1}^\infty, \quad \mbox{or} \\
  & \left(f(n)\right)_{n \in \N},
\end{align*}
depending on which author you read.  Worse, depending on the
situation, the same author (and this author) might use various
notations for a sequence!  In this textbook, I will usually write
$(a_n)$ if I want to speak of the sequence as a whole (think
\textit{gestalt}) and I will write $a_n$ if I am speaking of a
specific term in the sequence.

Let's summarize the preceding discussion in the following definition.
\begin{definition}\index{sequence} A \dfn{sequence}
  $(a_n)$ is, formally speaking, a real-valued function with domain
  \[
  \{ n \in \Z : n \geq N \},\quad \mbox{for some integer $N$.}
  \]
  Stated more humbly, a sequence assigns a real number to the
  integers starting with an index $N$.

  The ``outputs'' of a sequence are the \dfn{terms} of the
  sequence; the ``$n^{th}$ term'' is the real number that the
  sequence associates to the natural number $n$, and is usually
  written $a_n$. \index{sequence!term} The $n$ in the phrase
  ``$n^{th}$ term'' is called an
  \dfn{index}\index{sequence!index}; the plural of index is
  either indices or indexes, depending on who you ask.  The first
  index $N$ is called the \dfn{initial index}
\end{definition} 

Recall that the natural numbers $\N$ are the counting numbers $1, 2,
3, 4, \ldots$.  If we want our sequence to start at zero, we use
$\Z^{\ge 0}$ as the domain instead.  The fancy symbols $\Z^{\ge 0}$
refer to the non-negative integers, which include zero (since zero
is neither positive nor negative) and also positive integers (since
they certainly aren't negative).

To confuse matters further, some people---especially computer
scientists---might include zero in the natural numbers $\N$.
Mathematics is cultural.


\begin{warning}
  Usually the ``domain'' of a sequence is $\N$ and $\Z^{\ge 0}$.  But
  depending on the context, it may be convenient for a sequence to
  start somewhere else---perhaps with some negative number.  We
  shouldn't let the usual situation of $\N$ or $\Z^{\ge 0}$ get in the
  way of making the best choice for the problem at hand.
\end{warning}

As you can tell, there is a deep tension between precise definition
and a vague flexibility; as instructors, how we navigate that tension
will be a big part of whether we are successful in teaching the
course.  We need to invoke precision when we're tempted to be too
vague, and we need to reach for an extra helping of vagueness when the
formalism is getting in the way of our understanding.  It can be a
tough balance.

\end{document}
